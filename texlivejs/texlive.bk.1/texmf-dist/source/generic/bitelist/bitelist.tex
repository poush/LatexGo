\ProvidesFile{bitelist.tex}[2012/03/29 documenting bitelist.sty]
\title{\textsf{\huge bitelist.sty                   %% \huge 2012/03/19
       }\\---\\``Splitting" a List at a List Inside
       \\in \TeX's Mouth\thanks{This
       document describes version
       \textcolor{blue}{\UseVersionOf{\jobname.sty}}
       of \textsf{\jobname.sty} as of \UseDateOf{\jobname.sty}.}}
% \listfiles
{ \RequirePackage{makedoc} \ProcessLineMessage{}
  \MakeJobDoc{17}
  {\SectionLevelTwoParseInput}  }
\documentclass[fleqn]{article}%% TODO paper dimensions!?
version https://git-lfs.github.com/spec/v1
oid sha256:ded97341479a4e2387b08b2fca0a903a98527805994be66f525785a0cdabece0
size 1168
 %% shared formatting settings
% \ReadPackageInfos{bitelist}
\usepackage{bitelist}
\sloppy
\MDkeywords{macro programming, text filtering, substrings}
\begin{document}
\maketitle
\begin{MDabstract}
'bitelist.sty' provides commands for ``splitting" a token list 
at the first occurrence of a contained token list 
(i.e., for given $\sigma$, $\tau$, 
 return $\beta$ and shortest $\alpha$ s.t.\ $\tau=\alpha\sigma\beta$). 
 As opposed to other packages providing similar features, 
\ (\textit{i})\enspace the method uses \TeX's mechanism of reading 
delimited macro parameters;
\ (\textit{ii})\enspace the splitting macros work by pure expansion, 
without assignments, provided the macro doing the search has been 
defined before processing (e.g., a file);
\ (\textit{iii})\enspace instead of using one macro for a ``substring" 
test and another one to replace the ``substring"---which includes 
extracting corresponding prefix and suffix---, 
the \emph{same} macro that detects the occurrence returns 
the split;
\ (\textit{iv})\enspace 
\httpref{ctan.org/pkg/e-tex}{$\varepsilon$\hbox{-}\TeX} is not required.
\ (And \LaTeX\ is not required.)

This improves the author's \CtanPkgRef{fifinddo}{fifinddo.sty} 
(v0.51---and may once be used there). An elaborated approach 
(additionally to a simpler one) is provided that does not loose 
outer braces of prefix/suffix.

``Substring" detection and ``string" replacement are (implicitly) included 
with respect to certain representations of characters by tokens.
Counting occurrences and ``global" replacement could be achieved 
by applying the operation to earlier results, etc.---so 
this approach seems to be ``fundamental" for a certain larger 
set of list analysis tasks.

The documentation aims to prove the correctness of the methods 
with mathematical rigour.
\par\smallskip\noindent
\strong{Related packages:}\quad 
\ctanpkgref{datatool}, \ctanpkgref{stringstrings}, \ctanpkgref{ted}, 
\ctanpkgref{texapi}, \ctanpkgref{xstring}
\end{MDabstract}
\newpage
\tableofcontents

\section{Task, Background Reasoning, and Usage}
\subsection{The Task Quite Precisely}
\label{sec:task}

Perhaps I should not have written ``splitting" before, 
see Section~\ref{sec:name} why I did so though. 
Actually: 

At first we are dealing with token lists $\tau$ and $\sigma$ 
without braces 
(unless their category code has been changed appropriately)
that can be stored as macros without parameter or in token list registers. 
We want to find out whether $\tau$ contains $\sigma$ (``as a subword") 
in the sense that there are such token lists $\alpha$ and $\beta$ that 
$\tau$ is composed as $\alpha\sigma\beta$, i.e.,
\[\tau=\alpha\sigma\beta\]
and in this case 
we want to get $\alpha$ and $\beta$ of this kind with 
$\alpha$ being the \emph{shortest} possible. 
I.e., if there are such $\gamma$ and $\delta$ that $\tau$
is composed as $\gamma\sigma\delta$, $\alpha$ must be contained 
as a ``prefix" in $\gamma$, 
i.e., $\gamma$ is composed as $\alpha\eta$ for some token list $\eta$. 
The token lists $\alpha$, $\beta$, $\gamma$, $\delta$, $\eta$, 
$\sigma$, and $\tau$ are allowed to be empty throughout.

The task will be extended for some braces in Section~\ref{sec:braces}.

\subsection{Idea of Solution}

\TeX's mechanism of expanding macros (\TeX book Chapter~20)
at least has a built-in mechanism to return such $\alpha$ and $\beta$
\emph{provided} $\tau$ contains $\sigma$. Define
\[`\def<cmd>#1'\sigma`#2'\theta`{<replace-def>}'\]
where $\theta$ must be a token list (maybe of a single token)
that won't occur in $\tau$.\footnote{I am still following others in confusing 
    source code and tokens. I have better ideas, but must expand on them 
    elsewhere. Writing `&\def' rather indicates that it is source code, 
    then $\sigma$ etc. should be replaced by strings that are converted 
    into tokens $\sigma$ etc. 
    <cmd> sometimes is a \emph{string} starting with an escape character, 
    or it is an active character; but sometimes it rather is an ``active" 
    \emph{token} converted from such an escape string or an active character.}
This is a \strong{limitation} of the approach: 
It works for sets of such $\tau$ only that do not contain 
any of a small set of tokens or combinations of them.
('bitelist' will use `\BiteSep', `\BiteStop', and `\BiteCrit', 
 or any other three that can be chosen.)

On the other hand, \TeX's \emph{category codes} 
(\TeX book Chapter~7) can ensure this quite well. 
E.g., we may assume that input ``letters" always have category code 11
(or 12, or one of them), and for $\theta$ we can choose letters 
with \emph{different} category codes such as 3.
Without such tricks, you may often assume that nobody will input 
certain ``silly" commands such as `\BiteStop'. 
(But it may become difficult when you use a package for 
 replacement macros for generating its own documentation \dots)

With a <cmd> as defined above, \TeX\ will
\[\mbox{expand\quad}
    `<cmd>'\tau\theta
  \quad\mbox{to}\quad
    <replace>,\]
where <replace> will be the result of replacing 
\ (a)\enspace all occurrences of `#1' in <replace-def> by $\alpha$ as wanted and
\ (b)\enspace all occurrences of `#2' in <replace-def> by $\beta$ as wanted.
\
I.e., <cmd> returns $\alpha$ as its first argument and $\beta$ as its second argument.
The reason is that <cmd>'s first parameter is delimited by $\sigma$ and the second one by 
$\theta$ in the sense of The~\TeX book p.~203.
Our requirement to get the \emph{shortest} $\alpha$ for the composition of $\tau$ as 
$\alpha\sigma\beta$ is met because \TeX\ indeed looks for the \emph{first} occurrence of 
$\sigma$ at the right of <cmd>. 


\subsection{When We Don't Know \dots}
When $\sigma$ does \emph{not} occur in $\tau$ and we present $\tau\theta$ to <cmd> as 
before, \TeX\ will throw an error saying 
``Use of <cmd> doesn't match its definition."
When the purpose is ``substring detection" only, without returning $\beta$, 
many packages have solved the problem by issuing something like
\[`<cmd>'\tau\sigma\theta\]
Then (still provided $\theta$ does not occurr in $\tau$) 
<cmd>'s second argument is empty \emph{exactly} if $\sigma$ occurs in $\tau$.
This method has, e.g., been employed in \LaTeX's internal &\in@ mechanism 
(e.g., for dealing with package options) and by the \ctanpkgref{substr} package.
\ctanpkgref{datatool} has used the latter's substring test (for $\sigma$)
before calling a macro for replacing 
($\sigma$ by another token list, perhaps thinking of character tokens).

This way you get the wanted $\alpha$ as the first macro argument immediately indeed. 
An obstacle for getting $\beta$ is that <cmd>'s \emph{second} argument now contains 
an occurrence of $\sigma$ that is not an occurrence in $\tau$. 
In \CtanPkgRef{fifinddo}{fifinddo.sty} I didn't have a better idea than using 
another macro to remove the ``dummy text" from the second argument.
I considered it an advantage as compared with 'datatool' that 
\emph{one} macro could do this for \emph{all} replacement jobs, 
while 'datatool' uses \emph{two} macros with $\sigma$ as a delimiter 
for each $\sigma$ to be replaced.

But still, 'fifinddo' has used \emph{two} macros for each replacement, 
the extra one being for presenting $\tau$ to <cmd>, using a job identifier. 
This could be improved within 'fifinddo', but I could never afford 
to take the time for this.

\subsection{The Trick}
\label{sec:trick}

The solution presented here is not very ingenious, 
many students would have found it in an exercise for a math course.
My personal approach was looking at &\GetFileInfo from \LaTeX's 
\ctanpkgref{doc} package. There they try to get \emph{two} occurrences 
of a space token this way:\footnote{We are undoubling the hash marks 
                                    inside the definition text of 
                                    &\GetFileInfo.}
\[`\def\@tempb#1 #2 #3\relax#4\relax{%'\]
and &\@tempb is called as 
\[`\@tempb'\tau`\relax? ? \relax\relax'\]
or with $\tau=<list>$
\[`\@tempb<list>\relax? ? \relax\relax'\]
The final &\relax may not be removed, but for 'doc' it doesn't harm. 
It harms for \emph{me} when I don't want to have a `\relax' in a `.log' file list.
`\empty' would be better, however \dots

The idea is to use a \emph{three}-parameter macro for that \emph{single} occurrence 
of $\sigma$. We introduce a 
``dummy separator" $\zeta$ (or <sep>, `\BiteSep') 
between $\tau$ and the ``dummy text" and a 
``criterion" $\rho$ ($=<crit>$, `\BiteCrit') 
for determining occurrence of $\sigma$ ($=<find>$) in $\tau$ ($=<list>$).
Neither $\zeta$ nor $\rho$ must occur in $\tau$.
We will have definitions about as
\[`\def<cmd>#1'\sigma`#2'\zeta`#3'\theta`{<replace-def>}'\]
or
\[`\def<cmd>#1<find>#2<sep>#3<stop>{<replace-def>}'\]
and $\tau$ will be presented with context
\[`<cmd>'\tau\zeta\sigma\rho\zeta\theta
  \quad\mbox{or}\quad
  <cmd><list><sep><find><crit><sep><stop>
  \]
This ensures that <cmd> finds its parameter delimiters $\sigma$, $\zeta$, 
and $\theta$, in this order. $\sigma$ occurs in $\tau$ exactly if the second 
argument of <cmd> is $\rho$, and in this case the first occurrence 
of the second parameter delimiter $\zeta$ delimits $\tau$. 
Then <cmd>'s first argument is $\alpha$, and the second one is $\beta$, 
as wanted.

<cmd>'s \emph{third} parameter is delimited by the final $\theta$ (`\BiteStop'). 
When $\sigma$ occurs in $\tau$, <cmd>'s third argument starts after the first 
of the two $\zeta$, so it is $\sigma\rho\zeta$. 
It is just ignored, this way <cmd> removes all the ``dummy" material 
after $\tau$. When $\sigma$ does \emph{not} occur in $\tau$, 
we ignore all of its arguments, and the macro that invoked <cmd> 
must decide what to do next, e.g., keeping $\tau$ elsewhere 
for presenting it to another parsing macro resembling <cmd>.


\subsection{Installing and Calling}
The file 'bitelist.sty' is provided ready, installation only requires
putting it somewhere where \TeX\ finds it
(which may need updating the filename data
 base).\urlfoot{ukfaqref}{inst-wlcf}           %% corr. 2011/02/08

Below the `\documentclass' line(s) and above `\begin{document}',
you load 'bitelist.sty' (as usually) by
\begin{verbatim}
  \usepackage{bitelist}
\end{verbatim}
between the `\documentclass' line and `\begin{document}'; 
or by 
\begin{verbatim}
  \RequirePackage{bitelist}
\end{verbatim}
within a package file, or above or without the `\documentclass' line.
Moreover, the package should work \emph{without} \LaTeX\ and may be 
loaded by 
\begin{verbatim}
  \input bitelist.sty
\end{verbatim}
Actually, using the package for macro programming requires understanding 
of pp.~20f.\ of The~\TeX book. On the other hand, the package may be loaded
(without the user noticing it) automatically by a different package that 
uses programming tools from the present package.

\section{Implementation Part I}
\subsection{Package File Header (Legalize)}
version https://git-lfs.github.com/spec/v1
oid sha256:753d0f0b2d108493567f19d40184fd5d5bd86693f4e04b013981f137f230643a
size 14342


\section{Examples/Tests}
\label{sec:demo}
You should find a separate file `bitedemo.tex' 
with examples. It may be run separately with `tex' 
(Plain \TeX)---demonstrating that 'bitelist' is ``\strong{generic}", 
then finish by entering `\bye'. 
With ```latex bitedemo.tex'", end the job by entering `\stop'.
\strong{Expandability} is demonstrated by the `\BiteFind' commands 
running with `\typeout'.
\medskip
\noNiceVerb
\hrule
\verbatiminput{bitedemo.tex}
\hrule
\useNiceVerb

\section{The Package's Name}
\label{sec:name}

This package deals with \TeX's expansion mechanism. 
In Knuth's metaphor, this is \TeX's mouth. 
I am not entirely sure, I have never understood it, 
or I have understood it only for a few days or hours. 
However, the package deals with ``Lists in \TeX's Mouth" 
as described in Alan Jeffrey's 1990 
\tugbartref{tb11-2/tb28jeffrey}{\acro{TUG}boat paper} 
(Volume~11, No.~2, pp.~237--245).\foothttpurlref{% 
    tug.org/TUGboat/tb11-2/tb28jeffrey.pdf} 

``Splitting" in title and abstract is an attempt to describe 
the package brief{}ly without speaking Mathematicalese. 
It roughly refers to certain \Wikienref{string functions} 
in various programming languages\foothttpurlref{%
    en.wikipedia.org/wiki/String\string_functions\string#split}
with ```split'"  in their name.
However, there strings are splitted at separators such as commas. 
I am thinking here that a comma is a certain string ```,'", 
and this can be generalized to ``splitting" at any substring. 
With \TeX, the analogues are (a)~the token with the character code 
of the comma and category code 12, or the token list consisting of this 
single token,---and (b)~other lists of tokens~\dots

Anyway, calling a triple $(\alpha,\sigma,\beta)$ of token lists 
such that $\tau=\alpha\sigma\beta$ a ``split" of $\tau$ 
is not necessarily a bad idea.
Moreover, the blank space example (Section~\ref{sec:space})
is very close to the original idea of splitting at separators, 
a blank space is about as common as a separator as the comma is.

Finally, according to \urlhttpref{en.wiktionary.org}, 
the Proto-Indo-European origin of
\httpref{en.wiktionary.org/wiki/bite}{``to bite"}
just means ``to split."\foothttpurlref{en.wiktionary.org/wiki/bite\string#Etymology}
So in \TeX's mouth, splitting and biting is the same.


\end{document}

VERSION HISTORY

2012/03/26  for v0.1    started 
2012/03/27              pages of motivation etc.
2012/03/28              abstract: "mathematical rigour"; 
                        \section{Implementation}, \section{Task, ...}; 
                        \newpage, \LaTeX\; reference to sec:braces; 
                        "Examples/Tests" halfway; "Package's"; 
                        LaTeX not required, ...
2012/03/29              "Implementation Part I", label sec:demo; 
                        keywords etc. 
