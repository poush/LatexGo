\documentclass[aspectratio=54,10pt,xcolor=x11names]{beamer}
  \usepackage[french]{babel}
  \usepackage[autolanguage]{numprint}
  \usepackage{amsmath}
  \usepackage{booktabs,tabularx}
  \usepackage{metalogo}                % logo \XeLaTeX
  \usepackage{fontawesome}
  \usepackage{listings}
  \usepackage[export]{adjustbox}       % cadre serré autour d'une image
  \usepackage[overlay,absolute]{textpos}

  %% =============================
  %%  Informations de publication
  %% =============================
  \renewcommand{\year}{2016}
  \renewcommand{\month}{11-3}
  \newcommand{\ctanurl}{https://ctan.org/pkg/formation-latex-ul/}

  %% =======================
  %%  Apparence du document
  %% =======================

  %% Polices de caractères
  \usepackage{fontspec}
  \usepackage{unicode-math}
  \defaultfontfeatures{Ligatures=TeX,Scale=0.88}
  \setmainfont{Lucida Bright OT}
  \setsansfont{Lucida Sans OT}
  \setmathfont{Lucida Bright Math OT}
  \setmonofont{Lucida Grande Mono DK}
  \newfontfamily\titles{Myriad Pro}
  \newfontfamily\oldstyle[Numbers=OldStyle]{Myriad Pro}
  \usepackage{microtype}

  %% Couleurs
  \definecolor{comments}{rgb}{0.7,0,0}  % rouge foncé
  \definecolor{alert}{rgb}{0,0.7,0}     % vert foncé
  \definecolor{link}{rgb}{0,0.4,0.6}    % ~RoyalBlue de dvips
  \definecolor{url}{rgb}{0.6,0,0}       % rouge-brun
  \definecolor{rouge}{rgb}{0.85,0,0.07} % rouge bandeau identitaire
  \definecolor{or}{rgb}{1,0.8,0}        % or bandeau identitaire
  \colorlet{emphasis}{link}
  \colorlet{exercices}{gray}

  %% Hyperliens
  \hypersetup{%
    pdfauthor = {Vincent Goulet},
    pdftitle = {Rédaction avec LaTeX - Premiers pas},
    colorlinks = {true},
    linktocpage = {true},
    allcolors = {link},
    urlcolor = {url},
    pdfpagemode = {UseOutlines},
    pdfstartview = {Fit},
    bookmarksopen = {true},
    bookmarksnumbered = {true},
    bookmarksdepth = {subsection}}

  %% Paramètres de beamer
  \useinnertheme{default}
  \useoutertheme[width=10mm,height=2.2\baselineskip]{sidebar}
  \usefonttheme[onlylarge]{structurebold}
  \usefonttheme{professionalfonts}
  \addtobeamertemplate{headline}{\rule{0pt}{12pt}\par}{}
  \setbeamercolor{frametitle}{fg=white,bg=black}
  \setbeamerfont{frametitle}{family=\titles}
  \setbeamercolor{structure}{fg=emphasis,bg=white}
  \setbeamercolor{block title}{fg=black,bg=Snow2}
  \setbeamercolor{alerted text}{fg=black}
  \setbeamerfont{alerted text}{series=\bfseries}
  \setbeamercolor{block title alerted}{fg=white,bg=alert}
  \setbeamercolor{block title example}{fg=white,bg=exercices}
  \setbeamercolor{example text}{fg=exercices,bg=white}
  \setbeamertemplate{frametitle continuation}{}
  \setbeamertemplate{theorems}[numbered]
  \setbeamertemplate{section in toc}[sections numbered]
  \setbeamertemplate{navigation symbols}{}
  \setbeamertemplate{section in sidebar}{}
  \setbeamertemplate{subsection in sidebar}{}
  \setbeamertemplate{section in sidebar shaded}{}
  \setbeamertemplate{subsection in sidebar shaded}{}
  \AtBeginSection[]
  {
    \begin{frame}
      \frametitle{Sommaire}
      \small
      \begin{columns}[t]
        \begin{column}{.5\textwidth}
          \tableofcontents[sections={1-3},%
            currentsection,subsectionstyle=show/show/hide]
        \end{column}
        \begin{column}{.5\textwidth}
          \tableofcontents[sections={4-6},%
            currentsection,subsectionstyle=show/show/hide]
        \end{column}
      \end{columns}
    \end{frame}
  }

  %% Paramétrage de babel pour les guillemets
  \frenchbsetup{og=«, fg=»}

  %% Sections de code source
  \lstloadlanguages{[LaTeX]TeX}
  \lstset{language=[LaTeX]TeX,
    escapeinside=`',
    extendedchars=true,
    inputencoding=utf8/latin1,
    basicstyle=\small\ttfamily\NoAutoSpacing,
    commentstyle=\color{comments}\slshape,
    keywordstyle=\mdseries,
    emphstyle=\color{emphasis}\bfseries,
    backgroundcolor=\color{LightYellow1},
    frame=none,
    showstringspaces=false}

  %%% =========================
  %%%  Nouveaux environnements
  %%% =========================

  %% Exercices
  \theoremstyle{example}
  \newtheorem{exercice}[theorem]{Exercice}

  %% Environnements pour les demo de code; tirés du document
  %% principal.
  \newenvironment{demo}{%
    \begin{beamercolorbox}[wd=\linewidth,sep=6pt]{block body example}}
    {\end{beamercolorbox}}
  \newenvironment{texample}[1][0.45\linewidth]{%
    \noindent\begin{minipage}{#1}%
      \def\producing{\end{minipage}\hfill\begin{minipage}{\dimexpr0.9\linewidth-#1}%
        \hbox\bgroup\kern-.2pt%
        \vbox\bgroup\parindent0pt\relax
        % The 3pt is to cancel the -\lineskip from \displ@y
        \abovedisplayskip3pt \abovedisplayshortskip\abovedisplayskip
        \belowdisplayskip0pt \belowdisplayshortskip\belowdisplayskip
        \noindent}
    }{%
      \par
      % Ensure that a lonely \[\] structure doesn't take up width less than
      % \hsize.
      \hrule height0pt width\hsize
      \egroup\kern-.2pt\egroup
    \end{minipage}%
    \par
  }
  \renewenvironment{quote}{%
    \begin{beamercolorbox}[wd=\linewidth,sep=6pt]{block body example}}
    {\end{beamercolorbox}}
  \newenvironment{texoutput}[1]{%
    \begin{minipage}[t]{#1} \small}{%
    \end{minipage}}

  %% Rubriques conseil du TeXpert
  \newenvironment{conseil}{%
    \begin{alertblock}{Conseil du {\TeX}pert}
      \medskip
      \begin{minipage}[t]{0.1\textwidth}
        \color{alert}\raisebox{-1.2em}[0em][0em]{\huge\faThumbsOUp}
      \end{minipage}
      \begin{minipage}[t]{0.88\textwidth}}%
    {\end{minipage}\end{alertblock}}

  %% =====================
  %%  Nouvelles commandes
  %% =====================

  %% Noms de fonctions, code, environnement, etc.
  \newcommand{\fichier}[1]{\fbox{\texttt{#1}}}
  \newcommand{\class}[1]{\textbf{#1}}
  \newcommand{\pkg}[1]{\textbf{#1}}
  \newcommand{\link}[2]{\href{#1}{#2~\raisebox{-0.2ex}{\faExternalLink}}}
  \newcommand{\capsule}[2]{\href{#1}{\faYoutubePlay~#2}}

  %% «Bouton» de la page de copyright
  \newcommand{\browsebutton}{%
    \setlength{\fboxrule}{0.5pt}%
    \framebox[20mm][c]{%
      \makebox[2.5mm]{\raisebox{-1pt}{\footnotesize\faGithub}}\;%
      {\sffamily Voir sur GitHub}}}

  %% «Bouton» pour accéder à CTAN
  \newcommand{\ctanbutton}{%
    \setlength{\fboxrule}{1pt}%
    \framebox[60mm][l]{%
      \rule[-5pt]{0mm}{16pt}%
      \makebox[7mm]{\raisebox{-2.5pt}{\Large\faExternalLink}}\;%
      {\sffamily Accéder aux fichiers dans CTAN}}}

  %%% =======
  %%%  Varia
  %%% =======

  %% Nouveaux compteurs requis pour ajouter les numéros d'exercices
  %% dans la table des matières. Une page compte deux exercices;
  %% besoin de deux compteurs pour celle-là.
  \newcounter{exerciceref}
  \newcounter{exercicerefb}

  %% Longueurs pour la composition des pages couvertures avant et
  %% arrière.
  \newlength{\banderougewidth} \newlength{\banderougeheight}
  \newlength{\bandeorwidth}    \newlength{\bandeorheight}
  \newlength{\imageheight}
  \newlength{\logoheight}
  \newlength{\gapwidth}

\begin{document}

version https://git-lfs.github.com/spec/v1
oid sha256:de31d482a84a9a78085d21968d997046431a1ca3b0bea77bb439b1f7f911e12d
size 2408

version https://git-lfs.github.com/spec/v1
oid sha256:ec33d34cb7b3a220f1b42dc32af03430a2f7bfaaec9eb5b66cc0dc4a5a70f15e
size 1141

%%% Texte du contrat de licence au début des diapos

\begin{frame}[t,plain,fragile=singleslide]
  \tiny
  \vspace*{10mm}

  {\textcopyright} {\year} Vincent Goulet \\[4mm]

  \includegraphics[height=4mm,keepaspectratio=true]{by-sa} \\%

  Cette création est mise à disposition selon le contrat
  \href{http://creativecommons.org/licenses/by-sa/4.0/deed.fr}{%
    Attribution-Partage dans les mêmes conditions 4.0 International}
  de Creative Commons. En vertu de ce contrat, vous êtes libre de:
  \begin{itemize}
  \item[\color{black}\tiny$\blacktriangleright$]%
    \textbf{partager} --- reproduire, distribuer et communiquer
    l'{\oe}uvre;
  \item[\color{black}\tiny$\blacktriangleright$]%
    \textbf{remixer} --- adapter l'{\oe}uvre;
  \item[\color{black}\tiny$\blacktriangleright$]
    utiliser cette {\oe}uvre à des fins commerciales.
  \end{itemize}
  Selon les conditions suivantes:
  \vspace*{2mm}

  \begin{tabularx}{\linewidth}{@{}lX@{}}
    \raisebox{-5.5mm}[0mm][10mm]{%
      \includegraphics[height=7mm,keepaspectratio=true]{by}}
    & \textbf{Attribution} --- Vous devez créditer l'{\oe}uvre, intégrer
      un lien vers le contrat et indiquer si des modifications ont été
      effectuées à l'{\oe}uvre. Vous devez indiquer ces informations par
      tous les moyens possibles, mais vous ne pouvez suggérer que
      l'Offrant vous soutient ou soutient la façon dont vous avez
      utilisé son {\oe}uvre. \\
    \raisebox{-5.5mm}{\includegraphics[height=7mm,keepaspectratio=true]{sa}}
    & \textbf{Partage dans les mêmes conditions} --- Dans le cas où
      vous modifiez, transformez ou créez à partir du matériel composant
      l'{\oe}uvre originale, vous devez diffuser l'{\oe}uvre modifiée
      dans les même conditions, c'est à dire avec le même contrat avec
      lequel l'{\oe}uvre originale a été diffusée.
  \end{tabularx}
  \vfill

  \textbf{Code source} \\
  \begin{tabularx}{0.6\linewidth}{@{}Xl@{}}
    Le code source de ce document est conservé dans un dépôt
    Git public. &
                         \raisebox{-3pt}{%
                         \href{https://github.com/vigou3/formation-latex-ul/}{%
                         \browsebutton}}
  \end{tabularx}
  \vspace*{4mm}

  \textbf{Crédits} \\
  Concept original de la couverture: Marie-Ève Guérard. \\
  Photo: Olaf Leillinger via
  \href{https://commons.wikimedia.org/wiki/File:Suricata.suricatta.6861.jpg}{%
    Wikimedia Commons}. \\
  Lion de CTAN réalisé par Duane Bibby.
  \vfill
\end{frame}

%%% Local Variables:
%%% mode: latex
%%% TeX-engine: xetex
%%% TeX-master: "formation-latex-ul-diapos"
%%% coding: utf-8
%%% End:

version https://git-lfs.github.com/spec/v1
oid sha256:3a19c30e894717c7eafd54bbe5a2c9d33ae912133f544e27413468c2d5641c87
size 1332


\begin{frame}
  \frametitle{Sommaire}
  \small
  \begin{columns}[t]
    \begin{column}{.5\textwidth}
      \tableofcontents[sections={1-3},hideallsubsections]
    \end{column}
    \begin{column}{.5\textwidth}
      \tableofcontents[sections={4-6},hideallsubsections]
    \end{column}
  \end{columns}
\end{frame}

version https://git-lfs.github.com/spec/v1
oid sha256:ffc677eb9c3b21c2e4595db92b22973b2be484a7a6dfc8bb39c8e30edc5fb5ed
size 6563

version https://git-lfs.github.com/spec/v1
oid sha256:854b1a1c35377f484bd40467a68948719afe2c24d450894846d9d5260dacf6d5
size 11671

\section{Organisation d'un document}

\begin{frame}[plain]
  \begin{conseil}
    Utilisez impérativement les commandes {\LaTeX} pour identifier
    les différentes parties (la structure) d'un document.
  \end{conseil}
\end{frame}

\subsection{Parties d'un document}

\begin{frame}[fragile]
  \frametitle{Titre et page de titre}
  \begin{itemize}
  \item Mise en forme automatique
    \begin{lstlisting}
%% préambule
\title{`\textit{Titre du document}'}
\author{`\textit{Prénom Nom}'}
\date{`\textit{31 octobre 2014}'} % automatique si omis

%% corps du document
\maketitle
    \end{lstlisting}
  \item Mise en forme libre \\[6pt]
    \begin{minipage}{0.45\linewidth}
      \begin{block}{\small classes standards}
\begin{lstlisting}
\begin{titlepage}
  ...
\end{titlepage}
\end{lstlisting}
      \end{block}
    \end{minipage}
    \hfill
    \begin{minipage}{0.45\linewidth}
      \begin{block}{\small classe \class{memoir}}
\begin{lstlisting}
\begin{titlingpage}
  ...
\end{titlingpage}
\end{lstlisting}
      \end{block}
    \end{minipage}
  \end{itemize}
\end{frame}

\begin{frame}[fragile=singleslide]
  \frametitle{Résumé}
  \begin{itemize}
  \item Classes \class{article}, \class{report} ou \class{memoir}:
    résumé créé avec l'environnement
\begin{lstlisting}
\begin{abstract}

\end{abstract}
\end{lstlisting}
  \item Classe \class{ulthese}: résumés français et anglais traités
    comme des chapitres normaux (non numérotés)
  \end{itemize}
\end{frame}

\begin{frame}[fragile=singleslide]
  \frametitle{Sections}
  \begin{itemize}
  \item Découpage du document en sections avec les commandes
\begin{lstlisting}
\part{`\textit{titre}'}
\chapter{`\textit{titre}'}
\section{`\textit{titre}'}
\subsection{`\textit{titre}'}
\end{lstlisting}
\begin{lstlisting}
\subsubsection{`\textit{titre}'}     % à éviter dans un livre
\end{lstlisting}
\begin{lstlisting}
\paragraph{`\textit{titre}'}         % jamais (?) utilisé
\subparagraph{`\textit{titre}'}      % idem
\end{lstlisting}
  \item Numérotation automatique
  \item Commande suivie d'une \verb=*= = section non numérotée
  \item Titre «court» en argument optionnel
  \end{itemize}
\end{frame}

\begin{frame}[fragile=singleslide]
  \frametitle{Annexes}
  \begin{itemize}
  \item Les annexes sont des sections ou des chapitres avec une numérotation
    alphanumérique (A, A.1, ...)
  \item Sections suivantes identifiées comme des annexes par la
    commande
\begin{lstlisting}
\appendix
\end{lstlisting}
  \item Dans le titre, «Chapitre» changé pour «Annexe» le cas échéant
  \end{itemize}
\end{frame}

\begin{frame}[fragile]
  \frametitle{Structure logique d'un livre}
  \framesubtitle{(classes \class{book}, \class{memoir}, \class{ulthese})}
\begin{lstlisting}
\frontmatter
\end{lstlisting}
  \begin{itemize}
    \small
  \item préface, table des matières, etc.
  \item numérotation des pages en chiffres romains (i, ii, ...)
  \item chapitres non numérotés
  \end{itemize}
  \vfill

\begin{lstlisting}
\mainmatter
\end{lstlisting}
  \begin{itemize}
    \small
  \item le contenu à proprement parler
  \item numérotation des pages à partir de 1 en chiffres arabes
  \item chapitres numérotés
  \end{itemize}
  \vfill

\begin{lstlisting}
\backmatter
\end{lstlisting}
  \begin{itemize}
    \small
  \item tout le reste (bibliographie, index, etc.)
  \item numérotation des pages se poursuit
  \item chapitres non numérotés
  \end{itemize}
\end{frame}

\subsection{Table des matières}

\begin{frame}[fragile]
  \frametitle{Table des matières}
  \begin{itemize}
  \item Table des matières produite automatiquement avec
\begin{lstlisting}
\tableofcontents
\end{lstlisting}
  \item Requiert plusieurs compilations
  \item Sections non numérotées pas incluses
  \item Avec \pkg{hyperref}, produit également la table des
    matières du fichier PDF
  \item<2-> Classe \class{memoir} fournit également
\begin{lstlisting}
\tableofcontents*
\end{lstlisting}
    qui n'insère pas la table des matières dans la table des matières
  \item<3-> Aussi disponibles:
\begin{lstlisting}
\listoffigures
\listoftables
\end{lstlisting}
    (et leurs versions \verb=*= dans \class{memoir})
  \end{itemize}
\end{frame}

%%% >>>
\stepcounter{exerciceref}
\subsection{[~Exercice \theexerciceref~]}

\begin{frame}[fragile=singleslide,plain]
  \begin{exercice}
    Utiliser le fichier \fichier{exercice\_parties.tex}.
    \begin{enumerate}
    \item Étudier la structure du document dans le code source.
    \item Ajouter un titre et un auteur au document.
    \item Créer la table des matières du document en le compilant 2 à
      3 fois.
    \item Insérer deux ou trois titres de sections de différents niveaux
      dans le document.
    \item Vous remarquerez que la numérotation cesse à partir des
      sous-sections. C'est une particularité de la classe
      \class{memoir}.

      Recompiler le document après avoir ajouté au préambule la commande
\begin{lstlisting}
\maxsecnumdepth{subsection}
\end{lstlisting}
    \item Ajouter une annexe au document.
    \end{enumerate}
  \end{exercice}
\end{frame}
%%% <<<

\subsection{Renvois automatiques}

\begin{frame}[fragile=singleslide]
  \frametitle{Étiquettes et renvois automatiques}
  \framesubtitle{Parce que l'ordinateur le fera mieux que vous}

  \begin{itemize}
  \item Ne \alert{jamais} renvoyer manuellement à un numéro de
    section, d'équation, de tableau, etc.
  \item «Nommer» un élément avec \verb=\label=
  \item Faire référence par son nom avec \verb=\ref=
  \item Requiert 2 à 3 compilations
  \end{itemize}
\end{frame}

\begin{frame}[plain,fragile=singleslide]
  \begin{lstlisting}[emph={\label,\ref}]
\section{Définitions}
\label{sec:definitions}

Lorem ipsum dolor sit amet, consectetur
adipiscing elit. Duis in auctor dui. Vestibulum
ut, placerat ac, adipiscing vitae, felis.

\section{Historique}

Tel que vu à la section \ref{sec:definitions},
on a...
\end{lstlisting}
  \fbox{\includegraphics[viewport=124 550 484 664,clip=true,width=0.98\linewidth]{exemple-renvoi}}
\end{frame}

\begin{frame}[plain,fragile=singleslide]
  \begin{conseil}
    Adoptez une manière systématique et mnémotechnique de nommer les
    éléments dans un long document afin de vous y retrouver.

    \bigskip %
    Exemple:
\begin{lstlisting}
\label{chap:`\textit{chapitre}'}         % chapitre
\label{sec:`\textit{chapitre}':`\textit{section}'}  % section
\label{tab:`\textit{chapitre}':`\textit{tableau}'}  % tableau
\label{eq:`\textit{chapitre}':`\textit{equation}'}  % équation
\end{lstlisting}
  \end{conseil}
\end{frame}

\subsection{Hyperliens}

\begin{frame}[fragile]
  \frametitle{Renvois automatiques++}
  \begin{itemize}
  \item Paquetage \pkg{hyperref} insère des hyperliens vers les
    renvois dans les fichiers PDF
\begin{lstlisting}
Tel que vu à la section \ref{sec:definitions},
on a...
\end{lstlisting}
    \fbox{\includegraphics[viewport=124 550 484 564,clip=true,width=0.98\linewidth]{exemple-renvoi-hyperref}}
    \vfill
  \item<2-> Commande \verb=\autoref= permet de
    \begin{enumerate}
    \item nommer automatiquement le type de renvoi (section, équation,
      tableau, etc.)
    \item transformer en hyperlien le texte \textbf{et} le numéro
    \end{enumerate}
\begin{lstlisting}
Tel que vu à la \autoref{sec:definitions},
on a...
\end{lstlisting}
    \fbox{\includegraphics[viewport=124 550 484 564,clip=true,width=0.98\linewidth]{exemple-renvoi-autoref}}
  \end{itemize}
\end{frame}

%%% >>>
\stepcounter{exerciceref}
\subsection{[~Exercice \theexerciceref~]}

\begin{frame}[plain]
  \begin{exercice}
    Utiliser le fichier \fichier{exercice\_renvois.tex}.
    \begin{enumerate}
    \item Insérer dans le texte un renvoi au numéro d'une section.
    \item Activer le paquetage \pkg{hyperref} avec l'option
      \texttt{colorlinks} et comparer l'effet d'utiliser
      \texttt{{\textbackslash}ref} ou \texttt{{\textbackslash}autoref}
      pour le renvoi.
    \end{enumerate}
  \end{exercice}
\end{frame}
%%% <<<

%%% Local Variables:
%%% mode: latex
%%% TeX-engine: xetex
%%% TeX-master: "formation-latex-ul-diapos"
%%% End:

\section{Apparence du texte}

\subsection{Police de caractères}

\begin{frame}
  \frametitle{Police de caractères}
  \begin{itemize}
  \item Par défaut, tous les documents {\LaTeX} utilisent la même
    police, {\fontfamily{cmr}\selectfont Computer Modern}
  \item Aujourd'hui plus facile d'utiliser d'autres polices, surtout
    avec {\XeLaTeX}
    \begin{itemize}
    \item voir les fichiers d'exercices et les gabarits de
      \class{ulthese} pour des exemples
    \end{itemize}
  \item Privilégier les polices de grande qualité et très complètes
    (lettres accentuées, grand choix de symboles)
    \begin{itemize}
    \item polices Postscript standards ou leurs clones du projet
      TeX~Gyre
    \end{itemize}
  \item Peu de polices sont adaptées pour les mathématiques
    \begin{itemize}
    \item {\fontfamily{ppl}\selectfont Palatino},
      {\fontfamily{ptm}\selectfont Times}, \textrm{Lucida} (\$) sont des choix sûrs
    \end{itemize}
  \end{itemize}
\end{frame}

\begin{frame}[fragile]
  \frametitle{Changement d'attribut de la police}
  \begin{block}{famille}
    \vspace{-12pt}
    \begin{tabbing}
      \textsf{sans empattements} \qquad\= \verb=\sffamily= \qquad\=
      \verb=\textsf{texte}= \kill
      \small
      \textrm{romain} \> \verb=\rmfamily= \> \verb=\textrm{=\textit{texte}\verb=}= \\
      \texttt{largeur fixe} \> \verb=\ttfamily= \> \verb=\texttt{=\textit{texte}\verb=}= \\
      \textsf{sans empattements} \> \verb=\sffamily= \> \verb=\textsf{=\textit{texte}\verb=}=
    \end{tabbing}
  \end{block}
  \vfill
  \begin{block}{forme}
    \vspace{-12pt}
    \begin{tabbing}
      \textsf{sans empattements} \qquad\= \verb=\sffamily= \qquad\=
      \verb=\textsf{texte}= \kill
      \small
      \textup{\rmfamily droit} \> \verb=\upshape= \> \verb=\textup{=\textit{texte}\verb=}= \\
      \textit{\rmfamily italique} \> \verb=\itshape= \> \verb=\textit{=\textit{texte}\verb=}= \\
      \textsl{penché} \> \verb=\slshape= \> \verb=\textsl{=\textit{texte}\verb=}= \\
      \textsc{\rmfamily petites capitales} \> \verb=\scshape= \> \verb=\textsc{=\textit{texte}\verb=}=
    \end{tabbing}
  \end{block}
  \vfill
  \begin{block}{série}
    \vspace{-12pt}
    \begin{tabbing}
      \textsf{sans empattements} \qquad\= \verb=\sffamily= \qquad\=
      \verb=\textsf{texte}= \kill
      \rmfamily\small
      \textmd{\rmfamily moyen} \> \verb=\mdseries= \> \verb=\textmd{=\textit{texte}\verb=}= \\
      \textbf{\rmfamily gras} \> \verb=\bfseries= \> \verb=\textbf{=\textit{texte}\verb=}= \\
    \end{tabbing}
    \vfill
  \end{block}
  \begin{picture}(0,0)
    \thicklines\color{blue}
    \put(98,30){\dashbox{2}(62,190){}}
    \put(88,20){
      \begin{minipage}[t]{75\unitlength}
        \footnotesize\centering
        s'applique à tout le texte qui suit
      \end{minipage}}
  \end{picture}
  \begin{picture}(0,0)
    \thicklines\color{blue}
    \put(162,30){\dashbox{2}(82,190){}}
    \put(158,20){
      \begin{minipage}[t]{85\unitlength}
        \footnotesize\centering
        s'applique au texte en argument
      \end{minipage}}
  \end{picture}
\end{frame}

\subsection{Taille de la police}

\begin{frame}[fragile]
  \frametitle{Taille de la police}
  \vspace{-2pt}
  \begin{block}{commandes standards}
    \vspace{-10pt}
    \begin{tabbing}
      \verb=\footnotesize= \quad\= \kill
      \verb=\tiny= \> {\tiny vraiment petit} \\
      \verb=\scriptsize= \> {\scriptsize encore plus petit} \\
      \verb=\footnotesize= \> {\footnotesize plus petit} \\
      \verb=\small= \> {\small petit} \\
      \verb=\normalsize= \> {\normalsize normal} \\
      \verb=\large= \> {\large grand} \\
      \verb=\Large= \> {\Large plus grand} \\
      \verb=\LARGE= \> {\LARGE encore plus grand} \\
      \verb=\huge= \> {\huge énorme} \\
      \verb=\Huge= \> {\Huge encore plus énorme}
    \end{tabbing}
  \end{block}
  \vspace{-10pt}
  \pause
  \begin{block}{ajouts de \class{memoir} (et donc \class{ulthese})}
    \vspace{-10pt}
    \begin{tabbing}
      \verb=\footnotesize= \quad\= \kill
      \verb=\miniscule= \quad\> [$<$ \verb=\tiny=] \\
      \verb=\HUGE= \> [$>$ \verb=\Huge=] \\
    \end{tabbing}
  \end{block}
\end{frame}

\subsection{Italique}

\begin{frame}[fragile]
  \frametitle{Italique}
  \begin{itemize}
  \item<1-> Une des propriétés les \emph{plus utilisées} dans le texte
    \vfill
  \item<1-> Commande sémantique:
\begin{lstlisting}
\emph{`\textit{texte}'}
\end{lstlisting}
    \vfill
  \item<2-> Par défaut: texte en italique dans texte droit et vice versa
    \begin{demo}
      \small
      \begin{texample}[0.48\linewidth]
\begin{lstlisting}
C'était un peu \emph{rough}
par moments
\end{lstlisting}
        \producing
        C'était un peu \emph{rough} par moments
      \end{texample}

      \begin{texample}[0.48\linewidth]
\begin{lstlisting}
Il m'a dit: «\emph{Enough
\emph{poutine} for the
week!}»
\end{lstlisting}
        \producing
        Il m'a dit: «\emph{Enough \emph{poutine} for the week!}»
      \end{texample}
    \end{demo}
  \item<3-> Pas de commande pour souligner en {\LaTeX\dots} et ce n'est
    pas une omission!
  \end{itemize}
\end{frame}

\subsection{Listes}

\begin{frame}[fragile]
  \frametitle{Listes}
  \begin{itemize}
  \item Deux principales sortes de listes:
    \begin{enumerate}
    \item \alert{à puce} avec environnement \verb=itemize=
    \item \alert{numérotée} avec environnement \verb=enumerate=
    \end{enumerate}
  \item Possible de les imbriquer les unes dans les autres
  \item Marqueurs adaptés automatiquement jusqu'à 4 niveaux
  \end{itemize}
  \pause

\begin{lstlisting}
\begin{itemize}
\item Deux principales sortes de listes:
  \begin{enumerate}
  \item à puce avec environnement \verb=itemize=
  \item numérotée avec environnement \verb=enumerate=
  \end{enumerate}
\item Possible de les imbriquer les unes
  dans les autres
\item Marqueurs adaptés automatiquement jusqu'à 4 niveaux
\end{itemize}
\end{lstlisting}
\end{frame}

\begin{frame}[plain]
  \begin{conseil}
    \begin{itemize}
    \item {\LaTeX} permet de configurer à peu près toutes les facettes
      de la présentation des listes (puces, alignement, espacement).
    \item Plusieurs paquetages facilitent la configuration.
    \item Nous suggérons \pkg{enumitem} pour une configuration simple.
    \end{itemize}
  \end{conseil}
\end{frame}

\subsection{Notes de bas de page}

\begin{frame}[fragile]
  \frametitle{Notes de bas de page}
  \begin{itemize}
  \item Note de bas de page insérée avec la commande
\begin{lstlisting}
\footnote{`\textit{texte de la note}'}
\end{lstlisting}
  \item Commande doit suivre immédiatement le texte à annoter
  \item Méthode recommandée
\begin{lstlisting}[emph=footnote]
... fera remarquer que Pierre Lasou\footnote{%
  Spécialiste en ressources documentaires} %
fut d'une grande aide dans la préparation de ...
\end{lstlisting}
  \item Numérotation et disposition automatiques
  \end{itemize}
\end{frame}

\subsection{Code source}

\begin{frame}[fragile=singleslide]
  \frametitle{Code source}
  \begin{itemize}
  \item Environnement \verb=verbatim=
\begin{lstlisting}
\begin{verbatim}
Texte disposé exactement tel qu'il est tapé
dans une police à largeur fixe
\end{verbatim}
\end{lstlisting}
  \item Commande \verb=\verb= dont la syntaxe est
\begin{lstlisting}
\verb`\textit{c}'`\textit{source}'`\textit{c}'
\end{lstlisting}
    où \textit{c} est un caractère quelconque ne se trouvant pas dans
    \textit{source}
  \item Pour usage plus intensif, voir le paquetage \pkg{listings}
  \end{itemize}
\end{frame}

%%% >>>
\stepcounter{exerciceref}
\subsection{[~Exercice \theexerciceref~]}

\begin{frame}[plain,fragile=singleslide]
  \begin{exercice}
    \begin{enumerate}
    \item Ouvrir le fichier \fichier{exercice\_complet.tex} et en
      étudier le code source, puis le compiler.
    \item Supprimer l'option \texttt{article} au chargement de la
      classe et compiler de nouveau le document. Observer l'effet de
      cette option.
    \item Effectuer les modifications suivantes au document.
      \begin{enumerate}[a)]
      \item Dernier paragraphe de la première section, placer toute la
        phrase débutant par \texttt{«De simple dérivé»} à l'intérieur
        d'une commande \texttt{{\textbackslash}emph}.
      \item Changer la puce des listes en spécifiant le symbole
        \texttt{\$>\$} pour \verb=ItemLabeli= dans la commande
        \verb=\frenchbsetup= du préambule.
      \end{enumerate}
    \end{enumerate}
  \end{exercice}
\end{frame}
%%% <<<


%%% Local Variables:
%%% mode: latex
%%% TeX-engine: xetex
%%% TeX-master: "formation-latex-ul-diapos"
%%% End:

version https://git-lfs.github.com/spec/v1
oid sha256:72deadea01a856d708161717b6c23e76209fdcddb59fdd2d3f0344cf69412d0d
size 4103

version https://git-lfs.github.com/spec/v1
oid sha256:d8ed72a840d6f2ae5745987904112e9572636d466486f09b152fdd3b633a1e03
size 1751


\begin{frame}
  \frametitle{Et la suite?}

  \begin{columns}
    \begin{column}{.5\textwidth}
      Le document de référence couvre des concepts plus avancés:
      \begin{itemize}
        \small
      \item boîtes, tableaux et figures
      \item équations mathématiques élaborées
      \item bibliographie et citations
      \item commandes et environnement sur mesure
      \item changement de police
      \item diapositives
      \item rapports avec analyse intégrée
      \item etc.
      \end{itemize}
      Références additionnelles dans l'introduction.
    \end{column}
    \begin{column}{.5\textwidth}
      \includegraphics[width=\linewidth,frame]{formation-latex-ul}
    \end{column}
  \end{columns}
\end{frame}

{
  \setbeamercolor{background canvas}{bg=black}

  \begin{frame}[plain]
    \advance\textwidth-10mm
    \hsize\textwidth
    \columnwidth\textwidth %
    \scriptsize\color{lightgray} %
    \vfill
    \begin{center}
      \begin{minipage}{0.7\textwidth}
        \raggedright %
        Ce document a été produit avec le système de mise en page
        {\XeLaTeX} à partir de la classe \textbf{beamer}. Le texte
        principal est en Lucida Sans~OT, le code informatique en
        Lucida Grande Mono~DK et les titres en Adobe Myriad~Pro. Les
        icônes proviennent de la police Font~Awesome.
      \end{minipage}
    \end{center}
    \vfill
  \end{frame}
}

%%% Local Variables:
%%% mode: latex
%%% TeX-master: "formation-latex-ul-diapos"
%%% End:

\begingroup

\TPGrid{3}{36}
\textblockorigin{0mm}{0mm}
\setlength{\parindent}{0mm}
\setlength{\banderougewidth}{2\TPHorizModule}
\setlength{\bandeorwidth}{\TPHorizModule}
\setlength{\gapwidth}{0.75pt}
\addtolength{\bandeorwidth}{-\gapwidth}

\begin{frame}[plain]
  %% bandeau identitaire
  \begin{textblock*}{125mm}[0,1](0mm,30\TPVertModule)
    \textcolor{or}{\rule{\bandeorwidth}{\TPVertModule}}%      % bande or
    \rule{\gapwidth}{0pt}%                                    % filet
    \textcolor{rouge}{\rule{\banderougewidth}{\TPVertModule}} % bande rouge
  \end{textblock*}
\end{frame}
\endgroup

%%% Local Variables:
%%% mode: latex
%%% TeX-engine: xetex
%%% TeX-master: "formation-latex-ul-diapos"
%%% End:


\end{document}

%%% Local Variables:
%%% TeX-engine: xetex
%%% mode: latex
%%% TeX-master: t
%%% End:
