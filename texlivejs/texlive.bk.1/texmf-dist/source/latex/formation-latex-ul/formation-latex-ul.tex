\documentclass[letterpaper,11pt,x11names,english,french]{memoir}
  \usepackage{natbib,url,bibentry}
  \usepackage{babel}
  \usepackage[autolanguage]{numprint}
  \usepackage{amsmath,amsthm}
  \usepackage[shortlabels]{enumitem}
  \usepackage{graphicx}
  \usepackage{framed}                  % environnement titled-frame
  \usepackage{manfnt}                  % \mantriangleright (puce)
  \usepackage{dirtree}                 % arbre pour exercice sur \include
  \usepackage{metalogo}                % logo de \XeLaTeX logo
  \usepackage{mflogo}                  % logo de Metafont
  \usepackage{fontawesome}             % plusieurs icônes
  \usepackage{applekeys}               % touches Mac
  \usepackage{answers}                 % exercices et solutions
  \usepackage{listings}                % code informatique
  \usepackage[absolute]{textpos}       % disposition d'images
  \usepackage{pgf}                     % transparence pour couverture avant
  \usepackage{ifthen}                  % exécution conditionnelle


  %%% =============================
  %%%  Informations de publication
  %%% =============================
  \renewcommand{\year}{2016}
  \renewcommand{\month}{11-3}
  \newcommand{\ISBN}{978-2-9811416-7-5}
  \newcommand{\ctanurl}{https://ctan.org/pkg/formation-latex-ul/}

  %%% ===================
  %%%  Style du document
  %%% ===================

  %% Polices de caractères
  \usepackage{fontspec}
  \usepackage[bold-style=ISO]{unicode-math}
  \defaultfontfeatures{Scale=0.92}
  \setmainfont[Ligatures=TeX,Numbers=OldStyle]{Lucida Bright OT}
  \setmathfont{Lucida Bright Math OT}
  \setmonofont{Lucida Grande Mono DK}
  \setsansfont[Scale=1.0,Numbers=OldStyle]{Myriad Pro}
  \newfontfamily\fullcaps[Letters=Uppercase,Numbers=Uppercase]{Myriad Pro}
  \usepackage[babel=true]{microtype}
  \usepackage{icomma}

  %% Polices additionnelles pour le chapitre trucs et astuces
  \newfontfamily\CM{cmunrm.otf}                       % Computer Modern
  \newfontfamily\Times{texgyretermes-regular.otf}     % Times
  \newfontfamily\Palatino{texgyrepagella-regular.otf} % Palatino
  \newfontfamily\Bookman{texgyrebonum-regular.otf}    % Bookman
  \newfontfamily\NewCent{texgyreschola-regular.otf}   % New Cent. Sch.
  \newfontfamily\Charter{XCharter-Roman.otf}          % Charter
  \newfontfamily\Helvet{texgyreheros-regular.otf}     % Helvetica

  %% Couleurs
  \usepackage{xcolor}
  \definecolor{comments}{rgb}{0.7,0,0}    % rouge foncé
  \definecolor{link}{rgb}{0,0.4,0.6}      % ~RoyalBlue de dvips
  \definecolor{url}{rgb}{0.6,0,0}         % rouge-brun
  \definecolor{citation}{rgb}{0,0.5,0}    % vert foncé
  \definecolor{ULlinkcolor}{rgb}{0,0,0.3} % de ulthese.cls
  \definecolor{rouge}{rgb}{0.85,0,0.07}   % rouge bandeau identitaire
  \definecolor{or}{rgb}{1,0.8,0}          % or bandeau identitaire

  %% Hyperliens
  \usepackage{hyperref}
  \hypersetup{%
    pdfauthor = {Vincent Goulet},
    pdftitle = {Rédaction avec LaTeX},
    colorlinks = {true},
    linktocpage = {true},
    urlcolor = {url},
    linkcolor = {link},
    citecolor = {citation},
    pdfpagemode = {UseOutlines},
    pdfstartview = {Fit},
    bookmarksopen = {true},
    bookmarksnumbered = {true},
    bookmarksdepth = {subsubsection}}
  \setlength{\XeTeXLinkMargin}{1pt}

  %% Étiquettes de \autoref (redéfinitions compatibles avec babel).
  %% Attention! Les % à la fin des lignes sont importants sinon des
  %% blancs apparaissent dès que la commande \selectlanguage est
  %% utilisée, notamment dans la bibliographie.
  \addto\extrasfrench{%
    \def\subsectionautorefname{section}%
    \def\figureautorefname{figure}%
    \def\tableautorefname{tableau}%
    \def\exempleautorefname{exemple}%
    \def\exerciceautorefname{exercice}%
    \def\appendixautorefname{annexe}%
  }

  %% Table des matières (inspirée de classicthesis.sty)
  \renewcommand{\cftchapterleader}{\hspace{1.5em}}
  \renewcommand{\cftchapterafterpnum}{\cftparfillskip}
  \renewcommand{\cftsectionleader}{\hspace{1.5em}}
  \renewcommand{\cftsectionafterpnum}{\cftparfillskip}

  %% Titres des chapitres
  \chapterstyle{hangnum}
  \renewcommand{\chaptitlefont}{\normalfont\Huge\sffamily\bfseries\raggedright}

  %% Marges, entêtes et pieds de page
  \setlength{\marginparsep}{7mm}
  \setlength{\marginparwidth}{20mm}
  \setlength{\headwidth}{\textwidth}
  \addtolength{\headwidth}{\marginparsep}
  \addtolength{\headwidth}{\marginparwidth}
  \addtolength{\marginparwidth}{15mm} % plus d'espace pour titres de documentation

  %% Titres des sections et sous-sections
  \setsecheadstyle{\normalfont\Large\sffamily\bfseries\raggedright}
  \setsubsecheadstyle{\normalfont\large\sffamily\bfseries\raggedright}
  \maxsecnumdepth{subsection}
  \setsecnumdepth{subsection}

  %% Listes. Paramétrage avec enumitem.
  \setlist[enumerate]{leftmargin=*,align=left}
  \setlist[enumerate,2]{label=\alph*),labelsep=*,leftmargin=1.5em}
  \setlist[enumerate,3]{label=\roman*),labelsep=*,leftmargin=1.5em,align=right}
  \setlist[itemize]{leftmargin=*,align=left}

  %% Paramétrage de babel
  \frenchbsetup{%
    StandardItemizeEnv=true,       % format standard des listes
    ThinSpaceInFrenchNumbers=true, % espace fine dans les nombres
    ItemLabeli=\mantriangleright,  % puce premier niveau
    ItemLabelii=\textendash,       % puce second niveau
    og=«, fg=»                     % caractères « et » sont les guillemets
  }
  \def\frenchfigurename{{\scshape Fig.}}
  \def\frenchtablename{{\scshape Tab.}}

  %% Sections de code source
  \lstloadlanguages{[LaTeX]TeX}
  \lstset{language=[LaTeX]TeX,
    basicstyle=\ttfamily\NoAutoSpacing,
    keywordstyle=\mdseries,
    commentstyle=\color{comments}\slshape,
    emphstyle=\bfseries,
    escapeinside=`',
    extendedchars=true,
    showstringspaces=false,
    backgroundcolor=\color{LightYellow1},
    frame=lr, rulecolor=\color{LightYellow1},
    xleftmargin=3.4pt, xrightmargin=3.4pt}

  %%% =========================
  %%%  Nouveaux environnements
  %%% =========================

  %% Exemples
  \theoremstyle{definition}
  \newtheorem{exemple}{Exemple}[chapter]

  %% Exercices et réponses
  \Newassociation{sol}{solution}{solutions}
  \newcounter{exercice}[chapter]
  \renewcommand{\theexercice}{\thechapter.\arabic{exercice}}
  \newenvironment{exercice}[1][]{%
    \begin{list}{}{%
        \refstepcounter{exercice}
        \ifthenelse{\equal{#1}{nosol}}{%
          \renewcommand{\makelabel}{\bfseries\theexercice}}{%
          \hypertarget{ex:\theexercice}{}
          \Writetofile{solutions}{\protect\hypertarget{sol:\theexercice}{}}
          \renewcommand{\makelabel}{%
            \bfseries\protect\hyperlink{sol:\theexercice}{\theexercice}}}
        \settowidth{\labelwidth}{\bfseries\theexercice}
        \setlength{\leftmargin}{\labelwidth}
        \addtolength{\leftmargin}{\labelsep}
        \setlist[enumerate,1]{label=\alph*),labelsep=*,leftmargin=1.5em}
        \setlist[enumerate,2]{label=\roman*),labelsep=0.5em,align=right}}
      \item}
    {\end{list}}
  \renewenvironment{solution}[1]{%
    \begin{list}{}{%
        \renewcommand{\makelabel}{%
          \bfseries\protect\hyperlink{ex:#1}{#1}}
        \settowidth{\labelwidth}{\bfseries #1}
        \setlength{\leftmargin}{\labelwidth}
        \addtolength{\leftmargin}{\labelsep}
        \setlist[enumerate,1]{label=\alph*),labelsep=*,leftmargin=1.5em}
        \setlist[enumerate,2]{label=\roman*),labelsep=0.5em,align=right}}
      \item}
    {\end{list}}

  %% Démo de code LaTeX. Le code de 'texample' et 'eqxample' est
  %% repris de amsldoc.tex avec des petits ajustements.
  \newenvironment{demo}{%
    \begin{trivlist}\item}{%
    \end{trivlist}}
  \newenvironment{texample}[1][0.5\linewidth]{%
    \noindent\begin{minipage}{#1}%
      \def\producing{\end{minipage}\hfill\begin{minipage}{\dimexpr0.97\linewidth-#1}%
        \hbox\bgroup\kern-.2pt%
        \vbox\bgroup\parindent0pt\relax
        % The 3pt is to cancel the -\lineskip from \displ@y
        \abovedisplayskip3pt \abovedisplayshortskip\abovedisplayskip
        \belowdisplayskip0pt \belowdisplayshortskip\belowdisplayskip
        \noindent}
    }{%
      \par
      % Ensure that a lonely \[\] structure doesn't take up width less than
      % \hsize.
      \hrule height0pt width\hsize
      \egroup\kern-.2pt\egroup
    \end{minipage}%
    \par
  }
  \newenvironment{eqxample}{%
    \noindent\begin{minipage}{.5\columnwidth}%
      \def\producing{\end{minipage}\hfill\begin{minipage}{.45\columnwidth}%
        \hbox\bgroup\kern-.2pt\vrule width.2pt%
        \vbox\bgroup\parindent0pt\relax
        % The 3pt is to cancel the -\lineskip from \displ@y
        \abovedisplayskip3pt \abovedisplayshortskip\abovedisplayskip
        \belowdisplayskip0pt \belowdisplayshortskip\belowdisplayskip
        \noindent}
    }{%
      \par
      % Ensure that a lonely \[\] structure doesn't take up width less than
      % \hsize.
      \hrule height0pt width\hsize
      \egroup\vrule width.2pt\kern-.2pt\egroup
    \end{minipage}%
    \par
  }

  %% Un exemple du chapitre Trucs et astuces nécessite des
  %% environnements 'lstlisting' imbriqués, ce que ne digère pas
  %% LaTeX. La ruse consiste à définir un environnement équivalent qui
  %% porte simplement un autre nom.
  \lstnewenvironment{vglisting}{\lstset{deletetexcs={int,include}}}{}

  %% Exemples de notices bibliographiques
  \newenvironment{bibexample}[1][\linewidth]{%
    \begin{minipage}[t]{#1}%
      \begin{trivlist}}
      {\end{trivlist}\end{minipage}}

  %% Encadré générique pour les remarques importantes et autres
  %% comportant une icône sur la gauche. Argument: symbole à
  %% placer sur la gauche (obligatoire).
  \newenvironment{infobox}[1]{%
    \setlength{\FrameRule}{1pt}
    \begin{table}[h]%
      \begin{framed}%
        \noindent
        \begin{minipage}{0.1\linewidth}
          \raisebox{-1.5em}[0em][0em]{\HUGE#1}
        \end{minipage}
        \begin{minipage}[t]{0.88\linewidth}}%
        {\end{minipage}\end{framed}\end{table}}

  %% Remarques importantes
  \newenvironment{important}{%
    \begin{infobox}{\faExclamationCircle}}%
    {\end{infobox}}

  %% Informations
  \newenvironment{information}{%
    \begin{infobox}{\faInfoCircle}}%
    {\end{infobox}}

  %% Encadré avec titre (basé sur 'titled-frame' de framed pour les
  %% conseils du TeXpert. Cet environnement est laissé flottant.
  \newenvironment{conseil}{%
    \colorlet{TFFrameColor}{black}%
    \colorlet{TFTitleColor}{white}%
    \begin{table}%
      \begin{titled-frame}{\sffamily Conseil du {\TeX}pert}%
        \noindent
        \begin{minipage}{0.1\linewidth}
          \raisebox{-1.5em}[0em][0em]{\HUGE\faThumbsOUp}
        \end{minipage}
        \begin{minipage}[t]{0.88\linewidth}}%
        {\end{minipage}\end{titled-frame}\end{table}}

  %%% =======
  %%%  Index
  %%% =======
  \renewcommand{\preindexhook}{%
    Cet index contient des références aux commandes et environnements
    {\LaTeX}, ainsi qu'aux noms de paquetages et de classes. %
    Le premier numéro indique habituellement, mais pas toujours,
    la page où un concept est introduit, défini ou expliqué.%
    \vskip\onelineskip}
  \lstset{language=[AlLaTeX]TeX,
    morekeywords={align,align*,aligned,bmatrix,cases,equation*,%
      figure,gather,lstlisting,multline,quote,split,%
      table,tabular,tabularx},
    deletekeywords={document},   % répéter dans deletetexcs
    moretexcs={toprule,midrule,bottomrule,%
      includegraphics,reflectbox,resizebox,rotatebox,scalebox,%
      includepdf,frenchfigurename,frenchtablename,%
      newsubfloat,subcaption,%
      bm,dfrac,tfrac,iint,text,mathcal,mathbb,eqref,symbf,%
      citet,citep,citeauthor,citeyear,%
      setmainfont,setsansfont,setmonofont,setmathfont,%
      color,textcolor,definecolor,colorlet,hypersetup},
    deletetexcs={document,documentclass,usepackage,begin,end,LaTeX,TeX,%
      normalfont,bfseries,textbf,itshape,scshape,sffamily,ttfamily,texttt,%
      emph,small,Huge,raggedright,%
      hfill,def,a,b,c,d,em,i,j,l,r,t},
    index=[1][keywords],        % environnements
    indexstyle=[1]\ixenv,
    index=[2][texcs],           % commandes
    indexstyle=[2]\ixcmd}
  \newcommand{\ixenv}[1]{\index{#1 env@\Pe{#1} (environnement)}%
    \index{environnement!#1@\Pe{#1}}}
  \newcommand{\ixcmd}[1]{\index{#1@\string\cs{#1}}}
  \makeindex

  %%% =====================
  %%%  Nouvelles commandes
  %%% =====================

  %% Noms de fonctions, code, environnement, etc.
  \newcommand{\code}[1]{\texttt{#1}}
  \newcommand{\fichier}[1]{\texttt{#1}}
  \newcommand{\class}[1]{\textsf{#1}\index{#1 class@\textsf{#1} (classe)}%
    \index{classe!#1}}
  \newcommand{\pkg}[1]{\textbf{#1}\index{#1 pkg@\textbf{#1} (paquetage)}%
    \index{paquetage!#1}}
  \newcommand{\Pe}[1]{\code{#1}}         % tiré de la doc de memoir
  \newcommand{\Ie}[1]{\Pe{#1}\ixenv{#1}} % idem
  \newcommand{\mat}[1]{\symbf{#1}}       % en mode mathématique

  %% Modification de commandes tirées de memoir.cls servant à afficher
  %% des noms de commandes.
  %% - \cmdprint est modifiée pour que le nom de la commande ne soit
  %%   pas en italique;
  %% - \cmd est modifiée pour utiliser @ comme séparateur dans \index
  %%   et pour utiliser \cs plutôt que \cmdprint pour afficher le nom de
  %%   la commande (afin d'obtenir le même format d'entrée d'index
  %%   qu'avec \ixcmd ci-dessus);
  %% - \pixbsbs et \pixabang servent respectivement à afficher et
  %%   indexer \\ et \! ;
  %% - \pixbar sert à afficher et indexer \| avec un hack pour
  %%   contourner un problème d'insertion de l'hyperlien vers le
  %%   numéro de page: l'entrée est triée sur le symbole [ plutôt que
  %%   sur |.
  \renewcommand{\cmdprint}[1]{\textup{\texttt{\string#1}}}
  \makeatletter
  \renewcommand{\cmd}[1]{\cmdprint{#1}%
    \index{\expandafter\@gobble\string#1@\string\cs{\expandafter\@gobble\string#1}}}
  \makeatother
  \newcommand*{\pixbsbs}{\cmdprint{\\}\index{"\ @\string\cs{}\bs}}
  \newcommand*{\pixabang}{\cmdprint{\!}\index{"!@\string\cs{}\texttt{"!}}}
  \newcommand*{\pixbar}{\cmdprint{\|}\index{[@\string\cs{}\texttt{\textbar}}}

  %% Indications de capsule vidéo
  \newcommand{\capsule}[2]{\href{#1}{#2~\raisebox{-0.2ex}{\Large\faYoutubePlay}}}

  %% Hyperlien avec symbole de lien externe juste après
  \newcommand{\link}[2]{\href{#1}{#2~\raisebox{-0.2ex}{\faExternalLink}}}

  %% Lien vers documentation dans la marge
  %% usage: \doc[documentation]{nom_fichier}{url}
  \newcommand{\doc}[3][documentation]{\link{#3}{#1}%
    \ifthenelse{\equal{#2}{}}{}{\marginpar%
      [\hfill\faBookmark~\fichier{#2}]%
      {\faBookmark~\fichier{#2}}}}

  %% Suppression de l'hyperlien
  \newcommand{\nolink}[1]{\begin{NoHyper}#1\end{NoHyper}}

  %% «Bouton» de la page de copyright
  \newcommand{\browsebutton}{%
    \setlength{\fboxrule}{1pt}%
    \framebox[40mm][c]{%
      \rule[-5pt]{0mm}{16pt}%
      \makebox[7mm]{\raisebox{-2pt}{\LARGE\faGithub}}\;%
      {\sffamily Voir sur GitHub}}}

  %% Pour le tableau des commandes d'espacement en mode mathématique.
  %% Pris de la doc de amsmath.
  \newcommand{\lspx}{\mathord{\dashv\mkern-3mu}}
  \newcommand{\rspx}{\mathord{\mkern-2mu\vdash}}
  \newcommand{\spx}[1]{$\lspx #1\rspx$}

  %% Logo BIBTeX; tiré de http://bit.ly/1RQqUnG
  \newcommand{\BibTeX}{\rmfamily B\kern-.05em{\scshape i\kern-.025em b}\kern-.08em \TeX}

  %% Chapitre sur les bibliographies: des références bibliographiques
  %% sont insérées dans le texte avec \bibentry. Certaines commandes
  %% de francaisbst.tex sont alors utilisées, mais non encore
  %% définies. Répétées ici. De plus, il faut définir ici la commande
  %% \enquote plutôt que dans francais.bst. C'est pourquoi il y a une
  %% version modifiée de ce fichier dans ces sources.
  %% Voir http://bit.ly/1MORZmp
  \global\def\bbland{et}
  \global\def\bbledn{\'ed.}
  \global\def\bblfourtho{4{\ieme}}
  \global\def\bblth{{\ieme}}
  \global\def\bblvol{vol.}
  \def\bblno{\no{}}
  \def\bblpp{p.}
  \newcommand{\enquote}[1]{\guillemotleft#1\guillemotright}

  %%% =======
  %%%  Varia
  %%% =======

  %% Style de la bibliographie
  \bibliographystyle{francais}

  %% Longueurs pour la composition des pages couvertures avant et
  %% arrière.
  \newlength{\banderougewidth} \newlength{\banderougeheight}
  \newlength{\bandeorwidth}    \newlength{\bandeorheight}
  \newlength{\imageheight}
  \newlength{\logoheight}
  \newlength{\gapwidth}


  %%% ============
  %%%  Page titre
  %%% ============
  \title{\protect\raggedright%
    \sffamily\bfseries
    \fontsize{40}{40}\selectfont
    Rédaction avec \\
    \rmfamily\mdseries
    \fontsize{45}{45}\selectfont
    \raisebox{10pt}{{\textbackslash}title}%
    \fontsize{80}{80}\selectfont%
    \{%
    \fontsize{70}{70}\selectfont%
    \LaTeX
    \fontsize{80}{80}\selectfont%
    \}}
  \author{\protect\raggedright%
    \sffamily\bfseries
    \fontsize{17}{20}\selectfont
    Vincent Goulet \\
    \mdseries
    \fontsize{15}{20}\selectfont
    Professeur titulaire \textbar\ École d'actuariat}
  \date{%
    \sffamily\mdseries\fontsize{15}{20}\selectfont
    Édition {\fullcaps\year}.\month}

%  \includeonly{couverture-avant,frontispice}

\begin{document}

\frontmatter

%% Page couverture avant.
\pagestyle{empty}
version https://git-lfs.github.com/spec/v1
oid sha256:6c6f393045de1021274fb51f20353d75b5d6e407518d3f183158fe9f64972774
size 1572

\null\cleardoublepage           % cf. section 2.2 textpos.pdf

%% Page frontispice
\begingroup

\TPGrid{3}{36}
\textblockorigin{0mm}{0mm}

%% auteur
\begin{textblock*}{1.7\TPHorizModule}(0.3\TPHorizModule,5\TPVertModule)
  \begin{minipage}{\textwidth}
    \theauthor
  \end{minipage}
\end{textblock*}

%% titre
\begin{textblock*}{1.7\TPHorizModule}(0.3\TPHorizModule,14.2\TPVertModule)
  \begin{minipage}{\textwidth}
    \thetitle \\
  \end{minipage}
\end{textblock*}

%% édition
\begin{textblock*}{1.7\TPHorizModule}(0.3\TPHorizModule,30\TPVertModule)
  \begin{minipage}{\textwidth}
    \thedate
  \end{minipage}
\end{textblock*}

\endgroup

%%% Local Variables:
%%% mode: latex
%%% TeX-master: "formation-latex-ul"
%%% TeX-engine: xetex
%%% End:

\null\clearpage                 % idem

%% Page de copyright
\begingroup
\calccentering{\unitlength}
\begin{adjustwidth*}{\unitlength}{-\unitlength}
  \setlength{\parindent}{0pt}
  \setlength{\parskip}{\baselineskip}

  {\textcopyright} {\year} Vincent Goulet \\

  \includegraphics[height=7mm,keepaspectratio=true]{by-sa}\\%
  Cette création est mise à disposition selon le contrat
  \href{http://creativecommons.org/licenses/by-sa/4.0/deed.fr}{%
    Attribution-Partage dans les mêmes conditions 4.0 International} de
  Creative Commons. En vertu de ce contrat, vous êtes libre de:
  \begin{itemize}
  \item \textbf{partager} --- reproduire, distribuer et communiquer
    l'{\oe}uvre;
  \item \textbf{remixer} --- adapter l'{\oe}uvre;
  \item utiliser cette {\oe}uvre à des fins commerciales.
  \end{itemize}
  Selon les conditions suivantes:

  \begin{tabularx}{\linewidth}{@{}lX@{}}
    \raisebox{-9mm}[0mm][13mm]{%
    \includegraphics[height=11mm,keepaspectratio=true]{by}}
    & \textbf{Attribution} --- Vous devez créditer l'{\oe}uvre, intégrer
      un lien vers le contrat et indiquer si des modifications ont été
      effectuées à l'{\oe}uvre. Vous devez indiquer ces informations par
      tous les moyens possibles, mais vous ne pouvez suggérer que
      l'Offrant vous soutient ou soutient la façon dont vous avez utilisé
      son {\oe}uvre. \\
    \raisebox{-9mm}{\includegraphics[height=11mm,keepaspectratio=true]{sa}}
    & \textbf{Partage dans les mêmes conditions} --- Dans le cas où vous
      modifiez, transformez ou créez à partir du matériel composant
      l'{\oe}uvre originale, vous devez diffuser l'{\oe}uvre modifiée dans
      les même conditions, c'est à dire avec le même contrat avec lequel
      l'{\oe}uvre originale a été diffusée.
  \end{tabularx}
  \vspace*{4mm}

  ISBN {\ISBN} \\
  Dépôt légal -- Bibliothèque et Archives nationales du Québec, {\year} \\
  Dépôt légal -- Bibliothèque et Archives Canada, {\year}
  \vspace*{4mm}

  \textbf{Code source} \\[4pt]
  \begin{tabularx}{1.0\linewidth}{@{}Xl@{}}
    Le code source de ce document est conservé dans un dépôt
    Git public. &
                         \raisebox{-7pt}{%
                         \href{https://github.com/vigou3/formation-latex-ul/}{%
                         \browsebutton}}
  \end{tabularx}
  \vspace*{4mm}

  \textbf{Crédits} \\
  Concept original de la couverture: Marie-Ève Guérard. \\
  Photo: Olaf Leillinger via
  \href{https://commons.wikimedia.org/wiki/File:Suricata.suricatta.6861.jpg}{%
    Wikimedia Commons}.
\end{adjustwidth*}
\endgroup

%%% Local Variables:
%%% mode: latex
%%% TeX-master: "formation-latex-ul"
%%% coding: utf-8
%%% End:

\clearpage

%% Corps du document
\pagestyle{companion}

\chapter{Introduction}
\label{chap:introduction}

Le présent ouvrage tire son origine d'une formation sur la rédaction
de thèses et de mémoires avec {\LaTeX} développée pour la Bibliothèque
de l'Université Laval. La formation aborde les concepts de base pour
un nouvel utilisateur: processus d'édition, compilation,
visualisation; séparation du contenu et de l'apparence du texte; mise
en forme du texte; séparation du document en parties; rudiments du
mode mathématique. Transformée en prose, la série de diapositives qui
appuie la présentation correspond grosso modo aux quatre premiers
chapitres de l'ouvrage.

Les six autres chapitres visent à rendre l'utilisateur de {\LaTeX}
débutant ou intermédiaire autonome dans la rédaction de documents
relativement complexes comportant des tableaux, des figures, des
équations mathématiques élaborées, une bibliographie, etc. Nous avons
aussi émaillé le texte de conseils et d'astuces glanés au fil de nos
vingt années d'utilisation du système de mise en page.

Les nombreuses références à la classe de documents \class{ulthese}
s'adressent au premier public de l'ouvrage: les étudiantes et
étudiants de l'Université Laval occupés à la rédaction de leur thèse
ou de leur mémoire. Ils devront utiliser cette classe pour composer un
document conforme aux règles générales de présentation matérielle de
la Faculté des études supérieures et postdoctorales. Les autres
lecteurs pourront sans mal escamoter ces passages.

Chaque chapitre comporte quelques exercices. Les solutions se trouvent
en annexe. En consultation électronique, le numéro d'un exercice est,
le cas échéant, un hyperlien vers sa solution, et vice versa.

Un index en fin d'ouvrage collige les références aux commandes et
environnements {\LaTeX}, ainsi qu'aux noms de paquetages et de
classes.

\subsection*{Autres références utiles}

L'ouvrage n'a aucune prétention d'exhaustivité. La consultation de
documentation additionnelle pourra s'avérer nécessaire pour réaliser
des mises en page plus élaborées. À cet égard, nous recommandons
chaudement le livre de \citet{Kopka:latex:4e} --- il a servi
d'inspiration pour ce document à maints endroits. La très complète
documentation (plus de 600~pages!) de la classe \class{memoir}
\citep{memoir} constitue une autre référence de choix. Nous
recommandons également:
\begin{itemize}
\item \link{http://fr.wikibooks.org/wiki/LaTeX}{\emph{LaTeX} dans
    Wikilivre} pour de la documentation en ligne, en français et
  libre;
\item le très actif forum de discussion
  \link{http://tex.stackexchange.com}{{\TeX}--{\LaTeX} Stack Exchange}
  (avant de penser y poser une question, vérifier que la réponse ne se trouve
  pas déjà dans le forum\dots\ ce qui risque fort d'être le cas);
\item la très complète
  \link{http://www.tex.ac.uk/cgi-bin/texfaq2html}{%
    \emph{foire aux questions}} (en anglais) du groupe des
  utilisateurs de {\LaTeX} du Royaume-Uni.
\end{itemize}

\subsection*{Installation d'une distribution}

L'utilisation de {\LaTeX} requiert évidemment une distribution du
système. Nous recommandons la distribution {\TeX}~Live administrée par
le {\TeX} Users Group. Les hyperliens ci-dessous mènent vers des
vidéos qui expliquent comment installer cette distribution.
\begin{itemize}
\item \capsule{https://youtu.be/kA53EQ3Q47w}{%
    Installation sur macOS}
\item \capsule{https://youtu.be/7MfodhaghUk}{%
    Installation sur Windows}
\end{itemize}

\subsection*{Hyperliens vers la documentation}

Le texte comporte plusieurs renvois vers la documentation d'un
paquetage ou d'une classe, par exemple vers la %
\doc{memoir}{http://texdoc.net/pkg/memoir/} %
de la classe \class{memoir}. L'hyperlien mène vers la version en ligne
de la documentation dans le site %
\link{http://texdoc.net}{TeXdoc Online}. On trouve également dans la
marge le nom du fichier correspondant (sans l'extension \code{.pdf})
sur un système doté de {\TeX}~Live.

La plupart des logiciels intégrés de rédaction {\LaTeX} offrent une
interface pour accéder à cette documentation.
\begin{itemize}
\item TeXShop: menu \code{Aide|Afficher l'aide pour le
    package} (\optkey\,\cmdkey\, I).
\item Texmaker: menu \code{Aide|TeXDoc [selection]}.
\item GNU~Emacs: commande \code{TeX-doc} (\code{C-c ?}) du mode
  spécialisé AUC{\TeX}.
\end{itemize}
Le lecteur devrait consulter la rubrique d'aide de son éditeur pour
savoir s'il offre une interface au système de gestion de la
documentation Texdoc de {\TeX}~Live.

\subsection*{Fichiers d'accompagnement}

Ce document devrait être accompagné des fichiers nécessaires pour
compléter certains exercices figurant à la fin des chapitres, ainsi
que d'un gabarit \fichier{exercice\_gabarit.tex} pour composer les
solutions des autres exercices. Si ce n'est pas le cas, récupérer les
fichiers dans le site \href{\ctanurl}{\emph{Comprehensive TeX Archive
    Network}} (CTAN).

\begin{flushright}
  Vincent Goulet \\
  Québec, novembre \year
\end{flushright}

%%% Local Variables:
%%% mode: latex
%%% TeX-engine: xetex
%%% TeX-master: "formation-latex-ul"
%%% encoding: utf-8
%%% End:


\cleartorecto
\tableofcontents*

\mainmatter

version https://git-lfs.github.com/spec/v1
oid sha256:c8842f4dc4fc06895b455d91128c3c73066a9b742f7c5836a952b63c31a49984
size 825

version https://git-lfs.github.com/spec/v1
oid sha256:0abe9d68b82cc93075a643ca04a6ce78f09ca97a7b8aa50a531453917994182b
size 40156

\chapter{Organisation d'un document}
\label{chap:organisation}

La maîtrise des notions du chapitre précédent permet déjà de composer
un document simple avec {\LaTeX}. Toutefois, la puissance du système
de mise en page se manifeste vraiment lors de la préparation de
documents élaborés comportant plusieurs divisions internes, une table
des matières, des renvois, etc. Le présent chapitre aborde ces aspects
d'organisation d'un document.

\section{Choix d'une classe}
\label{sec:organisation:classe}

La toute première chose à faire au moment de se lancer dans la
rédaction d'un document avec {\LaTeX} consiste normalement à choisir
une classe. Nous avons déjà expliqué à la \autoref{sec:bases:classes}
comment spécifier la classe à utiliser. Cette section présente les
différences entre les classes ainsi que les principales options
disponibles.

Les classes standards sont \class{article}, \class{report},
\class{book}, \class{letter} et \class{slides}.

\begin{description}
\item[\normalfont\class{article}] Articles scientifiques et autres
  documents de longueur modérée ne nécessitant pas une mise en page
  élaborée. Le folio (numéro de page) est placé au centre du pied de
  page. Le titre apparaît dans le haut de la première page,
  immédiatement suivi du texte.
\item[\normalfont\class{report}] Rapports et autres documents plus
  longs pouvant être divisés en chapitres. Le titre apparaît sur une
  page de titre. La mise en page est autrement identique à celle de la
  classe \class{article}.
\item[\normalfont\class{book}] Longs documents divisés en chapitres.
  La mise en page est conçue pour une impression recto verso. L'entête
  de la page (autre que la première du chapitre) contient le folio sur
  le bord extérieur et le titre de chapitre (page paire) ou le titre
  de section (page impaire). Le titre apparaît sur une page de titre.
\item[\normalfont\class{letter}] Lettres et correspondance. Bien que
  puissante, cette classe est plus rarement utilisée. Nous n'en traitons
  pas davantage dans ce document.
\item[\normalfont\class{slides}] Diapositives simples pour des
  présentations. La \autoref{sec:trucs:diapositives} traite plus en
  détail de la production de diapositives.
\end{description}

On trouvera un sommaire des principales caractéristiques des classes
\class{article}, \class{report} et \class{book} dans le
\autoref{tab:organisation:classes}. De plus, les figures
\ref{fig:organisation:classes:article}--\ref{fig:organisation:classes:book}
fournissent des exemples de mise en page pour ces trois classes.

\begin{table}
  \caption{Caractéristiques des principales classes standards}
  \label{tab:organisation:classes}
  \begin{tabularx}{1.0\linewidth}{XXXXX}
    \toprule
    Classe & Divisions & Disposition & Entête & Pied de page \\
    \midrule
    \class{article} & parties,\newline sections, \dots
                       & recto & vide & folio centré \\
    \addlinespace[6pt]
    \class{report} & parties,\newline chapitres,\newline sections, \dots
                       & recto & vide & folio centré \\
    \addlinespace[6pt]
    \class{book} & parties,\newline chapitres,\newline sections, \dots
                       & recto verso & folio, titres & vide \\
    \bottomrule
  \end{tabularx}
\end{table}

\begin{figure}
  \begin{minipage}{0.49\linewidth}
    \fbox{\includegraphics[page=1,width=0.95\linewidth]{exemple-classe-article}}
  \end{minipage}
  \hfill
  \begin{minipage}{0.49\linewidth}
    \fbox{\includegraphics[page=2,width=0.95\linewidth]{exemple-classe-article}}
  \end{minipage}
  \caption{Exemple de mise en page avec la classe \class{article}}
  \label{fig:organisation:classes:article}
\end{figure}

\begin{figure}
  \begin{minipage}{0.49\linewidth}
    \fbox{\includegraphics[page=1,width=0.95\linewidth]{exemple-classe-report}}
  \end{minipage}
  \hfill
  \begin{minipage}{0.49\linewidth}
    \fbox{\includegraphics[page=2,width=0.95\linewidth]{exemple-classe-report}}
  \end{minipage}
  \caption{Exemple de mise en page avec la classe \class{report}}
  \label{fig:organisation:classes:report}
\end{figure}

\begin{figure}
  \begin{minipage}{0.49\linewidth}
    \fbox{\includegraphics[page=1,width=0.95\linewidth]{exemple-classe-book}}
  \end{minipage}
  \hfill
  \begin{minipage}{0.49\linewidth}
    \fbox{\includegraphics[page=2,width=0.95\linewidth]{exemple-classe-book}}
  \end{minipage}
  \caption{Exemple de mise en page avec la classe \class{book}}
  \label{fig:organisation:classes:book}
\end{figure}

Cet ouvrage fait une large place à la classe \class{memoir}
\citep{memoir}, une extension de la classe standard \class{book} qui
facilite à plusieurs égards la préparation de documents d'allure
professionnelle dans {\LaTeX}. Nous recommandons d'utiliser cette
classe en lieu et place de la classe \class{book}, ou même de la
classe \class{article} (voir ci-dessous).

La classe \class{memoir} incorpore d'office plus de 30 des paquetages
les plus populaires\footnote{%
  Consulter la section~18.24 de la documentation de \class{memoir}
  pour la liste ou encore le journal de la compilation (\emph{log})
  d'un document utilisant la classe.}. %
La classe fait partie des distributions {\LaTeX} modernes; elle
devrait être installée et disponible sur tout système. Elle est livrée
avec une %
\doc{memman}{http://texdoc.net/pkg/memoir} %
exhaustive: le manuel d'instructions fait près de 600~pages! Il peut
être utile de s'y référer de temps à autre pour réaliser une mise en
page particulière.

Les auteurs de thèses et de mémoires déposés à l'Université Laval
doivent utiliser la classe \class{ulthese}; voir l'\autoref{chap:ulthese}.

On rappelle que l'on charge une classe de document au début du
préambule avec la commande
\begin{lstlisting}
\documentclass`\oarg{options}\marg{classe}'
\end{lstlisting}
Les \meta{options} disponibles varient d'une classe à l'autre. Les
plus courantes sont les suivantes.
\begin{description}
\item[\mdseries \code{10pt}, \code{11pt}, \code{12pt}] Taille de la
  police du document en points. La valeur par défaut est \code{10pt}.
  Nous recommandons d'utiliser plutôt \code{11pt}. C'est la taille par
  défaut avec la classe \class{ulthese}.
\item[\mdseries \code{oneside}, \code{twoside}] Disposition du
  document en recto seulement ou en recto verso. Ces options ne sont
  utiles que pour modifier la disposition par défaut de la classe. Les
  thèses et mémoires de l'Université Laval sont produits en recto
  seulement.
\item[\mdseries \code{openright}, \code{openany}] Position de la
  première page des chapitres toujours à droite (page impaire) ou
  immédiatement après la dernière page du chapitre précédent. Avec la
  valeur par défaut, \code{openany}, {\LaTeX} insérera une page
  blanche dans le document si un chapitre se termine sur une page
  impaire.
\item[\mdseries \code{article} (classe \class{memoir} seulement)] Mise
  en page comme celle d'un article. Avec cette option, \class{memoir}
  peut remplacer la classe \class{article}, ce qui permet d'utiliser
  une seule et même classe pour les deux principaux types de document
  (article et livre).
\end{description}

D'autres options permettent de contrôler la position du titre, la
disposition en une ou deux colonnes, ou encore la position des
équations hors paragraphe. \citet{Thurnherr:class-options} offre une
présentation succincte des options standards. Le chapitre~1 de la %
\doc{memman}{http://texdoc.net/pkg/memoir} %
de \class{memoir} traite en plus des ajouts propres à cette classe.



\section{Parties d'un document}

Tout document de plus de quelques pages est normalement divisé en
chapitres, sections, sous-sections, etc. Il peut comporter une ou
plusieurs annexes et débuter par un résumé, notamment s'il s'agit d'un
article scientifique. Le document est habituellement coiffé d'un
titre, mais celui-ci est parfois affiché sur une page de titre
séparée.

Toutes ces considérations relevant essentiellement de la mise en page,
{\LaTeX} s'en charge pour nous. L'auteur n'a qu'à spécifier la
strucure logique du document à l'aide des commandes de la présente
section.

\subsection{Titre et page de titre}
\label{sec:organisation:parties:titre}

{\LaTeX} rend très simple la composition du titre d'un article
scientifique ou d'une page de titre simple (classes \class{report},
\class{book}, \class{memoir}). En premier lieu, on spécifie,
habituellement dans le préambule, le titre du document, le nom du ou
des auteurs et la date de publication avec les commandes suivantes:
\begin{lstlisting}
\title`\marg{Titre du document}'
\author`\marg{Prénom Nom {\bs\bs} Affiliation {\bs\bs} Adresse}'
\date`\marg{Date ou autre texte}'
\end{lstlisting}
Un long titre sera scindé automatiquement. L'auteur peut aussi scinder
le titre manuellement en insérant la commande {\bs\bs} aux points de
coupure. %
Si l'ouvrage comporte deux auteurs ou plus, on insère les informations
dans la commande \cmd{\author} les unes après les autres en séparant
chaque entrée par la commande \cmdprint{\and}. %
La commande \cmd{\date} insère le texte donné en argument (qu'il
s'agisse d'une date ou non) à l'endroit prévu à cet effet par
{\LaTeX}. Si l'on omet la commande, {\LaTeX} insère la date du jour au
moment de la compilation. Pour ne pas afficher la date, on laisse
l'argument vide:
\begin{lstlisting}
\date{}
\end{lstlisting}

Dans les articles scientifiques, le nom d'un auteur est fréquemment
suivi d'un appel de note renvoyant à des remerciements à un organisme
subventionnaire ou à quelque autre information additionnelle sur
l'auteur. On insère une telle note et son appel à l'endroit approprié
dans les commandes \cmd{\title} ou \cmd{\author} avec la commande
\begin{lstlisting}
\thanks`\marg{Texte}'
\end{lstlisting}

Les commandes ci-dessus ne permettent que de saisir les informations
relatives au titre. Pour produire le titre il faut, en second lieu,
insérér dans le corps du document la commande
\begin{lstlisting}
\maketitle
\end{lstlisting}

\begin{exemple}
  \label{ex:organisation:titre}
  Le code ci-dessous permet de créer un titre d'article standard. La
  page composée avec {\XeLaTeX} (puisque nous utilisons le paquetage
  \pkg{fontspec}) se trouve à la \autoref{fig:organisation:titre}.
  \begin{demo}
    \lstinputlisting[lastline=19]{exemple-titre.tex}
  \end{demo}

  La date de publication apparaissant dans l'illustration de la
  \autoref*{fig:organisation:titre} est celle de la compilation
  puisque la commande \cmdprint{\date} n'apparait pas dans le code
  source.

  On remarquera également que nous avons placé un symbole de
  commentaire \% immédiatement après le nom de l'auteur dans le code
  source. Tel qu'expliqué à la \autoref{sec:bases:caracteres:espaces},
  c'est pour éviter que {\LaTeX} ne transforme le retour à la ligne avant
  la commande \cmd{\thanks} en une espace entre le nom et l'appel de
  note.

  (Le paquetage \pkg{lorem} utilisé dans cet exemple permet
  d'insérer du faux texte \link{http://fr.lipsum.com}{Lorem Ipsum}
  dans un document \LaTeX.)
  \begin{figure}
    \centering
    \fbox{\includegraphics[width=0.7\linewidth]{exemple-titre}}
    \caption{Illustration d'un titre d'article standard.}
    \label{fig:organisation:titre}
  \end{figure}
  \qed
\end{exemple}

Nous avons mentionné plus haut que {\LaTeX} peut aussi produire
automatiquement la page de titre d'un rapport ou d'un livre. Il est
toutefois peu probable qu'elle convienne, surtout dans le cas d'un
livre. Une option plus flexible existe: les environnements
\Ie{titlepage} (classes standards) et \Ie{titlingpage} (classe
\class{memoir}) permettent de définir librement une page de
titre:
\begin{demo}
  \begin{minipage}{0.48\linewidth}
\begin{lstlisting}
\begin{titlepage}
  `\meta{Texte de la page de titre}'
\end{titlepage}
\end{lstlisting}
  \end{minipage}
  \hfill
  \begin{minipage}{0.48\linewidth}
\begin{lstlisting}
\begin{titlingpage}
  `\meta{Texte de la page de titre}'
\end{titlingpage}
\end{lstlisting}
  \end{minipage}
\end{demo}
L'auteur contrôle alors entièrement la disposition et la
composition des éléments de la page de titre. Nous renvoyons le
lecteur au chapitre~4 de la %
\doc{memman}{http://texdoc.net/pkg/memoir} %
de \class{memoir} pour une liste de bonnes pratiques en matière de
composition de page de titre et pour des exemples détaillés.

Comme c'est souvent le cas dans les universités, le format de la page
de titre des thèses et mémoires de l'Université Laval est
prédéterminé. La classe \class{ulthese} fournit donc ses propres
commandes de composition; consulter sa %
\doc{ulthese}{http://texdoc.net/pkg/ulthese}.

\subsection{Résumé}
\label{sec:organisation:parties:resume}

Les articles scientifiques et les rapports comportent souvent un
résumé, habituellement composé en retrait des marges gauche et droite
et dans une police plus petite. On produit le résumé avec
l'environnement \Ie{abstract} des classes \class{article},
\class{report} ou \class{memoir}:
\begin{lstlisting}
\begin{abstract}
  `\meta{Texte du résumé}'
\end{abstract}
\end{lstlisting}

Il est plutôt rare que les livres comportent un résumé ou alors, comme
pour les thèses et mémoires de l'Université Laval, celui-ci
est simplement traité comme un chapitre normal non numéroté.

\subsection{Sections}
\label{sec:organisation:parties:sections}

Les commandes ci-dessous servent à découper un document en sections
qui seront automatiquement numérotées par {\LaTeX} de manière
séquentielle:
\begin{lstlisting}
\part`\oarg{titre court}\marg{titre}'
\chapter`\oarg{titre court}\marg{titre}'
\section`\oarg{titre court}\marg{titre}'
\subsection`\oarg{titre court}\marg{titre}'
\subsubsection`\oarg{titre court}\marg{titre}'
\paragraph`\oarg{titre court}\marg{titre}'
\subparagraph`\oarg{titre court}\marg{titre}'
\end{lstlisting}
Les commandes forment, dans l'ordre où elles sont données, une
hiérarchie des titres d'un document\footnote{%
  Nous n'avons jamais utilisé les niveaux de division
  \cmdprint{\paragraph} et \cmdprint{\subparagraph}.}. %
Tel que mentionné précédemment, la commande \cmd{\chapter} n'est pas
disponible avec la classe \class{article}.

Chaque commande prend en argument obligatoire le \meta{titre} de la
section. Si celui-ci est très long, il peut être utile de fournir en
argument optionnel un \meta{titre court}; c'est ce dernier qui
apparaitra dans la table des matières et dans les entêtes de page, le cas
échéant.

Toutes les commandes existent en version suivie d'une \verb=*= qui
supprime la numérotation ainsi que l'insertion éventuelle dans la
table des matières (plus de détails à la \autoref{sec:organisation:tdm}).

\begin{conseil}
  Éviter d'utiliser des sous-sous-sections numérotées (commande
  \cmdprint{\subsubsection}) dans un livre. Cela résulte en une
  numérotation à quatre niveaux qui s'avère difficile à suivre pour le
  lecteur.
\end{conseil}

\subsection{Annexes}
\label{sec:organisation:parties:annexes}

Les annexes sont des sections ou des chapitres avec une numérotation
alphanumérique (A, A.1, ...) plutôt qu'entièrement numérique. On
informe {\LaTeX} que les sections suivantes doivent être traitées
comme des annexes en insérant dans le document la commande
%% [hack ci-dessous pour cacher la commande \appendix à reftex]
\begin{lstlisting}
`\verb=\appendix='
\end{lstlisting}
En plus de modifier le style de numérotation, la commande a pour effet
de changer le mot clé «Chapitre» pour «Annexe» dans les titres de
chapitre.

\subsection{Structure logique d'un livre}
\label{sec:oganisation:parties:livre}

Un livre se compose normalement de trois grandes parties logiques: les
pages liminaires (avant-propos, table des matières, tout ce qui
précède le chapitre premier); le corps du livre (chapitres et
annexes); les parties en fin d'ouvrage (bibliographie, index). Les
commandes
\begin{lstlisting}
\frontmatter
\mainmatter
\backmatter
\end{lstlisting}
qui sont disponibles avec les classes \class{book}, \class{memoir} et
\class{ulthese}, permettent d'identifier ces trois parties. Le
\autoref{tab:organisation:livre} résume l'effet de chaque commande.

\begin{table}
  \centering
  \caption{Commandes d'identification de la structure logique d'un
    livre et leurs effets}
  \label{tab:organisation:livre}
  \begin{tabular}{ll}
    \toprule
    Commande & Effets \\
    \midrule
    \cmd{\frontmatter} & numérotation des
                              pages en chiffres romains (i, ii, ...) \\
             & chapitres non numérotés \\
    \addlinespace[0.5\normalbaselineskip]
    \cmd{\mainmatter} & numérotation des pages
                             à partir de 1 en chiffres arabes \\
             & chapitres numérotés \\
    \addlinespace[0.5\normalbaselineskip]
    \cmd{\backmatter} & numérotation des pages
                             se poursuit \\
             & chapitres non numérotés \\
    \bottomrule
  \end{tabular}
\end{table}


\section{Table des matières}
\label{sec:organisation:tdm}

Dans la mesure où l'on a bien identifié les différentes divisions d'un
ouvrage avec les commandes mentionnées à la section précédente, la
production de la table des matières est on ne peut plus simple avec
{\LaTeX}: il suffit d'insérer la commande
\begin{lstlisting}
\tableofcontents
\end{lstlisting}
dans le corps du document à l'endroit où la table des matières doit
apparaître. C'est tout!

Lorsque le paquetage \pkg{hyperref} \citep{hyperref} est chargé, la
commande \cmdprint{\tableofcontents} produit également la table des
matières du fichier PDF. Cela permet de naviguer dans le document
directement depuis la visionneuse PDF. La
\autoref{fig:organisation:tdm-dans-pdf} illustre cette fonctionnalité
avec la visionneuse Aperçu de macOS.

\begin{figure}
  \resizebox{\textwidth}{!}{\includegraphics[angle=-90]{tdm-dans-pdf}}
  \caption{Consultation d'un document PDF avec la visionneuse Aperçu
    de macOS. La barre latérale de gauche affiche la table des
    matières du fichier PDF, ce qui permet de naviguer dans le
    document sans devoir revenir à celle du document.}
  \label{fig:organisation:tdm-dans-pdf}
\end{figure}

\begin{important}
  La production initiale de la table des matières et la prise en
  compte de toute modification requiert jusqu'à trois compilations
  consécutives du document.
\end{important}

Par défaut, une section non numérotée ne figure pas dans la table des
matières. Si l'on souhaite néanmoins l'y insérer, il faut utiliser la
commande suivante:
\begin{lstlisting}
\addtocontentsline{toc}`\marg{niveau}\marg{titre}'
\end{lstlisting}
où \meta{niveau} est le nom de la commande de division sans le
caractère {\bs} (\code{chapter}, \code{section}, etc.) et \meta{titre}
est le texte qui apparaitra dans la table des matières.

\begin{exemple}
  Selon les normes de présentation visuelle de la Faculté des études
  supérieures et postdoctorales pour les thèses et mémoires de
  l'Université Laval, les résumés, la liste des abbréviations et des
  sigles, les remerciements et l'avant-propos doivent être composés
  comme des chapitres normaux non numérotés. Cependant, ils doivent
  apparaître dans la table des matières. Pour ce faire, les gabarits
  fournis avec la classe \class{ulthese} comportent des lignes de la
  forme:
\begin{lstlisting}
\chapter*{Résumé}
\phantomsection\addcontentsline{toc}{chapter}{Résumé}
\end{lstlisting}
  La commande \cmd{\phantomsection} est rendue nécessaire (ou
  recommandée) par le paquetage \pkg{hyperref} qui est chargé dans le
  préambule. %
  \qed
\end{exemple}

Outre une table des matières, les ouvrages scientifiques
comportent parfois une liste des figures et une liste des tableaux. On
obtient celles-ci avec les commandes
\begin{lstlisting}
\listoffigures
\listoftables
\end{lstlisting}

Dans les classes \class{memoir} et \class{ulthese}, les commandes
ci-dessus insèrent leur propre titre de section dans la table des
matière (autrement dit, la table des matières apparait dans la table
des matières). Les versions
\begin{lstlisting}
\tableofcontents*
\listoffigures*
\listoftables*
\end{lstlisting}
adoptent le comportement des classes standards, soit d'omettre ces
parties dans la table des matières.


\section{Renvois automatiques}
\label{sec:organisation:renvois}

Un document d'une certaine ampleur contiendra souvent des renvois à
une section, un tableau, une équation, voire une page
spécifique. Évidemment, le numéro de la section, du tableau, de
l'équation ou de la page est susceptible de changer au fil de la
rédaction du document. C'est pourquoi l'auteur ne devrait
\emph{jamais} insérer les revois manuellement dans le texte. C'est une
tâche qu'il vaut mieux confier à l'ordinateur.

\subsection{Étiquettes et renvois}
\label{sec:organisation:renvois:etiquettes}

Les renvois automatiques dans {\LaTeX} reposent sur un système
d'étiquettes attribuées à des éléments de contenu et de référencement
par ces étiquettes. Ainsi, pour renvoyer le lecteur à un élément de
contenu, il faut d'abord le nommer en insérant la commande
\begin{lstlisting}
\label`\marg{nom}'
\end{lstlisting}
à proximité de l'élément. Le choix du \meta{nom} est tout à fait libre. Il
peut être constitué de toute combinaison de lettres, de chiffres et de
symboles autres que les symboles réservés mentionnés à la
\autoref{sec:bases:caracteres:reserves}.

Ensuite, il devient possible d'insérer un renvoi à cet élément de contenu
n'importe où dans le document avec la commande
\begin{lstlisting}
\ref`\marg{nom}'
\end{lstlisting}
Pour renvoyer à la page où se trouve l'élément, on utilise la commande
\begin{lstlisting}
\pageref`\marg{nom}'
\end{lstlisting}

\begin{exemple}
  \label{ex:organisation:renvoi}
  Le code ci-dessous permet d'insérer un renvoi automatique à un
  numéro de section dans un document. Le résultat se trouve à la
  \autoref{fig:organisation:renvoi}.
\begin{lstlisting}[emph={\label,\ref}]
\section{Définitions}
\label{sec:definitions}

Lorem ipsum dolor sit amet, consectetur adipiscing elit.
Duis in auctor dui. Vestibulum ut, placerat ac, adipiscing
vitae, felis.

\section{Historique}

Tel que vu à la section \ref{sec:definitions}, on a...
\end{lstlisting}

  \begin{figure}
    \fbox{\includegraphics[viewport=124 550 484 664,clip=true,width=0.98\linewidth]{exemple-renvoi}}
    \caption{Texte produit par le code de
      l'\autoref{ex:organisation:renvoi} illustrant un renvoi
      automatique standard}
    \label{fig:organisation:renvoi}
  \end{figure}
  \qed
\end{exemple}

\begin{conseil}
  Un long document contiendra vraisemblablement un grand nombre
  d'étiquettes et de renvois. Afin de s'y retrouver et pour éviter les
  doublons, adopter une manière systématique et mnémotechnique de
  nommer les éléments.

  Pour le présent document, nous avons utilisé un système de la forme
  suivante:
\begin{lstlisting}
\label{chap:`\meta{chapitre}'}`\phantom{\meta{}}'         % chapitre
\label{sec:`\meta{chapitre}':`\meta{section}'}  % section
\label{tab:`\meta{chapitre}':`\meta{tableau}'}  % tableau
\label{eq:`\meta{chapitre}':`\meta{equation}'}  % équation
\end{lstlisting}
\end{conseil}

\subsection{Production des renvois}
\label{sec:organisation:renvois:production}

La production des renvois requiert deux à trois compilations.
Tant que {\LaTeX} n'a pas complété les renvois, le journal de
compilation contient, vers la toute fin, le message
\begin{verbatim}
LaTeX Warning: Label(s) may have changed. Rerun to get
cross-references right.
\end{verbatim}

On prendra également garde aux alertes suivantes dans le journal de
compilation. Elles identifient des correctifs à apporter dans le
système d'étiquettes et de renvois. D'abord, le message
\begin{verbatim}
LaTeX Warning: There were undefined references.
\end{verbatim}
indique qu'une ou plusieurs commandes \cmd{\ref} renvoient à un
\meta{nom} qui n'est pas défini avec \cmd{\label}. On trouvera alors
dans le document le caractère \textbf{?} en lieu et place du renvoi. À
l'inverse, le message
\begin{verbatim}
LaTeX Warning: There were multiply-defined labels.
\end{verbatim}
indique que plusieurs commandes \cmdprint{\label} utilisent le même
\meta{nom}. Cela est susceptible de causer des renvois vers le mauvais
élément de contenu.

\subsection{Renvois avec hyperliens}
\label{sec:organisation:renvois:hyperliens}

Lorsque chargé dans le document, le paquetage \pkg{hyperref} insère
des hyperliens vers les renvois dans les fichiers PDF. C'est très
pratique en consultation électronique d'un document.

\begin{exemple}
  \label{ex:organisation:renvoi-hyperref}
  Avec l'ajout de
\begin{lstlisting}
\usepackage{hyperref}
\end{lstlisting}
  dans le préambule, le code de l'\autoref{ex:organisation:renvoi}
  produit maintenant le type de renvoi illustré à la
  \autoref{fig:organisation:renvoi-hyperref}. Le texte en couleur
  contrastante est un hyperlien vers le titre de la section.

  \begin{figure}
    \centering
    \fbox{\includegraphics[viewport=124 550 484 664,clip=true,width=0.98\linewidth]{exemple-renvoi-hyperref}}
    \caption{Texte produit par le code de
      l'\autoref{ex:organisation:renvoi} après l'ajout du paquetage
      \pkg{hyperref}}
    \label{fig:organisation:renvoi-hyperref}
  \end{figure}
  \qed
\end{exemple}

L'inconvénient avec la procédure illustrée dans l'exemple précédent,
c'est que seul le numéro de la référence est transformé en hyperlien.
La zone disponible pour cliquer s'en trouve plutôt restreinte. Une
fonctionnalité du paquetage \pkg{hyperref} --- à laquelle nous avons eu
recours dans le présent document --- permet d'agrandir cette zone. La
commande
\begin{lstlisting}
\autoref`\marg{nom}'
\end{lstlisting}
permet de nommer automatiquement le type de renvoi (section, équation,
tableau, etc.) et de transformer en hyperlien à la fois ce texte et le
numéro du renvoi.

\begin{exemple}
  \label{ex:organisation:renvoi-autoref}
  On reprend le texte de l'\autoref{ex:organisation:renvoi}, mais en
  utilisant cette fois la commande \cmd{\autoref} pour insérer un
  renvoi dans le texte. On remarque que le mot «section» a été
  supprimé du code source pour laisser à la commande le soin d'insérer
  le vocable approprié dans le document.
\begin{lstlisting}[emph={\label,\autoref}]
\section{Définitions}
\label{sec:definitions}

Lorem ipsum dolor sit amet, consectetur adipiscing elit.
Duis in auctor dui. Vestibulum ut, placerat ac, adipiscing
vitae, felis.

\section{Historique}

Tel que vu à la \autoref{sec:definitions}, on a...
\end{lstlisting}
  Comme on peut le voir à la
  \autoref{fig:organisation:renvoi-autoref}, le mot «section» ainsi
  que son numéro forment maintenant l'hyperlien.

  \begin{figure}
    \centering
    \fbox{\includegraphics[viewport=124 550 484 664,clip=true,width=0.98\linewidth]{exemple-renvoi-autoref}}
    \caption{Texte produit par le code de
      l'\autoref{ex:organisation:renvoi-autoref}}
    \label{fig:organisation:renvoi-autoref}
  \end{figure}
  \qed
\end{exemple}

La \autoref{sec:trucs:hyperliens} fournit des détails
additionnels sur la gestion des hyperliens dans un document PDF.


\section{Document contenu dans plusieurs fichiers}
\label{sec:organisation:include}

Lorsque le préambule et le corps du texte demeurent relativement
courts (peu de commandes spéciales et moins d'une vingtaine de pages
de texte), il demeure assez simple et convivial d'en faire l'édition
dans un seul fichier à l'aide de son éditeur de texte favori.

Cependant, si le préambule devient long et complexe ou, surtout,
lorsque l'ampleur du document augmente jusqu'à compter un grand nombre
de pages sur plusieurs chapitres, il convient de répartir les diverses
parties du document dans des fichiers séparés.

La segmentation en plusieurs fichiers rend l'édition du texte plus
simple et plus efficace. De plus, durant la phase de rédaction, elle
peut significativement accélérer la compilation des documents très
longs ou comptant plusieurs images.

\subsection{Insertion du contenu d'un autre fichier}
\label{sec:organisation:include:input}

La commande \cmd{\input} permet d'insérer le contenu d'un autre
fichier dans un document {\LaTeX}. La syntaxe de la commande est
\begin{lstlisting}
\input`\marg{fichier}'
\end{lstlisting}
où le nom du fichier à insérer est \meta{fichier}\code{.tex}. On
laisse donc tomber l'extension \code{.tex}, qui est implicite. Le
contenu du fichier est inséré tel quel dans le document, comme s'il
avait été saisi dans le fichier qui contient l'appel à \cmd{\input}.

Le procédé est surtout utile pour sauvegarder séparément des bouts de
code qui gênent l'édition du texte (figures, longs
tableaux) ou qui sont communs entre plusieurs documents (licence
d'utilisation, auteur et affiliation).

La commande peut aussi être utilisée dans le préambule pour charger
une partie ou l'ensemble de celui-ci. Cela permet de composer un même
préambule pour plusieurs documents. Il suffit alors de faire
d'éventuelles modifications à un seul endroit pour les voir prendre
effet dans tous les documents.

\subsection{Insertion de parties d'un document}
\label{sec:organisation:include:include}

Il est recommandé de segmenter tout document d'une certaine ampleur
dans des fichiers \verb=.tex= distincts pour chaque partie ---
habituellement un fichier par chapitre. Le document complet est
composé à l'aide d'un fichier maître qui contient le préambule
{\LaTeX} et un ensemble d'appels à la commande \cmdprint{\include}
pour réunir les parties dans un tout.

La syntaxe de \cmd{\include} est
\begin{lstlisting}
\include`\marg{fichier}'
\end{lstlisting}
où le nom du fichier à insérer est \meta{fichier}\code{.tex}. Ici
aussi on laisse tomber l'extension \code{.tex} qui est implicite.

Comme \cmd{\input}, la commande \cmd{\include} insère le contenu d'un
autre fichier dans un document. Toutefois, l'insertion d'un fichier
avec \cmd{\include} débute toujours une nouvelle page. On utilisera
donc cette commande principalement pour insérer des chapitres entiers
plutôt que seulement des portions de texte. De plus, un fichier inséré
avec \cmd{\include} peut contenir des appels à \cmd{\input}, mais pas
à \cmd{\include}.

\begin{exemple}
  La \autoref{fig:organisation:maitre} présente un exemple type de
  fichier maître. On y utilise la commande \cmd{\include} pour
  composer un document chapitre par chapitre.

  \begin{figure}
    \centering
    \begin{minipage}{0.75\linewidth}
\begin{lstlisting}[numbers=left, numberstyle=\tiny,
                   frame=single, rulecolor=\color{black}, framesep=6pt]
\documentclass{ulthese}
  [...]

\begin{document}

\frontmatter

\chapter{Introduction}
\label{chap:introduction}

Le présent ouvrage tire son origine d'une formation sur la rédaction
de thèses et de mémoires avec {\LaTeX} développée pour la Bibliothèque
de l'Université Laval. La formation aborde les concepts de base pour
un nouvel utilisateur: processus d'édition, compilation,
visualisation; séparation du contenu et de l'apparence du texte; mise
en forme du texte; séparation du document en parties; rudiments du
mode mathématique. Transformée en prose, la série de diapositives qui
appuie la présentation correspond grosso modo aux quatre premiers
chapitres de l'ouvrage.

Les six autres chapitres visent à rendre l'utilisateur de {\LaTeX}
débutant ou intermédiaire autonome dans la rédaction de documents
relativement complexes comportant des tableaux, des figures, des
équations mathématiques élaborées, une bibliographie, etc. Nous avons
aussi émaillé le texte de conseils et d'astuces glanés au fil de nos
vingt années d'utilisation du système de mise en page.

Les nombreuses références à la classe de documents \class{ulthese}
s'adressent au premier public de l'ouvrage: les étudiantes et
étudiants de l'Université Laval occupés à la rédaction de leur thèse
ou de leur mémoire. Ils devront utiliser cette classe pour composer un
document conforme aux règles générales de présentation matérielle de
la Faculté des études supérieures et postdoctorales. Les autres
lecteurs pourront sans mal escamoter ces passages.

Chaque chapitre comporte quelques exercices. Les solutions se trouvent
en annexe. En consultation électronique, le numéro d'un exercice est,
le cas échéant, un hyperlien vers sa solution, et vice versa.

Un index en fin d'ouvrage collige les références aux commandes et
environnements {\LaTeX}, ainsi qu'aux noms de paquetages et de
classes.

\subsection*{Autres références utiles}

L'ouvrage n'a aucune prétention d'exhaustivité. La consultation de
documentation additionnelle pourra s'avérer nécessaire pour réaliser
des mises en page plus élaborées. À cet égard, nous recommandons
chaudement le livre de \citet{Kopka:latex:4e} --- il a servi
d'inspiration pour ce document à maints endroits. La très complète
documentation (plus de 600~pages!) de la classe \class{memoir}
\citep{memoir} constitue une autre référence de choix. Nous
recommandons également:
\begin{itemize}
\item \link{http://fr.wikibooks.org/wiki/LaTeX}{\emph{LaTeX} dans
    Wikilivre} pour de la documentation en ligne, en français et
  libre;
\item le très actif forum de discussion
  \link{http://tex.stackexchange.com}{{\TeX}--{\LaTeX} Stack Exchange}
  (avant de penser y poser une question, vérifier que la réponse ne se trouve
  pas déjà dans le forum\dots\ ce qui risque fort d'être le cas);
\item la très complète
  \link{http://www.tex.ac.uk/cgi-bin/texfaq2html}{%
    \emph{foire aux questions}} (en anglais) du groupe des
  utilisateurs de {\LaTeX} du Royaume-Uni.
\end{itemize}

\subsection*{Installation d'une distribution}

L'utilisation de {\LaTeX} requiert évidemment une distribution du
système. Nous recommandons la distribution {\TeX}~Live administrée par
le {\TeX} Users Group. Les hyperliens ci-dessous mènent vers des
vidéos qui expliquent comment installer cette distribution.
\begin{itemize}
\item \capsule{https://youtu.be/kA53EQ3Q47w}{%
    Installation sur macOS}
\item \capsule{https://youtu.be/7MfodhaghUk}{%
    Installation sur Windows}
\end{itemize}

\subsection*{Hyperliens vers la documentation}

Le texte comporte plusieurs renvois vers la documentation d'un
paquetage ou d'une classe, par exemple vers la %
\doc{memoir}{http://texdoc.net/pkg/memoir/} %
de la classe \class{memoir}. L'hyperlien mène vers la version en ligne
de la documentation dans le site %
\link{http://texdoc.net}{TeXdoc Online}. On trouve également dans la
marge le nom du fichier correspondant (sans l'extension \code{.pdf})
sur un système doté de {\TeX}~Live.

La plupart des logiciels intégrés de rédaction {\LaTeX} offrent une
interface pour accéder à cette documentation.
\begin{itemize}
\item TeXShop: menu \code{Aide|Afficher l'aide pour le
    package} (\optkey\,\cmdkey\, I).
\item Texmaker: menu \code{Aide|TeXDoc [selection]}.
\item GNU~Emacs: commande \code{TeX-doc} (\code{C-c ?}) du mode
  spécialisé AUC{\TeX}.
\end{itemize}
Le lecteur devrait consulter la rubrique d'aide de son éditeur pour
savoir s'il offre une interface au système de gestion de la
documentation Texdoc de {\TeX}~Live.

\subsection*{Fichiers d'accompagnement}

Ce document devrait être accompagné des fichiers nécessaires pour
compléter certains exercices figurant à la fin des chapitres, ainsi
que d'un gabarit \fichier{exercice\_gabarit.tex} pour composer les
solutions des autres exercices. Si ce n'est pas le cas, récupérer les
fichiers dans le site \href{\ctanurl}{\emph{Comprehensive TeX Archive
    Network}} (CTAN).

\begin{flushright}
  Vincent Goulet \\
  Québec, novembre \year
\end{flushright}

%%% Local Variables:
%%% mode: latex
%%% TeX-engine: xetex
%%% TeX-master: "formation-latex-ul"
%%% encoding: utf-8
%%% End:

\tableofcontents*

\mainmatter
\include{historique}
\include{rappels}
\include{modele}
[...]

\end{document}
\end{lstlisting}
    \end{minipage}
    \caption{Structure type d'un fichier maître. Les fichiers
      \code{historique.tex}, \code{rappels.tex} et \code{modele.tex}
      contiennent le texte des trois premiers chapitres.}
    \label{fig:organisation:maitre}
  \end{figure}
  \qed
\end{exemple}

Le principal avantage de \cmd{\include} par rapport à \cmd{\input}
réside dans le fait que {\LaTeX} peut préserver entre les compilations
les informations telles que les numéros de pages, de sections ou
d'équations, ainsi que les renvois et les références bibliographiques.
Cela permet, par exemple, de compiler le texte d'un seul chapitre ---
plutôt que le document entier --- et néanmoins obtenir une image
représentative du chapitre. Procéder ainsi accélère significativement
la compilation des documents longs ou complexes.

\subsection{Compilation partielle}
\label{sec:organisation:include:compilation}

La commande \cmd{\includeonly}, que l'on utilise exclusivement dans le
préambule, sert à spécifier le ou les fichiers à compiler tout en
préservant la numérotation et les références. Sa syntaxe est
\begin{lstlisting}
\includeonly`\marg{liste\_fichiers}'
\end{lstlisting}
où \meta{liste\_fichiers} contient les noms des fichiers à
inclure dans la compilation, séparés par des virgules et sans
l'extension \code{.tex}.

\begin{conseil}
  Utiliser des noms de fichiers qui permettent de facilement
  identifier leur contenu. Par exemple, un nom comme
  \fichier{rappels.tex} identifie clairement le contenu du fichier et
  il résiste mieux aux changements à l'ordre des chapitres que
  \fichier{chapitre1.tex}.
\end{conseil}

Lors de l'utilisation de la commande \cmd{\includeonly}, toute la
numérotation dans les fichiers \meta{liste\_fichiers} suivra celle
établie lors de la compilation précédente. Si l'édition des fichiers
de \meta{liste\_fichiers} cause des changements dans la numérotation
et les références dans les autres parties du document, une nouvelle
compilation de l'ensemble ou d'une partie de celui-ci s'avérera
nécessaire.

\begin{exemple}
  Un document est composé en plusieurs parties avec les commandes
  suivantes:
\begin{lstlisting}
\include{historique}            % chapitre 1
\include{rappels}               % chapitre 2
\include{modele}                % chapitre 3
\end{lstlisting}
  Les chapitres débutent respectivement aux pages~1, 23 et 41.
  \begin{itemize}
  \item Si l'on ajoute au préambule du document la commande
\begin{lstlisting}
\includeonly{rappels}
\end{lstlisting}
    le numéro du chapitre sera toujours 2 et le folio de
    la première page sera toujours 23, même si les 22 pages
    précédentes ne se trouvent pas dans le document.
  \item Si l'on modifie le fichier \fichier{rappels.tex} de telle
    sorte que le chapitre se termine maintenant à la page 46, il
    faudra recompiler le document avec au moins les fichiers
    \fichier{rappels.tex} et \code{modele.tex} pour que les pages du
    chapitre~3 soient numérotées à partir de 47.
  \end{itemize}
  \qed
\end{exemple}

L'\autoref{ex:include} illustre mieux le cycle typique
d'utilisation des commandes \cmd{\include} et \cmd{\includeonly}.



%%%
%%% Exercices
%%%

\section{Exercices}
\label{sec:organisation:exercices}

\Opensolutionfile{solutions}[solutions-organisation]

\begin{Filesave}{solutions}
\section*{Chapitre \ref*{chap:organisation}}
\addcontentsline{toc}{section}{Chapitre \protect\ref*{chap:organisation}}

\end{Filesave}

\begin{exercice}[nosol]
  Utiliser le fichier \fichier{exercice\_parties.tex}.
  \begin{enumerate}
  \item Étudier la structure du document dans le code source.
  \item Ajouter un titre et un auteur au document à l'aide des
    commandes \cmdprint{\title} et \cmdprint{\author} se trouvant déjà
    dans le préambule.
  \item Créer la table des matières du document en le compilant 2 à 3
    fois.
  \item Insérer deux ou trois titres de sections de différents niveaux
    dans le document et recompiler.
  \item La numérotation cesse à partir des sous-sections. C'est une
    particularité de la classe \class{memoir}. Recompiler le document
    après avoir ajouté au préambule la commande
\begin{lstlisting}
\maxsecnumdepth{subsection}
\end{lstlisting}
  \item Ajouter une annexe au document.
  \end{enumerate}
\end{exercice}

\begin{exercice}[nosol]
  Utiliser le fichier \fichier{exercice\_renvois.tex}.
  \begin{enumerate}
  \item Insérer dans le texte un renvoi au numéro d'une section.
  \item Activer le paquetage \pkg{hyperref} avec l'option
    \code{colorlinks} et comparer l'effet d'utiliser \cmd{\ref} ou
    \cmd{\autoref} pour le renvoi.
  \end{enumerate}
\end{exercice}

\begin{exercice}[nosol]
  \label{ex:include}
  Cet exercice fait appel au fichier maître
  \fichier{exercice\_include.tex} et à plusieurs fichiers auxiliaires.
  Schématiquement, le document est composé ainsi:

  \medskip
  \begin{minipage}{\linewidth}
    \dirtree{%
      .1 exercice\_include.tex.
      .2 {\textbackslash}input pagetitre.tex.
      .2 {\textbackslash}include rpresentation.tex.
      .3 {\textbackslash}includegraphics console-screenshot.pdf.
      .2 {\textbackslash}include emacs.tex.
    }
  \end{minipage}
  \medskip

  La commande \cmd{\includegraphics} permet d'insérer une image dans
  un document {\LaTeX}. Elle provient du paquetage \pkg{graphicx}.

  \begin{enumerate}
  \item Étudier le code source du fichier maître, puis le compiler
    deux à trois fois jusqu'à ce que tous les renvois soient à jour.
    Il est normal à ce stade que la figure~1 du document soit vide.
  \item Ajouter dans le préambule du fichier maître la commande
\begin{lstlisting}
\includeonly{emacs}
\end{lstlisting}
    puis compiler le document.

    Observer que, malgré l'absence du chapitre~1, la numérotation et
    les références demeurent à jour, notamment la table des matières.
  \item Remplacer la commande ajoutée en b) dans le préambule du
    fichier maître par la commande
\begin{lstlisting}
\includeonly{rpresentation}
\end{lstlisting}
    Vers la fin du fichier \fichier{rpresentation.tex}, activer la
    commande
\begin{lstlisting}
\includegraphics[width=\textwidth]{console-screenshot}
\end{lstlisting}
    en supprimant le symbole \% au début de la ligne. Compiler de
    nouveau le document deux fois.

    Les modifications ont eu pour effet d'ajouter une page au
    chapitre~1. Observer que, selon la table des matières, le
    chapitre~2 débute toujours à la page~3 alors que celle-ci est
    maintenant occupée par la figure~1.
  \item Afin de corriger la table des matières, désactiver dans le
    préambule du fichier maître la commande \cmd{\includeonly}, puis
    compiler de nouveau le document quelques fois.
  \end{enumerate}
\end{exercice}

\begin{exercice}[nosol]
  Déplacer dans un fichier \fichier{preambule.tex} toutes les lignes
  du préambule du fichier \fichier{exercice\_include.tex} utilisé à
  l'exercice précédent, à l'exception de celles relatives à la page
  titre (titre, auteur, date). Insérer le préambule dans
  \fichier{exercice\_include.tex} avec la commande \cmd{\input}.
\end{exercice}

\Closesolutionfile{solutions}

%%% Local Variables:
%%% mode: latex
%%% TeX-engine: xetex
%%% TeX-master: "formation-latex-ul"
%%% coding: utf-8
%%% End:

version https://git-lfs.github.com/spec/v1
oid sha256:63bdd8292fea5691521a32972fd47b9b9e58fe984b40afc1b7086462f5a5f5bd
size 19777

\chapter{Boîtes}
\label{chap:boites}

Il arrive que l'on doive traiter de manière spéciale une aire
rectangulaire de texte; pour l'encadrer, la mettre en surbrillance ou
la mettre en exergue, par exemple.

Avec les traitements de texte, on aura souvent recours aux tableaux à
de telles fins. Or, les tableaux devraient être réservés à la
disposition d'information sous forme de lignes et de colonnes. Pour
disposer et mettre en forme tout autre type contenu se présentant sous
forme rectangulaire, {\LaTeX} offre la solution plus générale des
«boîtes».

Il existe trois sortes de boîtes en {\LaTeX}: les boîtes horizontales,
dont le contenu est disposé exclusivement côte à côte; les boîtes
verticales, qui peuvent contenir plusieurs lignes de contenu; les
boîtes de réglure pour former des lignes pleines de largeur et de
hauteur quelconques.

Il n'est pas inutile de savoir, au passage, que {\TeX} ne manipule que
cela, des boîtes. Pour {\TeX}, chaque caractère, chaque lettre n'est
qu'un rectangle d'une certaine largeur qui s'élève au-dessus de la
ligne de base (les lignes d'une feuille lignée) et qui, parfois, se
prolonge sous la ligne de base (pensons aux lettres \emph{p}, \emph{y}
ou \emph{Q}). Les commandes et environnements présentés ci-dessous
permettent simplement de créer d'autres boîtes dont le contrôle des
dimensions et du contenu est laissé à l'usager.

Une fois créée, une boîte ne peut être scindée en parties, notamment
entre les lignes ou entre les pages.


\section{Boîtes horizontales}
\label{sec:boites:lrbox}

Le concept de boîte le plus simple dans {\LaTeX} est celui de boîte
«horizontale», c'est-à-dire dont le contenu est disposé latéralement
de gauche à droite\footnote{%
  D'où l'appellation \emph{LR (left-right) box} en anglais.}. %
Le contenu est normalement du texte, mais conceptuellement ce pourrait
être n'importe quoi, y compris d'autres boîtes.

Les commandes de base pour créer des boîtes horizontales sont:
\begin{lstlisting}
\mbox`\marg{texte}'
\fbox`\marg{texte}'
\end{lstlisting}
Elles produisent une boîte de la largeur précise de \meta{texte}. Avec
la commande \cmd{\fbox}, le texte est au surplus \fbox{encadré}.

\begin{conseil}
  En usage courant, la commande \cmd{\mbox} sert principalement à deux
  choses:
  \begin{enumerate}
  \item réunir en un bloc du texte que l'on ne veut pas voir scindé
    entre les lignes ou entre les pages;
  \item \label{item:boites:mbox} créer une boîte vide avec
    \cs{mbox\{\}} afin de laisser croire à {\TeX} que du contenu
    apparaît à un endroit, sans toutefois qu'il n'occupe aucun espace.
  \end{enumerate}
  La seconde utilisation fait l'objet de
  l'\autoref{ex:boites:alignement-v}.
\end{conseil}

Il existe également des versions plus générales des commandes
\cmd{\mbox} et \cmd{\fbox}:
\begin{lstlisting}
\makebox`\oarg{largeur}\oarg{pos}\marg{texte}'
\framebox`\oarg{largeur}\oarg{pos}\marg{texte}'
\end{lstlisting}
Les arguments optionnels \meta{largeur} et \meta{pos} déterminent
respectivement la largeur de la boîte et la position du texte
dans la boîte. Les valeurs possibles de \meta{pos} sont: \code{l} pour
du texte aligné à gauche, \code{r} pour du texte aligné à droite et
\code{c} (la valeur par défaut) pour du texte centré. Ainsi, la commande
\begin{lstlisting}
\framebox[3.5cm][l]{aligné à gauche}
\end{lstlisting}
produit \framebox[3.5cm][l]{aligné à gauche}, alors que
\begin{lstlisting}
\makebox[3.5cm]{centré}
\end{lstlisting}
produit \makebox[3.5cm]{centré}.

Il est parfois nécessaire d'ajuster le positionnement vertical
d'éléments de contenu, notamment pour les symboles ou les images. La
commande
\begin{lstlisting}
\raisebox`\marg{déplacement}\marg{texte}'
\end{lstlisting}
produit une boîte horizontale dont le contenu \meta{texte} est
surélevé de la longueur \meta{déplacement} par rapport à la ligne de
base. Si \meta{déplacement} est négatif, la boîte est positionnée sous
la ligne de base.
\begin{demo}
  \begin{texample}[0.55\textwidth]
\begin{lstlisting}
Texte \raisebox{1ex}{au-dessus}
de la ligne de base.
\end{lstlisting}
    \producing
    Texte \raisebox{1ex}{au-dessus}
    de la ligne de base.
  \end{texample}

  \begin{texample}[0.55\textwidth]
\begin{lstlisting}
Texte \raisebox{-1ex}{au-dessous}
de la ligne de base.
\end{lstlisting}
    \producing
    Texte \raisebox{-1ex}{au-dessous}
    de la ligne de base.
  \end{texample}
\end{demo}

Attention, toutefois, de ne pas utiliser \cmd{\raisebox} pour placer
du texte en exposant ou en indice. Selon la nature du texte, employer
plutôt les commandes \cmd{\textsuperscript} et \cmd{\textsubscript},
les commandes de la famille \cmd{\ieme} de \pkg{babel} (section~1.1 de
la %
\doc{frenchb}{http://texdoc.net/pkg/babel-french/}) ou, pour des
symboles mathématiques, les commandes d'exposant et d'indice
spécifiques au mode mathématique (\autoref{sec:math:bases:exposants}).

\section{Boîtes verticales}
\label{sec:boites:parbox}

Les boîtes verticales se distinguent des boîtes horizontales par le
fait qu'elles peuvent contenir plusieurs lignes de contenu empilées
les unes au-dessus des autres. Lorsque le contenu en question est du
texte, on obtient des paragraphes\footnote{%
  D'où l'appellation de \emph{paragraph boxes} en anglais
  ou \emph{parboxes} dans le jargon {\LaTeX}.}. %

La commande de base pour créer une boîte verticale est:
\begin{lstlisting}
\parbox`\oarg{pos}\marg{largeur}\marg{texte}'
\end{lstlisting}
Ici, l'argument optionnel \meta{pos} permet d'ajuster l'alignement
vertical de la boîte avec la ligne de base: \code{b} ou \code{t} selon
que l'on souhaite aligner, respectivement, le bas ou le haut de la
boîte avec la ligne de base. Par défaut, la boîte est centrée avec la
ligne de base. Cet argument n'a aucun effet si la boîte est le seul
élément de contenu du paragraphe.

On remarquera que l'argument \meta{largeur} est ici obligatoire.
Autrement dit, on doit nécessairement définir la largeur des boîtes
verticales, un peu comme il faut bien définir la largeur de la page
pour le texte normal (la classe se charge de ce détail).

Les boîtes créées avec \cmd{\parbox} ne peuvent contenir de structures
«complexes» comme des listes ou des tableaux. Parce que plus général,
l'outil véritablement utile pour la création de boîtes verticales est
l'environnement \Pe{minipage}. Cet environnement peut contenir à peu
n'importe quel type de contenu. Comme son nom l'indique, c'est ni plus
ni moins qu'une page miniature à l'intérieur de la page standard.

La syntaxe de l'environnement \Ie{minipage} est la suivante:
\begin{lstlisting}
\begin{minipage}`\oarg{pos}\marg{largeur}'
  `\meta{texte}'
\end{minipage}
\end{lstlisting}
La signification des arguments \meta{largeur} et \meta{pos} est la
même que pour la commande \cmd{parbox}.

L'environnement \Pe{minipage} est fréquemment utilisé pour disposer
des éléments de contenu de manière spécifique sur la page, notamment
des tableaux ou des figures côte à côte ou en grille (voir
l'\autoref{exemple:tableaux:grille} à la
\autopageref{exemple:tableaux:grille}).

\begin{exemple}
  L'agencement de boîtes ci-dessous est produit avec le code qui suit
  immédiatement.  \\[0.5\baselineskip]
  \begin{minipage}{\textwidth}
    \makebox[0pt][l]{\color{lightgray}\rule{\textwidth}{0.7pt}}\relax
    \fbox{\begin{minipage}[b]{0.3\textwidth} La ligne inférieure de
        cette \emph{minipage} est alignée avec
      \end{minipage}} \hfill \fbox{\parbox{0.3\textwidth}{le centre de
        cette boîte verticale, qui est à son tour alignée avec}}
    \hfill \fbox{\begin{minipage}[t]{0.3\textwidth} la ligne
        supérieure de cette \emph{minipage}. Le filet horizontal grisé
        représente la ligne de base du paragraphe contenant les trois
        boîtes.
      \end{minipage}}
  \end{minipage}
\begin{lstlisting}
\begin{minipage}[b]{0.3\textwidth}
  La ligne inférieure de cette \emph{minipage} [...]
\end{minipage}
\hfill
\parbox{0.3\textwidth}{le centre de cette boîte [...] }
\hfill
\begin{minipage}[t]{0.3\textwidth}
  la ligne supérieure de cette \emph{minipage}. [...]
\end{minipage}
\end{lstlisting}
  \qed
\end{exemple}

La commande \cmd{\hfill} utilisée entre les boîtes dans l'exemple
ci-dessus indique à {\LaTeX} d'insérer de l'espace entre les éléments
de contenu de manière à remplir entièrement la ligne de texte. C'est
une commande très utile pour disposer automatiquement des éléments à
intervalles égaux sur la largeur du bloc de texte. Ainsi,
\begin{lstlisting}
\framebox[\linewidth]{gauche \hfill droite}
\end{lstlisting}
produit \\[0.5\baselineskip]
\framebox[\linewidth]{gauche \hfill droite} \\[0.5\baselineskip]
alors que
\begin{lstlisting}
\framebox[\linewidth]{gauche \hfill centre \hfill droite.}
\end{lstlisting}
produit \\[0.5\baselineskip]
\framebox[\linewidth]{gauche \hfill centre \hfill droite.}



\section{Boîtes de réglure}
\label{sec:boites:rulebox}

En imprimerie, une réglure est une ligne droite continue ou
pointillée. Une ligne n'étant jamais rien d'autre qu'un rectangle
plein, si mince fut-il, la réglure est le troisième type de
boîte\footnote{%
  \emph{Rule box}, en anglais} %
dans {\LaTeX}.

La commande
\begin{lstlisting}
\rule`\oarg{déplacement}\marg{largeur}\marg{hauteur}'
\end{lstlisting}
crée une réglure de dimensions \meta{largeur} $\times$ \meta{hauteur}.
Par défaut, la réglure s'appuie sur la ligne de base. Le résultat de
\begin{lstlisting}
\rule{2cm}{6pt}
\end{lstlisting}
est donc une ligne pleine de $2$~cm de long et de $6$~points d'épais:
\rule{2cm}{6pt}.

L'argument optionnel \meta{déplacement} permet de déplacer
verticalement la réglure au-dessus ou au-dessous de la ligne de base
selon que la longueur \meta{déplacement} est positive ou négative. Avec les deux
commandes
\begin{lstlisting}
\rule[3pt]{2cm}{6pt}
\rule[-3pt]{2cm}{6pt}
\end{lstlisting}
on crée respectivement les réglures \rule[3pt]{2cm}{6pt} et
\rule[-3pt]{2cm}{6pt}.

Un usage intéressant de la réglure consiste à faire croire à {\TeX}
qu'une ligne est plus haute qu'il n'y paraît en insérant dans celle-ci
une réglure de largeur nulle. Par exemple, la distance entre
\rule[-12pt]{0mm}{30pt}\relax la présente ligne et les autres du paragraphe est
plus grande que la normale parce que nous y avons inséré une réglure
invisible avec
\begin{lstlisting}
\rule[-12pt]{0mm}{30pt}
\end{lstlisting}
Ce truc est particulièrement utile pour augmenter la hauteur des
lignes dans un tableau; voir la \autoref{sec:tableaux:tableaux}.



%%%
%%% Exercices
%%%

\section{Exercices}
\label{sec:boites:exercices}

\Opensolutionfile{solutions}[solutions-boites]

\begin{Filesave}{solutions}
\section*{Chapitre \ref*{chap:boites}}
\addcontentsline{toc}{section}{Chapitre \protect\ref*{chap:boites}}

\end{Filesave}

\noindent%
Utiliser comme canevas le fichier \fichier{exercice\_gabarit.tex} pour
tous les exercices ci-dessous.

\begin{exercice}
  Une fois qu'une boîte est définie, {\TeX} n'y voit qu'une unité de
  contenu avec ses dimensions propres. Il est donc possible de définir
  une boîte à l'intérieur d'une autre, et ce, peu importe le type de
  boîte.

  Avec ceci en tête, définir la boîte suivante:
  \begin{center}
    \fbox{\fbox{%
        \parbox{10cm}{Ce bloc de texte est une boîte verticale de
          10~cm de large, doublement encadrée et centrée sur la
          ligne.}}}
  \end{center}

  \begin{sol}
    Une première boîte verticale de 10~cm de large contient le texte:
\begin{lstlisting}
\parbox{10cm}{Ce bloc [...] la ligne.}
\end{lstlisting}
    Cette boîte peut être placée dans une boîte horizontale encadrée
    avec \cmd{\fbox}. Celle-ci peut à son tour être placée dans une autre
    boîte horizontale encadrée, de manière à obtenir un cadrage
    double. Pour centrer le tout sur la ligne, on a recours à
    l'environnement \Ie{center}:
\begin{lstlisting}
\begin{center}
  \fbox{\fbox{\parbox{10cm}{Ce bloc [...] la ligne.}}}
\end{center}
\end{lstlisting}
  \end{sol}
\end{exercice}

\begin{exercice}
  \label{ex:boites:alignement-v}
  Réaliser l'agencement de boîtes verticales suivant:
  \begin{center}
    \begin{minipage}{0.8\linewidth}
      \makebox[0pt][l]{\color{lightgray}\rule{\linewidth}{0.7pt}}\relax
      \hfill
        \begin{minipage}[b]{0.95\linewidth}
          \small
          \parbox[t]{0.45\linewidth}{Deux boîtes verticales de
            hauteurs différentes placées côte à côte}
          \hfill
          \parbox[t]{0.45\linewidth}{alignées sur
            leurs premières lignes et le bas de la boîte
            la plus haute alignée sur la ligne de base (représentée
            ici par le filet horizontal grisé).} \\
          \mbox{}
        \end{minipage}
      \hfill
    \end{minipage}
  \end{center}

  La solution intuitive serait la suivante:
\begin{lstlisting}
\begin{minipage}[b]{...}
  \parbox[t]{...}{...} \hfill \parbox[t]{...}{...}
\end{minipage}
\end{lstlisting}
  Cependant, cette solution produit le résultat suivant (les boîtes
  sont rendues visibles par des cadres):
  \begin{center}
    \begin{minipage}{0.8\linewidth}
      \makebox[0pt][l]{\color{lightgray}\rule{\linewidth}{0.7pt}}\relax
      \hfill
      \fbox{%
        \begin{minipage}[b]{0.95\linewidth}
          \small
          \fbox{%
            \parbox[t]{0.45\linewidth}{Les deux boîtes sont
              correctement alignées l'une par rapport à l'autre}} \hfill
          \fbox{\parbox[t]{0.45\linewidth}{mais l'alignement avec la
              ligne de base est incorrect.}}
        \end{minipage}}
      \hfill
    \end{minipage}
  \end{center}
  La raison: pour {\TeX}, la \Ie{minipage} externe ne contient que
  deux «caractères» sur une seule ligne de «texte». La \Pe{minipage}
  est donc correctement alignée sur sa ligne du bas, mais celle-ci se
  trouve aussi être la ligne du haut.

  Pour parvenir au résultat escompté, utiliser la commande \cmd{\mbox}
  pour créer une seconde ligne (vide) dans la \Pe{minipage} externe.
  \begin{sol}
    L'idée consiste à créer une seconde ligne dans la \Pe{minipage}
    externe sans que celle-ci n'occupe aucun espace. Pour ce faire, on
    insère du contenu vide avec \cs{mbox\{\}}, tel qu'expliqué à la
    \autopageref{item:boites:mbox}. Le code
\begin{lstlisting}
\begin{minipage}[b]{...}
  \parbox[t]{...}{...} \hfill \parbox[t]{...}{...} \\
  \mbox{}
\end{minipage}
\end{lstlisting}
    produit donc le résultat voulu:
    \begin{center}
      \begin{minipage}{0.8\linewidth}
        \makebox[0pt][l]{\color{lightgray}\rule{\linewidth}{0.7pt}}\relax
        \hfill
        \fbox{%
          \begin{minipage}[b]{0.95\linewidth}
            \small
            \fbox{%
              \parbox[t]{0.45\linewidth}{Les boîtes sont rendues visibles
                par des cadres}} \hfill
            \fbox{\parbox[t]{0.45\linewidth}{et le filet horizontal grisé
                représente la ligne de base du paragraphe courant.}} \\
            \fbox{\mbox{}}
          \end{minipage}}
        \hfill
      \end{minipage}
  \end{center}
  (Sans le cadre, la boîte de la seconde ligne n'occupe aucun espace.)
  \end{sol}
\end{exercice}

\begin{exercice}
  Réaliser l'agencement de boîtes verticales ci-dessous. (La taille de
  la police est \cs{footnotesize}.)
  \begin{center}
    \begin{minipage}{120mm}
      \footnotesize
      \begin{minipage}[b]{80mm}
        \parbox[t]{30mm}{La première ligne de cette \emph{parbox} de
          $30$~mm de large est alignée avec celle de la boîte
          voisine.}
        \hfill
        \parbox[t]{45mm}{Cette \emph{parbox} de $45$~mm de large est
          positionnée de telle sorte que sa première ligne soit
          alignée avec le haut de la boîte à gauche et la dernière
          avec le bas de la boîte à droite. La solution intuitive
          consistant à placer côte à côte trois boîtes avec des
          arguments de positionnement \code{t}, \code{t} et \code{b}
          ne fonctionne pas.} \\
        \mbox{}
      \end{minipage}
      \hfill
      \parbox[b]{35mm}{Pour parvenir à cette disposition, il faut
        avoir recours à des lignes invisibles comme dans l'exercice
        précédent.}
    \end{minipage}
  \end{center}
  La troisième boîte fait $35$~mm de large et l'espace entre les
  boîtes, $5$~mm.
  \begin{sol}
    La solution la plus simple consiste à réunir les deux premières
    boîtes dans une \Pe{minipage} dans laquelle les deux boîtes seront
    alignées tel que désiré, puis à aligner la \Pe{minipage} avec la
    troisième boîte. Cependant, il faut insérer une seconde ligne
    invisible dans la \Pe{minipage} afin de pouvoir l'aligner par le
    bas avec la boîte de droite:
\begin{lstlisting}
\begin{minipage}[b]{80mm}
  \parbox[t]{30mm}{...} \hfill \parbox[t]{45mm}{...} \\
  \mbox{}
\end{minipage}
\hfill
\parbox[b]{35mm}{...}
\end{lstlisting}
  \end{sol}
\end{exercice}

\Closesolutionfile{solutions}

%%% Local Variables:
%%% mode: latex
%%% TeX-engine: xetex
%%% TeX-master: "formation-latex-ul"
%%% coding: utf-8
%%% End:

\chapter{Tableaux et figures}
\label{chap:tableaux}

Les tableaux et graphiques ne sont pas les éléments de texte les plus
simples et rapides à créer avec {\LaTeX}. Les traitements de texte
brillent, ici, avec leurs interfaces graphiques permettant de composer
un tableau ou un graphique simple pièce par pièce avec la souris.

En revanche, pour ce type de contenu comme pour tout autre, {\LaTeX}
fait ce qu'on lui demande, sans tenter de deviner notre pensée ou,
pire, de prétendre savoir mieux que nous ce que nous voulons faire. À
ce chapitre, les traitements de texte ne brillent plus! Quiconque a
déjà eu de la difficulté à contrôler les bordures d'un tableau, la
hauteur des lignes ou la largeur des colonnes dans un traitement de
texte comprendra combien l'exercice de composition d'un tableau avec
ces outils peut rapidement devenir frustrant.

Avant de discuter de la création ou de l'insertion de tableaux, de
graphiques et d'images dans un document {\LaTeX}, il convient de
présenter très succinctement quelques règles à suivre pour concevoir
des tableaux clairs et faciles à consulter.


\section{De la conception de beaux tableaux}
\label{sec:tableaux:booktabs}

Les tableaux servent à disposer de l'information sous forme de
grille. Par conséquent, le premier réflexe pour les mettre en forme
consiste souvent à mettre en évidence cette grille par le biais de
filets\footnote{%
  Terme typographique pour ce qui est communément appelés des «lignes»
  dans le langage courant ou des «bordures» dans les logiciels de
  traitement de texte. Dans la documentation en anglais, on parle de
  \emph{rules}.} %
horizontaux et verticaux.

C'est une mauvaise idée, une pratique à éviter. Vraiment!

Comparer les deux tableaux ci-dessous. Le premier est mis en forme
selon une approche classique supportée depuis toujours par {\LaTeX}:
filets doubles en entête et en pied de tableau, filets simples entre
chaque ligne et entre les colonnes.

\begin{center}
  \begin{tabular}{|>{$}c<{$}|>{$}r<{$}|>{$}r<{$}|>{$}r<{$}|>{$}c<{$}|>{$}c<{$}|}
    \hline\hline
    i &
    \multicolumn{1}{c|}{$v$} &
    \multicolumn{1}{c|}{$b_i$} &
    \lfloor v/b_i \rfloor & v \bmod b_i & x_i \\
    \hline
    0 & \nombre{91492} &  60 & \nombre{1524} & 52 & 52 \\
    \hline
    1 &  \nombre{1524} &  60 &           25  & 24 & 24 \\
    \hline
    2 &            25  &  24 &            1  &  1 &  1 \\
    \hline
    3 &             1  & 365 &            0  &  1 &  1 \\
    \hline\hline
  \end{tabular}
\end{center}

Le second tableau tire profit des fonctionnalités du paquetage
\pkg{booktabs} \citep{booktabs} et des recommandations de son auteur:
les filets horizontaux sont d'épaisseur différente selon qu'ils sont
situés dans l'entête et dans le pied du tableau ou entre les lignes,
l'espace autour des filets horizontaux est plus grand et, surtout, il
n'y a pas de filets verticaux.

\begin{center}
  \begin{tabular}{>{$}c<{$}>{$}r<{$}>{$}r<{$}>{$}r<{$}>{$}c<{$}>{$}c<{$}}
    \toprule
    i &
    \multicolumn{1}{c}{$v$} &
    \multicolumn{1}{c}{$b_i$} &
    \lfloor v/b_i \rfloor & v \bmod b_i & x_i \\
    \midrule
    0 & \nombre{91492} &  60 & \nombre{1524} & 52 & 52 \\
    1 &  \nombre{1524} &  60 &           25  & 24 & 24 \\
    2 &            25  &  24 &            1  &  1 &  1 \\
    3 &             1  & 365 &            0  &  1 &  1 \\
    \bottomrule
  \end{tabular}
\end{center}

La seconde version n'est-elle pas la plus aérée et la plus facile à
consulter? N'est-ce pas que, contrairement à ce que l'on pourrait
penser, les filets verticaux ne sont pas du tout requis pour bien
délimiter les colonnes?

Tel que mentionné ci-dessus, le paquetage \pkg{booktabs} ajoute des
fonctionnalités à {\LaTeX} pour améliorer la qualité typographique des
tableaux. Dans la %
\doc{booktabs}{http://texdoc.net/pkg/booktabs} %
du paquetage, son auteur énonce quelques règles à suivre pour la mise
en forme des tableaux:
\begin{enumerate}
\item ne \emph{jamais} utiliser de filets verticaux. Si l'information
  du côté gauche du tableau semble si différente de celle du côté
  droit qu'un filet vertical apparaît absolument nécessaire, scinder
  simplement l'information dans deux tableaux;
\item ne jamais utiliser de filets doubles;
\item placer les unités (\$, cm, {\textdegree}C, etc.) dans le titre
  de la colonne plutôt qu'après chaque valeur dans le corps du
  tableau;
\item toujours inscrire un chiffre du côté gauche du séparateur
  décimal: $0,1$ et non $,1$ (pratique plus répandue en anglais, où le
  séparateur décimal est le point);
\item ne pas utiliser un symbole pour représenter une valeur
  répétée (comme $''$ ou ---). Laisser un blanc ou répéter la
  valeur s'il subsiste une ambiguïté.
\end{enumerate}

Nous recommandons évidemment de suivre ces règles et c'est pourquoi la
présente documentation ainsi que les fichiers d'exemples font usage
des commandes de \pkg{booktabs}.

Les fonctionnalités de \pkg{booktabs} sont intégrées à la classe
\class{memoir} et, par conséquent, à \class{ulthese}. Il n'est donc pas
nécessaire de charger le paquetage avec ces deux classes.



\section{Tableaux}
\label{sec:tableaux:tableaux}

Peu importe l'outil informatique utilisé, la création d'un tableau
requiert toujours de préciser à l'ordinateur le nombre de colonnes que
contiendra le tableau, l'entête du tableau le cas échéant et le contenu
des différentes cellules. Cette dernière étape nécessite à son tour
une convention pour indiquer les passages à la colonne suivante
ainsi que le passage à la ligne suivante.

On crée des tableaux dans {\LaTeX} principalement avec les
environnements \Ie{tabular}, \Ie{tabular*} et \Ie{tabularx} (ce
dernier fourni par le paquetage \pkg{tabularx} ou par la classe
\class{memoir}). La syntaxe de ces environnements est:
\begin{lstlisting}
\begin{tabular}`\marg{format}' `\meta{lignes}' \end{tabular}
\begin{tabular*}`\marg{largeur}\marg{format}' `\meta{lignes}' \end{tabular*}
\begin{tabularx}`\marg{largeur}\marg{format}' `\meta{lignes}' \end{tabularx}
\end{lstlisting}
La signification des arguments\footnote{%
  Nous avons omis un argument optionnel à peu près jamais utilisé
  servant à spécifier l'alignement vertical du tableau par rapport à
  la ligne de base externe.} %
est la suivante. Nous ne traitons ici que les options les plus souvent
utilisées. Pour une liste plus exhaustive, consulter la %
\doc{memman}{http://texdoc.net/pkg/memman} %
de la classe \class{memoir} (chapitre 11) ou %
\citet[section
\link{http://fr.wikibooks.org/wiki/LaTeX/Tableaux}{Tableaux}]{wikilivres:latex}.

\begin{list}{}{%
    \setlength{\labelsep}{1.5ex}
    \settowidth{\labelwidth}{\meta{largeur}}
    \setlength{\leftmargin}{\labelwidth}
    \addtolength{\leftmargin}{\labelsep}
    \setlength{\parsep}{0.5ex plus0.2ex minus0.2ex}
    \setlength{\itemsep}{0.3ex}
    \renewcommand{\makelabel}[1]{\meta{#1}\hfill}}
%
\item[largeur] Largeur hors tout d'un tableau avec les environnements
  \Pe{tabular*} et \Pe{tabularx}. Autrement, avec l'environnement
  \Pe{tabular}, la largeur d'un tableau est déterminée automatiquement
  pour contenir tout le tableau, quitte à dépasser dans la marge de
  droite.

  La largeur du tableau est généralement exprimée en fraction de la
  largeur du bloc de texte (longueur \cmd{!textwidth}). Par exemple,
  les déclarations suivantes définissent respectivement des tableaux
  occupant toute la largeur d'une page et 80~\% de la largeur de la
  page:
\begin{lstlisting}
\begin{tabular*}{\textwidth}`\marg{format}'
\end{lstlisting}
\begin{lstlisting}
\begin{tabularx}{0.8\textwidth}`\marg{format}'
\end{lstlisting}
  L'environnement \Ie{tabular*} joue sur l'espace entre les colonnes
  pour parvenir à la largeur prescrite, alors que \Ie{tabularx} joue
  sur la largeur des colonnes (voir ci-dessous).
  %
\item[format] Le format des colonnes et, par le fait même, le nombre
  de colonnes puisque l'argument doit compter un symbole pour chaque
  colonne du tableau. Les principaux symboles de mise en forme des
  colonnes sont:
  \begin{description}
  \item[\normalfont\code{l}] contenu de la colonne aligné à gauche;
  \item[\normalfont\code{r}] contenu de la colonne aligné à droite;
  \item[\normalfont\code{c}] contenu de la colonne centré;
  \item[\normalfont\code{p\marg{lgr}}] contenu de la
    colonne traité comme un paragraphe de texte de largeur
    \meta{lgr};
  \item[\normalfont\code{X}] [environnement \Pe{tabularx} seulement]
    colonne dont la largeur peut être ajustée pour obtenir un tableau
    de la largeur prescrite; identique à \code{p} par ailleurs.
  \end{description}
  Par exemple, la déclaration
\begin{lstlisting}
\begin{tabular}{lrp{5cm}}
\end{lstlisting}
  définit un tableau à trois colonnes dont le contenu de la première
  est aligné à gauche; celui de la seconde est aligné à droite; celui de
  la troisième est en texte libre dans une cellule de 5~cm de largeur.

  Avec la déclaration
\begin{lstlisting}
\begin{tabularx}{\textwidth}{lrX}
\end{lstlisting}
  la largeur de la troisième colonne sera plutôt adaptée
  automatiquement pour que le tableau occupe toute la largeur de la
  page.

  Les symboles \verb=|= et \verb=||= dans \textit{format} servent à
  insérer des filets verticaux simples et doubles entre les colonnes,
  mais nous avons vu à la \autoref{sec:tableaux:booktabs} que c'est
  une pratique à proscrire.
  %
\item[lignes] Le contenu des cellules du tableau. Les entrées des
  cellules sont séparées par le symbole \verb=&= et les lignes par
  {\pixbsbs}. Une cellule peut être vide.
\end{list}

Outre du texte, les lignes de contenu peuvent contenir certaines
commandes spéciales pour contrôler la mise en forme du tableau. En
premier lieu, la commande
\begin{lstlisting}
\multicolumn`\marg{n}\marg{fmt}\marg{texte}'
\end{lstlisting}
permet de fusionner les \meta{n} cellules suivantes en une seule de
format \meta{fmt} et d'y placer \meta{texte}. Cette commande ne peut
apparaître qu'au début d'une ligne ou après un symbole de changement
de colonne \verb=&=. Elle est souvent utilisée avec une valeur de
\meta{n} égale à 1 pour changer le format d'une cellule, par exemple
pour centrer le titre d'une colonne qui est autrement alignée à gauche
ou à droite.

Ensuite, les commandes suivantes\footnote{%
  Ce sont les commandes de \pkg{booktabs} et \class{memoir} auxquelles
  nous faisions référence à la \autoref{sec:tableaux:booktabs}.} %
servent à insérer des filets horizontaux dans un tableau:
\begin{lstlisting}
\toprule
\midrule
\cmidrule`\marg{m-n}'
\bottomrule
\end{lstlisting}
La commande \cmd{\toprule} insère un filet horizontal épais suivi d'un
espace vertical au début d'un tableau; \cmd{\midrule} insère un filet
horizontal mince précédé et suivi d'un espace vertical entre deux
lignes; \cmd{\cmidrule}\marg{m-n} insère un filet horizontal comme
\cmdprint{\midrule} de la gauche de la colonne \meta{m} à la droite de
la colonne \meta{n}; enfin, \cmd{\bottomrule} insère un filet
horizontal épais précédé d'un espace vertical à la fin d'un tableau.
Une fin de ligne {\bs\bs} doit obligatoirement précéder chacune de ces
commandes, sauf évidemment \cmdprint{\toprule}.

La hauteur des lignes d'un tableau est déterminée automatiquement en
fonction du contenu de celles-ci.

\begin{exemple}
  \label{exemple:tableaux:tabular:1}
  Considérer le tableau suivant:
  \begin{center}
    \begin{tabular}{lrrr}
      \toprule
      Produit & Quantité & Prix unitaire (\$) & Prix (\$) \\
      \midrule
      Vis à bois    & 2 & 9,90 & 19,80 \\
      Clous vrillés & 5 & 4,35 & 21,75 \\
      \midrule
      TOTAL         & 7 &      & 41,55 \\
      \bottomrule
    \end{tabular}
  \end{center}
  La largeur du tableau est ajustée au contenu, la première colonne
  est alignée à gauche et toutes les autres, à droite. Le code
  ci-dessous permet de créer ce tableau. Remarquer comment les lignes
  de contenu sont définies.
\begin{lstlisting}
\begin{tabular}{lrrr}
  \toprule
  Produit & Quantité & Prix unitaire (\$) & Prix (\$) \\
  \midrule
  Vis à bois    & 2 & 9,90 & 19,80 \\
  Clous vrillés & 5 & 4,35 & 21,75 \\
  \midrule
  TOTAL         & 7 &      & 41,55 \\
  \bottomrule
\end{tabular}
\end{lstlisting}
  \qed
\end{exemple}

\begin{exemple}
  \label{exemple:tableaux:tabular:2}
  On souhaite modifier le tableau de
  l'\autoref{exemple:tableaux:tabular:1}  pour obtenir le tableau
  suivant:
  \begin{center}
    \begin{tabularx}{\textwidth}{Xrrr}
      \toprule
      \multicolumn{1}{c}{Produit} &
      \rule[-8pt]{0mm}{24pt} Quantité & Prix unitaire (\$) & Prix (\$) \\
      \midrule
      Vis à bois    & 2 & 9,90 & 19,80 \\
      Clous vrillés & 5 & 4,35 & 21,75 \\
      \midrule
      TOTAL         & 7 &      & 41,55 \\
      \bottomrule
    \end{tabularx}
  \end{center}
  Le tableau occupe désormais toute la largeur de la page, la largeur
  de la première colonne étant ajustée pour combler l'espace
  nécessaire. De plus, le titre de la première colonne est centré et
  la hauteur de l'entête est augmentée.

  Le code suivant permet de réaliser cette mise en forme.
\begin{lstlisting}
\begin{tabularx}{\textwidth}{Xrrr}
  \toprule
  \multicolumn{1}{c}{Produit} &
    \rule[-8pt]{0mm}{24pt} Quantité &
    Prix unitaire (\$) & Prix (\$) \\
  \midrule
  Vis à bois    & 2 & 9,90 & 19,80 \\
  Clous vrillés & 5 & 4,35 & 21,75 \\
  \midrule
  TOTAL         & 7 &      & 41,55 \\
  \bottomrule
\end{tabularx}
\end{lstlisting}
  L'environnement \Ie{tabularx} sert à créer un tableau de largeur
  définie et la commande \cmd{\multicolumn}, à centrer le titre de la
  première colonne. On augmente la hauteur de l'entête à l'aide d'une
  réglure invisible (\autoref{sec:boites:rulebox}). %
  \qed
\end{exemple}


\section{Figures et graphiques}
\label{sec:tableaux:figures}

Il est possible de tracer des figures simples directement avec
{\LaTeX}. Par «simple» on entend: des figures se limitant pour
l'essentiel à du texte, des lignes, des flèches, des ronds et des
ovales. C'est parfois amplement suffisant et, en définitive, assez
pratique puisque le code source d'une figure se trouve alors dans le
même format que le reste du document.

Pour la création de figures et de graphiques plus complexes, on aura généralement
recours à des logiciels spécialisés externes. {\LaTeX} est ensuite en
mesure d'importer des graphiques dans les formats standards tels que PDF, JPEG
ou PNG, voire même d'insérer dans un document une ou plusieurs pages
d'un document PDF.

Couvrir les détails de la création et de la manipulation d'images
dépasse largement la portée du présent document. Le reste de cette
section ne présente que les principales fonctionnalités. Le lecteur
qui souhaite en savoir plus pourra se référer aux sources de documentation habituelles figurant à la
bibliographie.


\subsection{Figures {\LaTeX}}
\label{sec:tableaux:figures:picture}

L'environnement \Ie{picture} permet de tracer des figures simples
comme des diagrammes à base de texte, des flux logiques ou des
organigrammes. Quelques logiciels spécialisés de création de
graphiques sont même en mesure d'exporter leurs graphiques dans le
format de \Pe{picture}.

Une fois conçues, les figures réalisées avec \Pe{picture} sont
simples à modifier; nul besoin de recourir à un logiciel externe pour
le moindre petit changement. Autre avantage: la police du texte
de la figure sera le même que celle du document.

Pour tracer une figure avec l'environnement \Pe{picture}, on crée
d'abord une grille (invisible) d'une dimension quelconque dans l'unité
de mesure de son choix (autrement dit: les lignes de la grille
peuvent être distantes aussi bien de \code{1pt} que de \code{1cm}).
Ensuite, on dispose des éléments sur la grille en donnant les
coordonnées du point d'ancrage et, le cas échéant, les dimensions de
l'élément, la distance à parcourir ou quelqu'autre information pour
compléter l'élément. C'est souvent plus simple d'esquisser d'abord un modèle au
crayon sur du papier quadrillé.

La figure ci-dessous illustre ce qu'il est possible de faire avec
l'environnement \Pe{picture}. La consultation du code commenté
correspondant devrait permettre de comprendre les principes de base de
la création de figures. Autrement, l'annexe~D de la %
\doc{memman}{http://texdoc.net/pkg/memoir} %
de \class{memoir} fournit une bonne introduction à \Pe{picture}.

(Nous avons tracé la grille en filigrane dans la figure afin de
faciliter la comparaison entre le code et le résultat.)

\setlength{\unitlength}{7mm}
\begin{center}
  \begin{picture}(15,9)
    \linethickness{0.3pt} \color{lightgray}
    \multiput(0,0)(1,0){16}{\line(0,1){9}}
    \multiput(0,0)(0,1){10}{\line(1,0){15}}
    \color{black}

    %% boîtes
    \put(0,7){%
      \framebox(5,1.5){
        \begin{minipage}{35mm}
          \centering L'environnement \\ \texttt{picture}
        \end{minipage}}}

    \put(1,4.5){\circle{2}}
    \put(1,4.5){\makebox(0,0){\small convient}}
    \put(4,3){\circle{2}}
    \put(4,3){\makebox(0,0){\small bien}}

    \put(8.5,5.7){pour les diagrammes}

    \thicklines
    \put(8,1){\dashbox{0.2}(7,1.5){et autres figures simples.}}

    %% lignes
    \thinlines
    \put(1,7){\vector(0,-1){1.5}}

    \put(14,5.75){\circle*{0.1}}
    \put(14,5.75){\vector(-1,-1){3.25}}

    \thicklines
    \put(1,3.5){\line(0,-1){0.5}}
    \put(1,3){\vector(1,0){2}}

    \qbezier(4,2)(5.5,-0.5)(7,4.25)
    \qbezier(7,4.25)(8.5,9)(10,6.5)
    \put(10,6.5){\vector(2,-3){0}}
  \end{picture}
\end{center}

\begingroup
  \small
\begin{lstlisting}
\setlength{\unitlength}{7mm}  % unité de mesure
\begin{picture}(15,9)         % grille 15 x 9
  %%%
  %%% On trace d'abord toutes les boîtes
  %%%
  %% Rectangle "L'environnement picture"
  \put(0,7){%                 % point d'ancrage (0, 7)
    \framebox(5,1.5){%        % rectangle 5 x 1,5 plein
      \begin{minipage}{35mm}  % contenu de la boîte
        \centering L'environnement \\ \texttt{picture}
      \end{minipage}}}

  %% Cercles "convient" et "bien"
  \put(1,4.5){\circle{2}}                     % cercle diamètre 2
  \put(1,4.5){\makebox(0,0){\small convient}} % texte centré
  \put(4,3){\circle{2}}                       % autre cercle
  \put(4,3){\makebox(0,0){\small bien}}       % texte

  %% Texte "pour les diagrammes"
  \put(8.5,5.7){pour les diagrammes} % point d'ancrage (8,5, 5,7)

  %% Rectangle pointillé "et autres figures simples."
  \thicklines                     % lignes grasses
  \put(8,1){\dashbox{0.2}(7,1.5){ % rectangle 7 x 1,5 pointillé
      et autres figures simples.}}

  %%%
  %%% On trace ensuite les lignes entre les boîtes
  %%%
  %% De "L'environnement picture" à "convient"
  \thinlines                    % retour aux lignes minces
  \put(1,7){\vector(0,-1){1.5}} % flèche vers le bas longueur 1,5
                                % [couple (0,-1) donne la pente]

  %% De "pour les diagrammes" à "et autres figures simples."
  \put(14,5.75){\circle*{0.1}}  % petit cercle plein
  \put(14,5.75){\vector(-1,-1){3.25}} % flèche vers sud-ouest
                                      % [3.25 = déplacement hor.]

  %% Entre les deux cercles; requiert deux segments
  \thicklines                   % lignes grasses
  \put(1,3.5){\line(0,-1){0.5}} % courte ligne vert. sans flèche
  \put(1,3){\vector(1,0){2}}    % flèche horizontale

  %% Entre "bien" et "pour les diagrammes"; requiert deux courbes
  %% de Bézier placées bout à bout pour produire une courbe en S
  \qbezier(4,2)(5.5,-0.5)(7,4.25)  % bas du S
  \qbezier(7,4.25)(8.5,9)(10,6.5)  % haut du S
  \put(10,6.5){\vector(2,-3){0}}   % pointe de flèche seule
\end{picture}
\end{lstlisting}
\endgroup


Il existe quelques outils pour tracer des figures plus complexes
directement avec {\TeX}, dont PSTricks \citep{pstricks}
ou le système Ti\emph{k}Z/\textsc{pgf} \citep{tikz}.
Ce dernier gagne beaucoup en popularité depuis quelques années.


\subsection{Importation d'images}
\label{sec:tableaux:figures:graphics}

Il est aujourd'hui simple d'importer des images de source externes
dans un document {\LaTeX} en utilisant l'un ou l'autre des paquetages
\pkg{graphics} ou \pkg{graphicx} \citep{graphicx} en combinaison avec
un moteur {\TeX} moderne tel que pdf{\LaTeX} ou {\XeLaTeX}. Les
fonctionnalités des deux paquetages sont les mêmes, seules les
syntaxes des commandes diffèrent. Nous présenterons les commandes de
\pkg{graphicx}, plus modernes et conviviales.

La commande de base pour importer des images dans un document {\LaTeX} est
\begin{lstlisting}
\includegraphics`\oarg{options}\marg{fichier}'
\end{lstlisting}
où \meta{fichier} est le nom du fichier à importer. Il n'est pas
nécessaire de préciser l'extension dans le nom de fichier pour les
types d'images usuelles. Avec les moteurs pdf{\LaTeX} et {\XeLaTeX},
les types d'images automatiquement reconnus sont au moins PDF, JPEG et
EPS.

Les \meta{options} de \cmd{\includegraphics}, nombreuses, permettent
de redimensionner une image, de la faire pivoter ou encore de n'en
importer qu'une partie. L'exemple ci-dessous présente les principales
fonctionnalités; consulter la %
\doc{grfguide}{http://texdoc.net/pkg/graphics/} %
pour les détails et d'autres options.

\begin{exemple}
  Le fichier \fichier{ul\_p.pdf} contenant le logo de l'Université
  Laval en couleur et en format vectoriel est distribué avec la
  présente documentation ainsi qu'avec la classe \class{ulthese}. La
  simple commande
\begin{lstlisting}
\includegraphics{ul_p}
\end{lstlisting}
  insère le fichier en pleine grandeur dans le document:
  \begin{demo}
    \includegraphics{ul_p}
  \end{demo}

  On peut redimensionner l'image en valeur relative avec l'option
  \code{scale} ou en valeur absolue avec les options \code{width} ou
  \code{height}:
  \begin{demo}
    \begin{texample}[0.62\linewidth]
\begin{lstlisting}
%% réduction à 40 % de taille réelle
\includegraphics[scale=0.4]{ul_p}
\end{lstlisting}
      \producing
      \includegraphics[scale=0.4]{ul_p}
    \end{texample}
    \medskip

    \begin{texample}[0.62\linewidth]
\begin{lstlisting}
%% réduction à 15 mm de haut
\includegraphics[height=15mm]{ul_p}
\end{lstlisting}
      \producing
      \includegraphics[height=15mm]{ul_p}
    \end{texample}
  \end{demo}
  (Il est préférable d'utiliser une seule de \code{width} ou
  \code{height}. Autrement, ajouter l'option
  \lstinline|keepaspectratio=true| pour éviter de déformer l'image.)

  L'option \code{angle} permet de faire pivoter l'image dans le sens
  inverse des aiguilles d'une montre autour du coin inférieur gauche
  de l'image:
  \begin{demo}
    \begin{texample}[0.72\linewidth]
\begin{lstlisting}
%% réduction à 25 % et rotation à 45 degrés
\includegraphics[angle=45,scale=0.25]{ul_p}
\end{lstlisting}
      \producing
      \includegraphics[angle=45,scale=0.25]{ul_p}
    \end{texample}
  \end{demo}

  Enfin, il y a diverses manières de sélectionner une partie seulement
  d'une image. L'option \code{bb} (pour \emph{Bounding Box}) prend
  quatre mesures en points PostScript (\autoref{tab:bases:longueurs})
  définissant le coin inférieur gauche et le coin supérieur droit de
  la zone à inclure:
  \begin{demo}
    \begin{texample}[0.72\linewidth]
\begin{lstlisting}
%% extraction du logo seul et réduction
\includegraphics[bb=0 0 102 129,clip=true,
  scale=0.4]{ul_p}
\end{lstlisting}
      \producing
      \includegraphics[bb=0 0 102 129, clip=true, scale=0.4]{ul_p}
    \end{texample}
  \end{demo}
  \qed
\end{exemple}

La commande \cmd{\includegraphics} permet d'appliquer certaines
transformations aux images importées. Ces transformations peuvent
également s'effectuer à l'aide de commandes externes \emph{après}
l'importation. L'avantage de ces commandes, c'est qu'elles sont
valides tout autant pour du texte que pour des images.

Le paquetage \pkg{graphicx} définit les commandes suivantes:
\begin{lstlisting}
\rotatebox`\oarg{options}\marg{angle}\marg{texte}'
\scalebox`\marg{échelle-h}\oarg{échelle-v}\marg{texte}'
\resizebox`\marg{dim-h}\marg{dim-v}\marg{texte}'
\reflectbox`\marg{texte}'
\end{lstlisting}
Dans tous les cas, \meta{texte} peut être du simple texte ou une boîte
quelconque, y compris le résultat de \cmd{\includegraphics}. Ainsi,
\begin{lstlisting}
\rotatebox{45}{\includegraphics{ul_p}}
\end{lstlisting}
et
\begin{lstlisting}
\includegraphics[angle=45]{ul_p}
\end{lstlisting}
donnent le même résultat.

Avec \cmd{\scalebox}, la mise à l'échelle \meta{échelle-h} s'applique
par défaut autant à l'horizontale qu'à la verticale. Autrement,
\meta{texte} est déformé.  Avec \cmd{\resizebox}, on peut spécifier
l'une de \meta{dim-h} ou \meta{dim-v} et \verb=!= pour l'autre valeur
pour éviter de déformer \meta{texte}.

\begin{exemple}
  Voici des exemples d'utilisation des commandes \cmd{\rotatebox},
  \cmd{\scalebox}, \cmd{\resizebox} et \cmd{\reflectbox} avec du texte:
  \begin{demo}
    \begin{texample}[0.55\linewidth]
\begin{lstlisting}
\rotatebox{135}{texte}
\end{lstlisting}
      \producing
      \rotatebox{135}{texte}
    \end{texample}

    \begin{texample}[0.55\linewidth]
\begin{lstlisting}
\scalebox{1.5}{texte}
\end{lstlisting}
      \producing
      \scalebox{1.5}{texte}
    \end{texample}

    \begin{texample}[0.55\linewidth]
\begin{lstlisting}
\scalebox{1.5}[0.75]{texte}
\end{lstlisting}
      \producing
      \scalebox{1.5}[0.75]{texte}
    \end{texample}

    \begin{texample}[0.55\linewidth]
\begin{lstlisting}
\resizebox{3cm}{!}{texte}
\end{lstlisting}
      \producing
      \resizebox{3cm}{!}{texte}
    \end{texample}

    \begin{texample}[0.55\linewidth]
\begin{lstlisting}
\reflectbox{texte}
\end{lstlisting}
      \producing
      \reflectbox{texte}
    \end{texample}
  \end{demo}
  \qed
\end{exemple}



\subsection{Insertion de documents PDF}
\label{sec:tableaux:figures:pdfpages}

Il est parfois utile d'insérer dans un document {\LaTeX} une ou
plusieurs pages d'un autre document en format PDF, et ce, sans avoir à
se soucier des marges respectives des deux documents. Si l'on utilise
les moteurs pdf{\LaTeX} ou {\XeLaTeX}, le très pratique paquetage
\pkg{pdfpages} \citep{pdfpages} fournit la commande
\begin{lstlisting}
\includepdf`\oarg{options}\marg{fichier}'
\end{lstlisting}
Les \meta{options} sont très nombreuses; consulter la %
\doc{pdfpages}{http://texdoc.net/pkg/pdfpages/}.

\begin{exemple}
  Il n'est pas rare que les couvertures avant et arrière d'un document
  soient réalisées dans un logiciel spécialisé de création graphique.
  Supposons que les deux couvertures sont sauvegardées en format PDF
  dans un fichier \fichier{couvertures.pdf}. Pour les utiliser dans le
  document, il suffit de placer aux endroits appropriés les commandes
\begin{lstlisting}
\includepdf[pages=1]{couvertures}
\includepdf[pages=2]{couvertures}
\end{lstlisting}
  \qed
\end{exemple}



\section{Éléments flottants}
\label{sec:tableaux:floats}

Dans la terminologie de {\LaTeX}, un élément flottant\footnote{%
  \emph{Float} en anglais.} %
est un bloc de contenu (une boîte, en fait) que le logiciel pourra
positionner sur la page et dans le document plus ou moins
automatiquement en fonction d'un algorithme prédéfini. C'est une
fonctionnalité très évoluée de {\LaTeX}.

Pourquoi voudrait-on laisser {\LaTeX} décider où un élément de contenu
devrait se retrouver dans notre document? D'abord et avant tout pour
les tableaux et les figures.

En effet, les tableaux et les figures occupent souvent beaucoup
d'espace vertical dans la page. S'il ne reste plus assez de place pour
y afficher un tel élément de contenu, {\LaTeX} devra le déplacer au
début de la page suivante et cela risque de produire une page
inesthétique car insuffisamment remplie\footnote{%
  \emph{Underful \cs{vbox}} dans le jargon de {\TeX}.}. %
Les traitements de texte génèrent sans rechigner des pages à demi
remplies dans de telles situations.

En définissant un élément comme flottant, on laisse plutôt à
{\LaTeX} la possibilité de le disposer au meilleur endroit en fonction
de la taille de l'élément, du contenu du document et de diverses
règles typographiques.

On crée des éléments flottants avec les environnements \Ie{table}
et \Ie{figure}:
\begin{lstlisting}
\begin{table}`\oarg{pos}'  `\meta{tableau}' \end{table}
\begin{figure}`\oarg{pos}' `\meta{figure}' \end{figure}
\end{lstlisting}
Ci-dessus, \meta{tableau} et \meta{figure} représentent le code source
d'un tableau ou d'une figure avec possiblement une commande
\cmd{caption}, tel que traité plus loin.

L'argument optionnel \meta{pos} permet d'indiquer à {\LaTeX}
la ou les positions \emph{souhaitées} pour le tableau ou la figure dans la page.
Lorsqu'il est question d'éléments flottants, il est très difficile de donner des
ordres fermes à {\LaTeX} et l'effet de l'argument \meta{pos} est
souvent déconcertant. Aussi vaut-il souvent mieux ne rien indiquer et
laisser {\LaTeX} faire à sa guise. Le résultat demeure assez
prévisible puisque {\LaTeX} tâchera d'insérer l'élément flottant dans
le document \emph{dès que possible} sous réserve des conditions
suivantes:
\begin{itemize}
\item l'élément flottant ne peut apparaître dans le document avant la
  page où l'élément est défini;
\item l'élément sera placé de préférence dans le haut de la page
  courante, puis dans le bas et enfin sur une page séparée ne pouvant
  contenir que des éléments flottants, mais pas de texte.
\end{itemize}

Si la décision de {\LaTeX} ne convient pas, il est possible de
l'infléchir avec une combinaison d'une ou plusieurs des lettres
suivantes dans l'argument \meta{pos};
\begin{description}
\item[\normalfont\code{b}] placer l'élément au bas (\emph{bottom}) de la page;
\item[\normalfont\code{h}] placer l'élément ici (\emph{here}), à
  l'endroit où il est défini dans le code source;
\item[\normalfont\code{p}] placer l'élément sur une page séparée;
\item[\normalfont\code{t}] placer l'élément au haut (\emph{top}) de la page;
\item[\normalfont\code{!}] essayer plus fort de placer l'élément à
  l'endroit spécifié dans le reste de l'argument.
\end{description}
La valeur par défaut de l'argument \meta{pos} est \code{tbp}. La
section~10.4 de la %
\doc{memman}{http://texdoc.net/pkg/memoir/} %
de \class{memoir} explique plus en détail la signification des valeurs
ci-dessus. Le lecteur qui voudrait vraiment \emph{tout} savoir sur la
disposition des éléments flottants pourra consulter
\cite{Mittelbach:floats:2014}.

\begin{exemple}
  On reprend le tableau de l'\autoref{exemple:tableaux:tabular:1}, mais cette
  fois défini à l'intérieur d'un environnement \Pe{table}:
\begin{lstlisting}
\begin{table}
  \centering
  \begin{tabular}{lrrr}
    \toprule
    Produit & Quantité & Prix unitaire (\$) & Prix (\$) \\
    \midrule
    Vis à bois    & 2 & 9,90 & 19,80 \\
    Clous vrillés & 5 & 4,35 & 21,75 \\
    \midrule
    TOTAL         & 7 &      & 41,55 \\
    \bottomrule
  \end{tabular}
\end{table}
\end{lstlisting}
  \begin{table}
    \centering
    \begin{tabular}{lrrr}
      \toprule
      Produit & Quantité & Prix unitaire (\$) & Prix (\$) \\
      \midrule
      Vis à bois    & 2 & 9,90 & 19,80 \\
      Clous vrillés & 5 & 4,35 & 21,75 \\
      \midrule
      TOTAL         & 7 &      & 41,55 \\
      \bottomrule
    \end{tabular}
  \end{table}
  Remarquer où {\LaTeX} a automatiquement placé le tableau dans le
  document en fonction des règles précitées. %
  \qed
\end{exemple}

Dans un document soigné, tout tableau et toute figure devrait
comporter une légende ainsi qu'un numéro afin de pouvoir les
annoncer et y faire référence dans le texte («comme l'illustre la
figure~3\dots»). Cela
permet à la fois de guider le lecteur au fil de sa lecture et de
construire une liste des tableaux et des figures\footnote{%
  Obtenues respectivement avec les commandes \cmd{\listoftables} et
  \cmd{\listoffigures} mentionnées à la
  \autoref{sec:organisation:tdm}.} %
dans les pages liminaires d'un long document.

Pour ajouter une légende à un tableau ou une figure, il suffit
d'utiliser à l'intérieur des environnements \Pe{table} et \Pe{figure}
la commande
\begin{lstlisting}
\caption`\oarg{texte\_court}\marg{texte}'
\end{lstlisting}
où \meta{texte} est le texte de la légende. Si celui-ci est long
(plus d'une ligne), on peut en fournir une version abrégée dans
l'argument optionnel \meta{texte\_court}. C'est cette version abrégée
qui sera utilisée dans la liste des tableaux ou dans la liste des
figures.

La commande \cmd{\caption} insère,  à l'endroit où elle
apparaît dans l'environnement, une légende de la forme «\textsc{Table}
\emph{n}~--~\meta{texte}» pour un tableau ou «\textsc{Figure}
\emph{n}~--~\meta{texte}» pour une figure. Le texte de la légende est
centré sur la page lorsqu'il fait moins d'une ligne; dans le cas
contraire il est disposé comme un paragraphe normal.

\begin{conseil}
  Les anciennes version du style français de \pkg{babel} utilisaient
  les étiquettes plus neutres «\textsc{Tab.}» et «\textsc{Fig.}» dans
  les légendes des tableaux et figures. Pour utiliser --- comme dans
  le présent document --- ces versions plutôt que les versions par
  défaut ajouter dans le préambule les commandes suivantes:
\begin{lstlisting}
\def\frenchtablename{{\scshape Tab.}}
\def\frenchfigurename{{\scshape Fig.}}
\end{lstlisting}
\end{conseil}

Pour faire référence à un tableau ou à une figure dans le texte, il
faut utiliser le système de renvois automatiques de {\LaTeX}
(\autoref{sec:organisation:renvois}). On attribue une étiquette à l'élément
flottant en plaçant la commande \cmd{\label} dans le texte de la
commande \cmd{\caption} ou dans son voisinage immédiat. Les commandes
\cmd{\ref} ou \cmd{\autoref} servent ensuite à insérer des renvois dans
le texte.

L'exemple suivant présente finalement la recette complète pour composer
un tableau et une figure dans {\LaTeX}, légende et renvoi inclus.

\begin{exemple}
  Le code source de la \autoref{fig:tableaux:captions} crée
  le \autoref{tab:tableaux:captions}.
  \begin{figure}
\begin{lstlisting}
\begin{table}
  \centering
  \caption{Tableau correspondant au code
    de la \autoref{fig:[...]}}
  \label{tab:[...]}
  \begin{tabular}{lrrr}
    \toprule
    Produit & Quantité & Prix unitaire (\$) & Prix (\$) \\
    \midrule
    Vis à bois    & 2 & 9,90 & 19,80 \\
    Clous vrillés & 5 & 4,35 & 21,75 \\
    \midrule
    TOTAL         & 7 &      & 41,55 \\
    \bottomrule
  \end{tabular}
\end{table}
\end{lstlisting}
    \caption{Code source pour créer le \autoref{tab:tableaux:captions}}
    \label{fig:tableaux:captions}
  \end{figure}
  \begin{table}
    \centering
    \caption{Tableau correspondant au code de la \autoref{fig:tableaux:captions}}
    \label{tab:tableaux:captions}
    \begin{tabular}{lrrr}
      \toprule
      Produit & Quantité & Prix unitaire (\$) & Prix (\$) \\
      \midrule
      Vis à bois    & 2 & 9,90 & 19,80 \\
      Clous vrillés & 5 & 4,35 & 21,75 \\
      \midrule
      TOTAL         & 7 &      & 41,55 \\
      \bottomrule
    \end{tabular}
  \end{table}
  \qed
\end{exemple}

Les environnements \Pe{table} et \Pe{figure} créent des éléments
flottants qui, par ailleurs, sont des boîtes verticales standards
(\autoref{sec:boites:parbox}). Il est donc permis d'y mettre à peu
près n'importe quoi, mais surtout plus d'un tableau ou plus d'une
figure (ou même une combinaison des deux). Les environnements
\Pe{minipage} (\autoref{sec:boites:parbox}) se révèlent alors
particulièrement utiles pour disposer les éléments de contenu dans la
boîte.

\begin{exemple}
  \label{exemple:tableaux:grille}
  La \autoref{fig:tableaux:grille} contient quatre images sous forme
  de grille $2 \times 2$. Le code ci-dessous démontre comment parvenir
  à cette disposition à l'aide de boîtes verticales créées avec
  l'environnement \Pe{minipage}.
  \begin{figure}
    \fcolorbox{lightgray}{white}{\begin{minipage}{0.45\linewidth}
      \includegraphics[scale=0.4]{ul_p}
    \end{minipage}}
    \hfill
    \fcolorbox{lightgray}{white}{\begin{minipage}{0.45\linewidth}
      \reflectbox{\includegraphics[scale=0.4]{ul_p}}
    \end{minipage}}
    \newline
    \fcolorbox{lightgray}{white}{\begin{minipage}{0.45\linewidth}
      \includegraphics[scale=0.4,angle=45]{ul_p}
    \end{minipage}}
    \hfill
    \fcolorbox{lightgray}{white}{\begin{minipage}{0.45\linewidth}
      \reflectbox{\includegraphics[scale=0.4,angle=45]{ul_p}}
    \end{minipage}}
  \caption{Exemple de disposition de plusieurs graphiques dans une
    même figure flottante. Les rectangles en grisé indiquent les
    limites des boîtes verticales.}
  \label{fig:tableaux:grille}
  \end{figure}
\begin{lstlisting}
\begin{figure}
  \begin{minipage}{0.45\linewidth}
    \includegraphics[scale=0.4]{ul_p}
  \end{minipage}
  \hfill
  \begin{minipage}{0.45\linewidth}
    \reflectbox{\includegraphics[scale=0.4]{ul_p}}
  \end{minipage}
  \newline
  \begin{minipage}{0.45\linewidth}
    \includegraphics[scale=0.4,angle=45]{ul_p}
  \end{minipage}
  \hfill
  \begin{minipage}{0.45\linewidth}
    \reflectbox{\includegraphics[scale=0.4,angle=45]{ul_p}}
  \end{minipage}
\end{figure}
\end{lstlisting}
  \qed
\end{exemple}

Lorsqu'une figure ou un tableau compte plusieurs éléments, comme à
l'exemple précédent, il peut être souhaitable d'ajouter pour chacun
une sous-légende. L'\autoref{ex:tableaux:subcaptions} explique
comment y parvenir. La section~10.9 de la %
  \doc{memman}{http://texdoc.net/pkg/memoir} %
de \class{memoir} comporte de nombreux détails additionnels sur les
sous-légendes.


%%%
%%% Exercices
%%%

\section{Exercices}
\label{sec:tableaux:exercices}

\Opensolutionfile{solutions}[solutions-tableaux+figures]

\begin{Filesave}{solutions}
\section*{Chapitre \ref*{chap:tableaux}}
\addcontentsline{toc}{section}{Chapitre \protect\ref*{chap:tableaux}}

\end{Filesave}

\begin{exercice}
  Reproduire le tableau ci-dessous à l'aide d'un environnement
  \Pe{tabular}. Utiliser le gabarit de document
  \fichier{exercice\_gabarit.tex}.

  La première colonne est alignée à gauche, la seconde est un bloc de
  texte de $7,5$~cm et la troisième est alignée à droite. Le symbole
  {\No} dans l'entête est produit par la commande \cmd{\No} de
  \pkg{babel}. Le dernier prix est composé avec la commande
  \cmd{\nombre} de \pkg{numprint}.
  \begin{center}
    \begin{tabular}{lp{7.5cm}r}
      \toprule
      {\No} lot & Description & Prix (\$) \\
      \midrule
      U-236 & Ordinateur portable MacBook Air 13~pouces mi-2013,
              processeur 1,3~GHz, 8~Go RAM, disque SSD 250~Go & 998 \\
      U-374 & Chaise de bureau ergonomique ajustable de 8 façons,
              revêtement de tissu gris foncé & 275 \\
      U-588 & Table de travail en L & \nombre{1125} \\
      \bottomrule
    \end{tabular}
  \end{center}
  \begin{sol}
    Les paquetages \pkg{babel} et \pkg{numprint} étant chargés dans le
    fichier de gabarit, le code pour créer le tableau est le suivant:
\begin{lstlisting}
\begin{tabular}{lp{7.5cm}r}
  \toprule
  {\No} lot & Description & Prix (\$) \\
  \midrule
  U-236 & Ordinateur [...] & 998 \\
  U-374 & Chaise [...] & 275 \\
  U-588 & Table [...] & \nombre{1125} \\
  \bottomrule
\end{tabular}
\end{lstlisting}
  \end{sol}
\end{exercice}

\begin{exercice}
  Apporter au tableau de l'exercice précédent les modifications
  suivantes: centrer le titre de la deuxième colonne; ajuster
  automatiquement la largeur du tableau au bloc de texte sur la page
  avec un environnement \Pe{tabularx}.
  \begin{sol}
    Pour effectuer les modifications demandées, il faut:
    \begin{enumerate}[i)]
    \item utiliser la commande \cmd{\multicolumn} dans l'entête du
      tableau pour centrer le titre de la deuxième colonne sans
      autrement centrer le contenu de la colonne;
    \item remplacer l'environnement \Pe{tabular} par l'environnement
      \Pe{tabularx} de \class{memoir}, spécifier une largeur de
      tableau \cs{textwidth}, changer le format de la deuxième colonne
      pour \code{X} afin que la largeur de celle-ci s'ajuste
      automatiquement pour combler celle du tableau.
    \end{enumerate}
\begin{lstlisting}
\begin{tabularx}{\textwidth}{lXr}
  \toprule
  {\No} lot & \multicolumn{1}{c}{Description}
            & Prix (\$) \\
  \midrule
  U-236 & Ordinateur [...] & 998 \\
  U-374 & Chaise [...] & 275 \\
  U-588 & Table [...] & \nombre{1125} \\
  \bottomrule
\end{tabularx}
\end{lstlisting}
  \end{sol}
\end{exercice}

\begin{exercice}
  \label{ex:tableaux:subcaptions}
  L'\autoref{exemple:tableaux:grille} montre comment intégrer
  plusieurs figures (ou tableaux) à l'intérieur d'un même
  environnement flottant en les disposant dans des boîtes verticales.
  Dans de tels cas, il peut être souhaitable de fournir une légende
  pour l'ensemble du flottant, mais aussi des sous-légendes pour
  chaque tableau ou figure.

  Avec les classes \class{ulthese} et \class{memoir}, la production de
  sous-légendes requiert d'abord de déclarer, dans le préambule du
  document, son intention d'en créer pour les environnements flottants
  \Pe{table} ou \Pe{figure} avec, selon le cas, les commandes
\begin{lstlisting}
\newsubfloat{table}
\newsubfloat{figure}
\end{lstlisting}
  Ensuite, on utilise la commande
\begin{lstlisting}
\subcaption`\marg{texte}'
\end{lstlisting}
  de la même manière que \cmd{\caption}.

  Le fichier \fichier{exercice\_subcaption.tex} contient la structure
  de base pour composer deux tableaux côte à côte. Ajouter des
  sous-légendes à l'intérieur de l'environnement flottant.
  \begin{sol}
    Tout d'abord, remarquer que la commande
\begin{lstlisting}
\newsubfloat{table}
\end{lstlisting}
    est déjà présente dans le préambule du fichier. Si l'on souhaite
    placer des sous-légendes au-dessus de chacun des deux tableaux, le
    code du tableau devient:
\begin{lstlisting}
\begin{table}
  \caption{Conversion du nombre décimal $23,31$
    en binaire.}
  \begin{minipage}[t]{0.45\linewidth}
    \subcaption`\marg{texte}'   % ajout
    \begin{tabular*}{\linewidth}{crrcc}
      ...
    \end{tabular*}
  \end{minipage}
  \hfill
  \begin{minipage}[t]{0.45\linewidth}
    \subcaption`\marg{texte}'   % ajout
    \begin{tabular*}{\linewidth}{ccccc}
      ...
    \end{tabular*}
  \end{minipage}
\end{table}
\end{lstlisting}
  \end{sol}
\end{exercice}

\begin{exercice}
  Utiliser le fichier \fichier{exercice\_gabarit.tex} pour composer un
  document qui insère, disons, la page couverture du présent document
  à l'aide des fonctionnalités du paquetage \pkg{pdfpages} décrites à
  la \autoref{sec:tableaux:figures:pdfpages}.
  \begin{sol}
    Le préambule du document devrait contenir la déclaration
\begin{lstlisting}
\usepackage{pdfpages}
\end{lstlisting}
    pour charger le paquetage \pkg{pdfpages}. Ensuite, à l'endroit où
    l'on souhaite insérer la couverture du présent document dans le
    document, il s'agit de placer la commande
\begin{lstlisting}
\includepdf[pages=1]{formation-latex-ul}
\end{lstlisting}
  \end{sol}
\end{exercice}

\begin{exercice}[nosol]
  Le document \fichier{exercice\_demo.tex} contient plusieurs éléments
  flottants, tableaux et figures. Examiner le code et modifier
  l'argument optionnel de position d'un flottant pour voir son effet
  sur la mise en page du document.
\end{exercice}

\Closesolutionfile{solutions}


%%% Local Variables:
%%% mode: latex
%%% TeX-engine: xetex
%%% TeX-master: "formation-latex-ul"
%%% coding: utf-8
%%% End:

\chapter{Mathématiques}
\label{chap:math}


S'il est un domaine où {\LaTeX} brille particulièrement, c'est bien
dans la préparation et la présentation d'équations mathématiques ---
des plus simples aux plus complexes. Après tout, l'amélioration de la
qualité typographique des équations mathématiques dans son ouvrage
phare \emph{The Art of Computer Programming} figurait parmi les
objectifs premiers de Knuth lorsqu'il a développé {\TeX}.


\section{Principes de base du mode mathématique}
\label{sec:math:base}

La mise en forme d'équations mathématiques requiert d'indiquer à
l'ordinateur, dans un langage spécial, le contenu des dites équations
et la position des symboles: en exposant, en indice, sous forme de
fraction, etc. L'ordinateur peut ensuite assembler le tout à partir de
règles typographiques portant, par exemple, sur la représentation des
variables et des constantes, l'espacement entre les symboles ou la
disposition des équations selon qu'elles apparaissent au fil du texte
ou hors d'un paragraphe.

On indique à {\LaTeX} que l'on change de «langage», par l'utilisation
d'un mode mathématique. Il y a deux grandes manière d'activer le mode
mathématique:
\begin{enumerate}
\item en insérant le code entre les symboles \verb=$ $= pour générer
  une équation «en ligne», ou au fil du texte;
  \begin{demo}
    \begin{texample}
\begin{lstlisting}
on sait que $(a + b)^2 =
a^2 + 2ab + b^2$, d'où
on obtient...
\end{lstlisting}
      \producing
      on sait que $(a + b)^2 = a^2 + 2ab + b^2,$ d'où on obtient...
    \end{texample}
  \end{demo}
\item en utilisant un environnement servant à créer une équation hors
  paragraphe;
  \begin{demo}
    \begin{texample}
\begin{lstlisting}
on sait que
\begin{equation*}
  (a + b)^2
  = a^2 + 2ab + b^2,
\end{equation*}
d'où on obtient...
\end{lstlisting}
      \producing
      on sait que
      \begin{equation*}
        (a + b)^2 = a^2 + 2ab + b^2,
      \end{equation*}
      d'où on obtient...
    \end{texample}
  \end{demo}
\end{enumerate}

Dans l'exemple ci-dessus, l'environnement \Pe{equation*} (tiré du
paquetage \pkg{amsmath}, voir la section suivante) crée une
équation hors paragraphe, centrée sur la ligne et non numérotée. Avec
l'environnement \Ie{equation} (donc sans \verb=*= dans le nom),
{\LaTeX} ajoute automatiquement un numéro d'équation séquentiel aligné
sur la marge de droite:
\begin{demo}
  \begin{texample}
\begin{lstlisting}
on sait que
\begin{equation}
  (a + b)^2 = a^2 + b^2,
\end{equation}
d'où on obtient...
\end{lstlisting}
    \producing
    on sait que
    \begin{equation}
      (a + b)^2 = a^2 + b^2,
    \end{equation}
    d'où on obtient...
  \end{texample}
\end{demo}
Cette disposition est la plus usuelle dans les ouvrages mathématiques.
Le type de numérotation diffère selon qu'un document comporte des
chapitres ou non.

En mode mathématique, les chiffres sont automatiquement considérés
comme des constantes, les lettres comme des variables et une suite de
lettres comme un produit de variables (nous verrons plus loin comment
représenter des fonctions mathématiques comme $\sin$, $\log$ ou
$\lim$). Ceci a trois conséquences principales:
\begin{enumerate}
\item conformément aux conventions typographiques, les chiffres sont
  représentés en caractère \textrm{romain} et les variables, en
  italique;
  \begin{demo}
    \begin{texample}
\begin{lstlisting}
$123xyz$
\end{lstlisting}
      \producing
      $123xyz$
    \end{texample}
  \end{demo}
\item l'espace entre les constantes, les variables et les opérateurs
  mathématiques est géré automatiquement;
  \begin{demo}
    \begin{texample}
\begin{lstlisting}
$z = 2 x + 3 x y$
\end{lstlisting}
    \producing
    $z = 2 x + 3 x y$
    \end{texample}
  \end{demo}
\item les espaces dans le code source n'ont aucun impact sur la
  disposition d'une équation.
  \begin{demo}
    \begin{texample}
\begin{lstlisting}
$z=2x + 3xy$
\end{lstlisting}
    \producing
    $z=2x + 3xy$
    \end{texample}
  \end{demo}
\end{enumerate}

Quant au langage retenu par {\LaTeX} pour décrire les équations
mathématiques, il est très similaire à celui que l'on utiliserait pour
le faire à voix haute. Il faut simplement recourir à des commandes
pour identifier les symboles mathématiques que l'on ne retrouve pas
sur un clavier usuel, comme les lettres grecques, les opérateurs
d'inégalité ou les symboles de sommes et d'intégrales.


\section{Un paquetage incontournable}
\label{sec:math:amsmath}

Le paquetage \pkg{amsmath} \citep{amsmath} produit par la prestigieuse
\emph{American Mathematical Society} fournit diverses extensions à
{\LaTeX} pour faciliter encore davantage la saisie d'équations
mathématiques complexes et en améliorer la présentation. L'utilisation
de ce paquetage doit être considérée incontournable pour tout document
contenant plus que quelques équations très simples.

Au chapitre des améliorations fournies par \pkg{amsmath}, notons
particulièrement:
\begin{itemize}
\item plusieurs environnements pour les équations hors paragraphe, en
  particulier pour les équations multilignes;
\item une meilleure gestion de l'espacement autour des symboles de
  relation (comme les signes d'égalité) dans les équations
  multilignes;
\item une commande pour faciliter l'entrée de texte à l'intérieur du
  mode mathématique;
\item un environnement pour la saisie des matrices et des coefficients
  binomiaux;
\item des commandes pour les intégrales multiples;
\item la possibilité de définir de nouveaux opérateurs mathématiques.
\end{itemize}
Nous décrivons certaines de ces fonctionnalités dans la suite, mais
tout utilisateur du paquetage devrait impérativement consulter sa %
\doc[documentation complète]{amsldoc}{http://texdoc.net/pkg/amsmath}.


\section{Principaux éléments du mode mathématique}
\label{sec:math:bases}

Cette section explique comment créer et assembler divers éléments
d'une formule mathématique: exposants, indices, fractions, texte, etc.
Les seuls symboles utilisés sont pour le moment les chiffres et les
lettres latines. La présentation d'une partie de l'éventail de
symboles mathématiques offerts par {\LaTeX} fera l'objet de la
\autoref{sec:math:symboles}.

\subsection{Exposants et indices}
\label{sec:math:bases:exposants}

{\LaTeX} permet de créer facilement et avec la bonne taille de
symboles n'importe quelle combinaison d'exposants et d'indices.

On place un caractère en \textsuperscript{exposant} avec la
commande \verb=^= et en \textsubscript{indice} avec la commande
\verb=_=. Les indices et exposants se combinent naturellement.
\begin{demo}
  \def\strut{\rule[-0.4ex]{0pt}{2ex}}
  \begin{minipage}{0.3\linewidth}
    \begin{texample}[0.6\linewidth]
\begin{lstlisting}
x^2
\end{lstlisting}
      \producing\strut $x^2$
    \end{texample}
  \end{minipage}
  \quad
  \begin{minipage}{0.3\linewidth}
    \begin{texample}[0.6\linewidth]
\begin{lstlisting}
a_n
\end{lstlisting}
      \producing\strut $a_n$
    \end{texample}
  \end{minipage}
  \quad
  \begin{minipage}{0.3\linewidth}
    \begin{texample}[0.63\linewidth]
\begin{lstlisting}
x_i^k
\end{lstlisting}
      \producing\strut $x_i^k$
    \end{texample}
  \end{minipage}
\end{demo}
(L'ordre de saisie n'a pas d'importance; le troisième exemple
donnerait le même résultat avec \verb=x^k_i=.)

Si l'exposant ou l'indice compte plus d'un caractère, il faut
regrouper le tout entre accolades \verb={ }=.
\begin{demo}
  \begin{minipage}{0.3\linewidth}
    \begin{texample}[0.6\linewidth]
\begin{lstlisting}
x^{2k + 1}
\end{lstlisting}
      \producing\strut $x^{2k + 1}$
    \end{texample}
  \end{minipage}
  \quad
  \begin{minipage}{0.3\linewidth}
    \begin{texample}[0.6\linewidth]
\begin{lstlisting}
x_{i,j}
\end{lstlisting}
      \producing\strut $x_{i,j}$
    \end{texample}
  \end{minipage}
  \quad
  \begin{minipage}{0.3\linewidth}
    \begin{texample}[0.63\linewidth]
\begin{lstlisting}
x_{ij}^{2n}
\end{lstlisting}
      \producing\strut $x_{ij}^{2n}$
    \end{texample}
  \end{minipage}
\end{demo}

Toutes les combinaisons d'exposants et d'indices sont possibles, y
compris les puissances de puissances ou les indices d'indices.
\begin{demo}
  \begin{minipage}{0.3\linewidth}
    \begin{texample}[0.6\linewidth]
\begin{lstlisting}
e^{-x^2}
\end{lstlisting}
      \producing\strut $e^{-x^2}$
    \end{texample}
  \end{minipage}
  \quad
  \begin{minipage}{0.6\linewidth}
    \begin{texample}[0.6\linewidth]
\begin{lstlisting}
A_{i_s, k^n}^{y_i}
\end{lstlisting}
      \producing\strut $A_{i_s,k^n}^{y_i}$
    \end{texample}
  \end{minipage}
  \quad \mbox{} % pour alignement avec bloc d'exemples ci-dessus
\end{demo}

\begin{important}
  Les commandes \verb=^= et \verb=_= sont permises dans le mode
  mathématique seulement. En fait, si {\TeX} rencontre l'une de ces
  commandes en mode texte, il tentera automatiquement de passer au
  mode mathématique après avoir émis l'avertissement
\begin{verbatim}
! Missing $ inserted.
\end{verbatim}
  Il est assez rare que le résultat soit celui souhaité.
\end{important}

\subsection{Fractions}
\label{sec:math:bases:fractions}

Il y a plusieurs façons de représenter une fraction selon qu'elle se
trouve au fil du texte, dans une équation hors paragraphe ou à
l'intérieur d'une autre fraction.

Pour les fractions au fil du texte, il vaut souvent mieux utiliser
simplement la barre oblique \verb=/= pour séparer le numérateur du
dénominateur, quitte à utiliser des parenthèses. Ainsi, on utilise
\verb=$(n + 1)/2$= pour obtenir $(n + 1)/2$.

De manière plus générale, la commande
\begin{lstlisting}
\frac`\marg{numérateur}\marg{dénominateur}'
\end{lstlisting}
dispose \meta{numérateur} au-dessus de \meta{dénominateur}, séparé par
une ligne horizontale. La taille des caractères s'ajuste
automatiquement selon que la fraction se trouve au fil du texte ou
dans une équation hors paragraphe, ainsi que selon la position de la
fraction dans l'équation.
\begin{demo}
  \begin{texample}
\begin{lstlisting}
% taille au fil du texte
On a $z_1 = \frac{x}{y}$ et
$z_2 = xy$.
\end{lstlisting}
    \producing On a $z_1 = \frac{x}{y}$ et $z_2 = xy$.
  \end{texample}

  \begin{texample}
\begin{lstlisting}
% taille hors paragraphe
On a
\begin{equation*}
  z_1 = \frac{x}{y}
\end{equation*}
et $z_2 = xy$.
\end{lstlisting}
    \producing On a
    \begin{equation*}
      z_1 = \frac{x}{y}
    \end{equation*}
    et $z_2 = xy$.
  \end{texample}

  \begin{texample}
\begin{lstlisting}
% deux tailles combinées
Soit
\begin{equation*}
  z = \frac{\frac{x}{2}
    + 1}{y}.
\end{equation*}
\end{lstlisting}
    \producing Soit
    \begin{equation*}
      z = \frac{\frac{x}{2} + 1}{y}.
    \end{equation*}
  \end{texample}
\end{demo}

Les commandes
\begin{lstlisting}
\dfrac`\marg{numérateur}\marg{dénominateur}'
\tfrac`\marg{numérateur}\marg{dénominateur}'
\end{lstlisting}
de \pkg{amsmath} permettent de forcer une fraction à adopter la taille
d'une fraction hors paragraphe (\emph{displayed}) dans le cas de
\cmd{\dfrac}, ou de celle d'une fraction au fil du texte (\emph{text})
dans le cas de \cmd{\tfrac}. Consulter l'\autoref{ex:math:matrices} à
la \autopageref{ex:math:matrices} pour visualiser l'effet de la
commande \cmd{\dfrac}.

\begin{conseil}
  Il est parfois visuellement plus intéressant, surtout au fil du
  texte, d'écrire une fraction comme $1/x$ sous la forme $x^{-1}$.
\end{conseil}

\subsection{Racines}
\label{sec:math:bases:racines}

La commande
\begin{lstlisting}
\sqrt`\oarg{n}\marg{radicande}'
\end{lstlisting}
construit un symbole de radical autour de \meta{radicande}, par défaut
la racine carrée. Si l'argument optionnel \meta{n} est spécifié, c'est
plutôt un symbole de racine d'ordre $n$ qui est tracé. La longueur et
la hauteur du radical s'adapte toujours à celles du radicande.
\begin{demo}
  \begin{minipage}{0.3\linewidth}
    \begin{texample}[0.6\linewidth]
\begin{lstlisting}
\sqrt{2}
\end{lstlisting}
      \producing $\sqrt{2}$
    \end{texample}
  \end{minipage}
  \quad
  \begin{minipage}{0.3\linewidth}
    \begin{texample}[0.6\linewidth]
\begin{lstlisting}
\sqrt{625}
\end{lstlisting}
      \producing
      $\sqrt{625}$
    \end{texample}
  \end{minipage}
  \quad
  \begin{minipage}{0.3\linewidth}
    \begin{texample}[0.6\linewidth]
\begin{lstlisting}
\sqrt[3]{8}
\end{lstlisting}
      \producing
      $\sqrt[3]{8}$
    \end{texample}
  \end{minipage}

  \begin{texample}
\begin{lstlisting}
\sqrt[n]{x + y + z}
\end{lstlisting}
    \producing
    $\sqrt[n]{x + y + z}$
  \end{texample}

  \begin{texample}
\begin{lstlisting}
\sqrt{\frac{x + y}{x^2 - y^2}}
\end{lstlisting}
    \producing
    $\displaystyle \sqrt{\frac{x + y}{x^2 - y^2}}$
  \end{texample}
\end{demo}

\subsection{Sommes et intégrales}
\label{sec:math:bases:sommes-et-integrales}

Les sommes et les intégrales requièrent un symbole spécial ainsi que
des limites inférieures et supérieures, le cas échéant.

Les commandes \cmd{\sum} et \cmd{\int} servent respectivement à tracer
les symboles de somme $\sum$ et d'intégrale $\int$. Le paquetage
\pkg{amsmath} fournit également des commandes comme \cmd{\iint} et
\cmd{\iiint} pour obtenir des symboles d'intégrales multiples finement
disposés ($\iint$ et $\iiint$).

On entre les éventuelles limites inférieures et supérieures comme des
indices et des exposants.
\begin{demo}
  \begin{texample}
\begin{lstlisting}
\sum_{i = 0}^n x_i
\end{lstlisting}
    \producing $\displaystyle \sum_{i = 0}^n x_i$
  \end{texample}

  \begin{texample}
\begin{lstlisting}
\int_0^{10} f(x)\, dx
\end{lstlisting}
    \producing $\displaystyle \int_0^{10} f(x)\, dx$
  \end{texample}

  \begin{texample}
\begin{lstlisting}
\iint_D f(x, y)\, dx dy
\end{lstlisting}
    \producing $\displaystyle \iint_D f(x, y)\, dx dy$
  \end{texample}
\end{demo}

La taille des symboles et la position des limites s'ajustent
automatiquement selon le contexte. Au fil du texte, la somme et
l'intégrale simple ci-dessus apparaîtraient comme $\sum_{i = 0}^n x_i$
et $\int_0^{10} f(x)\, dx$.

Dans une intégrale il est recommandé de séparer l'intégrande de
l'opérateur de différentiation $dx$ par une espace fine. C'est ce à
quoi sert la commande \cmd{\,} ci-dessus; voir aussi le
\autoref{tab:math:espaces} de la \autopageref{tab:math:espaces}.

\subsection{Points de suspension}
\label{sec:math:bases:dots}

Les formules mathématiques comportent fréquemment des points de
suspension dans des suites de variables ou d'opérations. On recommande
d'éviter de les entrer comme trois points finaux consécutifs, car
l'espacement entre les points sera trop petit et le résultat, jugé
disgracieux\footnote{%
  Le résultat exact dépend de la police utilisée.} %
d'un point de vue typographique: $...$

Le \autoref{tab:math:dots} fournit les commandes {\LaTeX} servant à
générer divers types de points de suspension.

\begin{table}
  \caption{Points de suspension}
  \label{tab:math:dots}
  \centering
  \begin{tabular}{lll}
    \toprule
    commande & type de points & exemple \\
    \midrule
    \cmd{\dots} &  sélection automatique \\
    \cmd{\ldots} & points à la ligne de base & $x_1, \ldots, x_n$ \\
    \cmd{\cdots} & points centrés & $x_1 + \cdots + x_n$ \\
    \addlinespace[0.5\normalbaselineskip]
    \cmd{\vdots} & points verticaux & $
                                      \begin{matrix}
                                        x_1 \\ \vdots \\ x_n
                                      \end{matrix}$ \\
    \addlinespace[0.5\normalbaselineskip]
    \cmd{\ddots} & points diagonaux & $
                                      \begin{matrix}
                                        x_1 &&\\ &\ddots& \\ && x_n
                                      \end{matrix}$ \\
    \bottomrule
  \end{tabular}
\end{table}

Avec \pkg{amsmath}, la commande \cmd{\dots} tâche de sélectionner
automatiquement entre les points à la ligne de base ou les points
centrés selon le contexte. Comme le résultat est en général le bon,
nous recommandons d'utiliser principalement cette commande pour
insérer des points de suspension en mode mathématique.
\begin{demo}
  \begin{texample}
\begin{lstlisting}
$x_1, \dots, x_n$
\end{lstlisting}
    \producing $x_1, \dots, x_n$
  \end{texample}

  \begin{texample}
\begin{lstlisting}
$x_1, \ldots, x_n$
\end{lstlisting}
    \producing $x_1, \ldots, x_n$
  \end{texample}

  \begin{texample}
\begin{lstlisting}
$x_1 + \dots + x_n$
\end{lstlisting}
    \producing $x_1 + \dots + x_n$
  \end{texample}

  \begin{texample}
\begin{lstlisting}
$x_1 + \cdots + x_n$
\end{lstlisting}
    \producing $x_1 + \cdots + x_n$
  \end{texample}
\end{demo}
Le paquetage définit également les commandes sémantiques suivantes:
\begin{itemize}
\item \cmd{\dotsc} pour des «points avec des virgules» (\emph{commas});
\item \cmd{\dotsb} pour des «points avec des opérateurs binaires»;
\item \cmd{\dotsm} pour des «points de multiplication»;
\item \cmd{\dotsi} pour des «points avec des intégrales»;
\item \cmd{\dotso} pour d'«autres points» (\emph{other}).
\end{itemize}

\subsection{Texte et espaces}
\label{sec:math:bases:texte}

On l'a vu, en mode mathématique {\LaTeX} traite les lettres comme des
variables et gère automatiquement l'espacement entre les divers
symboles. Or, il n'est pas rare que des formules mathématiques
contiennent du texte (notamment des mots comme «où», «si», «quand»).
De plus, il est parfois souhaitable de pouvoir ajuster les blancs
entre des éléments.

La commande de \pkg{amsmath}
\begin{lstlisting}
\text`\marg{texte}'
\end{lstlisting}
insère \meta{texte} dans une formule mathématique. Le texte est inséré
tel quel, sans aucune gestion des espaces avant ou après le texte. Si
des espaces sont nécessaires, ils doivent faire partie de
\meta{texte}.
\begin{demo}
  \begin{texample}
\begin{lstlisting}
f(x) = a e^{-ax}
\text{ pour } x > 0
\end{lstlisting}
    \producing $f(x) = a e^{-ax} \text{ pour } x > 0$
  \end{texample}
\end{demo}

Les commandes
\begin{lstlisting}
\quad
\qquad
\end{lstlisting}
insèrent un blanc de largeur variable selon la taille de la police en
vigueur. La commande \cmd{\quad} insère un blanc de $1$~em (la largeur de
la lettre M dans la police en vigueur), alors que \cmd{\qquad}
insère le double de cette longueur\footnote{%
  Bien qu'elles soient surtout utilisées dans le mode mathématique,
  les commandes \cmd{\quad} et \cmd{\qquad} sont également valides
  dans le mode texte.}.
\begin{demo}
  \begin{texample}
\begin{lstlisting}
f(x) = a e^{-ax},
\quad x > 0
\end{lstlisting}
    \producing
    $f(x) = a e^{-ax}, \quad x > 0$
  \end{texample}
\end{demo}

Le \autoref{tab:math:espaces} répertorie et compare les différentes
commandes qui permettent d'insérer des espaces plus ou moins fines
entre des éléments dans le mode mathématique.

\begin{table}
  \caption{Espaces dans le mode mathématique}
  \label{tab:math:espaces}
  \centering
  \begin{tabular}{lll}
    \toprule
    Commande & Longueur & Exemple \\
    \midrule
             & pas d'espace & \spx{} \\
    \cmd{\,} & $3/18$ de \cmdprint{quad} & \spx{\,} \\
    \cmd{\:} & $4/18$ de \cmdprint{quad} & \spx{\:} \\
    \cmd{\;} & $5/18$ de \cmdprint{quad} & \spx{\;} \\
    \pixabang & $-3/18$ de \cmdprint{quad} & \spx{\!} \\
    \cmd{\quad} & $1$~em & \spx{\quad} \\
    \cmd{\qquad} & $2$~em & \spx{\qquad} \\
    \bottomrule
  \end{tabular}
\end{table}

\subsection{Fonctions et opérateurs}
\label{sec:math:bases:fonctions}

Les règles de typographie pour les équations mathématiques exigent que
les variables apparaissent en \textit{italique}, mais que les noms de
fonctions, eux, apparaissent en \textrm{romain}, comme le texte
standard. Pensons, ici, à des fonctions comme $\sin$ ou $\log$.

On sait que {\LaTeX} interprétera un nom de fonction saisi tel quel
comme un produit de variables:
\begin{demo}
  \begin{minipage}{0.45\linewidth}
    \begin{texample}
\begin{lstlisting}
$2 sin 30$
\end{lstlisting}
      \producing
      $2 sin 30$
    \end{texample}
  \end{minipage}
  \hfill
  \begin{minipage}{0.45\linewidth}
    \begin{texample}
\begin{lstlisting}
$3 log 2$
\end{lstlisting}
      \producing
      $3 log 2$
    \end{texample}
  \end{minipage}
\end{demo}
La commande \cmdprint{\text} permettrait d'afficher les noms de
fonction en romain, mais l'utiliser à répétition se révélerait
rapidement peu pratique. De plus, un nom de fonction n'est pas du
texte à proprement parler, mais plutôt un opérateur mathématique.

La solution de {\LaTeX} consiste à définir des commandes pour un grand
nombre de fonctions et d'opérateurs mathématiques standards:
\begin{lstlisting}
\arccos   \cosh    \det    \inf      \limsup   \Pr     \tan
\arcsin   \cot     \dim    \ker      \ln       \sec    \tanh
\arctan   \coth    \exp    \lg       \log      \sin
\arg      \csc     \gcd    \lim      \max      \sinh
\cos      \deg     \hom    \liminf   \min      \sup
\end{lstlisting}
L'espacement autour des fonctions et opérateurs est géré par {\LaTeX}.
\begin{demo}
  \begin{minipage}{0.45\linewidth}
    \begin{texample}
\begin{lstlisting}
$2 \sin30$
\end{lstlisting}
      \producing
      $2 \sin30$
    \end{texample}
  \end{minipage}
  \hfill
  \begin{minipage}{0.45\linewidth}
    \begin{texample}
\begin{lstlisting}
$3 \log 2$
\end{lstlisting}
      \producing
      $3 \log 2$
    \end{texample}
  \end{minipage}
\end{demo}

Certaines des fonctions ci-dessus, notamment \cmd{\lim}, acceptent des
limites comme les symboles de somme et d'intégrale.
\begin{demo}
  \begin{texample}
\begin{lstlisting}
% au fil du texte
\lim_{x \to 0} x = 0
\end{lstlisting}
    \producing $\lim_{x \to 0} x = 0$
  \end{texample}

  \begin{texample}
\begin{lstlisting}
% hors paragraphe
\lim_{x \to 0} x = 0
\end{lstlisting}
    \producing $\displaystyle \lim_{x \to 0} x = 0$
  \end{texample}
\end{demo}

La commande \cmd{\DeclareMathOperator} de \pkg{amsmath} permet de
définir de nouveaux opérateurs mathématiques lorsque nécessaire;
consulter la %
\doc{amsldoc}{http://texdoc.net/pkg/amsmath} %
du paquetage (section~5.1) pour les détails.


\begin{exemple}
  Le matériel passé en revue jusqu'à maintenant permet déjà
  de composer des équations élaborées --- sous réserve qu'elles
  tiennent sur une seule ligne comme dans le présent exemple.

  On présente ci-dessous, pièce par pièce, le code {\LaTeX} pour créer
  l'équation suivante:
  \begin{equation*}
    \int_x^\infty (y - x) f_{X|X > x}(y)\, dy =
    \frac{1}{1 - F_X(x)} \int_x^\infty (y - x) f_X(y)\, dy.
  \end{equation*}
  \begin{demo}
    \begin{texample}[0.58\linewidth]
\begin{lstlisting}
\begin{equation*}
\end{lstlisting}
      \producing
      \emph{équation hors paragraphe}
    \end{texample}

    \begin{texample}[0.58\linewidth]
\begin{lstlisting}
\int_x^\infty
\end{lstlisting}
      \producing
      $\displaystyle \int_x^\infty$
    \end{texample}

    \begin{texample}[0.58\linewidth]
\begin{lstlisting}
(y - x) f_{X|X > x}(y)\, dy =
\end{lstlisting}
      \producing
      $(y - x) f_{X|X > x}(y)\, dy =$
    \end{texample}

    \begin{texample}[0.58\linewidth]
\begin{lstlisting}
\frac{1}{1 - F_X(x)}
\end{lstlisting}
      \producing
      $\dfrac{1}{1 - F_X(x)}$
    \end{texample}

    \begin{texample}[0.58\linewidth]
\begin{lstlisting}
\int_x^\infty (y - x) f_X(y)\, dy
\end{lstlisting}
      \producing
      $\displaystyle \int_x^\infty (y - x) f_X(y)\, dy$
    \end{texample}

    \begin{texample}[0.58\linewidth]
\begin{lstlisting}
\end{equation*}
\end{lstlisting}
      \producing
      \emph{fin de l'environnement}
    \end{texample}
  \end{demo}
  \qed
\end{exemple}


\section{Symboles mathématiques}
\label{sec:math:symboles}

Outre les chiffres et les lettres de l'alphabet, les claviers
d'ordinateurs ne comptent normalement que les symboles
mathématiques suivants:
\begin{center}
%\begin{verbatim}
\verb0+  -  =  <  >  /  :  !  '  |  [  ]  (  ) { }0
%\end{verbatim}
\end{center}
Ils s'emploient directement dans les équations, sauf les accolades
\verb={ }= qui sont des symboles réservés par {\LaTeX}. Il faut donc
entrer celles-ci avec \cmdprint{\{} et \cmdprint{\}}, comme dans du
texte normal.

Pour représenter les innombrables autres symboles mathématiques, on
aura recours à des commandes qui débutent, comme d'habitude, par le
symbole {\bs} et dont le nom est habituellement dérivé de la
signification mathématique du symbole.

Si un symbole mathématique a été utilisé quelque part dans une
publication, il y a de fortes chances que sa version existe dans
{\LaTeX}. Il serait donc utopique de tenter de faire ici une recension
des symboles disponibles. Nous nous contenterons d'un avant goût des
principales catégories.

L'ouvrage de référence pour connaître les symboles disponibles dans
{\LaTeX} est la bien nommée \emph{Comprehensive {\LaTeX} Symbol List}
\citep{comprehensive}. La liste comprend près de \nombre{6000}
symboles répartis sur plus de 160 pages! On y trouve de tout, des
symboles mathématiques aux pictogrammes de météo ou d'échecs, en
passant par\dots\ des figurines des Simpsons.

\begin{important}
  Les moteurs {\XeTeX} et {\LuaTeX} supportent nativement le code
  source en format Unicode UTF-8 \citep{Unicode:5.0}. Ce standard
  contient des définitions pour plusieurs symboles mathématiques
  \citep{wikipedia:unicode-math}. Cela signifie qu'il est possible
  d'entrer une partie au moins des équations mathématiques avec des
  caractères visibles à l'écran, plutôt qu'avec des commandes
  {\LaTeX}. Nous ne saurions toutefois recommander cette pratique qui
  rend les fichiers source moins compatibles d'un système à un autre.
\end{important}

\subsection{Lettres grecques}
\label{sec:math:symboles:grecques}

On obtient les lettres grecques en {\LaTeX} avec des commandes
correspondant au nom de chaque lettre. Lorsque la commande débute par
une capitale, on obtient une lettre grecque majuscule. Les commandes
de certaines lettres grecques majuscules n'existent pas lorsque celles-ci
sont identiques aux lettres romaines.

Les tableaux \ref{tab:math:grecques} et \ref{tab:math:Grecques}
présentent l'ensemble des lettres grecques disponibles dans {\LaTeX}.

\begin{table}
  \caption{Lettres grecques minuscules}
  \label{tab:math:grecques}
  \begin{tabularx}{1.0\linewidth}{lXlXlXlX}
    $\alpha$      & \cmd{\alpha}      & $\theta$    & \cmd{\theta} &
    $o$           & o                 & $\tau$      & \cmd{\tau} \\
    $\beta$       & \cmd{\beta}       & $\vartheta$ & \cmd{\vartheta} &
    $\pi$         & \cmd{\pi}         & $\upsilon$  & \cmd{\upsilon} \\
    $\gamma$      & \cmd{\gamma}      & $\iota$     & \cmd{\iota} &
    $\varpi$      & \cmd{\varpi}      & $\phi$      & \cmd{\phi} \\
    $\delta$      & \cmd{\delta}      & $\kappa$    & \cmd{\kappa} &
    $\rho$        & \cmd{\rho}        & $\varphi$   & \cmd{\varphi} \\
    $\epsilon$    & \cmd{\epsilon}    & $\lambda$   & \cmd{\lambda} &
    $\varrho$     & \cmd{\varrho}     & $\chi$      & \cmd{\chi} \\
    $\varepsilon$ & \cmd{\varepsilon} & $\mu$       & \cmd{\mu} &
    $\sigma$      & \cmd{\sigma}      & $\psi$      & \cmd{\psi} \\
    $\zeta$       & \cmd{\zeta}       & $\nu$       & \cmd{\nu} &
    $\varsigma$   & \cmd{\varsigma}   & $\omega$    & \cmd{\omega} \\
    $\eta$        & \cmd{\eta}        & $\xi$       & \cmd{\xi}
  \end{tabularx}
\end{table}

\begin{table}
  \caption{Lettres grecques majuscules}
  \label{tab:math:Grecques}
  \begin{tabularx}{1.0\linewidth}{lXlXlXlX}
    $\Gamma$    & \cmd{\Gamma}   &
    $\Lambda$   & \cmd{\Lambda}  &
    $\Sigma$    & \cmd{\Sigma}   &
    $\Psi$      & \cmd{\Psi}     \\
    $\Delta$    & \cmd{\Delta}   &
    $\Xi$       & \cmd{\Xi}      &
    $\Upsilon$  & \cmd{\Upsilon} &
    $\Omega$    & \cmd{\Omega}   \\
    $\Theta$    & \cmd{\Theta}   &
    $\Pi$       & \cmd{\Pi}      &
    $\Phi$      & \cmd{\Phi}
  \end{tabularx}
\end{table}

\subsection{Lettres modifiées}
\label{sec:math:symboles:mathcal}

Les lettres de l'alphabet, principalement en majuscule, servent
parfois en mathématiques dans des versions modifiées pour représenter
des quantités, notamment les ensembles.

La commande \cmd{\mathcal} permet de transformer un ou plusieurs
caractères en version dite «calligraphique» dans le mode mathématique.
\begin{demo}
  \begin{texample}
\begin{lstlisting}
\mathcal{ABC\, xyz}
\end{lstlisting}
    \producing
    %% pris de lucidaot.tex; apparemment lent
    \setmathfont[RawFeature=+ss04]{Lucida Bright Math OT}
    $\mathscr{ABC\, xyz}$
  \end{texample}
\end{demo}

La commande \cmd{\mathbb} fournie par les paquetages \pkg{amsfonts} et
\pkg{unicode-math}, entre autres, génère des versions majuscule
ajourée (\emph{blackboard bold}) des lettres de l'alphabet. Elles sont
principalement utilisée pour représenter les ensembles de nombres.
\begin{demo}
  \begin{texample}
\begin{lstlisting}
\mathbb{NZRC}
\end{lstlisting}
    \producing $\mathbb{NZRC}$
  \end{texample}
\end{demo}

Le tableau~213 de la %
\doc[\emph{Comprehensive {\LaTeX} Symbol List}]{comprehensive}{http://texdoc.net/pkg/comprehensive} %
présente plusieurs autres alphabets spéciaux disponibles en mode
mathématique.

\begin{conseil}
  Certaines polices OpenType contiennent plusieurs
  versions des symboles mathématiques. Par exemple, la police utilisée
  dans le présent document contient deux versions de la police
  calligraphique, celle présentée ci-contre et celle-ci: %
  \setmathfont[RawFeature=-ss04]{Lucida Bright Math OT}
  $\mathscr{ABC\, xyz}$. Consulter éventuellement la documentation de
  la police pour les détails.
\end{conseil}

\subsection{Opérateurs binaires et relations}
\label{sec:math:symboles:binaires+relations}

Les opérateurs binaires combinent deux quantités pour en former une
troisième; pensons simplement aux opérateurs d'addition $+$ et de
soustraction $-$ que l'on retrouve sur un clavier d'ordinateur normal.
Les relations, quant à elles, servent pour la comparaison entre deux
quantités, comme $<$ et $>$. Le \autoref{tab:math:binaires}
présente une sélection d'opérateurs binaires et
le \autoref{tab:math:relations}  une sélection de relations.

La %
\doc[\emph{Comprehensive {\LaTeX} Symbol List}]{}{http://texdoc.net/pkg/comprehensive} %
consacre plus d'une dizaine de tableaux aux opérateurs binaires et
près d'une quarantaine aux relations. C'est dire à quel point les
tableaux \ref*{tab:math:binaires} et \ref*{tab:math:relations} de la
\autopageref{tab:math:binaires} ne présentent que les principaux
éléments à titre indicatif.

Certaines relations existent directement en version opposée, ou
négative (comme $\neq$ ou $\notin$) soit dans {\LaTeX} de base, soit
avec \pkg{amsmath} ou un autre paquetage. Autrement, il est possible
de préfixer toute relation de \cmd{\not} pour y superposer une barre
oblique $/$.

\begin{table}[p]
  \caption{Quelques opérateurs binaires}
  \label{tab:math:binaires}
  \begin{tabularx}{1.0\linewidth}{lXlXlXlX}
    $\times$    & \cmd{\times} &
    $\div$      & \cmd{\div}   &
    $\pm$       & \cmd{\pm}    &
    $\cdot$     & \cmd{\cdot}    \\
    $\cup$      & \cmd{\cup} &
    $\cap$      & \cmd{\cap} &
    $\setminus$ & \cmd{\setminus} &
    $\circ$     & \cmd{\circ}  \\
    $\wedge$    & \cmd{\wedge} &
    $\vee$      & \cmd{\vee} &
    $\oplus$    & \cmd{\oplus} &
    $\otimes$   & \cmd{\otimes} \\
    $\ast$      & \cmd{\ast} &
    $\star$     & \cmd{\star} &
    $\boxplus$  & \cmd{\boxplus}$^\dagger$ &
    $\boxtimes$ & \cmd{\boxtimes}$^\dagger$ \\
    \addlinespace
  \end{tabularx}
  \hspace*{1em}{\footnotesize $^\dagger$ requiert \pkg{amsmath}}
\end{table}

\begin{table}[p]
  \caption{Quelques relations et leur négation}
  \label{tab:math:relations}
  \begin{tabularx}{1.0\linewidth}{lXlXlXlX}
    $\leq$      & \cmd{\leq} &
    $\geq$      & \cmd{\geq}   &
    $\neq$      & \cmd{\neq}    &
    $\equiv$    & \cmd{\equiv}    \\
    $\subset$   & \cmd{\subset} &
    $\subseteq$ & \cmd{\subseteq}  &
    $\in$       & \cmd{\in} &
    $\notin$    & \cmd{\notin} \\
    $\nless$    & \cmd{\nless}$^\dagger$ &
    $\ngtr$     & \cmd{\ngtr}$^\dagger$   &
    $\nleq$     & \cmd{\nleq}$^\dagger$    &
    $\ngeq$     & \cmd{\ngeq}$^\dagger$ \\
    \addlinespace
  \end{tabularx}
  \hspace*{1em}{\footnotesize $^\dagger$ requiert \pkg{amsmath}}
\end{table}

\subsection{Flèches}
\label{sec:math:symboles:fleches}

Les flèches de différents types sont souvent utilisées en notation
mathématique, notamment dans les limites ou pour les expressions
logiques. Le \autoref{tab:math:fleches} en présente une sélection.

On retrouve les flèches utilisables en notation mathématique dans les
tableaux 102 à 119 de la %
\doc[\emph{Comprehensive {\LaTeX} Symbol List}]{}{http://texdoc.net/pkg/comprehensive}. %
Le document contient divers autres types de flèches, mais celles-ci ne
sont généralement pas appropriées pour les mathématiques (pensons à
{\manerrarrow} ou {\faArrowRight}).

Le paquetage \pkg{amsmath} fournit plusieurs flèches additionnelles
ainsi que la négation des plus communes. Ces dernières apparaissent
d'ailleurs dans le \autoref{tab:math:fleches}.

\begin{table}[p]
  \caption{Quelques flèches et leur négation}
  \label{tab:math:fleches}
  \begin{tabularx}{1.0\linewidth}{lXlX}
    $\gets$                & \cmd{\leftarrow}\quad \cmd{\gets} &
    $\longleftarrow$       & \cmd{\longleftarrow}              \\
    $\Leftarrow$           & \cmd{\Leftarrow}                  &
    $\Longleftarrow$       & \cmd{\Longleftarrow}              \\
    $\to$                  & \cmd{\rightarrow}\quad \cmd{\to}  &
    $\longrightarrow$      & \cmd{\longrightarrow}             \\
    $\Rightarrow$          & \cmd{\Rightarrow}                 &
    $\Longrightarrow$      & \cmd{\Longrightarrow}             \\
    $\uparrow$             & \cmd{\uparrow}                    &
    $\downarrow$           & \cmd{\downarrow}                  \\
    $\Uparrow$             & \cmd{\Uparrow}                    &
    $\Downarrow$           & \cmd{\Downarrow}                  \\
    $\updownarrow$         & \cmd{\updownarrow}                &
    $\Updownarrow$         & \cmd{\Updownarrow}                \\
    $\leftrightarrow$      & \cmd{\leftrightarrow}             &
    $\longleftrightarrow$  & \cmd{\longleftrightarrow}         \\
    $\Leftrightarrow$      & \cmd{\Leftrightarrow}             &
    $\Longleftrightarrow$  & \cmd{\Longleftrightarrow}         \\
    $\nleftarrow$          & \cmd{\nleftarrow}$^\dagger$         &
    $\nleftrightarrow$     & \cmd{\nleftrightarrow}$^\dagger$    \\
    $\nrightarrow$         & \cmd{\nrightarrow}$^\dagger$        &
    $\nLeftarrow$          & \cmd{\nLeftarrow}$^\dagger$         \\
    $\nLeftrightarrow$     & \cmd{\nLeftrightarrow}$^\dagger$    &
    $\nRightarrow$         & \cmd{\nRightarrow}$^\dagger$        \\
    \addlinespace
  \end{tabularx}
  \hspace*{1em}{\footnotesize $^\dagger$ requiert \pkg{amsmath}}
\end{table}

\subsection{Accents et autres symboles utiles}
\label{sec:math:symboles:autres}

Le \autoref{tab:math:autres} présente quelques uns des accents
disponibles dans le mode mathématique, ainsi que divers symboles
fréquemment utilisés en mathématiques.

Pour connaître l'ensemble des accents du mode mathématique de
{\LaTeX}, consulter le tableau~164 de la %
\doc[\emph{Comprehensive {\LaTeX} Symbol List}]{}{http://texdoc.net/pkg/comprehensive}. %
Les versions extensibles de certains accents se trouvent au
tableau~169. Quant aux symboles mathématiques divers, on en trouve de
toutes les sortes dans les tableaux 201--212.

\begin{table}[p]
  \caption{Accents et symboles mathématiques divers}
  \label{tab:math:autres}
  \begin{tabularx}{1.0\linewidth}{lXlXlXlX}
    $\hat{a}$ & \cmd{\hat}\verb={a}= &
    $\bar{a}$ & \cmd{\bar}\verb={a}= &
    $\tilde{a}$ & \cmd{\tilde}\verb={a}= &
    $\ddot{a}$ & \cmd{\ddot}\verb={a}= \\
    $\infty$ & \cmd{\infty} &
    $\nabla$ & \cmd{\nabla} &
    $\partial$ & \cmd{\partial} &
    $\ell$ & \cmd{\ell} \\
    $\forall$ & \cmd{\forall} &
    $\exists$ & \cmd{\exists} &
    $\emptyset$ & \cmd{\emptyset} &
    $\prime$ & \cmd{\prime} \\
    $\neg$ & \cmd{\neg} &
    $\backslash$ & \cmd{\backslash} &
    $\|$ & \pixbar &
    $\angle$ & \cmd{\angle}
  \end{tabularx}
\end{table}


\begin{exemple}
  L'équation suivante contient plusieurs des éléments présentés dans
  cette section et la précédente:
  \begin{equation*}
    \frac{\Gamma(\alpha)}{\lambda^\alpha} =
    \sum_{j = 0}^\infty \int_j^{j + 1} x^{\alpha - 1} e^{-\lambda x}\,
    dx,
    \quad
    \alpha > 0 \text{ et } \lambda > 0.
  \end{equation*}
  \begin{demo}
    \begin{texample}[0.58\linewidth]
\begin{lstlisting}
\begin{equation*}
\end{lstlisting}
      \producing
      \emph{équation hors paragraphe}
    \end{texample}

    \begin{texample}[0.58\linewidth]
\begin{lstlisting}
\frac{\Gamma(\alpha)}{
  \lambda^\alpha} =
\end{lstlisting}
      \producing
      $\dfrac{\Gamma(\alpha)}{\lambda^\alpha} =$
    \end{texample}

    \begin{texample}[0.58\linewidth]
\begin{lstlisting}
\sum_{j = 0}^\infty \int_j^{j + 1}
\end{lstlisting}
      \producing
      $\displaystyle \sum_{j = 0}^\infty \int_j^{j + 1}$
    \end{texample}

    \begin{texample}[0.58\linewidth]
\begin{lstlisting}
x^{\alpha - 1} e^{-\lambda x}\, dx
\end{lstlisting}
      \producing
      $\displaystyle x^{\alpha - 1} e^{-\lambda x}\, dx$
    \end{texample}

    \begin{texample}[0.58\linewidth]
\begin{lstlisting}
, \quad \alpha > 0 \text{ et }
\lambda > 0.
\end{lstlisting}
      \producing
      $, \quad \alpha > 0 \text{ et } \lambda > 0.$
    \end{texample}

    \begin{texample}[0.58\linewidth]
\begin{lstlisting}
\end{equation*}
\end{lstlisting}
      \producing
      \emph{fin de l'environnement}
    \end{texample}
  \end{demo}
  \qed
\end{exemple}


\section{Équations sur plusieurs lignes et numérotation}
\label{sec:math:align}

Dans ce qui précède, nous n'avons présenté que des équations tenant
sur une seule ligne en mode hors paragraphe. Cette section se penche
sur la manière de représenter des groupes d'équations du type
\begin{align}
  y &= 2x + 4 \\
  y &= 6x - 1
\end{align}
ou des suites d'équations comme
\begin{align*}
  x_{\text{max}}
  &= \sum_{i = 0}^{m - 1} (b - 1) b^i \\
  &= (b - 1) \sum_{i = 0}^{m - 1} b^i \\
  &= b^m - 1.
\end{align*}

Nous recommandons fortement les environnements de \pkg{amsmath} pour
les équations sur plusieurs lignes: ils sont plus polyvalents, plus
simples à utiliser et leur rendu est meilleur. Le
\autoref{tab:math:displays} --- repris presque intégralement de la
documentation de ce paquetage --- compare les différents
environnements pour les équations hors paragraphe.

\begin{table}[p]
  \caption{Comparaison des environnements pour les équations hors
    paragraphe de \pkg{amsmath} (les lignes verticales indiquent les
    marges logiques).}
  \label{tab:math:displays}
  \renewcommand{\theequation}{\arabic{equation}}
  \begin{eqxample}
\begin{lstlisting}
\begin{equation*}
  a = b
\end{equation*}
\end{lstlisting}
    \producing
    \begin{equation*}
      a = b
    \end{equation*}
  \end{eqxample}

  \begin{eqxample}
\begin{lstlisting}
\begin{equation}
  a = b
\end{equation}
\end{lstlisting}
    \producing
    \begin{equation}
      a = b
    \end{equation}
  \end{eqxample}

  \begin{eqxample}
\begin{lstlisting}
\begin{equation}
  \label{eq:5}
  \begin{split}
    a &= b + c - d \\
    &\phantom{=} + e - f \\
    &= g + h \\
    &= i
  \end{split}
\end{equation}
\end{lstlisting}
    \producing
    \begin{equation}\label{eq:math:5}
      \begin{split}
        a& =b+c-d\\
        &\phantom{=} +e-f\\
        & =g+h\\
        & =i
      \end{split}
    \end{equation}
  \end{eqxample}

  \begin{eqxample}
\begin{lstlisting}
\begin{multline}
  a + b + c + d + e + f \\
  + i + j + k + l + m + n
\end{multline}
\end{lstlisting}
    \producing
    \begin{multline}
      a+b+c+d+e+f\\
      +i+j+k+l+m+n
    \end{multline}
  \end{eqxample}

  \begin{eqxample}
\begin{lstlisting}
\begin{gather}
  a_1 = b_1 + c_1 \\
  a_2 = b_2 + c_2 - d_2 + e_2
\end{gather}
\end{lstlisting}
    \producing
    \begin{gather}
      a_1=b_1+c_1\\
      a_2=b_2+c_2-d_2+e_2
    \end{gather}
  \end{eqxample}

  \begin{eqxample}
\begin{lstlisting}
\begin{align}
  a_1 &= b_1 + c_1 \\
  a_2 &= b_2 + c_2 - d_2 + e_2
\end{align}
\end{lstlisting}
    \producing
    \begin{align}
      a_1& =b_1+c_1\\
      a_2& =b_2+c_2-d_2+e_2
    \end{align}
  \end{eqxample}

  \begin{eqxample}
\begin{lstlisting}
\begin{align}
  a_{11} &= b_{11} &
  a_{12} &= b_{12} \\
  a_{21} &= b_{21} &
  a_{22} &= b_{22} + c_{22}
\end{align}
\end{lstlisting}
    \producing
    \begin{align}
      a_{11}& =b_{11}&
      a_{12}& =b_{12}\\
      a_{21}& =b_{21}&
      a_{22}& =b_{22}+c_{22}
    \end{align}
  \end{eqxample}
\end{table}

\begin{itemize}
\item L'environnement de base pour les équations alignées sur un
  symbole de relation (en une ou plusieurs colonnes) est \Ie{align}.
  C'est l'environnement le plus utilisé en mode mathématique hormis
  \Pe{equation}.
\item Les environnements \Pe{multline}, \Pe{gather} et \Pe{align}
  existent également en version étoilée (\Ie{multline*}, \Ie{gather*},
  \Ie{align*}) qui ne numérotent pas les équations.
\item Dans les environnements \Pe{align} et \Ie{split}, les équations
  successives sont alignées sur le caractère se trouvant immédiatement
  après le marqueur de colonne \verb=&=.
\item Comme dans les tableaux, la commande {\pixbsbs} sert à
  délimiter les lignes de la suite d'équations.
\item Remarquer, dans le troisième exemple du
  \autoref{tab:math:displays}, comment la commande \cmd{\phantom} sert
  à insérer un blanc exactement de la largeur du symbole $=$ au début
  de la seconde ligne de la suite d'égalités.
\item Pour supprimer la numérotation d'une ligne dans une série
  d'équations numérotées, placer la commande \cmd{\notag} juste avant
  la commande de changement de ligne {\bs\bs}.
  \begin{demo}
    \begin{texample}[0.53\linewidth]
\begin{lstlisting}
\begin{align}
  a_1 &= b_1 + c_1 \notag \\
  a_2 &= b_2 + c_2 - d_2 + e_2
\end{align}
\end{lstlisting}
      \producing
      \begin{align}
        a_1& =b_1+c_1 \notag \\
        a_2& =b_2+c_2-d_2+e_2
      \end{align}
    \end{texample}
  \end{demo}
\item Les renvois vers des équations numérotées fonctionnent, comme
  partout ailleurs en {\LaTeX}, avec le système d'étiquettes et de
  références (\autoref{sec:tableaux:floats}). Le paquetage
  \pkg{amsmath} fournit également la pratique commande \cmd{\eqref}
  qui place automatiquement le numéro d'équation entre parenthèses.
  \begin{demo}
    \begin{texample}
\begin{lstlisting}
L'équation \eqref{eq:5} du
tableau `\ref*{tab:math:displays}' démontre que...
\end{lstlisting}
      \producing
      L'équation \eqref{eq:math:5} du tableau
      \ref*{tab:math:displays} démontre que\dots
    \end{texample}
  \end{demo}
\item L'environnement \Ie{split} sert à apposer un seul numéro à une
  équation affichée sur plusieurs lignes. Il doit être employé à
  l'intérieur d'un autre environnement d'équations hors paragraphe.
\end{itemize}
Consulter le chapitre 3 de la %
\doc{amsldoc}{http://texdoc.net/pkg/amsmath} %
du paquetage \pkg{amsmath} pour les détails sur
l'utilisation des environnements du \autoref*{tab:math:displays}.

\begin{conseil}
  Veillez à respecter les règles suivantes pour la composition des
  équations.
  \begin{enumerate}
  \item Qu'elles apparaissent en ligne ou hors paragraphe, les
    équations font partie intégrante de la phrase. Ainsi, les règles
    de ponctuation usuelles s'appliquent-elles aux équations.
  \item Lorsqu'une équation s'étend sur plus d'une ligne, couper
    chaque ligne \emph{avant} un opérateur de sorte que chaque ligne
    constitue une expression mathématique complète (voir les troisième
    et quatrième exemples du \autoref{tab:math:displays}).
  \item Ne numéroter que les équations d'un document auxquelles le
    texte fait référence.
  \end{enumerate}
\end{conseil}

\begin{exemple}
  Nous avons réalisé les deux suites d'équations au début de la
  section avec les extraits de code ci-dessous, dans l'ordre.
\begin{lstlisting}
\begin{align}
  y &= 2x + 4 \\
  y &= 6x - 1
\end{align}
\end{lstlisting}
\begin{lstlisting}
\begin{align*}
  x_{\text{max}}
  &= \sum_{i = 0}^{m - 1} (b - 1) b^i \\
  &= (b - 1) \sum_{i = 0}^{m - 1} b^i \\
  &= b^m - 1.
\end{align*}
\end{lstlisting}
  \qed
\end{exemple}


\section{Délimiteurs de taille variable}
\label{sec:math:delimiteurs}

Les délimiteurs en mathématiques sont des symboles généralement
utilisés en paire tels que les parenthèses $(~)$, les crochets $[~]$
ou les accolades $\{~\}$ et qui servent à regrouper des termes d'une
équation. La taille des délimiteurs doit s'adapter au contenu entre
ceux-ci afin d'obtenir, par exemple,
\begin{equation*}
  \left( 1 + \frac{1}{x} \right)
\end{equation*}
plutôt que la peu esthétique composition
\begin{equation*}
  (1 + \frac{1}{x}).
\end{equation*}

La paire de commandes
\begin{lstlisting}
\left`\meta{delim\_g}'  ...  \right`\meta{delim\_d}'
\end{lstlisting}
définit un délimiteur gauche \meta{delim\_g} et un délimiteur droit
\meta{delim\_d} dont la taille s'ajustera automatiquement au contenu
entre les deux commandes.
\begin{demo}
  \begin{texample}
\begin{lstlisting}
\left( 1 + \frac{1}{x} \right)
\end{lstlisting}
    \producing
    \begin{equation*}
      \left( 1 + \frac{1}{x} \right)
    \end{equation*}
  \end{texample}

  \begin{texample}
\begin{lstlisting}
\left(
  \sum_{i = 1}^n x_i^2
\right)^{1/2}
\end{lstlisting}
    \producing
    \begin{equation*}
      \left(
        \sum_{i = 1}^n x_i^2
      \right)^{1/2}
    \end{equation*}
  \end{texample}
\end{demo}
Les commandes \cmd{\left} et \cmd{\right} doivent toujours former une
paire, c'est-à-dire qu'à \emph{toute} commande \cmdprint{\left} doit
absolument correspondre une commande \cmdprint{\right}. Cette
contrainte est facile à oublier!

\begin{itemize}
\item Il est possible d'imbriquer des paires de commandes les unes à
  l'intérieur des autres, pour autant que l'expression compte toujours
  autant de \cmdprint{\left} que de \cmdprint{\right}.
  \begin{demo}
    \begin{texample}
\begin{lstlisting}
\left[
  \int
  \left(
    1 + \frac{x}{k}
  \right) dx
\right]
\end{lstlisting}
      \producing
      \begin{equation*}
        \left[
          \int
          \left(
            1 + \frac{x}{k}
          \right) dx
        \right]
      \end{equation*}
    \end{texample}
  \end{demo}
%
\item Les symboles \meta{delim\_g} et \meta{delim\_d} n'ont pas à
  former une paire logique; toute combinaison de délimiteurs est
  valide.
  \begin{demo}
    \begin{texample}
\begin{lstlisting}
\int_0^1 x\, dx =
\left[
  \frac{x^2}{2}
\right|_0^1
\end{lstlisting}
      \producing
      \begin{equation*}
        \int_0^1 x\, dx =
        \left[
          \frac{x^2}{2}
        \right|_0^1
      \end{equation*}
    \end{texample}
  \end{demo}
%
\item Il arrive qu'un seul délimiteur soit nécessaire. Pour respecter
  la règle de la paire ci-dessus, on aura recours dans ce cas à un
  délimiteur \emph{invisible} représenté par le caractère «\verb=.=».
  \begin{demo}
    \begin{texample}
\begin{lstlisting}
f(x) =
\left\{
  \begin{aligned}
    1 - x, &\quad x < 1 \\
    x - 1, &\quad x \geq 1
  \end{aligned}
\right.
\end{lstlisting}
      \producing
      \begin{equation*}
        f(x) =
        \left\{
          \begin{aligned}
            1 - x, &\quad x < 1 \\
            x - 1, &\quad x \geq 1
          \end{aligned}
        \right.
      \end{equation*}
    \end{texample}
  \end{demo}
  (L'environnement \Ie{aligned} utilisé ci-dessus provient de
  \pkg{amsmath}.) On notera au passage que l'environnement \Ie{cases}
  de \pkg{amsmath} rend plus simple la réalisation de constructions
  comme celle ci-dessus.
  \begin{demo}
    \begin{texample}
\begin{lstlisting}
f(x) =
\begin{cases}
  1 - x, & x < 1 \\
  x - 1, & x \geq 1
\end{cases}
\end{lstlisting}
      \producing
      \begin{equation*}
        f(x) =
        \begin{cases}
          1 - x, & x < 1 \\
          x - 1, & x \geq 1
        \end{cases}
      \end{equation*}
    \end{texample}
  \end{demo}
%
\item La règle de la paire est tout spécialement délicate dans les
  équations sur plusieurs lignes car elle s'applique à chaque ligne
  d'une équation. Par conséquent, si la paire de délimiteurs s'ouvre
  sur une ligne et se referme sur une autre, il faudra ajouter un
  délimiteur invisible à la fin de la première ligne ainsi qu'au début
  de la seconde.
  \begin{demo}
    \begin{texample}
\begin{lstlisting}
\begin{align*}
  a
  &= \left(
    b + \frac{c}{d}
    \right. \\
  &\phantom{=} \left.
    + \frac{e}{d} - f
    \right)
\end{align*}
\end{lstlisting}
      \producing
      \begin{align*}
        a
        &= \left(
          b + \frac{c}{d}
          \right. \\
        &\phantom{=} \left.
          + \frac{e}{d} - f
          \right)
      \end{align*}
    \end{texample}
  \end{demo}
%
\item Quand les choix de taille de délimiteurs de {\LaTeX} ne
  conviennent pas pour une raison ou pour une autre, on peut
  sélectionner leur taille avec les commandes %
  \cmd{\big}, %
  \cmd{\Big}, %
  \cmd{\bigg} et %
  \cmd{\Bigg}. %
  Ces commandes s'utilisent comme \cmdprint{\left} et
    \cmdprint{\right} en les faisant immédiatement suivre d'un
  délimiteur. Le \autoref{tab:math:big_et_al} contient des exemples de
  délimiteurs pour chaque taille.

  \begin{table}
    \centering
    \caption{Tailles des délimiteurs mathématiques}
    \label{tab:math:big_et_al}
    \begin{tabular}{ll}
      \toprule
      taille standard & $(~) \quad [~] \quad \{~\}$ \\
      \addlinespace[0.5\normalbaselineskip]
      \cmd{\big} & $\big(~\big) \quad \big[~\big] \quad \big\{~\big\}$ \\
      \addlinespace[0.5\normalbaselineskip]
      \cmd{\Big} & $\Big(~\Big) \quad \Big[~\Big] \quad \Big\{~\Big\}$ \\
      \addlinespace[0.5\normalbaselineskip]
      \cmd{\bigg} & $\bigg(~\bigg) \quad \bigg[~\bigg] \quad \bigg\{~\bigg\}$ \\
      \addlinespace[0.5\normalbaselineskip]
      \cmd{\Bigg} & $\Bigg(~\Bigg) \quad \Bigg[~\Bigg] \quad \Bigg\{~\Bigg\}$ \\
      \addlinespace[4pt]\bottomrule
    \end{tabular}
  \end{table}
\end{itemize}

La section~14 de la %
\doc{amsldoc}{http://texdoc.net/pkg/amsmath} %
de \pkg{amsmath} traite de divers enjeux typographiques en lien
avec les délimiteurs et on y introduit des nouvelles commandes pour
contrôler leur taille. C'est une lecture suggérée.

\begin{exemple}
  Le développement de la formule d'approximation de Simpson comporte
  plusieurs des éléments discutés jusqu'à maintenant:
  \begin{align*}
    \int_a^b f(x)\, dx
    &\approx \sum_{j = 0}^{n - 1}
      \int_{x_{2j}}^{x_{2(j + 1)}} f(x)\, dx \\
    &= \frac{h}{3} \sum_{j = 0}^{n - 1}
      \left[
      f(x_{2j}) + 4 f(x_{2j + 1}) + f(x_{2(j + 1)})
      \right]
    \displaybreak[0] \\
    &= \frac{h}{3}
      \left[
      f(x_0) +
      \sum_{j = 1}^{n - 1} f(x_{2j}) +
      4 \sum_{j = 0}^{n - 1} f(x_{2j + 1})
      \right. \displaybreak[0] \\
    &\phantom{=}  + \left.
      \sum_{j = 0}^{n - 2} f(x_{2(j + 1)}) +
      f(x_{2n})
      \right] \\
    &= \frac{h}{3}
      \left[
      f(a) +
      2 \sum_{j = 1}^{n - 1} f(x_{2j}) +
      4 \sum_{j = 0}^{n - 1} f(x_{2j + 1}) +
      f(b)
      \right].
  \end{align*}
  On compose ce bloc d'équations avec le code source ci-dessous.
\begin{lstlisting}
\begin{align*}
  \int_a^b f(x)\, dx
  &\approx \sum_{j = 0}^{n - 1}
    \int_{x_{2j}}^{x_{2(j + 1)}} f(x)\, dx \\
  &= \frac{h}{3} \sum_{j = 0}^{n - 1}
    \left[
    f(x_{2j}) + 4 f(x_{2j + 1}) + f(x_{2(j + 1)})
    \right] \\
  &= \frac{h}{3}
    \left[
    f(x_0) +
    \sum_{j = 1}^{n - 1} f(x_{2j}) +
    4 \sum_{j = 0}^{n - 1} f(x_{2j + 1})
    \right. \\
  &\phantom{=} + \left.
    \sum_{j = 0}^{n - 2} f(x_{2(j + 1)}) +
    f(x_{2n})
    \right] \\
  &= \frac{h}{3}
    \left[
    f(a) +
    2 \sum_{j = 1}^{n - 1} f(x_{2j}) +
    4 \sum_{j = 0}^{n - 1} f(x_{2j + 1}) +
    f(b)
    \right].
\end{align*}
\end{lstlisting}
  \qed
\end{exemple}

\begin{conseil}
  De manière générale, il est déconseillé de scinder une suite
  d'équations entre deux pages. Le chargement du paquetage
  \pkg{amsmath} désactive d'ailleurs cette possibilité. Cependant,
  c'est parfois inévitable pour les longs blocs d'équations.

  La commande \cmd{\displaybreak}, placée immédiatement avant {\bs\bs}
  dans un environnement comme \Ie{align} permet d'indiquer à {\LaTeX}
  la possibilité d'insérer un saut de page après la ligne courante
  dans le bloc d'équations.

  La commande accepte en argument optionnel un entier entre $0$ et $4$
  indiquant à quel point un saut de page est désiré:
  \cmdprint{\displaybreak[0]} signifie «il est permis de changer de
  page ici» sans que ce ne soit obligatoire;
  \cmdprint{\displaybreak[4]} ou, de manière équivalente,
  \cmdprint{\displaybreak} force un saut de page.

  À moins d'en être vraiment aux toutes dernières étapes d'édition
  d'un document, il est conseillé d'utiliser la commande
  \cmdprint{\displaybreak} avec parcimonie et avec un argument
  optionnel faible.
\end{conseil}


\section{Caractères gras en mathématiques}
\label{sec:math:gras}

Les caractères gras sont parfois utilisés en mathématiques,
particulièrement pour représenter les vecteurs et les matrices:
\begin{equation*}
  \symbf{A} \symbf{x} = \symbf{b} \Leftrightarrow
  \symbf{x} = \symbf{A}^{-1} \symbf{b}
\end{equation*}

Pourquoi consacrer une section spécialement à cette convention
typographique? Parce que la création de symboles mathématiques en
gras doit certainement figurer parmi les questions les plus
fréquemment posées par les utilisateurs de {\LaTeX}\dots\ et que la
réponse n'est pas unique!

La commande
\begin{lstlisting}
\mathbf`\marg{symbole}'
\end{lstlisting}
place \meta{symbole} en caractère gras en mode mathématique. C'est
donc l'analogue de la commande \cmd{\textbf} du mode texte. Dans
{\LaTeX} de base, la commande n'a toutefois un effet que sur les
lettres latines et, parfois, les lettres grecques majuscules.
\begin{demo}
  \begin{texample}[0.6\linewidth]
\begin{lstlisting}
\theta \mathbf{\theta} +
\Gamma \mathbf{\Gamma} \mathbf{+}
A \mathbf{A}
\end{lstlisting}
    \producing
    %% ok, ici il faut vraiment tricher pour reproduire avec
    %% unicode-math ce qui se produirait avec pdflatex standard
    $\theta {\theta} + \Gamma \symbf{\Gamma} {+} A \symbfup{A}$
  \end{texample}
\end{demo}
On remarquera aussi que \verb=\mathbf{A}= produit une lettre
majuscule droite plutôt qu'en italique.

La manière la plus standard et robuste d'obtenir des symboles
mathématiques (autres que les lettres) en gras semble être, au moment
d'écrire ces lignes, via la commande
\begin{lstlisting}
\bm`\marg{symbole}'
\end{lstlisting}
fournie par le paquetage \pkg{bm} \citep{bm}.
\begin{demo}
  \begin{texample}[0.6\linewidth]
\begin{lstlisting}
\theta \bm{\theta} +
\Gamma \bm{\Gamma} \bm{+}
A \bm{A}
\end{lstlisting}
    \producing
    %% ok, ici il faut vraiment tricher pour reproduire avec
    %% unicode-math ce qui se produirait avec pdflatex standard
    $\theta \symbf{\theta} + \Gamma \symbfup{\Gamma} \symbf{+} A
    \symbf{A}$
  \end{texample}
\end{demo}

Les utilisateurs de {\XeLaTeX} chargent normalement le paquetage
\pkg{unicode-math} \citep{unicode-math} pour sélectionner la police
pour les mathématiques. Ce paquetage fournit la commande
\begin{lstlisting}
\symbf`\marg{symbole}'
\end{lstlisting}
pour placer un \meta{symbole} mathématique en gras. Le paquetage offre
différentes combinaisons de lettres latines et grecques droites ou
italiques en gras selon la valeur de l'option \code{bold-style};
consulter la section~5 de la %
\doc{unicode-math}{http://texdoc.net/pkg/unicode-math}. %
\begin{demo}
  \begin{texample}[0.6\linewidth]
\begin{lstlisting}
% XeLaTeX + paquetage unicode-math
% avec l'option bold-style=ISO
\theta \symbf{\theta} +
\Gamma \symbf{\Gamma} \symbf{+}
A \symbf{A}
\end{lstlisting}
    \producing
    $\theta \symbf{\theta} + \Gamma \symbf{\Gamma} \symbf{+} A
    \symbf{A}$
  \end{texample}
\end{demo}

\begin{conseil}
  \label{conseil:math:mat}
  Si le gras est fréquemment utilisé dans un document pour une
  notation particulière, il est fortement recommandé de définir une
  nouvelle commande\footnotemark\
  sémantique plutôt que d'utiliser à répétition l'une ou l'autre des
  commandes ci-dessus.

  Par exemple, si le gras est utilisé pour les vecteurs et matrices,
  on pourrait définir une nouvelle commande \cmdprint{\mat} en
  insérant dans le préambule du document
\begin{lstlisting}
\newcommand[1]{\mat}{\bm{#1}}
\end{lstlisting}
\end{conseil}
%
\footnotetext{%
  La définition de nouvelles commandes est couvert plus en détail au
  \autoref{chap:commandes}.}


\begin{information}
  On l'a vu ci-dessus: le paquetage \pkg{unicode-math} offre l'option
  \code{ISO} pour le traitement du gras dans les mathématiques. En
  fait, c'est toute la composition des mathématiques qui fait l'objet
  d'un standard ISO!

  Plusieurs prescriptions du standard --- comme les noms de variables
  en italique --- sont déjà prises en compte par {\LaTeX}. Le respect
  de certaines autres règles est moins répandu, notamment celle qui
  veut que les constantes mathématiques dont la valeur de change pas
  (ce sont des constantes, après tout) soient représentées en police
  droite. On pense, par exemple, au nombre d'Euler $\mathrm{e}$, au
  nombre imaginaire $\mathrm{i} = \sqrt{-1}$. Même chose pour les
  opérateurs mathématiques, dont le $\mathrm{d}$ dans les dérivées et
  les intégrales.

  Voici un exemple d'intégrale composée avec ces règles:
  \begin{equation*}
    \int_C \frac{\mathrm{e}^z}{z}\, \mathrm{d}z = 2 \pi \mathrm{i}.
  \end{equation*}
  Pour en savoir plus sur l'utilisation du standard ISO dans {\LaTeX},
  consulter \citet{Beccari:ISO} ou, pour un sommaire rapide, l'%
  \link{https://nickhigham.wordpress.com/2016/01/28/typesetting-mathematics-according-to-the-iso-standard/}{entrée de blogue} %
  de Nick Higham.
\end{information}

\begin{exemple}
  \label{ex:math:matrices}
  Le paquetage \pkg{amsmath} fournit quelques environnements qui
  facilitent la mise en forme de matrices; ils diffèrent simplement
  par le type de délimiteur autour de la matrice.

  Considérer le bloc d'équations suivant:
  \begin{align*}
    \mat{J}(\mat{\theta})
    &=
      \begin{bmatrix}
        \dfrac{\partial f_1(\mat{\theta})}{\partial \theta_1} &
        \dfrac{\partial f_1(\mat{\theta})}{\partial \theta_2} \\[12pt]
        \dfrac{\partial f_2(\mat{\theta})}{\partial \theta_1} &
        \dfrac{\partial f_2(\mat{\theta})}{\partial \theta_2}
      \end{bmatrix} \\
    &=
      \left[
      \frac{\partial f_i(\mat{\theta})}{\partial \theta_j}
      \right]_{2 \times 2}, \quad i, j = 1, 2.
  \end{align*}
  Le code pour composer ces équations est:
\begin{lstlisting}
\begin{align*}
  \mat{J}(\mat{\theta})
  &=
  \begin{bmatrix}
    \dfrac{\partial f_1(\mat{\theta})}{\partial \theta_1} &
    \dfrac{\partial f_1(\mat{\theta})}{\partial \theta_2}
    \\[12pt] % augmenter l'espace entre les lignes
    \dfrac{\partial f_2(\mat{\theta})}{\partial \theta_1} &
    \dfrac{\partial f_2(\mat{\theta})}{\partial \theta_2}
  \end{bmatrix} \\
  &=
  \left[
  \frac{\partial f_i(\mat{\theta})}{\partial \theta_j}
  \right]_{2 \times 2}, \quad i, j = 1, 2.
\end{align*}
\end{lstlisting}
  On remarquera l'utilisation de la commande \cmd{\dfrac}
  (\autoref{sec:math:bases:fractions}) pour composer des grandes
  fractions à l'intérieur des matrices. Nous avons également eu
  recours à la commande \cmdprint{\mat} mentionnée dans la
  rubrique Conseil du {\TeX}pert de la \autopageref{conseil:math:mat}. %
  \qed
\end{exemple}


%%%
%%% Exercices
%%%

\section{Exercices}
\label{sec:math:exercices}

\Opensolutionfile{solutions}[solutions-mathematiques]

\begin{Filesave}{solutions}
\section*{Chapitre \ref*{chap:math}}
\addcontentsline{toc}{section}{Chapitre \protect\ref*{chap:math}}

\end{Filesave}

\begin{exercice}
  Utiliser le gabarit de document \fichier{exercice\_gabarit.tex} pour
  reproduire le texte suivant:
  \begin{quote}
    La dérivée de la fonction composée $f \circ g(x) = f[g(x)]$ est
    $\{f[g(x)]\}^\prime = f^\prime[g(x)] g^\prime(x)$. La dérivée
    seconde du produit des fonctions $f$ et $g$ est
    \begin{equation*}
      [f(x) g(x)]^{\prime\prime} =
      f^{\prime\prime}(x) g(x) + 2 f^\prime(x) g^\prime(x) +
      f(x) g^{\prime\prime}(x).
    \end{equation*}
  \end{quote}
  \begin{sol}
    On trouve la commande pour produire le symbole $\circ$ dans le
    \autoref{tab:math:binaires}. On peut produire les symboles de
    dérivée $\prime$ avec la commande \cmd{\prime}
    (\autoref{tab:math:autres}) ou simplement avec le caractère
    \verb='=.
\begin{lstlisting}
La dérivée de la fonction composée $f \circ g(x) =
f[g(x)]$ est $\{f[g(x)]\}^\prime = f^\prime[g(x)]
g^\prime(x)$. La dérivée seconde du produit des
fonctions $f$ et $g$ est
\begin{equation*}
  [f(x) g(x)]^{\prime\prime} =
  f^{\prime\prime}(x) g(x) + 2 f^\prime(x) g^\prime(x) +
  f(x) g^{\prime\prime}(x).
\end{equation*}
\end{lstlisting}
  \end{sol}
\end{exercice}

\begin{exercice}
  Composer l'équation suivante avec l'environnement \Pe{align*}:
  \begin{align*}
    f(x +& h, y + k)
    = f(x, y) +
      \left\{
      \frac{\partial f(x, y)}{\partial x} h +
      \frac{\partial f(x, y)}{\partial y} k
      \right\} \\
    &+
      \frac{1}{2}
      \left\{
      \frac{\partial^2 f(x, y)}{\partial x^2} h^2 +
      \frac{\partial^2 f(x, y)}{\partial x \partial y} kh +
      \frac{\partial^2 f(x, y)}{\partial y^2} k^2
      \right\} \\
    &+
      \frac{1}{6} \{\cdots\} + \dots + \frac{1}{n!} \{\cdots\} + R_n.
  \end{align*}
  Aligner les deuxième et troisième lignes de l'équation sur divers
  caractères de la première ligne afin que l'équation ne dépasse pas
  les marges du document.
  \begin{sol}
    Nous avons aligné les lignes de l'équation juste à droite du
    premier symbole $+$ à la première ligne. Remarquer l'usage des
    commandes \cmd{\cdots} et \cmd{\dots} dans la dernière ligne:
    {\LaTeX} choisit correctement la position centrée des points entre les
    opérateurs d'addition, mais pas entre les accolades.
\begin{lstlisting}
\begin{align*}
  f(x +& h, y + k) = f(x, y) +
    \left\{
    \frac{\partial f(x, y)}{\partial x} h +
    \frac{\partial f(x, y)}{\partial y} k
    \right\} \\
  &+
    \frac{1}{2}
    \left\{
    \frac{\partial^2 f(x, y)}{\partial x^2} h^2 +
    \frac{\partial^2 f(x, y)}{\partial x \partial y} kh +
    \frac{\partial^2 f(x, y)}{\partial y^2} k^2
    \right\} \\
  &+
    \frac{1}{6} \{\cdots\} + \dots +
    \frac{1}{n!} \{\cdots\} + R_n.
\end{align*}
\end{lstlisting}
  \end{sol}
\end{exercice}

\begin{exercice}
  Composer à l'aide de l'environnement \Ie{cases}
  (\autoref{sec:math:delimiteurs}) la définition de la fonction
  $\tilde{f}(x)$:
  \begin{equation*}
    \tilde{f}(x) =
    \begin{cases}
      0, & x \leq c_0\\
      \dfrac{F_n(c_j) - F_n(c_{j-1})}{c_j - c_{j-1}} =
      \dfrac{n_j}{n (c_j - c_{j - 1})}, &  c_{j-1} < x \leq c_j\\
      0, & x > c_r.
    \end{cases}
  \end{equation*}
  Il est nécessaire d'imposer la taille des fractions dans la seconde
  branche de la définition à l'aide des fonctions de la
  \autoref{sec:math:bases:fractions}.
  \begin{sol}
    Il faut utiliser \cmd{\dfrac} pour obtenir des fractions dans une
    branche de \Pe{cases} de la même taille que dans une équation hors
    paragraphe:
\begin{lstlisting}
\begin{equation*}
  \tilde{f}(x) =
  \begin{cases}
    0, & x \leq c_0\\
    \dfrac{F_n(c_j) - F_n(c_{j-1})}{c_j - c_{j-1}} =
    \dfrac{n_j}{n (c_j - c_{j - 1})}, &
      c_{j-1} < x \leq c_j \\
    0, & x > c_r.
  \end{cases}
\end{equation*}
\end{lstlisting}
  \end{sol}
\end{exercice}

\begin{exercice}[nosol]
  Le fichier \fichier{exercice\_mathematiques.tex} contient un exemple
  complet de développement mathématique. Étudier le contenu du fichier
  puis compiler celui-ci tel quel avec {pdf\LaTeX} ou {\XeLaTeX}.
  Effectuer ensuite les modifications suivantes.
  \begin{enumerate}
  \item Charger le paquetage \pkg{amsfonts} dans le préambule, puis
    remplacer \verb=$R^+$= par \verb=$\mathbb{R}^+$= à la ligne
    débutant par «Le domaine».
  \item Dans l'équation du Jacobien de la transformation, remplacer
    successivement l'environnement \Ie{vmatrix} par %
    \Ie{pmatrix}, %
    \Ie{bmatrix}, %
    \Ie{Bmatrix} et %
    \Ie{Vmatrix}. Observer l'effet sur les délimiteurs de la matrice.
  \item Toujours dans la même matrice, composer successivement les
    deux fractions avec les commandes \cmd{\frac}, \cmd{\tfrac} et
    \cmd{\dfrac}. Observer le résultat.
  \item Réduire l'espacement de part et d'autre du symbole
    $\Leftrightarrow$ dans la seconde équation hors paragraphe.
  \item À l'aide de la fonction Rechercher et remplacer de l'éditeur
    de texte, remplacer toutes les occurrences du symbole $\theta$ par
    $\lambda$.
  \end{enumerate}
\end{exercice}

\Closesolutionfile{solutions}

%%% Local Variables:
%%% mode: latex
%%% TeX-engine: xetex
%%% TeX-master: "formation-latex-ul"
%%% coding: utf-8
%%% End:

version https://git-lfs.github.com/spec/v1
oid sha256:bde34e0d67701852574dce9d46b6551edf73017845c601bdb936b6e993de1b93
size 34993

version https://git-lfs.github.com/spec/v1
oid sha256:3cab9dad9c9aa96de811e3293fec99d28d04f509f2d40a058a663b9c49a40560
size 17069

\chapter{Trucs et astuces divers}
\label{chap:trucs}

En clôture de l'ouvrage, ce chapitre traite de différents sujets que
même une personne débutant avec {\LaTeX} voudra assez rapidement
aborder, comme le contrôle des sauts de ligne et des sauts de page, la
modification de la police du document, l'utilisation de la couleur ou
l'insertion d'hyperliens dans le fichier de sortie PDF. Nous offrons
également de courtes introductions à des usages plus spécialisés de
{\LaTeX} comme la mise en page de code informatique, la production de
diapositives ou la programmation lettrée. Enfin, nous expliquons
sommairement comme assurer de manière efficace la gestion des versions
de ses documents, surtout dans un contexte de travail collaboratif.

\section{Contrôle de la disposition du texte}
\label{sec:trucs:controle}

Des algorithmes élaborés permettent généralement à {\LaTeX} de
disposer le texte harmonieusement sur la page. Néanmoins, il
est parfois nécessaire  d'effectuer soi-même de menus ajustements,
notamment pour les sauts de page et la coupure de mots.

\subsection{Sauts de ligne et de page}
\label{sec:trucs:controle:sauts}

Il est assez rarement nécessaire avec {\LaTeX} de devoir forcer les
retours à la ligne. Chose certaine, l'on devrait toujours utiliser une
ligne blanche dans le code source pour identifier un changement de
paragraphe.

Cela dit, les commandes suivantes permettent d'insérer un saut de
ligne manuellement lorsque requis:
\begin{lstlisting}
\\
\\`\oarg{longueur}'
\newline
\end{lstlisting}
La commande {\pixbsbs} est connue: elle sert aussi à délimiter les
lignes dans les tableaux (\autoref{sec:tableaux:tableaux}) et les lignes
d'une suite d'équations (\autoref{sec:math:align}). L'argument optionnel
\oarg{longueur} permet d'insérer un blanc entre les deux lignes; la
\autoref{sec:bases:longueurs} explique comment spécifier une
longueur.

Généralement équivalente à {\bs\bs}, la commande \cmd{\newline} est
parfois nécessaire, notamment pour insérer un changement de ligne à
l'intérieur d'une cellule d'un tableau ou à l'intérieur d'un titre de
section. Quand {\bs\bs} ne fonctionne pas, essayer
\cmdprint{\newline}.

\begin{exemple}
  La commande {\bs\bs} est particulièrement utile --- voire nécessaire
  --- pour disposer des boîtes à l'intérieur d'une figure.
  L'utilisation de l'argument \meta{longueur} permet alors de
  contrôler l'espacement vertical entre les éléments. Comparer les
  deux exemples ci-dessous.
  \begin{demo}
    \begin{texample}[0.55\linewidth]
\begin{lstlisting}
\begin{minipage}{1.0\linewidth}
  \framebox[\linewidth]{texte}
\end{minipage} \\
\begin{minipage}{1.0\linewidth}
  \framebox[\linewidth]{texte}
\end{minipage}
\end{lstlisting}
      \producing
      \begin{minipage}{1.0\linewidth}
        \framebox[\linewidth]{texte}
      \end{minipage}
      \begin{minipage}{1.0\linewidth}
        \framebox[\linewidth]{texte}
      \end{minipage}
    \end{texample}

    \begin{texample}[0.55\linewidth]
\begin{lstlisting}
\begin{minipage}{1.0\linewidth}
  \framebox[\linewidth]{texte}
\end{minipage} \\`\textbf{[6pt]}'
\begin{minipage}{1.0\linewidth}
  \framebox[\linewidth]{texte}
\end{minipage}
\end{lstlisting}
      \producing
      \begin{minipage}{1.0\linewidth}
        \framebox[\linewidth]{texte}
      \end{minipage} \\[6pt]
      \begin{minipage}{1.0\linewidth}
        \framebox[\linewidth]{texte}
      \end{minipage}
    \end{texample}
  \end{demo}
  \qed
\end{exemple}

Les commandes
\begin{lstlisting}
\newpage
\clearpage
\cleartorecto              % classe memoir seulement
\cleartoverso              % classe memoir seulement
\end{lstlisting}
permettent d'insérer manuellement un saut de page pour éviter une
coupure malheureuse. La commande de base pour insérer un saut
n'importe où dans la page est \cmd{\newpage}. La commande
\cmd{\clearpage}, quant à elle, va également s'assurer d'afficher tous
les éléments flottants (\autoref{sec:tableaux:floats}) en attente de
disposition.

Les commandes \cmd{\cleartorecto} et \cmd{\cleartoverso}, propres à la
classe \class{memoir} permettent respectivement de passer
automatiquement à une page recto ou à une page verso. Évidemment, elles
n'ont d'utilité que dans les documents recto verso.

Moins directives, les commandes
\begin{lstlisting}
\pagebreak`\oarg{n}'
\enlargethispage`\marg{longueur}'
\end{lstlisting}
permettent de seulement aider {\LaTeX} à gérer les sauts de page à un
endroit précis. La commande \cmd{\pagebreak} est intéressante
lorsque utilisée avec son argument optionnel \meta{n}: celui-ci
indique, par le biais d'un nombre entier entre 0 et 4, à quel point
nous \emph{recommandons} à {\LaTeX} d'insérer un saut de page à
l'endroit où la commande apparaît (0 étant une faible recommandation
et 4, une forte).

La commande \cmd{\enlargethispage}, comme son nom l'indique, permet
d'allonger une page de \meta{longueur} pour y faire tenir plus de
texte. C'est une commande particulièrement utile pour éviter que la
toute dernière ligne d'un chapitre ou d'un document se retrouve seule sur
une page.

\subsection{Coupure de mots}
\label{sec:trucs:controle:coupure}

La coupure automatique des mots en fin de ligne est toujours active
avec {\LaTeX}. C'est d'ailleurs pourquoi il importe d'indiquer à
{\LaTeX} dans quelle langue est le texte (lorsque ce n'est pas en
anglais) avec les commandes du paquetage \pkg{babel}.

Il existe deux façons de contrôler la coupure de mots. La première,
principalement utilisée lorsque {\LaTeX} refuse de couper un mot en
fin de ligne, consiste à insérer des \emph{suggestions} d'endroits où
couper le mot avec la commande \cmd{\-}. Par exemple, en écrivant
\verb=vrai\-sem\-blance=, on indique à {\LaTeX} qu'il est possible de
diviser le mot en \emph{vrai-semblance} ou \emph{vraisem-blance}.

La seconde méthode, celle-là surtout utilisée lorsque {\LaTeX} ne
reconnaît pas des mots qui reviennent souvent dans le document,
consiste à fournir dans le préambule une liste d'exceptions avec la
commande
\begin{lstlisting}
\hyphenation`\marg{liste}'
\end{lstlisting}
La \meta{liste} est une suite de mots, séparés par des virgules, des
blancs ou des retours à la ligne, dans lesquels les points de coupure
sont identifiés par un trait d'union.

\begin{exemple}
  La commande suivante, insérée dans le préambule, permet d'ajouter
  des points de coupure aux mots «puisque», «constante» et
  «vraisemblance» pour l'ensemble du document.
\begin{lstlisting}
\hyphenation{puis-que,cons-tante,vrai-sem-blance}
\end{lstlisting}
  \qed
\end{exemple}

\begin{important}
  Règle générale, garder les opérations d'ajustements de la mise en
  page --- position des éléments flottants, sauts de page, lignes trop
  longues, etc. --- pour la toute fin de la rédaction.
\end{important}


\section{Au-delà de la police Computer Modern}
\label{sec:trucs:polices}

{\CM%
  Les documents {\LaTeX} standards sont facilement reconnaissables par
  leur police par défaut, Computer Modern --- celle utilisée dans ce
  paragraphe. Pour qui souhaitait briser la relative monotonie induite
  par cette uniformité, il a longtemps été difficile d'utiliser une
  autre police. Fort heureusement, la situation a beaucoup évolué et
  il est aujourd'hui assez simple de produire des documents {\LaTeX}
  utilisant des polices variées.}

Avant d'aller plus loin, une mise en garde: si un document contient
plus que quelques équations mathématiques très simples, le choix de
police devient très restreint. La raison: peu de polices comprennent
des symboles mathématiques et les informations nécessaires pour les
assembler selon les hauts standards qualité usuels de {\LaTeX}.

Cela dit, pour qui souhaite aller au-delà de la police Computer Modern
sans trop se compliquer la vie, il existe deux solutions principales.

\begin{enumerate}
\item Utiliser l'une ou l'autre des polices PostScript standards
  convenant pour du texte normal
  ({\Times Times}, %
   {\Palatino Palatino}, %
   {\Bookman Bookman}, %
   {\Charter Charter}, %
   {\NewCent New Century Schoolbook}, %
   {\Helvet Helvetica}). %
  C'est très simple avec toute distribution {\LaTeX} moderne: il
  suffit de charger le paquetage approprié. Consulter la %
  \doc{psnfss2e}{http://texdoc.net/pkg/psnfss/} %
  de l'ensemble de paquetages PSNFSS.
\item Utiliser une police OpenType ou TrueType présente sur son
  système avec le moteur {\XeLaTeX}. Seule une poignée de ces polices
  offre toutefois un support approprié pour les mathématiques. La
  gestion des polices avec {\XeLaTeX} se fait avec le paquetage
  standard \pkg{fontspec}; consulter sa %
  \doc{fontspec}{http://texdoc.net/pkg/fontspec/}.
\end{enumerate}

Pour les thèses et mémoires de l'Université Laval, la Faculté des
études supérieures et postdoctorales accepte les polices %
\begin{quote}
  \begin{tabbing}
    Computer Modern \qquad \=  ABCDEF abcdef 1234567890 \kill
    {\CM Computer Modern} \> {\CM ABCDEF abcdef 1234567890} \\
    {\Times Times} \> {\Times ABCDEF abcdef 1234567890} \\
    {\Palatino Palatino} \> {\Palatino ABCDEF abcdef 1234567890}
  \end{tabbing}
\end{quote}
Pour utiliser ces deux dernières avec pdf{\LaTeX}, charger
respectivement les paquetages \pkg{mathptmx} ou \pkg{mathpazo}. Avec
{\XeLaTeX}, utiliser les polices Termes et Pagella du projet %
\link{http://www.gust.org.pl/projects/e-foundry/tex-gyre/}{TeX~Gyre}.
Ce sont des polices très similaires à Times et Palatino, disponibles
en version OpenType et qui fournissent un bon support pour les
mathématiques via le projet frère %
\link{http://www.gust.org.pl/projects/e-foundry/tg-math/}{TeX~Gyre
  Math}.

\begin{exemple}
  \label{ex:trucs:palatino}
  Pour utiliser la police PostScript classique Palatino avec {\LaTeX}
  tant pour le texte que pour les mathématiques, il suffit d'insérer
  dans le préambule de son document la commande
\begin{lstlisting}
\usepackage{mathpazo}
\end{lstlisting}

  Avec le moteur {\XeLaTeX}, il est possible d'utiliser n'importe
  quelle police OpenType et TrueType installée dans le système
  d'exploitation de l'ordinateur. Pour obtenir un résultat équivalent
  à celui de \pkg{mathpazo}, installer les polices TeX~Gyre dans le
  système, puis insérer dans le préambule les commandes
\begin{lstlisting}
\usepackage{fontspec}
\setmainfont{TeX Gyre Pagella}
\setmathfont{TeX Gyre Pagella Math}
\end{lstlisting}
  \qed
\end{exemple}

Le texte principal du présent document est en %
\link{http://tug.org/store/lucida/}{Lucida Bright~OT}, %
une police commerciale de très haute qualité offrant également un
excellent support pour les mathématiques. Ses auteurs ont toujours été
proches de la communauté {\LaTeX}. La Bibliothèque de l'Université
Laval détient une licence d'utilisation de cette police. Les étudiants
et le personnel de l'Université peuvent s'en procurer une copie
gratuitement en écrivant à
\href{mailto:lucida@bibl.ulaval.ca}{lucida@bibl.ulaval.ca}.



\section{Couleurs}
\label{sec:trucs:couleurs}

L'utilisation de couleur dans un document {\LaTeX} requiert de charger
le paquetage \pkg{xcolor} \citep{xcolor}. Celui-ci définit d'abord
plusieurs couleurs que l'on peut utiliser directement; consulter la %
\doc{xcolor}{http://texdoc.net/pkg/xcolor} %
pour en connaître les différentes listes. Le
\autoref{tab:trucs:couleurs} fournit celle des couleurs toujours
disponibles.

\begin{table}
  \centering
  \caption{Couleurs toujours disponibles lorsque le paquetage
    \pkg{xcolor} est chargé}
  \label{tab:trucs:couleurs}
  \begin{tabularx}{1.0\linewidth}{XlXll}
    \toprule
    \fcolorbox{black}{black}{\phantom{xx}}\, black &
    \fcolorbox{black}{blue}{\phantom{xx}}\, blue &
    \fcolorbox{black}{brown}{\phantom{xx}}\, brown &
    \fcolorbox{black}{cyan}{\phantom{xx}}\, cyan &
    \fcolorbox{black}{darkgray}{\phantom{xx}}\, darkgray \\
    \addlinespace[3pt]
    \fcolorbox{black}{gray}{\phantom{xx}}\, gray &
    \fcolorbox{black}{green}{\phantom{xx}}\, green &
    \fcolorbox{black}{lightgray}{\phantom{xx}}\, lightgray &
    \fcolorbox{black}{lime}{\phantom{xx}}\, lime &
    \fcolorbox{black}{magenta}{\phantom{xx}}\, magenta \\
    \addlinespace[3pt]
    \fcolorbox{black}{olive}{\phantom{xx}}\, olive &
    \fcolorbox{black}{orange}{\phantom{xx}}\, orange &
    \fcolorbox{black}{pink}{\phantom{xx}}\, pink &
    \fcolorbox{black}{purple}{\phantom{xx}}\, purple &
    \fcolorbox{black}{red}{\phantom{xx}}\, red \\
    \addlinespace[3pt]
    \fcolorbox{black}{teal}{\phantom{xx}}\, teal &
    \fcolorbox{black}{violet}{\phantom{xx}}\, violet &
    \fcolorbox{black}{white}{\phantom{xx}}\, white &
    \fcolorbox{black}{yellow}{\phantom{xx}}\, yellow \\
    \bottomrule
  \end{tabularx}
\end{table}

Un peu comme pour les changements d'attributs de police, il existe
deux commandes pour modifier la couleur du texte:
\begin{lstlisting}
\color`\marg{nom}'
\textcolor`\marg{nom}\marg{texte}'
\end{lstlisting}
La première modifie la couleur de tout ce qui suit (à moins d'en
limiter la portée avec des accolades) et la seconde, seulement pour
\meta{texte}. Dans les deux cas, \meta{nom} est le nom d'une couleur.
\begin{demo}
  \begin{texample}
\begin{lstlisting}
texte {\color{red} en rouge}
et \textcolor{blue}{en bleu}
\end{lstlisting}
  \producing
  texte {\color{red} en rouge} et \textcolor{blue}{en bleu}
  \end{texample}
\end{demo}

La commande \cmd{\definecolor} permet de définir de nouvelles couleurs
selon plusieurs systèmes de codage. Le plus usuel demeure \emph{Rouge,
  vert, bleu} (RVB ou RGB, en anglais) où une couleur est représentée
par une combinaison de teintes --- exprimées par un nombre entre $0$
et $1$ --- de rouge, de vert et de bleu. Dans ce cas, la syntaxe de
\cmd{\definecolor} est
\begin{lstlisting}
\definecolor`\marg{nom}'{rgb}`\marg{valeur\_r,valeur\_v,valeur\_b}
\end{lstlisting}
où \meta{valeur\_r}, \meta{valeur\_v} et \meta{valeur\_b} sont
respectivement les teintes de rouge, de vert et de bleu.

\begin{exemple}
  La commande
\begin{lstlisting}
\definecolor{acier}{rgb}{0.3,0.4,0.6}
\end{lstlisting}
  définit une nouvelle couleur nommée \code{acier} composée de rouge
  30~\%, de vert 40~\% et de bleu 60~\%: %
  \fcolorbox{black}[rgb]{0.3,0,0}{\phantom{xx}} $+$ %
  \fcolorbox{black}[rgb]{0,0.4,0}{\phantom{xx}} $+$ %
  \fcolorbox{black}[rgb]{0,0,0.6}{\phantom{xx}} $=$ %
  \fcolorbox{black}[rgb]{0.3,0.4,0.6}{\phantom{xx}}. %
  Une fois la couleur  \code{acier} définie dans le préambule, elle
  devient disponible pour être utilisée dans les commandes
  \cmdprint{\color} et \cmdprint{\textcolor}. %
  \qed
\end{exemple}

La commande \cmd{\colorlet}, dont la syntaxe simplifiée est
\begin{lstlisting}
\colorlet`\marg{nom}\marg{couleur}'
\end{lstlisting}
permet de faire référence à la \meta{couleur} déjà existante par
\meta{nom}. C'est pratique pour assigner un nom sémantique à une
couleur.



\section{Hyperliens et métadonnées de documents PDF}
\label{sec:trucs:hyperliens}

Nous en avons déjà traité à quelques reprises, notamment à la
\autoref{sec:organisation:renvois}: le paquetage \pkg{hyperref}
\citep{hyperref} permet de transformer les références dans le
texte en hyperliens cliquables lorsque le document est produit avec
pdf{\LaTeX} ou {\XeLaTeX}. C'est très pratique lors de la consultation
électronique d'un document.

Le paquetage offre une multitudes d'options de configuration; nous
n'en présenterons que quelques unes. On accède aux options de
configuration de \pkg{hyperref} via la commande \cmd{\hypersetup} dans
le préambule. Celle-ci prend en arguments des paires
\code{option=valeur} séparées par des virgules.

Une des principales choses que l'on pourra souhaiter configurer, c'est
le comportement et la couleur des divers types d'hyperliens. On
trouvera ci-dessous les options de configuration pertinentes, leur
valeur (avec en gras la valeur par défaut) ainsi qu'une brève
explication de chacune.

\begin{table}[h]
  \begin{tabularx}{1.0\linewidth}{@{}p{7em}p{6em}X@{}}
    \code{colorlinks} & \code{true}|\code{\textbf{false}} & colorer les
                                                            liens selon
                                                            leur type \\
    \code{linktocpage} & \code{true}|\code{\textbf{false}} & faire du
                                                             numéro de
                                                             page
                                                             plutôt que
                                                             du titre l'hyperlien dans
                                                             la table
                                                             des
                                                             matières \\
    \code{linkcolor} & \meta{couleur} & couleur des liens internes \\
    \code{urlcolor}  & \meta{couleur} & couleur des URL externes \\
    \code{citecolor} & \meta{couleur} & couleur des citations \\
    \code{allcolor}  & \meta{couleur} & couleur pour tous les types d'hyperliens
  \end{tabularx}
\end{table}

Lorsque la valeur admissible d'une option est \code{true} ou
\code{false}, sa seule mention équivaut à \code{true}. La valeur
\meta{couleur} est le nom d'une couleur telle que définie par
\pkg{xcolor} (\autoref{sec:trucs:couleurs}).

Les fichiers PDF peuvent contenir diverses métadonnées sur leur
contenu. Le paquetage \pkg{hyperref} permet d'en définir certaines,
notamment le titre ou l'auteur du document.

\begin{table}[!h]
  \begin{tabularx}{1.0\linewidth}{@{}p{7em}p{6em}X@{}}
    \code{pdftitle}  & texte & titre du document PDF \\
    \code{pdfauthor} & texte & auteur du document PDF
  \end{tabularx}
\end{table}

Consulter la section~3.7 de la %
\doc{hyperref}{http://texdoc.net/pkg/hyperref} %
de \pkg{hyperref} pour obtenir la liste complète des options de
configuration des métadonnées.

\begin{exemple}
  \label{ex:trucs:couleurs}
  Le présent document fait appel aux définitions de couleurs
  et aux options de configuration de \pkg{hyperref} suivantes:
\begin{lstlisting}
\definecolor{link}{rgb}{0,0.4,0.6}   % ~RoyalBlue de dvips
\definecolor{url}{rgb}{0.6,0,0}      % rouge foncé
\definecolor{citation}{rgb}{0,0.5,0} % vert foncé
\hypersetup{%
  pdfauthor = {Vincent Goulet},
  pdftitle = {Rédaction avec LaTeX},
  colorlinks = true,
  linktocpage = true,
  urlcolor = url,
  linkcolor = link,
  citecolor = citation}
\end{lstlisting}
  \qed
\end{exemple}

\begin{exemple}
  Les gabarits de thèses et de mémoires livrés avec la classe
  \class{ulthese} contiennent dans le préambule la ligne suivante:
\begin{lstlisting}
\hypersetup{colorlinks,allcolors=ULlinkcolor}
\end{lstlisting}
  Tous les liens de la thèse ou du mémoire seront donc de la même
  couleur, \code{ULlinkcolor} \fcolorbox{black}{ULlinkcolor}{\phantom{xx}},
  une couleur définie par la classe. %
  \qed
\end{exemple}

\begin{conseil}
  L'interaction du paquetage \pkg{hyperref} avec les autres est
  parfois (voire souvent) délicate. Pour cette raison, il est
  fortement recommandé que \pkg{hyperref} soit le tout dernier
  paquetage chargé dans le préambule.
\end{conseil}



\section{Présentation de code informatique}
\label{sec:trucs:listings}

L'environnement standard \Ie{verbatim} de {\LaTeX} permet de présenter
du texte tel qu'il est entré dans le code source du document. C'est un
environnement particulièrement utile pour afficher du code
informatique puisque le texte est composé en police non
proportionnelle et que sa disposition exacte est respectée.

\begin{demo}
  \begin{texample}
\begin{lstlisting}[deletetexcs={int,include}]
\begin{verbatim}
/* Hello World en C */
#include <stdio.h>

int main()
{
    printf("Hello world\n");
    return 0;
}
\end{verbatim}
\end{lstlisting}
    \producing
\begin{verbatim}
/* Hello World en C */
#include <stdio.h>

int main()
{
    printf("Hello world\n");
    return 0;
}
\end{verbatim}
  \end{texample}
\end{demo}

Si un document doit contenir beaucoup de code informatique et que l'on
souhaite exercer un fin contrôle sur sa disposition et sa mise en
forme, il vaut mieux se tourner vers un paquetage spécialisé comme
\pkg{listings} \citep{listings}. La %
\doc{listings}{http://texdoc.net/pkg/listings} %
du paquetage compare ses fonctionnalités à celles de plusieurs autres
paquetages similaires.

Le paquetage \pkg{listings} peut effectuer automatiquement le marquage
des mots-clés de nombreux langages de programmation, ajouter des
numéros de ligne, importer du code de fichiers externes ou même
indexer les mots-clés des extraits de code. À titre d'illustration, en
utilisant l'environnement \Ie{lstlisting} de \pkg{listings} plutôt que
\Ie{verbatim}, l'extrait de code C ci-dessus pourrait se présenter
ainsi:
\begin{demo}
  \begin{texample}
    \begin{vglisting}
\begin{lstlisting}
/* Hello World en C */
#include <stdio.h>

int main()
{
  printf("Hello world\n");
  return 0;
}
\end{lstlisting}
    \end{vglisting}
    \producing
\begin{lstlisting}[language=C,%
  deletetexcs={int,include},
  frame=single,%
  backgroundcolor=\color{Azure2},%
  rulecolor=\color{black},%
  numbers=left,%
  numberstyle=\tiny\sffamily,%
  stringstyle=\color{orange}\itshape,%
  identifierstyle=\color{cyan}\mdseries,%
  xleftmargin=12pt,%
  keywordstyle=\color{blue}\bfseries,
  index=]
/* Hello World en C */
#include <stdio.h>

int main()
{
  printf("Hello world\n");
  return 0;
}
\end{lstlisting}
  \end{texample}
\end{demo}

Il serait trop long et nettement hors de la portée du présent ouvrage
d'expliquer les nombreuses fonctionnalités de \pkg{listings}.
Précisons simplement que le paquetage a servi pour en composer les
extraits de code et pour construire une grande partie de
l'index.

\begin{exemple}
  \label{ex:trucs:listings}
  Pour parvenir à la présentation des extraits de code source {\LaTeX}
  de ce document, le paquetage \pkg{listings} est configuré dans le
  préambule de la manière suivante:
\begin{lstlisting}
%% Couleurs
\definecolor{comments}{rgb}{0.7,0,0}

%% Configuration de listings
\lstset{language = [LaTeX]TeX,
  basicstyle = \ttfamily\NoAutoSpacing,
  keywordstyle = \mdseries,
  commentstyle = \color{comments}\slshape,
  extendedchars = true,
  showstringspaces = false,
  backgroundcolor = \color{LightYellow1},
  frame = lr,
  rulecolor = \color{LightYellow1},
  xleftmargin = 3.4pt,
  xrightmargin = 3.4pt}
\end{lstlisting}
  (La couleur \code{LightYellow1} est définie par \pkg{xcolor} lorsque
  le paquetage est chargé avec l'option \code{x11names}.)
  \qed
\end{exemple}



\section{Production de rapports avec l'analyse intégrée}
\label{sec:trucs:sweave}

Les publications scientifiques reposent souvent sur une forme ou une
autre d'analyse numérique ou statistique, la production de code
informatique, une simulation stochastique, etc. La portion
développement et analyse est alors produite avec un certain outil
et la publication, avec un outil d'édition séparé --- {\LaTeX} dans le cas
qui nous occupe. Or, tous les auteurs ont vécu cette situation: les
résultats de l'analyse changent et il faut modifier le rapport en
conséquence, refaire les tableaux et les graphiques, retracer cette
valeur isolée au fil du texte directement tirée de l'analyse\dots\
Seule la quantité de temps perdu rivalise avec le risque d'erreur.

Il existe pourtant une meilleure façon de travailler.

Cette meilleure façon de faire, tirée du concept de
\emph{programmation lettrée}, consiste à combiner dans un seul et même
document l'analyse et le rapport, puis de produire automatiquement une
partie du second à partir de la première.

Les utilisateurs du système statistique R bénéficient d'une mise en
œuvre simple et élégante du concept ci-dessus avec l'outil Sweave
\citep{Sweave}. Un fichier Sweave est à la base un document {\LaTeX}
dans lequel on a inséré du code R à l'intérieur de balises spéciales
\verb|<<>>=| et \verb|@| tirée de la syntaxe \code{noweb}
\citep{noweb}, comme ceci:
\begin{lstlisting}
\section{Commandes R}

L'utilisateur de R interagit avec l'interprète en entrant
des commandes à l'invite de commande:
<<>>=
2 + 3
pi
cos(pi/4)
@
La commande \verb=exp(1)= donne \Sexpr{exp(1)},
la valeur du nombre $e$.
\end{lstlisting}

Par convention, on enregistre un tel document sous un nom se terminant
par l'extension \code{.Rnw}. Sa compilation s'effectue
en deux étapes:
\begin{enumerate}
\item Le fichier \code{.Rnw} est passé à la commande \code{Sweave()}
  de R. Celle-ci retrace les extraits de code et les remplace par des
  environnements {\LaTeX} contenant, par défaut, les expressions R et
  leur résultat dans des environnements \Ie{Sinput} et \Ie{Soutput}.
  Elle évalue également les expressions R se trouvant dans les
  commandes \cmd{\Sexpr} pour les remplacer par leur résultat. Cela
  produit un fichier \code{.tex}:
\begin{lstlisting}
\section{Commandes R}

L'utilisateur de R interagit avec l'interprète en entrant
des commandes à l'invite de commande:
\begin{Schunk}
\begin{Sinput}
> 2 + 3
\end{Sinput}
\begin{Soutput}
[1] 5
\end{Soutput}
\begin{Sinput}
> pi
\end{Sinput}
\begin{Soutput}
[1] 3.141593
\end{Soutput}
\begin{Sinput}
> cos(pi/4)
\end{Sinput}
\begin{Soutput}
[1] 0.7071068
\end{Soutput}
\end{Schunk}
La commande \verb=exp(1)= donne 2.71828182845905,
la valeur du nombre $e$.
\end{lstlisting}
\item On compile le fichier \code{.tex} comme d'habitude.
\end{enumerate}

Sweave se révèle particulièrement utile pour créer des graphiques à
partir de R: tout ce que l'on doit conserver dans son fichier
\code{.Rnw}, c'est le code pour créer le graphique. Il est également
possible de contrôler l'exécution des blocs de code et l'affichage du
code source et des résultats par le biais d'options placées à
l'intérieur de la balise d'ouverture \verb|<<>>=|. Consulter la %
\link{http://leisch.userweb.mwn.de/Sweave/Sweave-manual.pdf}{documentation} %
de Sweave pour les détails.

\link{http://mpastell.com/pweave/}{Pweave} est un système similaire à
Sweave pour le langage Python.

Inspiré de Sweave, knitr \citep{knitr} permet également d'entrelacer
du code {\LaTeX} et du code R. Cet outil offre plus d'options de
traitement que Sweave, mais au prix d'une complexité accrue.

En terminant, rappelons que Sweave n'est qu'un exemple de système de
programmation lettrée. Il en existe plusieurs autres. Il appartient
dès lors au lecteur de trouver le système qui correspond le mieux à
ses besoins.

\begin{information}
  On doit le concept de programmation lettrée au créateur de {\TeX},
  Donald Knuth.  En fait, tout le code source de {\TeX} est écrit
  en programmation lettrée! La %
  \link{https://fr.wikipedia.org/wiki/Programmation_lettrée}{page Wikipedia} %
  consacrée au sujet offre un très bon survol de l'historique et de la
  nature du concept.

  Le lecteur intéressé pourra consulter, par exemple, le %
  \link{http://www.ctan.org/pkg/ulthese}{code source} %
  de la classe \class{ulthese}. La documentation de la classe, le
  code {\LaTeX}, les gabarits, tout se trouve entrelacé
  dans le seul fichier \code{ulthese.dtx}.
\end{information}


\section{Diapositives}
\label{sec:trucs:diapositives}

Il n'est pas rare qu'une publication scientifique fasse l'objet d'une
présentation dans le cadre d'un colloque ou d'un séminaire. Lorsque le
texte a été rédigé avec {\LaTeX}, il est tout naturel de souhaiter le
réutiliser pour la préparation de diapositives --- surtout si le texte
comporte de nombreuses équations mathématiques qu'il serait
extrêmement long de retranscrire dans un logiciel de présentation
comme PowerPoint.

Fort heureusement, il est tout à fait possible de composer ses
diapositives avec {\LaTeX}. La classe standard \class{slides} produit
des diapositives élégantes, quoique minimalistes, qui conviendront à
l'auteur qui ne recherche rien de plus que du texte noir sur fond
blanc.

Cependant, l'outil devenu le standard \emph{de facto} pour la
production de diapositives est la classe \class{beamer}
\citep{beamer}. Celle-ci compte un grand nombre de thèmes et de
gabarits élaborés, rend très simple l'insertion d'animations d'une
diapositive à l'autre, gère automatiquement la table des matières
et\dots\ produit des diapositives en couleur.

Quelle que soit la classe utilisée, les diapositives produites avec
{\LaTeX} se présentent sous forme de fichier PDF. On les projette avec
une liseuse PDF en mode plein écran.

Nous n'irons pas plus loin sur le sujet dans ce document. Consulter
la %
\doc{beamer}{http://texdoc.net/pkg/beameruserguide} %
de \class{beamer} pour apprendre à utiliser la classe. Produire des
diapositives de grande qualité avec les gabarits fournis avec la
classe se révèle simple et rapide.



\section{Gestion des versions et travail collaboratif}
\label{sec:trucs:cvs}

Plusieurs personnes travaillent sur un même fichier, ou encore, une
seule personne y travaille, mais de plusieurs postes de travail
différents. Quelle est la plus récente version du fichier? Un ajout
fait hier dans le fichier a-t-il été pris en compte par une collègue
aujourd'hui? Une modification apportée au fichier n'est plus
nécessaire; comment retourner en arrière facilement?

Les informaticiens ont résolu ce genre de problèmes il y a des
dizaines d'années avec les systèmes de gestion de versions. Les
systèmes les plus populaires en ce moment sont Subversion
\citep{subversion} et Git \citep{git}.

Bien que développés à l'origine pour la gestion du code source de
logiciels, les systèmes de contrôle de versions conviennent
parfaitement pour les sources {\LaTeX}. L'utilisation d'un tel système
permet de:
\begin{itemize}
\item toujours savoir quelle est la plus récente version d'un fichier;
\item travailler à plusieurs personnes simultanément sur un même
  fichier;
\item revenir aisément à une version antérieure d'un fichier;
\item comparer deux versions d'un fichier pour connaître les
  modifications qui y ont été apportées;
\item gérer automatiquement les éventuels conflits de modification
  d'un fichier;
\item disposer en tout temps d'une copie de secours de son travail
  lorsque l'on a recours à un serveur central (obligatoire avec
  Subversion; optionnel avec Git\footnote{%
    Si ce n'est que pour le volet de sauvegarde, nous recommandons
    d'utiliser avec Git un dépôt central tel que
    \link{https://github.com}{GitHub}, un serveur public qui héberge
    déjà des millions de projets.}.)
\end{itemize}
Un système de gestion de versions est un outil qui permet d'augmenter
considérablement sa productivité ou celle de son équipe de
travail au moment de rédiger un ouvrage scientifique.

La %
\link{https://fr.wikipedia.org/wiki/Gestion_de_versions}{page Wikipedia} %
sur le sujet offre une bonne introduction aux systèmes de gestion de
versions. Les membres de la communauté de l'Université Laval qui
souhaitent mettre sur pied un dépôt pour un projet pourront
consulter leur équipe de soutien informatique facultaire.

\begin{information}
  La présente documentation est conservée dans un dépôt Subversion
  public d'où l'on peut toujours obtenir la plus récente version.
  Consulter la page des notices de copyright au début du document pour
  accéder au dépôt.
\end{information}



%%%
%%% Exercices
%%%

\section{Exercices}
\label{sec:trucs:exercices}

\Opensolutionfile{solutions}[solutions-trucs+astuces]

\begin{Filesave}{solutions}
\section*{Chapitre \ref*{chap:trucs}}
\addcontentsline{toc}{section}{Chapitre \protect\ref*{chap:trucs}}

\end{Filesave}

\noindent%
Pour les exercices
\nolink{\ref{exercice:trucs:1}}--\nolink{\ref{exercice:trucs:n}},
utiliser le fichier \fichier{exercice\_trucs.tex}. Celui-ci reprend
une partie de la documentation de la classe \class{ulthese}. De plus,
on remarquera que le document:
\begin{itemize}
\item doit être compilé avec pdf{\LaTeX} puisqu'il charge les
  paquetages \pkg{inputenc} et \pkg{fontenc};
\item définit des nouvelles commandes \cmdprint{\class} et
  \cmdprint{\fichier} pour composer, respectivement, les noms de
  classe et les noms de fichier;
\item utilise la commande \cmdprint{\doc} de
  l'\autoref{ex:commandes:doc};
\item charge le paquetage \pkg{hyperref}, ce qui transforme les titres
  de la table des matières, les renvois aux notes de bas de page et
  les liens externes en hyperliens.
\end{itemize}
\medskip

\begin{exercice}
  \label{exercice:trucs:1}
  Compiler et visualiser le fichier sans aucune modification. Le texte
  est composé dans la police par défaut Computer Modern.

  Ensuite, modifier le préambule du document pour composer le document
  dans la police Palatino, tel qu'expliqué à
  l'\autoref{ex:trucs:palatino}. Charger également le paquetage
  \pkg{helvet} afin d'utiliser Helvetica pour le texte en police sans
  empattements (\cmdprint{\textsf}).
  \begin{sol}
    On doit ajouter dans le préambule les commandes
\begin{lstlisting}
\usepackage{mathpazo}
\usepackage{helvet}
\end{lstlisting}
    La police Helvetica est très grande. Tel que mentionné dans la %
    \doc{psnfss2e}{http://texdoc.net/pkg/psnfss/} %
    de PSNFSS (section~4), il est généralement préférable de charger
    le paquetage \pkg{helvet} avec, par exemple,
\begin{lstlisting}
\usepackage[scaled=0.92]{helvet}
\end{lstlisting}
    pour que le texte en Helvetica se marie mieux à celui dans une
    autre police.
  \end{sol}
\end{exercice}

\begin{exercice}
  Configurer le paquetage \pkg{hyperref} pour que les hyperliens dans
  la table des matières soient ancrés aux numéros de page plutôt
  qu'aux titres de section.
  \begin{sol}
    Tel qu'expliqué à la \autoref{sec:trucs:hyperliens}, on doit
    insérer dans le préambule du document la commande
\begin{lstlisting}
\hypersetup{linktocpage=true}
\end{lstlisting}
    ou, plus simplement,
\begin{lstlisting}
\hypersetup{linktocpage}
\end{lstlisting}
  \end{sol}
\end{exercice}

\begin{exercice}
  Charger le paquetage \pkg{xcolor} et ajouter l'option
  \code{colorlinks} à \pkg{hyperref}, puis recompiler le document.
  Observer les changements.
  \begin{sol}
    En conservant l'ajout de l'exercice précédent, on a dans le
    préambule la commande
\begin{lstlisting}
\hypersetup{colorlinks, linktocpage}
\end{lstlisting}
    Les hyperliens se présentent maintenant en couleur selon les
    paramètres par défaut de \pkg{hyperref}.
  \end{sol}
\end{exercice}

\begin{exercice}
  En s'inspirant de l'\autoref{ex:trucs:couleurs}, modifier la couleur
  des liens internes et externes.
  \begin{sol}
    On peut soit utiliser des couleurs prédéfinies de \pkg{xcolor}
    (\autoref{tab:trucs:couleurs}), soit en  définir de nouvelles avec
    \cmd{\definecolor}. On fait ensuite appel aux couleurs choisies
    pour les options \code{linkcolor} (liens internes) et
    \code{urlcolor} (liens externes) de \pkg{hyperref}.

    Exemple utilisant des couleurs prédéfinies:
\begin{lstlisting}
\hypersetup{colorlinks, linktocpage,
  linkcolor=brown, urlcolor=blue}
\end{lstlisting}

    Exemple avec de nouvelles couleurs:
\begin{lstlisting}
\definecolor{link}{rgb}{0,0.4,0.6}
\definecolor{url}{rgb}{0.6,0,0}
\hypersetup{colorlinks, linktocpage,
  urlcolor=url, linkcolor=link}
\end{lstlisting}
  \end{sol}
\end{exercice}

\begin{exercice}
  \label{exercice:trucs:n}
  Charger le paquetage \pkg{listings} et modifier l'environnement
  \Pe{verbatim} que l'on trouve dans le document pour un environnement
  \Pe{lstlisting}. En s'inspirant de l'\autoref{ex:trucs:listings},
  configurer la présentation des extraits de code pour utiliser une
  police non proportionnelle (\cmdprint{\ttfamily}) et un arrière-plan
  de la couleur standard \code{lightgray}.
  \begin{sol}
    S'il y avait plusieurs extraits de code dans le document, mieux
    vaudrait les configurer tous à l'identique dans le préambule du
    document avec
\begin{lstlisting}
\lstset{basicstyle=\ttfamily,
  backgroundcolor=\color{lightgray}}
\end{lstlisting}
    Ensuite,
\begin{vglisting}
\begin{lstlisting}
latex ulthese.ins
\end{lstlisting}
\end{vglisting}
    donne le résultat demandé.

    Pour un seul extrait, il est également possible de simplement
    charger le paquetage dans le préambule et d'effectuer la
    configuration à l'ouverture de l'environnement, comme ceci:
\begin{vglisting}
\begin{lstlisting}[basicstyle=\ttfamily,
  backgroundcolor=\color{lightgray}]
latex ulthese.ins
\end{lstlisting}
\end{vglisting}
  \end{sol}
\end{exercice}

\Closesolutionfile{solutions}


%%% Local Variables:
%%% mode: latex
%%% TeX-engine: xetex
%%% TeX-master: "formation-latex-ul"
%%% coding: utf-8
%%% End:


\appendix
version https://git-lfs.github.com/spec/v1
oid sha256:1ae860479f70bc51be429ddc5cb7a2192ed0d0bd7a682f6d702ab12b30b60fb6
size 16486
               % éléments spécifiques à ulthese
\chapter{Solutions des exercices}
\label{chap:solutions}

\input{solutions-bases}
% \input{solutions-organisation}
% \input{solutions-apparence}
\input{solutions-boites}
\input{solutions-tableaux+figures}
\input{solutions-mathematiques}
\input{solutions-bibliographie}
\input{solutions-commandes}
\input{solutions-trucs+astuces}

%%% Local Variables:
%%% mode: latex
%%% TeX-master: "formation-latex-ul"
%%% coding: utf-8
%%% End:


\bibliography{formation-latex-ul}

\cleardoublepage
\printindex

\cleartoverso

version https://git-lfs.github.com/spec/v1
oid sha256:bdccf2263ba9aff9456fb8d468ba75a11889349a6fe4baa14784e4639ff7437d
size 19420


\cleartoverso

%% Page couverture arrière.
\pagestyle{empty}
version https://git-lfs.github.com/spec/v1
oid sha256:dced4a1be889087e17b0107742eea041b4bad7428125899b98aad27534a30700
size 833


\end{document}

%%% Local Variables:
%%% mode: latex
%%% TeX-engine: xetex
%%% TeX-master: t
%%% coding: utf-8
%%% End:
