\chapter{Classe pour les thèses et mémoires de
  l'Université Laval}
\label{chap:ulthese}

La classe \class{ulthese} \citep{ulthese} permet de composer des
thèses et mémoires immédiatement conformes aux règles générales de
présentation matérielle de la Faculté des études supérieures et
postdoctorales de l'Université Laval. Ces règles définissent
principalement la présentation des pages de titre des thèses et
mémoires ainsi que la disposition du texte sur la page. La classe est
compatible tant avec {\LaTeX} qu'avec {\XeLaTeX}.

La classe \class{ulthese} est basée sur la classe \class{memoir}.
L'intégralité des fonctionnalités de \class{memoir} se retrouve donc
disponible dans \class{ulthese}. Les deux classes sont livrées avec la
distribution {\TeX}~Live.

Outre \class{memoir}, la classe charge par défaut un certain nombre de
paquetages, notamment les essentiels pour la rédaction en français. Il
n'est donc pas nécessaire de charger de nouveau les paquetages
suivants: \pkg{babel}, \pkg{numprint}, \pkg{natbib} \pkg{fontspec}
(moteur {\XeLaTeX} seulement), \pkg{graphicx}, \pkg{xcolor},
\pkg{textcomp}. Le paquetage \pkg{geometry} est incompatible avec la
classe à cause de sa mauvaise interaction avec \class{memoir}. Son
chargement dans le préambule du document cause une erreur lors de la
compilation.

Afin de faciliter la rédaction, la classe est livrée avec un ensemble
de gabarits sur lesquels se baser pour:
\begin{itemize}
\item les fichiers maîtres de divers types de thèses et mémoires
  (standard, sur mesure, en cotutelle, en bidiplomation, en
  extension, etc.);
\item les fichiers des parties les plus usuelles (résumés français
  et anglais, avant-propos, introduction, chapitres, conclusion,
  etc.).
\end{itemize}
Ces gabarits comportent des commentaires succincts pour guider
l'utilisateur dans la préparation de son document. La
\doc{ulthese}{http://texdoc.net/pkg/ulthese} %
de la classe explique en détail le contenu des gabarits.


%%% Local Variables:
%%% mode: latex
%%% TeX-engine: xetex
%%% TeX-master: "formation-latex-ul"
%%% coding: utf-8
%%% End:
