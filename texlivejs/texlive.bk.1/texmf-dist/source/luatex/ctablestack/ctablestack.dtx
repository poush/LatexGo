% \iffalse meta-comment
%
% Copyright (C) 2015 David Carlisle and Joseph Wright
%
% It may be distributed and/or modified under the conditions of
% the LaTeX Project Public License (LPPL), either version 1.3c of
% this license or (at your option) any later version.  The latest
% version of this license is in the file:
%
%   http://www.latex-project.org/lppl.txt
%
%<package>\ifx
%<package>  \ProvidesPackage\undefined\begingroup\def\ProvidesPackage
%<package>    #1#2[#3]{\endgroup\immediate\write-1{Package: #1 #3}}
%<package>\fi
%<package>\ProvidesPackage{ctablestack}
%<*driver>
\ProvidesFile{ctablestack.dtx}
%</driver>
%<*package>
  [2015/10/01 v1.0 Catcode table stable support]
%</package>
%<*driver>
\documentclass{ltxdoc}
\GetFileInfo{ctablestack.dtx}
\begin{document}
\title{\filename\\Catcode table stable support}
\author{David Carlisle and Joseph Wright}
\date{\filedate}
\maketitle
\setcounter{tocdepth}{2}
\tableofcontents
\DocInput{\filename}
\end{document}
%</driver>
% \fi
%
% \section{Overview}
%
% This small package adds support for a stack of category code tables to
% the core support for Lua\TeX{} provided by the \LaTeX{} kernel and
% available for plain users as |ltluatex.tex|. As such, the
% code here may be used with both plain \TeX{} and \LaTeX{}, and requires
% either an up-to-date \LaTeX{} kernel (2016 onward), use of \textsf{latexrelease}
% with older kernels or loading |ltluatex.tex| for plain users.
%
% The commands here are aimed mainly for use by package authors to develop
% environments needing specific catcode regimes. As such the package does
% not define any user level commands.
%
% \noindent
% \DescribeMacro{\@setrangecatcode}
% |\@setrangecatcode{|\meta{start}|}{|\meta{end}|}{|\meta{catcode}|}|\\
% Sets all characters in the range \meta{start}--\meta{end} inclusive to
% have the \meta{catcode} specified.
%
% \noindent
% \DescribeMacro{\@pushcatcodetable}
% \DescribeMacro{\@popcatcodetable}
% |\@pushcatcodetable|\\
% |\@popcatcodetable|\\
% This pair of commands enable the current category code r\'{e}gime to
% be saved and restored meaning that arbitrary catcode changes can be made.
% This functionality will normally be used in concert with applying
% catcode tables. For example
% \begin{verbatim}
% \catcode`\Z=4 %
% \@pushcatcodetable
% \catcodetable\catcodetable@latex
% % Code here
% \@popcatcodetable
% \showthe\catcode`\Z
% \end{verbatim}
% will ensure that the `content' is set with normal category codes but
% allow restoration of the non-standard codes at the conclusion. Importantly,
% it does not require that anything is known about the catcode situation in
% advance (\emph{cf.}~a more traditional approach to saving the state of
% targeting characters).
%
% \StopEventually{}
%
% \section{Implementation}
%
%    \begin{macrocode}
%<*package>
%    \end{macrocode}
%
%    \begin{macrocode}
\edef\ctstackatcatcode{\the\catcode`\@}
\catcode`\@=11
%    \end{macrocode}
%
% Check for \textsf{ltluatex} functionality using \cs{newluafunction} as a marker.
%    \begin{macrocode}
\ifx\newluafunction\@undefined
  version https://git-lfs.github.com/spec/v1
oid sha256:ce5da7a9f4fc9424af06d1269ef3ee551bb7260a5c0f4635e1bae23771b50624
size 58674
%
\fi
%    \end{macrocode}
%
% \begin{macro}{\@setcatcodetable}
% Save a catcode table specified in |#1| using the catcode settings specified in |#2|.
% These settings are executed in a local group to avoid affecting surrounding code.
% (Saving a catcode table is always a global operation.)
%    \begin{macrocode}1
\def\@setcatcodetable#1#2{%
  \begingroup
    #2%
    \savecatcodetable#1%
  \endgroup
}
%    \end{macrocode}
% \end{macro}
%
% \begin{macro}{\@setrangecatcode}
%  Set a range of characters from |#1| to |#2| inclusive to the catcode specified in |#3|.
%    \begin{macrocode}
\def\@setrangecatcode#1#2#3{%
  \ifnum#1>#2 %
    \expandafter\@gobble
  \else
    \expandafter\@firstofone
  \fi
    {%
      \catcode#1=#3 %
      \expandafter\@setrangecatcode\expandafter
        {\number\numexpr#1+1\relax}{#2}{#3}%
    }%
}
%    \end{macrocode}
% \end{macro}
%
% \begin{macro}{\@catcodetablelist}
% \begin{macro}{\@catcodetablestack}
%   Data structures for a stack: a list of free tables in the stack and
%   the stack record itself.
%    \begin{macrocode}
\def\@catcodetablelist{}
\def\@catcodetablestack{}
%    \end{macrocode}
% \end{macro}
% \end{macro}
%
% \begin{macro}{\@catcodetablestackcnt}
%   A count for adding to the list of scratch tables.
%    \begin{macrocode}
\newcount\@catcodetablestackcnt
%    \end{macrocode}
% \end{macro}
%
% \begin{macro}{\@pushcatcodetable}
% \begin{macro}{\@pushctbl}
%   To push a table, first check there is a free one in the pool and if
%   not create one. Then take the top table in the pool and use it to save
%   the current table.
%    \begin{macrocode}
\def\@pushcatcodetable{%
  \ifx\@catcodetablelist\empty
    \global\advance\@catcodetablestackcnt by\@ne
    \edef\@tempa{\csname ct@\the\@catcodetablestackcnt\endcsname}%
    \expandafter\newcatcodetable\@tempa
    \xdef\@catcodetablelist{\@tempa}%
  \fi
  \expandafter\@pushctbl\@catcodetablelist\@nil
}
%    \end{macrocode}
%
%    \begin{macrocode}
\def\@pushctbl#1#2\@nil{%
  \gdef\@catcodetablelist{#2}%
  \xdef\@catcodetablestack{#1\@catcodetablestack}%
  \savecatcodetable#1%
}
%    \end{macrocode}
% \end{macro}
% \end{macro}
%
% \begin{macro}{\@popcatcodetable}
% \begin{macro}{\@popctbl}
%   Much the same in reverse.
%    \begin{macrocode}
\def\@popcatcodetable{%
  \if!\@catcodetablestack!%
    \errmessage{Attempt to pop empty catcodetable stack}%
  \else
    \expandafter\@popctbl\@catcodetablestack\@nil
  \fi
}
%    \end{macrocode}
%
%    \begin{macrocode}
\def\@popctbl#1#2\@nil{%
  \gdef\@catcodetablestack{#2}%
  \xdef\@catcodetablelist{\@catcodetablelist#1}%
  \catcodetable#1%
}
%    \end{macrocode}
% \end{macro}
% \end{macro}
%
%    \begin{macrocode}
\catcode`\@\ctstackatcatcode\relax
%    \end{macrocode}
%    \begin{macrocode}
%</package>
%    \end{macrocode}
%
% \Finale
