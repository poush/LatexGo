% \iffalse meta-comment
%
% Copyright 1993-2018
% The LaTeX3 Project and any individual authors listed elsewhere
% in this file.
%
% This file is part of the LaTeX base system.
% -------------------------------------------
%
% It may be distributed and/or modified under the
% conditions of the LaTeX Project Public License, either version 1.3c
% of this license or (at your option) any later version.
% The latest version of this license is in
%    https://www.latex-project.org/lppl.txt
% and version 1.3c or later is part of all distributions of LaTeX
% version 2005/12/01 or later.
%
% This file has the LPPL maintenance status "maintained".
%
% The list of all files belonging to the LaTeX base distribution is
% given in the file `manifest.txt'. See also `legal.txt' for additional
% information.
%
% The list of derived (unpacked) files belonging to the distribution
% and covered by LPPL is defined by the unpacking scripts (with
% extension .ins) which are part of the distribution.
%
% \fi
%
% \iffalse
%%% From File: ltspace.dtx
%<*driver>
% \fi
\ProvidesFile{ltspace.dtx}
             [2018/10/10 v1.3i LaTeX Kernel (spacing)]
% \iffalse
\documentclass{ltxdoc}
\GetFileInfo{ltspace.dtx}
\title{\filename}
\date{\filedate}
 \author{%
  Johannes Braams\and
  David Carlisle\and
  Alan Jeffrey\and
  Leslie Lamport\and
  Frank Mittelbach\and
  Chris Rowley\and
  Rainer Sch\"opf}
\begin{document}
 \MaintainedByLaTeXTeam{latex}
 \maketitle
 \DocInput{\filename}
\end{document}
%</driver>
% \fi
%
%
% \changes{v1.1a}{1994/05/16}{(ASAJ) Split from ltinit.dtx.}
% \changes{v1.2f}{1995/05/25}{Macros moved to ltlists.dtx}
% \changes{v1.2n}{1996/04/22}{Documentation Improvements}
% \changes{v1.2o}{1996/06/22}{Documentation of problems added}
% \changes{v1.2q}{1996/07/27}{Further documentation of problems}
% \changes{v1.2r}{1996/07/27}{Correct documentation of problems}
% \changes{v1.2w}{1998/08/17}{Documentation fixes.}
%
%
% \section{Spacing}
%
% This section deals with spacing, and line- and page-breaking.
%
% \subsection{User Commands}
%
% \DescribeMacro\nopagebreak\oarg{i} : \meta{i} = 0,...,4.
%
% Default argument = 4.  Puts a penalty into the vertical list
% output as follows:\\
%                   0 : penalty = 0\\
%                   1 : penalty = |\@lowpenalty|\\
%                   2 : penalty = |\@medpenalty|\\
%                   3 : penalty = |\@highpenalty|\\
%                   4 : penalty = 10000
%
% \DescribeMacro\pagebreak\oarg{i} :
%          same as \nopagebreak except negatives of its penalty
%
% \DescribeMacro\linebreak\oarg{i} : analog of the above
%
% \DescribeMacro\nolinebreak\oarg{i} : analog of the above
%
% \DescribeMacro\samepage :
% inhibits page breaking most places by setting the
% following penalties to 10000:\\
%                    |\interlinepenalty|\\
%                    |\postdisplaypenalty|\\
%                    |\interdisplaylinepenalty|\\
%                    |\@beginparpenalty|\\
%                    |\@endparpenalty|\\
%                    |\@itempenalty|\\
%                    |\@secpenalty|\\
%                    |\interfootnotelinepenalty|
%
% \DescribeMacro{\\}         : initially defined to be |\newline|
%
% |\\|\oarg{length} : initially defined to be
%              |\vspace|\marg{length}|\newline|\\
%              Note: |\\*| adds a |\vadjust{\penalty 10000}|
%
% OBSOLETE COMMANDS (which never made it into the manual):
%
% |\obeycr|    : defines <CR> == |\\\relax|\\
% |\restorecr| : restores <CR> to its usual meaning.
%
%
% \StopEventually{}
%
% \subsection{Chris' comments}
%
% There are several aspects of the handling of space in horizontal
% mode that are inconsistent or do not work well in some cases.
% These are largely concerned with ignoring the effect of space
% tokens that would otherwise typeset an inter-word space.
%
% Negating the effect of such space tokens is achieved by two
% mechanisms:
% \begin{itemize}
% \item |\unskip| is used to remove the glue just added by
% a space that has already had its effect; it is sometimes
% invoked after an |\ifdim| test on |\lastskip| (see below);
% \item |\ignorespaces| is used to ignore space-tokens yet to come.
% \end{itemize}
%
% The test done on |\lastskip| is sometimes for equality with zero and
% sometimes for being positive.  Recall also that the test is only on
% the natural length of the glue and that no glue cannot be
% distinguished from glue whose natural length is zero: to summarise,
% a pretty awful test.  It is not clear why these tests are not all
% the same; I think that they should all be for equality.  One place
% where |\unskip| is often used is just before a |\par| (which itself
% internally does an |\unskip|) and one bit of code (in |\@item|) even
% has two |\unskips| before a |\par|.  These uses may be fossil code
% but if they are necessary, maybe |\@killglue| would be even safer.
%
% Such removal of glue by |\unskip| may sometimes have the wrong result,
% removing not the glue from a space-token but other explicit glue;
% this is sometimes not what is intended.
%
% A common way to prevent such removal is to add an |\hskip\z@| after
% the glue that should not be removed.  This protects that glue
% against one |\unskip| with no test but not against more than one.
% It does work for `tested |\unskip|s'.   This is used
% by |\hspace*| but not by |\hspace|; this is inconsistent as the star
% is supposed to prevent removal only at the beginning of a line, not
% at the end, or in a tabular, etc.
%
% If this reason for removing glue were the only consideration then a
% tested-|\unskip| and protection by |\hskip\z@| would suffice but
% would need to be consistently implemented.
%
% However, the class of invisibles, commands and environments tries to
% be even cleverer: one of these tries to leave only one inter-word
% space whenever there is one before it and one after it; and it does
% this quite well.
%
% But problems can arise when there is not a space-token on
% both sides of it; in particular, when an invisible appears at the
% beginning or end of a piece of text the method still leaves one space
% token whereas usually in these cases it should leave none.
%
% Also, the current rules do not work well when more than one such
% command appears consecutively, separated by space-tokens; it leaves
% glue between every other invisible.
%
% There is also a question about what these commands should do when
% they occur next to spaces that do not come from space tokens but,
% for example, from |\hspace|.  Should they still produce `just one
% space'?  If so, which one?  It is good to note that the manual
% is sufficiently cautious about invisibles that we are not obliged to
% make anything work.
%
% Another interesting side-road to explore is whether the space-tokens
% either side of an |\hspace{...}| should be ignored.
%
% One alternative to the current algorithm that is often suggested is
% that all glue around the invisible  should be consolidated into a
% space after it (usually without stating how much glue should be put
% there).  The command |\nolinebreak| is implemented this way (and
% |\linebreak| should also be). This does not work correctly for the
% following common case:
% \begin{verbatim}
%   ... some text
%   \index{some-word}
%   some-word and more text.
% \end{verbatim}
% This is optimal coding since it is normal to index a word that gets
% split across a page-break on its starting page.  This would, on the
% other hand, fix another common (and documented) failure of the
% current system: when the invisible is the last thing in a paragraph
% the space before it is not removed and, worse, it is also hidden
% from the paragraph-ending mechanism so that an `empty' line can be
% created at the end of the paragraph.
%
% Another deficiency (I think) of the current system is that the
% following is treated as having the |\index| command between the
% paragraphs, which is probably not what the author intended (since
% there is no empty line after it).
% \begin{verbatim}
%
%   \index{beginnings}
%   Beginnings of paragraphs ...
% \end{verbatim}
%
% I know of no algorithm that will handle satisfactorily even
% all the most common cases; note that it could be that the best
% algorithm may be different for different invisibles since,
% for example, the common uses and expected behaviour of
% |\index|, |\marginpar|, |\linebreak|, |\pagebreak| and
% |\vspace| are somewhat different.  [For example, is
% |\vspace| ever used in the middle of a paragraph?]
%
% One method that can (and is) used to make invisible commands produce
% no space when used at the beginning of text is to put in some glue
% that is nearly enough the same as no glue or glue of zero length in
% all respects except for the precise test for not being exactly equal
% to zero; examples of such glue are |\hskip 1sp| and, possibly better
% but more complex, |\hskip -1sp \hskip 1sp|.  However, this only works
% when it is known that user-supplied text is about to start.
%
% Some similar concerns apply to the handling of space and penalties
% in vertical mode; there is an extra hurdle here as |\unskip| does
% not work on the main vertical list.  The complexity of the tests done
% by |\addvspace| have never been explained.
%
% The implementation of space hacks etc for vertical mode is another
% major area that needs further attention; my earlier experiments
% did not produce much improvement over the current unsatisfactory
% situation.
%
% One particular problem is what happens when the following very
% natural coding is used (part of the problem here is that this looks
% like an hmode problem, but it is not):
% \begin{verbatim}
%   ... end of text.
%
%   \begin{enumerate}
%     \item \label{item:xxx} Item text.
%   \end{enumerate}
% \end{verbatim}
%
% \subsection{Some immediate actions}
%
% \begin{itemize}
% \item Fix bug in |\linebreak|.
% \item Fix bug in |\\*|.
% \item Reimplement |\\|, etc, removing extra |\vadjust|s and getting
% better error trapping (this seems to involve a lot more tokens).
% \item Investigate whether |\\|, etc need to be errors in vmode; I
% think that they could be noops (maybe with a warning).
% \item Make all(?) |\unskips| include test for zero skip (rather than
%   other tests or no test).
% \item Consider replacing |\hskip 1sp| by something better (here
%   called an `infinitesimal' skip).
% \item Look at all |\hskip\z@| (or similar) to see if they should be
%   changed to an `infinitesimal' skip.
% \item Resolve the inconsistency between |\hspace| and |\hspace*|.
% \item Remove unnecessary |\unskips|.
% \item Investigate and rationalise the `newline' code.
% \item Find better algorithms for all sorts of things or, easier(?),
%   fix \TeX{} itself.
% \end{itemize}
%
% \subsection{The code}
%
%    \begin{macrocode}
%<*2ekernel>
\message{spacing,}
%    \end{macrocode}
%
%  \begin{macro}{\pagebreak}
%  \begin{macro}{\nopagebreak}
% \changes{v1.2h}{1995/07/05}{Reimplemented both using \cs{@no@pgbk}}
% \changes{v1.2j}{1995/10/16}{(DPC) Use \cs{@testopt} /1911}
%    \begin{macrocode}
\def\pagebreak{\@testopt{\@no@pgbk-}4}
\def\nopagebreak{\@testopt\@no@pgbk4}
%    \end{macrocode}
%  \end{macro}
%  \end{macro}
%
%
%  \begin{macro}{\@no@pgbk}
% \changes{v1.2h}{1995/07/05}{Macro replaces \cs{@pgbk}
%                 and \cs{@nopgbk}}
%    \begin{macrocode}
\def\@no@pgbk #1[#2]{%
  \ifvmode
    \penalty #1\@getpen{#2}%
  \else
    \@bsphack
    \vadjust{\penalty #1\@getpen{#2}}%
    \@esphack
  \fi}
%    \end{macrocode}
%  \end{macro}
%
%  \begin{macro}{\linebreak}
%  \begin{macro}{\nolinebreak}
% \changes{v1.2u}{1996/10/29}{Reimplemented both using \cs{@no@lnbk}}
% \changes{v1.2j}{1995/10/16}{(DPC) Use \cs{@testopt} /1911}
%    \begin{macrocode}
\def\linebreak{\@testopt{\@no@lnbk-}4}
\def\nolinebreak{\@testopt\@no@lnbk4}
%    \end{macrocode}
%  \end{macro}
%  \end{macro}
%  \begin{macro}{\@no@lnbk}
% \changes{v1.2u}{1996/10/29}{Macro replaces \cs{@lnbk}
%                 and \cs{@nolnbk}}
%    \begin{macrocode}
\def\@no@lnbk #1[#2]{%
  \ifvmode
    \@nolnerr
  \else
    \@tempskipa\lastskip
    \unskip
    \penalty #1\@getpen{#2}%
    \ifdim\@tempskipa>\z@
      \hskip\@tempskipa
      \ignorespaces
    \fi
  \fi}
%    \end{macrocode}
%  \end{macro}
%
%  \begin{macro}{\samepage}
%    \begin{macrocode}
\def\samepage{\interlinepenalty\@M
   \postdisplaypenalty\@M
   \interdisplaylinepenalty\@M
   \@beginparpenalty\@M
   \@endparpenalty\@M
   \@itempenalty\@M
   \@secpenalty\@M
   \interfootnotelinepenalty\@M}
%    \end{macrocode}
%  \end{macro}
%
%
%  \begin{macro}{\\}
% \changes{v1.2a}{1994/11/11}{(DPC) Make robust}
% \changes{v1.2d}{1994/11/14}{(DPC) Macro modified}
%
% \changes{v1.2u}{1996/10/29}{Corrected and rationalised code}
% The purpose of the new code is to fix a few bugs; however, it also
% attempts to optimize the following, in order of priority:
%     \begin{enumerate}
%       \item efficient execution of plain |\\|;
%       \item efficient execution of |\\[...]|;
%       \item memory use;
%       \item name-space use.
%     \end{enumerate}
% The changes should make no difference to the typeset output.
% It appears to be safe to use |\reserved@e| and |\reserved@f| here
% (other reserved macros are somewhat disastrous).
%
% These changes made |\newline| even less robust than it had been,
% so now it is explicitly robust, like |\\|.
%  \begin{macro}{\@normalcr}
% The internal definition of the `normal' definition of |\\|.
%    \begin{macrocode}
\DeclareRobustCommand\\{%
  \let \reserved@e \relax
  \let \reserved@f \relax
  \@ifstar{\let \reserved@e \vadjust \let \reserved@f \nobreak
             \@xnewline}%
          \@xnewline}
\expandafter\let\expandafter\@normalcr
     \csname\expandafter\@gobble\string\\ \endcsname
%    \end{macrocode}
%  \end{macro}
%  \end{macro}
%  \begin{macro}{\newline}
% A simple form of the `normal' definition of |\\|.
% \changes{v1.2v}{1997/05/07}{Made completely robust.}
%    \begin{macrocode}
\DeclareRobustCommand\newline{\@normalcr\relax}
%    \end{macrocode}
%  \end{macro}
%
%  \begin{macro}{\@xnewline}
%    \begin{macrocode}
\def\@xnewline{\@ifnextchar[% ] bracket matching
                  \@newline
                 {\@gnewline\relax}}
%    \end{macrocode}
%  \end{macro}
%
%  \begin{macro}{\@newline}
%    \begin{macrocode}
\def\@newline[#1]{\let \reserved@e \vadjust
                   \@gnewline {\vskip #1}}
%    \end{macrocode}
%  \end{macro}
%
%  \begin{macro}{\@gnewline}
% \changes{v1.2u}{1996/10/29}{Added macro}
% The |\nobreak| added to prevent null lines when |\\|
% ends an overfull line.  Change made 24 May 89 as suggested by
% Frank Mittelbach and Rainer Sch\"opf
% \changes{v1.2h}{1995/07/05}{Use \cs{break}}
%    \begin{macrocode}
\def\@gnewline #1{%
  \ifvmode
    \@nolnerr
  \else
    \unskip \reserved@e {\reserved@f#1}\nobreak \hfil \break
  \fi}
%    \end{macrocode}
%  \end{macro}
%
%
% \begin{macro}{\@getpen}
%    \begin{macrocode}
\def\@getpen#1{\ifcase #1 \z@ \or \@lowpenalty\or
         \@medpenalty \or \@highpenalty
         \else \@M \fi}
%    \end{macrocode}
%  \end{macro}
%
%
% \begin{macro}{\if@nobreak}
% \changes{v1.2p}{1996/07/26}{put \cs{global} inside definition}
% Switch used to avoid page breaks caused by |\label| after a
%      section heading, etc. It should be \textbf{GLOBALLY} set true
%      after the  |\nobreak| and \textbf{globally} set false by the
%      next invocation of |\everypar|.
%
%      Commands that reset |\everypar| should globally set it false if
%      appropriate.
%    \begin{macrocode}
\def\@nobreakfalse{\global\let\if@nobreak\iffalse}
\def\@nobreaktrue {\global\let\if@nobreak\iftrue}
\@nobreakfalse
%    \end{macrocode}
% \end{macro}
%
%
%
% \begin{macro}{\@savsk}
% \begin{macro}{\@savsf}
% Registers used to save the space factor and last skip.
%    \begin{macrocode}
\newdimen\@savsk
\newcount\@savsf
%    \end{macrocode}
% \end{macro}
% \end{macro}
%
%
%
%
%  \begin{macro}{\@bsphack}
% \changes{LaTeX2e}{1993/12/08}
%         {Command reimplemented; late birthday present for Chris}
% \changes{LaTeX2e}{1993/12/08}{Command reimplemented}
% \changes{LaTeX2e}{1993/12/16}{Corrected optimisation :-)}
%  |\@bsphack| and |\@esphack|
%  used by macros such as |\index| and
% |\begin{@float}| \ldots |\end{@float}|
%  that want to be invisible --- i.e.,
%  not leave any extra space when used in the middle of text.  Such
%  a macro should begin with |\@bsphack| and end with |\@esphack|
%  The macro in question should not create any text, nor change the
%  mode.
%
% Before giving the current definition we give an extended definition
% that is currently not used (because it doesn't work as advertised:-)
%
%    These are generalised hacks which attempt to do sensible things
%    when `invisible commands' appear in vmode too.
%
%    They need to cope with space in both hmode (plus spacefactor) and
%    vmode, and also cope with breaks etc.  In vmode this means
%    ensuring that any following |\addvspace|, etc sees the correct
%    glue in |\lastskip|.
%
%    In fact, these improved versions should be used for other cases
%    of `whatsits, thingies etc' which should be invisible.  They are
%    only for commands, not environments (see notes on |\@Esphack|).
%
%    BTW, anyone know why the standard hacks are surrounded by
%    |\ifmmode\else| rather than simply |\ifhmode|?
%
%    And are there any cases where saving the spacefactor is
%    essential?  I have some extensions where it is, but it does not
%    appear to be so in the standard uses.
%\begin{verbatim}
%\def \@bsphack{%
%  \relax \ifvmode
%    \@savsk \lastskip
%    \ifdim \lastskip=\z@
%    \else
%      \vskip -\lastskip
%    \fi
%  \else
%    \ifhmode
%      \@savsk \lastskip
%      \@savsf \spacefactor
%    \fi
%  \fi
%}
%\end{verbatim}
%    I think that, in vmode, it is the safest to put
%    in a |\nobreak| immediately after such things since writes,
%    inserts etc followed by glue give valid breakpoints and, in
%    general, it is possible to create breaks but impossible to
%    destroy them.
%\begin{verbatim}
%\def \@esphack{%
%   \relax \ifvmode
%     \nobreak
%     \ifdim \@savsk=\z@
%     \else
%       \vskip\@savsk
%     \fi
%   \else
%     \ifhmode
%       \spacefactor \@savsf
%       \ifdim \@savsk>\z@
%         \ignorespaces
%       \fi
%     \fi
%   \fi
%}
%\end{verbatim}
%    For the moment we are going to ignore the vertical versions until
%    they are correct.
% \changes{LaTeX2e}{1993/12/19}{There seem to be problems with selfmade
%                           birthday presents}
%    \begin{macrocode}
\def\@bsphack{%
  \relax
  \ifhmode
    \@savsk\lastskip
    \@savsf\spacefactor
  \fi}
%    \end{macrocode}
% \end{macro}
%
% \begin{macro}{\@esphack}
%   Companion to |\@bsphack|.  If this command is not properly paired
%   with |\@bsphack| one might end up with a low-level \TeX{} error:
%   ``BAD spacefactor''. One possible cause is calling |\@bsphack| in
%   vertical mode, then doing something that gets you (sometimes) into
%   horizontal mode and finally calling |\@esphack|. Even if no error
%   is generated that is wrong, because |\@esphack| will then use the
%   saved values for |\@savsk| and |\@savsf| from some earlier
%   invocation of |\@bsphack| which will have nothing to do with the
%   current situation.
% \changes{v1.3d}{2015/01/11}{Allow hyphenation (Donald Arseneau pr/3498) (latexrelease)}
%    \begin{macrocode}
%</2ekernel>
%<latexrelease>\IncludeInRelease{2018/10/10}%
%<latexrelease>                 {\@esphack}{hyphenation and nobreak after space hack}%
%<*2ekernel|latexrelease>
\def\@esphack{%
  \relax
  \ifhmode
    \spacefactor\@savsf
    \ifdim\@savsk>\z@
%    \end{macrocode}
% \changes{v1.3f}{2015/11/07}
%         {Only space if there is no space at the end of the hlist latex/4443}
%    \begin{macrocode}
      \ifdim\lastskip=\z@ 
        \nobreak \hskip\z@skip
      \fi
      \ignorespaces
    \fi
%    \end{macrocode}
% \changes{v1.3i}{2018/10/10}
%         {Don't introduce breakpoints if @nobreak is true and after sections}
%    \begin{macrocode}
  \else
    \ifvmode
      \if@nobreak\nobreak\else\if@noskipsec\nobreak\fi\fi
    \fi
%    \end{macrocode}
%
%    \begin{macrocode}
  \fi}%
%</2ekernel|latexrelease>
%<latexrelease>\EndIncludeInRelease
%<latexrelease>\IncludeInRelease{2015/10/01}%
%<latexrelease>                 {\@esphack}{hyphenation and nobreak after space hack}%
%<latexrelease>\def\@esphack{%
%<latexrelease>  \relax
%<latexrelease>  \ifhmode
%<latexrelease>    \spacefactor\@savsf
%<latexrelease>    \ifdim\@savsk>\z@
%<latexrelease>      \ifdim\lastskip=\z@ 
%<latexrelease>        \nobreak \hskip\z@skip
%<latexrelease>      \fi
%<latexrelease>      \ignorespaces
%<latexrelease>    \fi
%<latexrelease>  \fi}%
%<latexrelease>\EndIncludeInRelease
%<latexrelease>\IncludeInRelease{2015/01/01}%
%<latexrelease>                 {\@esphack}{hyphenation and nobreak after space hack}%
%<latexrelease>\def\@esphack{%
%<latexrelease>  \relax
%<latexrelease>  \ifhmode
%<latexrelease>    \spacefactor\@savsf
%<latexrelease>    \ifdim\@savsk>\z@
%<latexrelease>      \nobreak \hskip\z@skip
%<latexrelease>      \ignorespaces
%<latexrelease>    \fi
%<latexrelease>  \fi}%
%<latexrelease>\EndIncludeInRelease
%<latexrelease>\IncludeInRelease{0000/00/00}%
%<latexrelease>                 {\@esphack}{hyphenation and nobreak after space hack}%
%<latexrelease>\def\@esphack{%
%<latexrelease>  \relax
%<latexrelease>  \ifhmode
%<latexrelease>    \spacefactor\@savsf
%<latexrelease>    \ifdim\@savsk>\z@
%<latexrelease>      \ignorespaces
%<latexrelease>    \fi
%<latexrelease>  \fi}%
%<latexrelease>\EndIncludeInRelease
%<*2ekernel>
%    \end{macrocode}
% \end{macro}
%
% \begin{macro}{\@Esphack}
% A variant of |\@esphack| that sets the |@ignore| switch to
%     true (as |\@esphack| used to do previously).
% This is currently used only for floats and similar environments.
% \changes{v1.2s}{1996/08/02}{Remove \cs{global} before \cs{@ignore...}}
% \changes{v1.3d}{2015/01/11}{Allow hyphenation (Donald Arseneau pr/3498) (latexrelease)}w
%    \begin{macrocode}
%</2ekernel>
%<latexrelease>\IncludeInRelease{2015/01/01}%
%<latexrelease>                 {\@Esphack}{hyphenation after space hack}%
%<*2ekernel|latexrelease>
\def\@Esphack{%
  \relax
  \ifhmode
    \spacefactor\@savsf
    \ifdim\@savsk>\z@
      \nobreak \hskip\z@skip
      \@ignoretrue
      \ignorespaces
    \fi
   \fi}%
%</2ekernel|latexrelease>
%<latexrelease>\EndIncludeInRelease
%<latexrelease>\IncludeInRelease{0000/00/00}%
%<latexrelease>                 {\@Esphack}{hyphenation after space hack}%
%<latexrelease>\def\@Esphack{%
%<latexrelease>  \relax
%<latexrelease>  \ifhmode
%<latexrelease>    \spacefactor\@savsf
%<latexrelease>    \ifdim\@savsk>\z@
%<latexrelease>      \@ignoretrue
%<latexrelease>      \ignorespaces
%<latexrelease>    \fi
%<latexrelease>   \fi}%
%<latexrelease>\EndIncludeInRelease
%<*2ekernel>
%    \end{macrocode}
%  \end{macro}
%
%
%  \begin{macro}{\@vbsphack}
% \changes{LaTeX2e}{1993/12/08}{Command added}
%    Another variant which is useful for invisible things which should
%    not live in vmode (this is how some people feel about marginals).
%
%    If it occurs in vmode then it enters hmode and ensures that
%    |\@savsk| is nonzero so that the |\ignorespaces| is put in later.
%    It is not used at present.
% \changes{v1.2f}{1995/05/25}{(CAR) not used so `removed'.}
%\begin{verbatim}
% \def \@vbsphack{ %
%    \relax \ifvmode
%      \leavevmode
%      \@savsk 1sp
%      \@savsf \spacefactor
%    \else
%      \ifhmode
%        \@savsk \lastskip
%        \@savsf \spacefactor
%      \fi
%    \fi
% }
%\end{verbatim}
%  \end{macro}
%
%
% \subsection{Vertical spacing}
%
%
% \LaTeX\ supports the plain \TeX\ commands
% |\smallskip|, |\medskip| and |\bigskip|.
% However, it redefines them using |\vspace| instead of |\vskip|.
%
% Extra vertical space is added by the command
% |\addvspace|\marg{skip},
% which adds a vertical skip of \meta{skip} to the document.
% The sequence\\
%         |\addvspace|\marg{s1} |\addvspace|\marg{s2}
% is equivalent to\\
%         |\addvspace|\marg{maximum of s1, s2}.
%
% |\addvspace| should be used only in vertical mode, and gives an
% error if it's not.  The |\addvspace| command does \emph{not} add
% vertical space if |@minipage| is true. The minipage environment uses
% this to inhibit the addition of extra vertical space at the beginning.
%
% Penalties are put into the vertical list with the
% |\addpenalty|\marg{penalty}
% command.  It works properly when |\addpenalty| and |\addvspace|
% commands are mixed.
%
% The |@nobreak| switch is set true used when in vertical mode and no
% page break should occur.  (Right now, it is used only by the section
% heading commands to inhibit page breaking after a heading.)
%
%
%\begin{verbatim}
% \addvspace{SKIP} ==
%  BEGIN
%   if vmode
%     then if @minipage
%            else if \lastskip =0
%                    then  \vskip SKIP
%                    else  if \lastskip < SKIP
%                             then  \vskip -\lastskip
%                                   \vskip SKIP
%                             else if SKIP < 0 and \lastskip >= 0
%                                    then \vskip -\lastskip
%                                         \vskip \lastskip + SKIP
%          fi      fi       fi      fi
%     else useful error message (CAR).
%   fi
%  END
%\end{verbatim}
%
% \begin{macro}{\@xaddvskip}
% Internal macro for |\vspace| handling the case that space has
% previously been added.
%    \begin{macrocode}
\def\@xaddvskip{%
  \ifdim\lastskip<\@tempskipb
    \vskip-\lastskip
    \vskip\@tempskipb
  \else
    \ifdim\@tempskipb<\z@
      \ifdim\lastskip<\z@
      \else
        \advance\@tempskipb\lastskip
        \vskip-\lastskip
        \vskip \@tempskipb
      \fi
    \fi
  \fi}
%    \end{macrocode}
% \end{macro}
%
%
%  \begin{macro}{\addvspace}
% \changes{v1.2b}{1994/11/12}{Corrected error message}
% \changes{v1.2c}{1994/11/13}{Recorrected error message}
%  Add vertical space taking into account space already added, as
%  described above.
%    \begin{macrocode}
\def\addvspace#1{%
  \ifvmode
     \if@minipage\else
       \ifdim \lastskip =\z@
         \vskip #1\relax
       \else
       \@tempskipb#1\relax
         \@xaddvskip
       \fi
     \fi
  \else
    \@noitemerr
  \fi}
%    \end{macrocode}
%  \end{macro}
%
%  \begin{macro}{\addpenalty}
% \changes{v1.2b}{1994/11/12}{Corrected error message}
% \changes{v1.2c}{1994/11/13}{Recorrected error message}
% \changes{v1.1h}{2015/01/09}{Donald Arseneau's fix from PR/377703 (latexrelease)}
%    \begin{macrocode}
%</2ekernel>
%<latexrelease>\IncludeInRelease{2015/01/01}%
%<latexrelease>                 {\addpenalty}{\addpenalty}%
%<*2ekernel|latexrelease>
%    \end{macrocode}
%   Fix provided by Donald (though the original fix was not good
%   enough).  In 2005 Plamen Tanovski discovered that this fix wasn't
%   good enough either as the \cs{vskip} kept getting bigger if
%   several \cs{addpenalty} commands followed each other. Donald
%   kindly send a new fix.
%    \begin{macrocode}
\def\addpenalty#1{%
  \ifvmode
    \if@minipage
    \else
      \if@nobreak
      \else
        \ifdim\lastskip=\z@
          \penalty#1\relax
        \else
          \@tempskipb\lastskip
%    \end{macrocode}
%    We have to make sure the final \cs{vskip} seen by \TeX\ is the
%    correct one, namely \cs{@tempskipb}. However we may have to
%    adjust for \cs{prevdepth} when placing the penalty but that
%    should not affect the skip we pass on to \TeX.
% \changes{v1.3e}{2015/01/14}{Avoid adding redundant skips (DPC)}
%    \begin{macrocode}
          \begingroup
            \@tempskipa\@tempskipb
            \advance \@tempskipb
              \ifdim\prevdepth>\maxdepth\maxdepth\else
%    \end{macrocode}
%    If |\prevdepth| is -1000pt due to |\nointerlineskip| we better
%    not add it!
%    \begin{macrocode}
                 \ifdim \prevdepth = -\@m\p@ \z@ \else \prevdepth \fi
               \fi
             \vskip -\@tempskipb
             \penalty#1%
             \ifdim\@tempskipa=\@tempskipb
%    \end{macrocode}
% Do nothing if the |\prevdepth| check made no adjustment.
%    \begin{macrocode}
             \else
%    \end{macrocode}
% Combine the prevdepth adjustment into a single skip.
%    \begin{macrocode}
               \advance\@tempskipb -\@tempskipa
               \vskip \@tempskipb
             \fi
%    \end{macrocode}
% The final skip is always the specified length.
%    \begin{macrocode}
             \vskip \@tempskipa
          \endgroup
        \fi
      \fi
    \fi
  \else
    \@noitemerr
  \fi}%
%    \end{macrocode}
%
%    \begin{macrocode}
%</2ekernel|latexrelease>
%<latexrelease>\EndIncludeInRelease
%<latexrelease>\IncludeInRelease{0000/00/00}%
%<latexrelease>                 {\addpenalty}{\addpenalty}%
%<latexrelease>\def\addpenalty#1{%
%<latexrelease>  \ifvmode
%<latexrelease>    \if@minipage
%<latexrelease>    \else
%<latexrelease>      \if@nobreak
%<latexrelease>      \else
%<latexrelease>        \ifdim\lastskip=\z@
%<latexrelease>          \penalty#1\relax
%<latexrelease>        \else
%<latexrelease>          \@tempskipb\lastskip
%<latexrelease>          \vskip -\lastskip
%<latexrelease>          \penalty#1%
%<latexrelease>          \vskip\@tempskipb
%<latexrelease>        \fi
%<latexrelease>      \fi
%<latexrelease>    \fi
%<latexrelease>  \else
%<latexrelease>    \@noitemerr
%<latexrelease>  \fi}%
%<latexrelease>\EndIncludeInRelease
%<*2ekernel>
%    \end{macrocode}
%  \end{macro}
%
% \begin{macro}{\vspace}
% \changes{v1.2m}{1996/01/20}{Made robust}
% \begin{macro}{\@vspace}
% \begin{macro}{\@vspacer}
% \changes{v1.2f}{1995/05/25}
%         {(CAR) macros modified to be more efficient}
% \changes{v1.2f}{1995/05/25}{(CAR) \cs{@restorepar} added to avoid
%   possible infinite tail recursion caused by a typo in the argument.}
%    The new code for these commands depends on the following facts:
%    \begin{itemize}
%      \item The value of prevdepth is changed only when a box or rule
%        is created and added to a vertical list;
%      \item The value of prevdepth is used only when a box is created
%        and added to a vertical list;
%      \item The value of prevdepth is always local to the building of
%        one vertical list.
%    \end{itemize}
%    \begin{macrocode}
\DeclareRobustCommand\vspace{\@ifstar\@vspacer\@vspace}
\def\@vspace #1{%
  \ifvmode
    \vskip #1
    \vskip\z@skip
   \else
     \@bsphack
     \vadjust{\@restorepar
              \vskip #1
              \vskip\z@skip
              }%
     \@esphack
   \fi}
%    \end{macrocode}
%
%    \begin{macrocode}
\def\@vspacer#1{%
  \ifvmode
    \dimen@\prevdepth
    \hrule \@height\z@
    \nobreak
    \vskip #1
    \vskip\z@skip
    \prevdepth\dimen@
  \else
    \@bsphack
    \vadjust{\@restorepar
             \hrule \@height\z@
             \nobreak
             \vskip #1
             \vskip\z@skip}%
    \@esphack
  \fi}
%    \end{macrocode}
% \end{macro}
% \end{macro}
% \end{macro}
%
% \begin{macro}{\smallskip}
% \begin{macro}{\medskip}
% \begin{macro}{\bigskip}
%    \begin{macrocode}
\def\smallskip{\vspace\smallskipamount}
\def\medskip{\vspace\medskipamount}
\def\bigskip{\vspace\bigskipamount}
%    \end{macrocode}
% \end{macro}
% \end{macro}
% \end{macro}
%
% \begin{macro}{\smallskipamount}
% \begin{macro}{\medskipamount}
% \begin{macro}{\bigskipamount}
%    \begin{macrocode}
\newskip\smallskipamount \smallskipamount=3pt plus 1pt minus 1pt
\newskip\medskipamount   \medskipamount  =6pt plus 2pt minus 2pt
\newskip\bigskipamount   \bigskipamount =12pt plus 4pt minus 4pt
%    \end{macrocode}
% \end{macro}
% \end{macro}
% \end{macro}
%
%
%
% \subsection{Horizontal space (and breaks)}
%
% \begin{macro}{\nobreakdashes}
% \changes{v1.3}{2004/02/04}{(Macro added}
% \changes{v1.3a}{2004/02/15}{(Added spacefactor setting}
%    This idea is borrowed from the \textsf{amsmath} package but
%    here we define a robust command.
%
%    This command is a low-level command designed for use only before
%    hyphens or dashes (such as |-|, |--|, or |---|).
%
%    It could probably be better implemented: it may need its own
%    private token register and temporary command.
%
%    Setting the hyphen in a box and then unboxing it means that the
%    normal penalty will not be added after it---and if the penalty is
%    not there a break will not be taken (unless an explicit penalty
%    or glue follows, thus the final \verb=\nobreak=).
%
%    Note that even if it is not followed by a `-', it still leaves
%    vmode and sets the spacefactor; so use it carefully!
%
%    \begin{macrocode}
\DeclareRobustCommand{\nobreakdashes}{%
  \leavevmode
  \toks@{}%
  \def\reserved@a##1{\toks@\expandafter{\the\toks@-}%
                     \futurelet\@let@token \reserved@b}%
  \def\reserved@b   {\ifx\@let@token -%
                        \expandafter\reserved@a
                     \else
                       \setbox\z@ \hbox{\the\toks@\nobreak}%
                       \unhbox\z@
                       \spacefactor\sfcode`\-
                     \fi}%
  \futurelet\@let@token \reserved@b
}
%    \end{macrocode}
% \end{macro}
%
% \begin{macro}{\nobreakspace}
% \changes{v1.2k}{1995/12/04}{(Macro added}
% \begin{macro}{\@xobeysp}
% \changes{v1.2t}{1996/09/28}{Moved from ltmiscen.dtx and redefined to
%                 use \cs{nobreakspace }}
%
%   This is a robust command that produces a horizontal space at
%   which, in paragraph-mode, a line-break is not possible.  We then
%   define an active |~| to expand to it since this is the documented
%   behaviour of |~|.  One reason for introducing this is that some
%   8-bit input encodings have a slot for such a space and we do not
%   want to use active characters as the \LaTeX{} internal commands.
%
%   The braces in the definition of |~| are needed to ensure that a
%   following space is preserved when reading to/from internal files.
% \changes{v1.2l}{1995/12/04}{(braces added to definition of tilde}
%
%   We need to keep \cs{@xobeysp} as it is widely used; so here it is
%   let to the non-robust command \cs{nobreakspace }.
%
%    \begin{macrocode}
\DeclareRobustCommand{\nobreakspace}{%
   \leavevmode\nobreak\ }
\catcode `\~=13
\def~{\nobreakspace{}}
\expandafter\let\expandafter\@xobeysp\csname nobreakspace \endcsname
%    \end{macrocode}
% \end{macro}
% \end{macro}
%
%    \begin{macrocode}
%    \end{macrocode}
%
%
% \begin{macro}{\,}
%   Used in paragraph mode produces a |\thinspace|.  It has the
%   ordinary definition in math mode.  Useful for quotes inside quotes,
%   as in  |``\,`Foo', he said.''|
% \changes{v1.0o}{1994/05/11}{Use \cs{DeclareRobustCommand}. ASAJ.}
%    \begin{macrocode}
\DeclareRobustCommand{\,}{%
   \relax\ifmmode\mskip\thinmuskip\else\thinspace\fi
}
%    \end{macrocode}
% \end{macro}
%
% \begin{macro}{\@}
%     Placed before a '.', makes it a sentence-ending period.  Does the
%     right thing for other punctuation marks as well.  Does this by
%     setting spacefactor to 1000.
% \changes{v1.3b}{2014/12/30}{\cs{@} discards spaces when moving
%             (pr3039)(latexrelease)}
%    \begin{macrocode}
%</2ekernel>
%<latexrelease>\IncludeInRelease{2015/01/01}%
%<latexrelease>                 {\@}{Space after \@}%
%<*2ekernel|latexrelease>
%    \end{macrocode}
%    \begin{macrocode}
\def\@{\spacefactor\@m{}}%
%</2ekernel|latexrelease>
%<latexrelease>\EndIncludeInRelease
%<latexrelease>\IncludeInRelease{0000/00/00}%
%<latexrelease>                 {\@}{Space after \@}%
%<latexrelease>\def\@{\spacefactor\@m}%
%<latexrelease>\EndIncludeInRelease
%<*2ekernel>
%    \end{macrocode}
% \end{macro}
%
%
% \begin{macro}{\hspace}
% \changes{v1.0o}{1994/05/11}{Use \cs{DeclareRobustCommand}. ASAJ.}
%    \begin{macrocode}
\DeclareRobustCommand\hspace{\@ifstar\@hspacer\@hspace}
%    \end{macrocode}
% \end{macro}
%
% \begin{macro}{\@hspace}
% \changes{LaTeX2e}{1993/08/05}
%    {(RmS) Removed superfluous \cs{leavevmode} in \cs{@hspace} and
%               \cs{@hspacer}, as suggested by CAR.}
%    \begin{macrocode}
\def\@hspace#1{\hskip #1\relax}
%    \end{macrocode}
% \end{macro}
%
%
% \begin{macro}{\@hspacer}
% extra |\hskip 0pt| added 1985/17/12 to guard
% against a following |\unskip|
% |\relax| added 13 Oct 88 for usual \TeX\ lossage
% replaced both changes by |\hskip\z@skip| 27 Nov 91
%    \begin{macrocode}
\def\@hspacer#1{\vrule \@width\z@\nobreak
                \hskip #1\hskip \z@skip}
%    \end{macrocode}
% \end{macro}
%
%
%
% \begin{macro}{\fill}
%    \begin{macrocode}
\newskip\fill
\fill = 0pt plus 1fill
%    \end{macrocode}
% \end{macro}
%
%
%
% \begin{macro}{\stretch}
%    \begin{macrocode}
\def\stretch#1{\z@ \@plus #1fill\relax}
%    \end{macrocode}
% \end{macro}
%
%    
%    \begin{macrocode}
%</2ekernel>
%<*2ekernel|latexrelease>
%<latexrelease>\IncludeInRelease{2018/12/01}%
%<latexrelease>                 {\thinspace}{Start LR-mode}%
%    \end{macrocode}
%
%
% \begin{macro}{\thinspace}
% \begin{macro}{\negthinspace}
% \begin{macro}{\enspace}
% \changes{v1.3h}{2018/09/24}{Start LR-mode if necessary (git/49)}   
%    \begin{macrocode}
\def\thinspace{\leavevmode@ifvmode\kern .16667em }
\def\negthinspace{\leavevmode@ifvmode\kern-.16667em }
\def\enspace{\leavevmode@ifvmode\kern.5em }
%    \end{macrocode}
% \end{macro}
% \end{macro}
% \end{macro}
%
%  \begin{macro}{\leavevmode@ifvmode}
%    Leave vmode but only if we are really in vmode, otherwise the
%    expansion is empty (which is not the case with the default definition).
% \changes{v1.3h}{2018/09/24}{Macro added (git/49)}   
%    \begin{macrocode}
\protected\def\leavevmode@ifvmode{\ifvmode\expandafter\indent\fi}
%    \end{macrocode}
%  \end{macro}
%
%    \begin{macrocode}
%</2ekernel|latexrelease>
%<latexrelease>\EndIncludeInRelease
%<latexrelease>\IncludeInRelease{0000/00/00}%
%<latexrelease>                 {\thinspace}{Start LR-mode}%
%<latexrelease>\def\thinspace{\kern .16667em }
%<latexrelease>\def\negthinspace{\kern-.16667em }
%<latexrelease>\def\enspace{\kern.5em }
%<latexrelease>\let\leavevmode@ifvmode\@undefined
%<latexrelease>\EndIncludeInRelease
%<*2ekernel>
%    \end{macrocode}
%
%
%
% \begin{macro}{\enskip}
% \begin{macro}{\quad}
% \begin{macro}{\qquad}
%    \begin{macrocode}
\def\enskip{\hskip.5em\relax}
\def\quad{\hskip1em\relax}
\def\qquad{\hskip2em\relax}
%    \end{macrocode}
% \end{macro}
% \end{macro}
% \end{macro}
%
% \begin{macro}{\obeycr}
% \begin{macro}{\restorecr}
% The following definitions will probably get deleted or moved to
% compatibility mode soon.
%
% \changes{v1.2g}{1995/06/11}
%                {(CAR) \cs{relax} added to stop silent eating of *.}
%    \begin{macrocode}
{\catcode`\^^M=13 \gdef\obeycr{\catcode`\^^M13 \def^^M{\\\relax}%
    \@gobblecr}%
{\catcode`\^^M=13 \gdef\@gobblecr{\@ifnextchar
\@gobble\ignorespaces}}
\gdef\restorecr{\catcode`\^^M5 }}
%    \end{macrocode}
% \end{macro}
% \end{macro}
%
%    \begin{macrocode}
%</2ekernel>
%    \end{macrocode}
%
% \Finale
\endinput

