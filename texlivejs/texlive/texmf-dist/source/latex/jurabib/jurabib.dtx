version https://git-lfs.github.com/spec/v1
oid sha256:74496d27ae4c60d65f87f444a461ea9df52a260761d6dcae2de8ffc4606cef5a
size 442924
