% \iffalse meta-comment
%
% File: hypdoc.dtx
% Version: 2018/09/07 v1.14
% Info: Hyper extensions for doc.sty
%
% Copyright (C) 2002, 2006, 2007, 2009-2011 by
%    Heiko Oberdiek <heiko.oberdiek at googlemail.com>
%    2016
%    https://github.com/ho-tex/oberdiek/issues
%
% This work may be distributed and/or modified under the
% conditions of the LaTeX Project Public License, either
% version 1.3c of this license or (at your option) any later
% version. This version of this license is in
%    http://www.latex-project.org/lppl/lppl-1-3c.txt
% and the latest version of this license is in
%    http://www.latex-project.org/lppl.txt
% and version 1.3 or later is part of all distributions of
% LaTeX version 2005/12/01 or later.
%
% This work has the LPPL maintenance status "maintained".
%
% This Current Maintainer of this work is Heiko Oberdiek.
%
% This work consists of the main source file hypdoc.dtx
% and the derived files
%    hypdoc.sty, hypdoc.pdf, hypdoc.ins, hypdoc.drv.
%
% Distribution:
%    CTAN:macros/latex/contrib/oberdiek/hypdoc.dtx
%    CTAN:macros/latex/contrib/oberdiek/hypdoc.pdf
%
% Unpacking:
%    (a) If hypdoc.ins is present:
%           tex hypdoc.ins
%    (b) Without hypdoc.ins:
%           tex hypdoc.dtx
%    (c) If you insist on using LaTeX
%           latex \let\install=y% \iffalse meta-comment
%
% File: hypdoc.dtx
% Version: 2018/09/07 v1.14
% Info: Hyper extensions for doc.sty
%
% Copyright (C) 2002, 2006, 2007, 2009-2011 by
%    Heiko Oberdiek <heiko.oberdiek at googlemail.com>
%    2016
%    https://github.com/ho-tex/oberdiek/issues
%
% This work may be distributed and/or modified under the
% conditions of the LaTeX Project Public License, either
% version 1.3c of this license or (at your option) any later
% version. This version of this license is in
%    http://www.latex-project.org/lppl/lppl-1-3c.txt
% and the latest version of this license is in
%    http://www.latex-project.org/lppl.txt
% and version 1.3 or later is part of all distributions of
% LaTeX version 2005/12/01 or later.
%
% This work has the LPPL maintenance status "maintained".
%
% This Current Maintainer of this work is Heiko Oberdiek.
%
% This work consists of the main source file hypdoc.dtx
% and the derived files
%    hypdoc.sty, hypdoc.pdf, hypdoc.ins, hypdoc.drv.
%
% Distribution:
%    CTAN:macros/latex/contrib/oberdiek/hypdoc.dtx
%    CTAN:macros/latex/contrib/oberdiek/hypdoc.pdf
%
% Unpacking:
%    (a) If hypdoc.ins is present:
%           tex hypdoc.ins
%    (b) Without hypdoc.ins:
%           tex hypdoc.dtx
%    (c) If you insist on using LaTeX
%           latex \let\install=y% \iffalse meta-comment
%
% File: hypdoc.dtx
% Version: 2018/09/07 v1.14
% Info: Hyper extensions for doc.sty
%
% Copyright (C) 2002, 2006, 2007, 2009-2011 by
%    Heiko Oberdiek <heiko.oberdiek at googlemail.com>
%    2016
%    https://github.com/ho-tex/oberdiek/issues
%
% This work may be distributed and/or modified under the
% conditions of the LaTeX Project Public License, either
% version 1.3c of this license or (at your option) any later
% version. This version of this license is in
%    http://www.latex-project.org/lppl/lppl-1-3c.txt
% and the latest version of this license is in
%    http://www.latex-project.org/lppl.txt
% and version 1.3 or later is part of all distributions of
% LaTeX version 2005/12/01 or later.
%
% This work has the LPPL maintenance status "maintained".
%
% This Current Maintainer of this work is Heiko Oberdiek.
%
% This work consists of the main source file hypdoc.dtx
% and the derived files
%    hypdoc.sty, hypdoc.pdf, hypdoc.ins, hypdoc.drv.
%
% Distribution:
%    CTAN:macros/latex/contrib/oberdiek/hypdoc.dtx
%    CTAN:macros/latex/contrib/oberdiek/hypdoc.pdf
%
% Unpacking:
%    (a) If hypdoc.ins is present:
%           tex hypdoc.ins
%    (b) Without hypdoc.ins:
%           tex hypdoc.dtx
%    (c) If you insist on using LaTeX
%           latex \let\install=y% \iffalse meta-comment
%
% File: hypdoc.dtx
% Version: 2018/09/07 v1.14
% Info: Hyper extensions for doc.sty
%
% Copyright (C) 2002, 2006, 2007, 2009-2011 by
%    Heiko Oberdiek <heiko.oberdiek at googlemail.com>
%    2016
%    https://github.com/ho-tex/oberdiek/issues
%
% This work may be distributed and/or modified under the
% conditions of the LaTeX Project Public License, either
% version 1.3c of this license or (at your option) any later
% version. This version of this license is in
%    http://www.latex-project.org/lppl/lppl-1-3c.txt
% and the latest version of this license is in
%    http://www.latex-project.org/lppl.txt
% and version 1.3 or later is part of all distributions of
% LaTeX version 2005/12/01 or later.
%
% This work has the LPPL maintenance status "maintained".
%
% This Current Maintainer of this work is Heiko Oberdiek.
%
% This work consists of the main source file hypdoc.dtx
% and the derived files
%    hypdoc.sty, hypdoc.pdf, hypdoc.ins, hypdoc.drv.
%
% Distribution:
%    CTAN:macros/latex/contrib/oberdiek/hypdoc.dtx
%    CTAN:macros/latex/contrib/oberdiek/hypdoc.pdf
%
% Unpacking:
%    (a) If hypdoc.ins is present:
%           tex hypdoc.ins
%    (b) Without hypdoc.ins:
%           tex hypdoc.dtx
%    (c) If you insist on using LaTeX
%           latex \let\install=y\input{hypdoc.dtx}
%        (quote the arguments according to the demands of your shell)
%
% Documentation:
%    (a) If hypdoc.drv is present:
%           latex hypdoc.drv
%    (b) Without hypdoc.drv:
%           latex hypdoc.dtx; ...
%    The class ltxdoc loads the configuration file ltxdoc.cfg
%    if available. Here you can specify further options, e.g.
%    use A4 as paper format:
%       \PassOptionsToClass{a4paper}{article}
%
%    Programm calls to get the documentation (example):
%       pdflatex hypdoc.dtx
%       makeindex -s gind.ist hypdoc.idx
%       pdflatex hypdoc.dtx
%       makeindex -s gind.ist hypdoc.idx
%       pdflatex hypdoc.dtx
%
% Installation:
%    TDS:tex/latex/oberdiek/hypdoc.sty
%    TDS:doc/latex/oberdiek/hypdoc.pdf
%    TDS:source/latex/oberdiek/hypdoc.dtx
%
%<*ignore>
\begingroup
  \catcode123=1 %
  \catcode125=2 %
  \def\x{LaTeX2e}%
\expandafter\endgroup
\ifcase 0\ifx\install y1\fi\expandafter
         \ifx\csname processbatchFile\endcsname\relax\else1\fi
         \ifx\fmtname\x\else 1\fi\relax
\else\csname fi\endcsname
%</ignore>
%<*install>
\input docstrip.tex
\Msg{************************************************************************}
\Msg{* Installation}
\Msg{* Package: hypdoc 2018/09/07 v1.14 Hyper extensions for doc.sty (HO)}
\Msg{************************************************************************}

\keepsilent
\askforoverwritefalse

\let\MetaPrefix\relax
\preamble

This is a generated file.

Project: hypdoc
Version: 2018/09/07 v1.14

Copyright (C) 2002, 2006, 2007, 2009-2011 by
   Heiko Oberdiek <heiko.oberdiek at googlemail.com>

This work may be distributed and/or modified under the
conditions of the LaTeX Project Public License, either
version 1.3c of this license or (at your option) any later
version. This version of this license is in
   http://www.latex-project.org/lppl/lppl-1-3c.txt
and the latest version of this license is in
   http://www.latex-project.org/lppl.txt
and version 1.3 or later is part of all distributions of
LaTeX version 2005/12/01 or later.

This work has the LPPL maintenance status "maintained".

This Current Maintainer of this work is Heiko Oberdiek.

This work consists of the main source file hypdoc.dtx
and the derived files
   hypdoc.sty, hypdoc.pdf, hypdoc.ins, hypdoc.drv.

\endpreamble
\let\MetaPrefix\DoubleperCent

\generate{%
  \file{hypdoc.ins}{\from{hypdoc.dtx}{install}}%
  \file{hypdoc.drv}{\from{hypdoc.dtx}{driver}}%
  \usedir{tex/latex/oberdiek}%
  \file{hypdoc.sty}{\from{hypdoc.dtx}{package}}%
  \nopreamble
  \nopostamble
%  \usedir{source/latex/oberdiek/catalogue}%
%  \file{hypdoc.xml}{\from{hypdoc.dtx}{catalogue}}%
}

\catcode32=13\relax% active space
\let =\space%
\Msg{************************************************************************}
\Msg{*}
\Msg{* To finish the installation you have to move the following}
\Msg{* file into a directory searched by TeX:}
\Msg{*}
\Msg{*     hypdoc.sty}
\Msg{*}
\Msg{* To produce the documentation run the file `hypdoc.drv'}
\Msg{* through LaTeX.}
\Msg{*}
\Msg{* Happy TeXing!}
\Msg{*}
\Msg{************************************************************************}

\endbatchfile
%</install>
%<*ignore>
\fi
%</ignore>
%<*driver>
\NeedsTeXFormat{LaTeX2e}
\ProvidesFile{hypdoc.drv}%
  [2018/09/07 v1.14 Hyper extensions for doc.sty (HO)]%
\documentclass{ltxdoc}
\usepackage{holtxdoc}[2011/11/22]
\begin{document}
  \DocInput{hypdoc.dtx}%
\end{document}
%</driver>
% \fi
%
%
% \CharacterTable
%  {Upper-case    \A\B\C\D\E\F\G\H\I\J\K\L\M\N\O\P\Q\R\S\T\U\V\W\X\Y\Z
%   Lower-case    \a\b\c\d\e\f\g\h\i\j\k\l\m\n\o\p\q\r\s\t\u\v\w\x\y\z
%   Digits        \0\1\2\3\4\5\6\7\8\9
%   Exclamation   \!     Double quote  \"     Hash (number) \#
%   Dollar        \$     Percent       \%     Ampersand     \&
%   Acute accent  \'     Left paren    \(     Right paren   \)
%   Asterisk      \*     Plus          \+     Comma         \,
%   Minus         \-     Point         \.     Solidus       \/
%   Colon         \:     Semicolon     \;     Less than     \<
%   Equals        \=     Greater than  \>     Question mark \?
%   Commercial at \@     Left bracket  \[     Backslash     \\
%   Right bracket \]     Circumflex    \^     Underscore    \_
%   Grave accent  \`     Left brace    \{     Vertical bar  \|
%   Right brace   \}     Tilde         \~}
%
% \GetFileInfo{hypdoc.drv}
%
% \title{The \xpackage{hypdoc} package}
% \date{2018/09/07 v1.14}
% \author{Heiko Oberdiek\thanks
% {Please report any issues at https://github.com/ho-tex/oberdiek/issues}\\
% \xemail{heiko.oberdiek at googlemail.com}}
%
% \maketitle
%
% \begin{abstract}
% This package adds hyper features to the package
% \xpackage{doc} that is used in the documentation
% system of \LaTeXe. Bookmarks are added and references
% are linked as far as possible.
% \end{abstract}
%
% \tableofcontents
%
% \section{Features}
%
% The package adds the following features:
% \begin{description}
% \item[Bookmarks:]
%   For supporting bookmarks package \verb|hyperref| is loaded.
%   Some \TeX\ code inside section titles can cause problems
%   during bookmark processing; these can be warnings, ugly
%   bookmark titles or even errors. With the help of
%   \verb|\pdfstringdefDisableCommands| or \verb|\texorpdfstring|
%   you can manage these things, see \verb|hyperref|'s package
%   documentation.
% \item[Additional bookmarks:]
%   Unnumbered bookmark entries are generated for:
%   \begin{itemize}
%   \item Table of contents
%   \item List of figures
%   \item List of tables
%   \item Index
%   \item Glossary
%   \end{itemize}
%   Option \texttt{numbered} controls, whether these section or
%   chapters should be numbered. The exception is the table
%   of contents if this it would be the first numbered
%   section. Then just a bookmark is added.
% \item[Metadata:]
%   The loaded package \verb|hyperref|, enforced by
%   option \verb|pdfusetitle|, tries to detect \verb|\title|
%   and \verb|\author| and sets the corresponding PDF
%   information fields. In case of problems see the previous
%   item about bookmarks.
%   These enries can be overwritten or new ones added by
%   \verb|\hypersetup| after the packages is loaded, see
%   documentation of package \verb|hyperref|.
% \item[References, links:] Because of package \verb|\hyperref|
%   |\ref| become links, also \verb|\url|, footnote referencing
%   are supported.
% \item[Index:]
%   The main part of this package deals with index links in order
%   to support the kind of index that package \verb|doc| provides.
% \item[Thumbnails:]
%   Package \verb|thumbpdf| may additionally be loaded for the case that the
%   PDF file also should contain thumbnails, see the documentation
%   of this package. But thumbnail embedding is not really
%   necessary, because nowadays AcrobatReader is able to
%   generate and view thumbnails on the fly.
% \end{description}
%
% \section{Usage}
%
% The purpose of this packages is to add PDF features to
% the documentation of LaTeX packages. Thus just load
% the package in the driver file that generates the
% documentation:
% \begin{quote}
%   \verb|\usepackage{hypdoc}|
% \end{quote}
% Package \verb|doc| is loaded by \verb|hypdoc|, thus you
% also can replace a \verb|\usepackage{doc}| with this line.
% At any case, the package must be loaded before macros of
% package \verb|doc| such as \verb|\CodelineIndex| or
% \verb|\PageIndex| are used.
%
% If the class \verb|ltxdoc| is used, you can also use
% the configuration file for this class to add
% package \verb|hypdoc|. Add the following line
% to the file \verb|ltxdoc.cfg| (or create a new one):
% \begin{quote}
%\begin{verbatim}
%\AtEndOfClass{\RequirePackage{hypdoc}}
%\end{verbatim}
% \end{quote}
%
% \subsection{Options}
%
% There is one option \texttt{numbered}. It controls,
% whether the starred versions of section or chapter
% inside \cs{tableofcontents}, \cs{listoffigures}, \dots
% should be replaced by the non-star versions.
%
% \subsection{Simple example}
% Example of a simple driver file \verb|foobar.drv|
% of a package \verb|foobar|. Often the driver file
% is packed in the file \verb|foobar.dtx| and can
% be extracted by docstrip or if an instruction file
% for docstrip is provided by:
% \begin{quote}
% \verb|tex foobar.ins|
% \end{quote}
% Note that although docstrip doesn't rely on \LaTeX,
% sometimes you need to run the \verb|.ins| file through
% \LaTeX. Oddly enough, often the only \LaTeX\ feature
% in use is \verb|\NeedsTeXFormat{LaTeX2e}| in the first line.
% \begin{quote}
%\begin{verbatim}
%\documentclass{ltxdoc}
%\usepackage{hypdoc}
%\RecordChanges
%\EnableCrossrefs
%\CodelineIndex
%\begin{document}
%\title{Title for package \texttt{foobar}}
%\author{Author of package \texttt{foobar}}
%\date{...}
%\maketitle
%\tableofcontents
%\DocInput{foobar.dtx}
%\PrintChanges
%\PrintIndex
%\end{document}
%\end{verbatim}
% \end{quote}
%
% Then compile it, for example with pdf\LaTeX.
% \begin{quote}
%\begin{verbatim}
%pdflatex foobar.drv
%makeindex -s gind.ist foobar.idx
%makeindex -s gglo.ist -o foobar.gls foobar.glo
%pdflatex foobar.drv
%\end{verbatim}
% \end{quote}
% You will need several cycles until the cross references
% and similar things are correct.
%
% \subsection{Example \texttt{source2e.tex}}
%
% The source code of the base of \LaTeX\ is available as
% a bunch of \verb|.dtx| files. \LaTeX\ provides the file
% \verb|source2e.tex| to merge them all together.
%
% Now you can either load package \verb|hypdoc| in a private copy
% of this file (please, respect the LPPL) or use the configuration
% file \verb|ltxdoc.cfg|. Example that also sets A4 paper size:
% \begin{quote}
%\begin{verbatim}
%\PassOptionsToClass{a4paper}{article}
%\AtEndOfClass{\RequirePackage{hypdoc}}
%\end{verbatim}
% \end{quote}
%
% Then \verb|source2e.tex| can be compiled by pdf\TeX, for example:
% \begin{quote}
%\begin{verbatim}
%pdflatex source2e
%makeindex -s gind.ist source2e.idx
%makeindex -s gglo.ist -o source2e.glo source2e.gls
%pdflatex source2e
%makeindex -s gind.ist source2e.idx
%makeindex -s gglo.ist -o source2e.glo source2e.gls
%pdflatex source2e
%\end{verbatim}
% \end{quote}
%
% \StopEventually{
% }
%
% \section{Implementation}
%
%    \begin{macrocode}
%<*package>
\NeedsTeXFormat{LaTeX2e}
\ProvidesPackage{hypdoc}%
  [2018/09/07 v1.14 Hyper extensions for doc.sty (HO)]%
%    \end{macrocode}
%    The package identification is done at the top of the |.dtx| file
%    in order to use only one identification string.
%
%    For unique command names this package uses |HD@| as prefix
%
% \subsection{Options}
%
%    \begin{macro}{\ifHD@numbered}
%    \begin{macrocode}
\newif\ifHD@numbered
%    \end{macrocode}
%    \end{macro}
%    \begin{macrocode}
\DeclareOption{numbered}{\HD@numberedtrue}
\ProcessOptions*\relax
%    \end{macrocode}
%
% \subsection{Package loading}
%
%    \begin{macrocode}
\RequirePackage{atveryend}[2010/03/24]
\RequirePackage{doc}
%    \end{macrocode}
%    \begin{macrocode}
\ifx\numexpr\@undefined
  \RequirePackage{calc}%
\fi
%    \end{macrocode}
%    \begin{macrocode}
\RequirePackage[%
  hyperindex=false,%
  pdfusetitle,%
  colorlinks,%
  pdfpagelabels%
]{hyperref}[2002/05/09]%
\hypersetup{%
  plainpages=false,%
  bookmarksopen,%
  bookmarksnumbered,%
  bookmarksopenlevel=0,%
  pdfstartview={FitBH \hypercalcbp{%
    \paperheight-\topmargin-1in-\headheight-\headsep
  }},%
}
%    \end{macrocode}
%
%    \begin{macrocode}
\newcounter{HD@unique}
%    \end{macrocode}
%    \begin{macro}{\theHD@unique}
%    \begin{macrocode}
\renewcommand{\theHD@unique}{%
  \number\value{HD@unique}%
}
%    \end{macrocode}
%    \end{macro}
%
%    \begin{macrocode}
\g@addto@macro\Hy@UseMaketitleInfos{%
  \@ifundefined{fileinfo}{}{%
    \ifx\@pdfsubject\@empty
      \hypersetup{pdfsubject={\fileinfo}}%
    \fi
  }%
}
%    \end{macrocode}
%
%    \begin{macrocode}
\pdfstringdefDisableCommands{%
  \let\thanks\@gobble
  \let\footnote\@gobble
  \def\cs#1{\textbackslash #1}%
  \let\normalfont\@empty
  \let\scshape\@empty
  \def\and{and }%
}
%    \end{macrocode}
%
% \subsubsection{Patch for bug latex/4096}
%
%    \begin{macrocode}
\begingroup\expandafter\expandafter\expandafter\endgroup
\expandafter\ifx\csname mod@math@codes\endcsname\relax
\else
  \def\HD@ModFix#1{%
    \mathcode`#1=`#1\relax
  }%
  \g@addto@macro\mod@math@codes{%
    \HD@ModFix\-%
    \HD@ModFix\+%
    \HD@ModFix\:%
    \HD@ModFix\=%
    \HD@ModFix\*%
  }%
\fi
%    \end{macrocode}
%
% \subsection{Bookmarks}
%
%    \begin{macro}{\HD@sectionpatch}
%    \begin{macrocode}
\def\HD@sectionpatch{%
  \ifHD@numbered
    \HD@@sectionpatch{section}%
    \HD@@sectionpatch{chapter}%
  \else
    \let\Hy@writebookmark\HD@disable@writebookmark
  \fi
}
%    \end{macrocode}
%    \end{macro}
%    \begin{macro}{\HD@@sectionpatch}
%    \begin{macrocode}
\def\HD@@sectionpatch#1{%
  \expandafter\let\csname HDorg@@#1\expandafter\endcsname
    \csname #1\endcsname
  \@namedef{#1}{%
    \@ifstar{\@nameuse{HDorg@@#1}}{\@nameuse{HDorg@@#1}}%
  }%
}
%    \end{macrocode}
%    \end{macro}
%    \begin{macro}{\HD@disable@writebookmark}
%    \begin{macrocode}
\def\HD@disable@writebookmark#1#2#3#4#5{}
%    \end{macrocode}
%    \end{macro}
%
%    \begin{macro}{\HD@guesstoclevel}
%    A wild guessing of the toclevel. There is the hope, that
%    the macro |#1| starts with a sectioning command.
%    \begin{macrocode}
\def\HD@guesstoclevel#1{%
  \expandafter\HD@@guesstoclevel#1\section\HD@end
}
%    \end{macrocode}
%    \end{macro}
%    \begin{macro}{\HD@@guesstoclevel}
%    \begin{macrocode}
\def\HD@@guesstoclevel#1#2\HD@end{%
  \@ifundefined{toclevel@\expandafter\@gobble\string#1\@empty}{%
    1%
  }{%
    \@nameuse{toclevel@\expandafter\@gobble\string#1\@empty}%
  }%
}
%    \end{macrocode}
%    \end{macro}
%
% \subsection{\cs{tableofcontents}}
%
%    \begin{macro}{\HDorg@tableofcontents}
%    \begin{macrocode}
\let\HDorg@tableofcontents\tableofcontents
%    \end{macrocode}
%    \end{macro}
%    \begin{macro}{\tableofcontents}
%    As first section the table of contents shouldn't
%    go into the toc itself. Also a number looks too funny then.
%    Just a bookmark is enough in this case.
%    \begin{macrocode}
\def\tableofcontents{%
  \begingroup
    \@ifundefined{c@section}{%
      \def\c@section{0}%
    }{}%
  \expandafter\endgroup
  \ifcase\c@section
    \stepcounter{HD@unique}%
    \pdfbookmark[\HD@guesstoclevel{\HDorg@tableofcontents}]%
        {\contentsname}{toc\theHD@unique}%
    \HDorg@tableofcontents
  \else
    \ifHD@numbered
    \else
      \stepcounter{HD@unique}%
      \pdfbookmark[\HD@guesstoclevel{\HDorg@tableofcontents}]%
          {\contentsname}{toc\theHD@unique}%
    \fi
    \begingroup
      \HD@sectionpatch
      \HDorg@tableofcontents
    \endgroup
  \fi
}
%    \end{macrocode}
%    \end{macro}
%
% \subsection{\cs{listoffigures}}
%
%    \begin{macro}{\HDorg@listoffigures}
%    \begin{macrocode}
\let\HDorg@listoffigures\listoffigures
%    \end{macrocode}
%    \end{macro}
%    \begin{macro}{\listoffigures}
%    \begin{macrocode}
\def\listoffigures{%
  \ifHD@numbered
  \else
    \stepcounter{HD@unique}%
    \pdfbookmark[\HD@guesstoclevel{\HDorg@listoffigures}]%
        {\listfigurename}{toc\theHD@unique}%
  \fi
  \begingroup
    \HD@sectionpatch
    \HDorg@listoffigures
  \endgroup
}
%    \end{macrocode}
%    \end{macro}
%
% \subsection{\cs{listoftables}}
%
%    \begin{macro}{\HDorg@listoftables}
%    \begin{macrocode}
\let\HDorg@listoftables\listoftables
%    \end{macrocode}
%    \end{macro}
%    \begin{macro}{\listoftables}
%    \begin{macrocode}
\def\listoftables{%
  \ifHD@numbered
  \else
    \stepcounter{HD@unique}%
    \pdfbookmark[\HD@guesstoclevel{\HDorg@listoftables}]%
        {\listtablename}{toc\theHD@unique}%
  \fi
  \begingroup
    \HD@sectionpatch
    \HDorg@listoftables
  \endgroup
}
%    \end{macrocode}
%    \end{macro}
%
% \subsection{\cs{thebibliography}}
%
%    \begin{macro}{\HDorg@thebibliography}
%    \begin{macrocode}
\let\HDorg@thebibliography\thebibliography
%    \end{macrocode}
%    \end{macro}
%    \begin{macro}{\thebibliography}
%    \begin{macrocode}
\def\thebibliography{%
  \ifHD@numbered
  \else
    \stepcounter{HD@unique}%
    \pdfbookmark[\HD@guesstoclevel{\HDorg@thebibliography}]%
      {\@ifundefined{chapter}{\refname}{\bibname}}{bib\theHD@unique}%
  \fi
  \HD@sectionpatch
  \HDorg@thebibliography
}
%    \end{macrocode}
%    \end{macro}
%
% \subsection{\cs{theindex}}
%
%    \begin{macro}{\HDorg@theindex}
%    \begin{macrocode}
\let\HDorg@theindex\theindex
%    \end{macrocode}
%    \end{macro}
%    \begin{macro}{\theindex}
%    \begin{macrocode}
\def\theindex{%
  \let\HDorg@index@prologue\index@prologue
  \let\HDorg@writebookmark\Hy@writebookmark
  \HD@sectionpatch
  \g@addto@macro\IndexParms{%
    \ifHD@numbered
    \else
      \let\Hy@writebookmark\HDorg@writebookmark
    \fi
    \let\bfseries\HD@bfseries
  }%
  \edef\HD@toclevel@index{\HD@guesstoclevel{\index@prologue}}%
  \begingroup
    \count@=\HD@toclevel@index\relax
    \advance\count@ by 1 %
    \xdef\HD@toclevel@subindex{\the\count@}%
  \endgroup
  \def\index@prologue{%
    \ifHD@numbered
    \else
      \begingroup
        \stepcounter{HD@unique}%
        \let\Hy@writebookmark\HDorg@writebookmark
        \pdfbookmark[\HD@toclevel@index]%
            {\indexname}{index\theHD@unique}%
      \endgroup
    \fi
    \HDorg@index@prologue
  }%
  \HDorg@theindex
}
%    \end{macrocode}
%    \end{macro}
%    \begin{macro}{\HDorg@bfseries}
%    \begin{macrocode}
\let\HDorg@bfseries\bfseries
%    \end{macrocode}
%    \end{macro}
%    \begin{macro}{\HD@bfseries}
%    \begin{macrocode}
\def\HD@bfseries{\futurelet\HD@hfil\HD@@bfseries}%
%    \end{macrocode}
%    \end{macro}
%    \begin{macro}{\HD@@bfseries}
%    \begin{macrocode}
\def\HD@@bfseries{%
  \HDorg@bfseries
  \ifx\HD@hfil\hfil
    \expandafter\HD@@@bfseries
  \fi
}
%    \end{macrocode}
%    \end{macro}
%    \begin{macro}{\HD@@@bfseries}
%    \begin{macrocode}
\def\HD@@@bfseries\hfil#1\hfil{%
  \ifx\\#1\\%
  \else
    \raisebox{\baselineskip}[0pt]{%
      \kern-\HD@margin\relax
      \pdfbookmark[\HD@toclevel@subindex]{#1}{HD.#1}%
      \kern\HD@margin\relax
    }%
  \fi
  \hfil#1\hfil
}%
%    \end{macrocode}
%    \end{macro}
%    \begin{macro}{\HD@margin}
%    \begin{macrocode}
\def\HD@margin{1mm}
%    \end{macrocode}
%    \end{macro}
%
% \subsection{\cs{theglossary}}
%
%    \begin{macro}{\HDorg@theglossary}
%    \begin{macrocode}
\let\HDorg@theglossary\theglossary
%    \end{macrocode}
%    \end{macro}
%    \begin{macro}{\theglossary}
%    \begin{macrocode}
\def\theglossary{%
  \let\HDorg@glossary@prologue\glossary@prologue
  \let\HDorg@writebookmark\Hy@writebookmark
  \HD@sectionpatch
  \def\glossary@prologue{%
    \ifHD@numbered
    \else
      \begingroup
        \let\Hy@writebookmark\HDorg@writebookmark
        \stepcounter{HD@unique}%
        \pdfbookmark[\HD@guesstoclevel{\HDorg@glossary@prologue}]%
            {\changehistoryname}{glossary\theHD@unique}%
      \endgroup
    \fi
    \HDorg@glossary@prologue
  }%
  \HDorg@theglossary
}
%    \end{macrocode}
%    \end{macro}
%    \begin{macro}{\glossaryname}
%    \begin{macrocode}
\providecommand*{\glossaryname}{Change History}
%    \end{macrocode}
%    \end{macro}
%
%    \begin{macro}{\glossaryname}
%    \begin{macrocode}
\providecommand*{\changehistoryname}{\glossaryname}
%    \end{macrocode}
%    \end{macro}
%
% \subsection{Index}
%
%    \begin{macrocode}
\newcounter{HD@hypercount}
%    \end{macrocode}
%    \begin{macro}{\HD@helplength}
%    \begin{macrocode}
\newlength{\HD@helplength}
%    \end{macrocode}
%    \end{macro}
%    \begin{macro}{\ifHD@savedest}
%    \begin{macrocode}
\newif\ifHD@savedest
\HD@savedesttrue
%    \end{macrocode}
%    \end{macro}
%    \begin{macrocode}
\AtBeginDocument{%
  \HD@savedest@init
}
%    \end{macrocode}
%    \begin{macrocode}
\RequirePackage{rerunfilecheck}[2009/12/10]
%    \end{macrocode}
%    \begin{macro}{\HD@savedest@init}
%    \begin{macrocode}
\begingroup
  \catcode`\^^M=\active
\@firstofone{\endgroup
  \def\HD@savedest@init{%
    \ifHD@savedest
      \let\HDorg@tilde~%
      \let~\HD@savedest@entry
      \edef\HDorg@catcodeM{\the\catcode`\^^M}%
      \catcode`\^^M=\active
      \def^^M{}%
      \InputIfFileExists{\jobname.hd}{}{}%
      \let~\HDorg@tilde
      \catcode`\^^M=\HDorg@catcodeM\relax
      \newwrite\HD@savedest@out
      \RerunFileCheck{\jobname.hd}{%
        \immediate\closeout\HD@savedest@out
      }{}%
      \immediate\openout\HD@savedest@out=\jobname.hd\relax
    \else
      \let\HD@savedest@add\@gobble
    \fi
  }%
%    \end{macrocode}
%    \end{macro}
%    \begin{macro}{\HD@savedest@entry}
%    \begin{macrocode}
  \def\HD@savedest@entry#1^^M{%
    \@namedef{HD.#1}{}%
  }%
}
%    \end{macrocode}
%    \end{macro}
%    \begin{macro}{\HD@savedest@add}
%    \begin{macrocode}
\def\HD@savedest@add#1{%
  \immediate\write\HD@savedest@out{\string~#1}%
}
%    \end{macrocode}
%    \end{macro}
%    \begin{macro}{\ifHD@targetraise}
%    \begin{macrocode}
\newif\ifHD@targetraise
\HD@targetraisetrue
%    \end{macrocode}
%    \end{macro}
%    \begin{macro}{\HD@target}
%    \begin{macrocode}
\def\HD@target{%
  \ifHD@targetraise
    \ifhmode
      \settoheight{\HD@helplength}{[H]}%
      \addtolength{\HD@helplength}{0.4ex}%
      \raisebox{\HD@helplength}[0pt][0pt]{%
        \HD@maketarget
      }%
    \else
      \HD@maketarget
    \fi
  \else
    \HD@maketarget
  \fi
}
%    \end{macrocode}
%    \end{macro}
%    \begin{macro}{\HD@maketarget}
%    \begin{macrocode}
\def\HD@maketarget{%
  \stepcounter{HD@hypercount}%
  \def\HD@next{%
    \llap{\hypertarget{HD.\the\c@HD@hypercount}{}\,}%
  }%
  \ifHD@savedest
    \expandafter\ifx\csname HD.\the\c@HD@hypercount\endcsname\@empty
    \else
      \let\HD@next\relax
    \fi
  \fi
  \HD@next
}
%    \end{macrocode}
%    \end{macro}
%
%    \begin{macro}{\CodelineIndex}
%    \begin{macrocode}
\g@addto@macro\CodelineIndex{%
  \let\special@index\HD@codeline@wrindex
}
%    \end{macrocode}
%    \end{macro}
%    \begin{macro}{\PageIndex}
%    \begin{macrocode}
\g@addto@macro\PageIndex{%
  \let\special@index\HD@page@wrindex
}
%    \end{macrocode}
%    \end{macro}
%
%    \begin{macro}{\HD@codeline@wrindex}
%    \begin{macrocode}
\def\HD@codeline@wrindex#1{%
  \begingroup
    \let\HDorg@encapchar\encapchar
    \def\encapchar##1\encapchar##2\@nil{%
      \HDorg@encapchar
      hdclindex{\the\c@HD@hypercount}{##1}%
    }%
    \codeline@wrindex{%
      #1\encapchar\encapchar\@nil
    }%
  \endgroup
}
%    \end{macrocode}
%    \end{macro}
%    \begin{macro}{\HD@page@wrindex}
%    \begin{macrocode}
\def\HD@page@wrindex#1{%
  \begingroup
    \let\HDorg@encapchar\encapchar
    \def\encapchar##1\encapchar##2\@nil{%
      \HDorg@encapchar
      hdpindex{##1}%
    }%
    \HDorg@index{#1\encapchar\encapchar\@nil}%
  \endgroup
}
%    \end{macrocode}
%    \end{macro}
%
%    \begin{macro}{\hdclindex}
%    \begin{macrocode}
\def\hdclindex#1#2#3{%
  \csname\ifx\\#2\\relax\else#2\fi\endcsname{%
    \hyperlink{HD.#1}{#3}%
    \HD@savedest@add{#1}%
  }%
}
%    \end{macrocode}
%    \end{macro}
%    \begin{macro}{\hdpindex}
%    \begin{macrocode}
\def\hdpindex#1#2{%
  \csname\ifx\\#1\\relax\else#1\fi\endcsname{%
    \hyperpage{#2}%
  }%
}
%    \end{macrocode}
%    \end{macro}
%
% \subsection{\cs{PrintDescribeMacro}}
%
%    \begin{macro}{\HDorg@PrintDescribeMacro}
%    \begin{macrocode}
\let\HDorg@PrintDescribeMacro\PrintDescribeMacro
%    \end{macrocode}
%    \end{macro}
%    \begin{macro}{\PrintDescribeMacro}
%    \begin{macrocode}
\renewcommand\PrintDescribeMacro[1]{%
  \begingroup
    \let\HDorg@MacroFont\MacroFont
    \def\MacroFont{%
      \HD@target
      \HDorg@MacroFont
    }%
    \HDorg@PrintDescribeMacro{#1}%
  \endgroup
}
%    \end{macrocode}
%    \end{macro}
%
% \subsection{\cs{PrintMacroName}}
%
%    \begin{macro}{\HDorg@PrintMacroName}
%    \begin{macrocode}
\let\HDorg@PrintMacroName\PrintMacroName
%    \end{macrocode}
%    \end{macro}
%    \begin{macro}{\PrintMacroName}
%    \begin{macrocode}
\renewcommand\PrintMacroName[1]{%
  \begingroup
    \let\HDorg@MacroFont\MacroFont
    \def\MacroFont{%
      \HD@target
      \HDorg@MacroFont
    }%
    \HDorg@PrintMacroName{#1}%
  \endgroup
}
%    \end{macrocode}
%    \end{macro}
%
% \subsection{\cs{theCodelineNo}}
%
%    \begin{macro}{\HDorg@theCodelineNo}
%    \begin{macrocode}
\let\HDorg@theCodelineNo\theCodelineNo
%    \end{macrocode}
%    \end{macro}
%    \begin{macro}{\theCodelineNo}
%    \begin{macrocode}
\renewcommand*\theCodelineNo{%
  \settowidth{\HD@helplength}{\HDorg@theCodelineNo\ }%
  \ifdim\HD@helplength<\MacroIndent
    \addtolength{\HD@helplength}{-\MacroIndent}%
    \llap{%
      \begingroup\HD@target\endgroup
      \hspace*{-\HD@helplength}%
    }%
  \else
    \HD@target
  \fi
  \HDorg@theCodelineNo
}
%    \end{macrocode}
%    \end{macro}
%
%    \begin{macro}{\HDorg@index}
%    \begin{macrocode}
\AtBeginDocument{%
  \let\HDorg@index\index
}
%    \end{macrocode}
%    \end{macro}
%
% \subsection{\cs{SpecialUsageIndex}}
%
%    \begin{macro}{\HDorg@SpecialUsageIndex}
%    \begin{macrocode}
\let\HDorg@SpecialUsageIndex\SpecialUsageIndex
\renewcommand*\SpecialUsageIndex[1]{%
  \@bsphack
  \begingroup
    \HD@target
    \let\index\HDorg@index
    \let\HDorg@encapchar\encapchar
    \edef\encapchar usage{%
       \HDorg@encapchar hdclindex{\the\c@HD@hypercount}{usage}%
    }%
    \HDorg@SpecialUsageIndex{#1}%
  \endgroup
  \@esphack
}
%    \end{macrocode}
%    \end{macro}
%
% \subsection{\cs{autoindex}}
%
%    \begin{macrocode}
\begingroup\expandafter\expandafter\expandafter\endgroup
\expandafter\ifx\csname autoindex\endcsname\relax
\else
%    \end{macrocode}
%    \begin{macro}{\HDorg@autoindex}
%    \begin{macrocode}
  \let\HDorg@autoindex\autoindex
%    \end{macrocode}
%    \end{macro}
%    \begin{macro}{\autoindex}
%    \begin{macrocode}
  \renewcommand*\autoindex[1]{%
    \@bsphack
    \begingroup
      \HD@target
      \let\index\HDorg@index
      \let\HDorg@encapchar\encapchar
      \edef\encapchar usage{%
         \HDorg@encapchar hdclindex{\the\c@HD@hypercount}{usage}%
      }%
      \HDorg@autoindex{#1}%
    \endgroup
    \@esphack
  }%
\fi
%    \end{macrocode}
%    \end{macro}
%
% \subsection{\cs{SpecialEnvIndex}}
%
%    \begin{macro}{\HDorg@SpecialEnvIndex}
%    \begin{macrocode}
\let\HDorg@SpecialEnvIndex\SpecialEnvIndex
%    \end{macrocode}
%    \end{macro}
%    \begin{macro}{\SpecialEnvIndex}
%    \begin{macrocode}
\renewcommand*\SpecialEnvIndex[1]{%
  \@bsphack
  \begingroup
    \HD@target
    \let\HDorg@encapchar\encapchar
    \edef\encapchar usage{%
      \HDorg@encapchar hdclindex{\the\c@HD@hypercount}{usage}%
    }%
    \HDorg@SpecialEnvIndex{#1}%
  \endgroup
  \@esphack
}
%    \end{macrocode}
%    \end{macro}
%
% \subsection{\cs{SortIndex}}
%
%    \begin{macro}{\HDorg@SortIndex}
%    \begin{macrocode}
\let\HDorg@SortIndex\SortIndex
%    \end{macrocode}
%    \end{macro}
%    \begin{macro}{\SortIndex}
%    \begin{macrocode}
\renewcommand*{\SortIndex}[2]{%
  \@bsphack
  \begingroup
    \let\index\HD@page@wrindex
    \HDorg@SortIndex{#1}{#2}%
  \endgroup
  \@esphack
}
%    \end{macrocode}
%    \end{macro}
%
% \subsection{\cs{@wrglossary}}
%
%    \begin{macro}{\HDorg@wrglossary}
%    \begin{macrocode}
\let\HDorg@wrglossary\@wrglossary
%    \end{macrocode}
%    \end{macro}
%    \begin{macro}{\@wrglossary}
%    \begin{macrocode}
\def\@wrglossary#1{%
  \let\HDorg@encapchar\encapchar
  \def\encapchar##1\encapchar##2\@nil{%
    \HDorg@encapchar
    hdpindex{##1}%
  }%
  \HDorg@wrglossary{#1\encapchar\encapchar\@nil}%
}
%    \end{macrocode}
%    \end{macro}
%
% \subsection{\cs{MacroIndent}}
%
%    \cs{MacroIndent} automatically calculated.
%    \begin{macrocode}
\AfterLastShipout{%
  \if@filesw
    \immediate\write\@auxout{%
      \string\providecommand\string\HD@SetMacroIndent[1]{}%
    }%
    \immediate\write\@auxout{%
      \string\HD@SetMacroIndent{\number\value{CodelineNo}}%
    }%
  \fi
}
%    \end{macrocode}
%    \begin{macro}{\HD@SetMacroIndent}
%    \begin{macrocode}
\def\HD@SetMacroIndent#1{%
  \begingroup
    \value{CodelineNo}=#1\relax
    \settowidth{\MacroIndent}{\HDorg@theCodelineNo\ }%
    \global\MacroIndent\MacroIndent
  \endgroup
}
%    \end{macrocode}
%    \end{macro}
%
%    \begin{macrocode}
%</package>
%    \end{macrocode}
%
% \section{Installation}
%
% \subsection{Download}
%
% \paragraph{Package.} This package is available on
% CTAN\footnote{\url{http://ctan.org/pkg/hypdoc}}:
% \begin{description}
% \item[\CTAN{macros/latex/contrib/oberdiek/hypdoc.dtx}] The source file.
% \item[\CTAN{macros/latex/contrib/oberdiek/hypdoc.pdf}] Documentation.
% \end{description}
%
%
% \paragraph{Bundle.} All the packages of the bundle `oberdiek'
% are also available in a TDS compliant ZIP archive. There
% the packages are already unpacked and the documentation files
% are generated. The files and directories obey the TDS standard.
% \begin{description}
% \item[\CTAN{install/macros/latex/contrib/oberdiek.tds.zip}]
% \end{description}
% \emph{TDS} refers to the standard ``A Directory Structure
% for \TeX\ Files'' (\CTAN{tds/tds.pdf}). Directories
% with \xfile{texmf} in their name are usually organized this way.
%
% \subsection{Bundle installation}
%
% \paragraph{Unpacking.} Unpack the \xfile{oberdiek.tds.zip} in the
% TDS tree (also known as \xfile{texmf} tree) of your choice.
% Example (linux):
% \begin{quote}
%   |unzip oberdiek.tds.zip -d ~/texmf|
% \end{quote}
%
% \paragraph{Script installation.}
% Check the directory \xfile{TDS:scripts/oberdiek/} for
% scripts that need further installation steps.
% Package \xpackage{attachfile2} comes with the Perl script
% \xfile{pdfatfi.pl} that should be installed in such a way
% that it can be called as \texttt{pdfatfi}.
% Example (linux):
% \begin{quote}
%   |chmod +x scripts/oberdiek/pdfatfi.pl|\\
%   |cp scripts/oberdiek/pdfatfi.pl /usr/local/bin/|
% \end{quote}
%
% \subsection{Package installation}
%
% \paragraph{Unpacking.} The \xfile{.dtx} file is a self-extracting
% \docstrip\ archive. The files are extracted by running the
% \xfile{.dtx} through \plainTeX:
% \begin{quote}
%   \verb|tex hypdoc.dtx|
% \end{quote}
%
% \paragraph{TDS.} Now the different files must be moved into
% the different directories in your installation TDS tree
% (also known as \xfile{texmf} tree):
% \begin{quote}
% \def\t{^^A
% \begin{tabular}{@{}>{\ttfamily}l@{ $\rightarrow$ }>{\ttfamily}l@{}}
%   hypdoc.sty & tex/latex/oberdiek/hypdoc.sty\\
%   hypdoc.pdf & doc/latex/oberdiek/hypdoc.pdf\\
%   hypdoc.dtx & source/latex/oberdiek/hypdoc.dtx\\
% \end{tabular}^^A
% }^^A
% \sbox0{\t}^^A
% \ifdim\wd0>\linewidth
%   \begingroup
%     \advance\linewidth by\leftmargin
%     \advance\linewidth by\rightmargin
%   \edef\x{\endgroup
%     \def\noexpand\lw{\the\linewidth}^^A
%   }\x
%   \def\lwbox{^^A
%     \leavevmode
%     \hbox to \linewidth{^^A
%       \kern-\leftmargin\relax
%       \hss
%       \usebox0
%       \hss
%       \kern-\rightmargin\relax
%     }^^A
%   }^^A
%   \ifdim\wd0>\lw
%     \sbox0{\small\t}^^A
%     \ifdim\wd0>\linewidth
%       \ifdim\wd0>\lw
%         \sbox0{\footnotesize\t}^^A
%         \ifdim\wd0>\linewidth
%           \ifdim\wd0>\lw
%             \sbox0{\scriptsize\t}^^A
%             \ifdim\wd0>\linewidth
%               \ifdim\wd0>\lw
%                 \sbox0{\tiny\t}^^A
%                 \ifdim\wd0>\linewidth
%                   \lwbox
%                 \else
%                   \usebox0
%                 \fi
%               \else
%                 \lwbox
%               \fi
%             \else
%               \usebox0
%             \fi
%           \else
%             \lwbox
%           \fi
%         \else
%           \usebox0
%         \fi
%       \else
%         \lwbox
%       \fi
%     \else
%       \usebox0
%     \fi
%   \else
%     \lwbox
%   \fi
% \else
%   \usebox0
% \fi
% \end{quote}
% If you have a \xfile{docstrip.cfg} that configures and enables \docstrip's
% TDS installing feature, then some files can already be in the right
% place, see the documentation of \docstrip.
%
% \subsection{Refresh file name databases}
%
% If your \TeX~distribution
% (\teTeX, \mikTeX, \dots) relies on file name databases, you must refresh
% these. For example, \teTeX\ users run \verb|texhash| or
% \verb|mktexlsr|.
%
% \subsection{Some details for the interested}
%
% \paragraph{Attached source.}
%
% The PDF documentation on CTAN also includes the
% \xfile{.dtx} source file. It can be extracted by
% AcrobatReader 6 or higher. Another option is \textsf{pdftk},
% e.g. unpack the file into the current directory:
% \begin{quote}
%   \verb|pdftk hypdoc.pdf unpack_files output .|
% \end{quote}
%
% \paragraph{Unpacking with \LaTeX.}
% The \xfile{.dtx} chooses its action depending on the format:
% \begin{description}
% \item[\plainTeX:] Run \docstrip\ and extract the files.
% \item[\LaTeX:] Generate the documentation.
% \end{description}
% If you insist on using \LaTeX\ for \docstrip\ (really,
% \docstrip\ does not need \LaTeX), then inform the autodetect routine
% about your intention:
% \begin{quote}
%   \verb|latex \let\install=y\input{hypdoc.dtx}|
% \end{quote}
% Do not forget to quote the argument according to the demands
% of your shell.
%
% \paragraph{Generating the documentation.}
% You can use both the \xfile{.dtx} or the \xfile{.drv} to generate
% the documentation. The process can be configured by the
% configuration file \xfile{ltxdoc.cfg}. For instance, put this
% line into this file, if you want to have A4 as paper format:
% \begin{quote}
%   \verb|\PassOptionsToClass{a4paper}{article}|
% \end{quote}
% An example follows how to generate the
% documentation with pdf\LaTeX:
% \begin{quote}
%\begin{verbatim}
%pdflatex hypdoc.dtx
%makeindex -s gind.ist hypdoc.idx
%pdflatex hypdoc.dtx
%makeindex -s gind.ist hypdoc.idx
%pdflatex hypdoc.dtx
%\end{verbatim}
% \end{quote}
%
% \section{Catalogue}
%
% The following XML file can be used as source for the
% \href{http://mirror.ctan.org/help/Catalogue/catalogue.html}{\TeX\ Catalogue}.
% The elements \texttt{caption} and \texttt{description} are imported
% from the original XML file from the Catalogue.
% The name of the XML file in the Catalogue is \xfile{hypdoc.xml}.
%    \begin{macrocode}
%<*catalogue>
<?xml version='1.0' encoding='us-ascii'?>
<!DOCTYPE entry SYSTEM 'catalogue.dtd'>
<entry datestamp='$Date$' modifier='$Author$' id='hypdoc'>
  <name>hypdoc</name>
  <caption>Hyper extensions for doc.sty.</caption>
  <authorref id='auth:oberdiek'/>
  <copyright owner='Heiko Oberdiek' year='2002,2006,2007,2009-2011'/>
  <license type='lppl1.3'/>
  <version number='1.14'/>
  <description>
    This package adds hypertext features to the package
    <xref refid='doc'>doc</xref> that is used in the documentation
    system of <xref refid='latex'>LaTeX2e</xref>. Bookmarks are added
    and references are linked as far as possible.
    <p/>
    The package is part of the <xref refid='oberdiek'>oberdiek</xref>
    bundle.
  </description>
  <documentation details='Package documentation'
      href='ctan:/macros/latex/contrib/oberdiek/hypdoc.pdf'/>
  <ctan file='true' path='/macros/latex/contrib/oberdiek/hypdoc.dtx'/>
  <miktex location='oberdiek'/>
  <texlive location='oberdiek'/>
  <install path='/macros/latex/contrib/oberdiek/oberdiek.tds.zip'/>
</entry>
%</catalogue>
%    \end{macrocode}
%
% \begin{History}
%   \begin{Version}{2002/05/10 v1.0}
%   \item
%     First standalone version.
%   \end{Version}
%   \begin{Version}{2006/02/20 v1.1}
%   \item
%     Option \xoption{numbered} added.
%   \item
%     \cs{HD@savedest@init} fixed (empty \xext{hd} file triggered
%     first page with fl ligature).
%   \item
%     Bookmarks for \cs{listoffigures} and \cs{listoftables}.
%   \item
%     Documentation: features and usage added.
%   \item
%     LPPL 1.3.
%   \end{Version}
%   \begin{Version}{2006/04/27 v1.2}
%   \item
%     \cs{HD@target} fixed (\xfile{multicol.dtx}).
%   \end{Version}
%   \begin{Version}{2006/06/01 v1.3}
%   \item
%     Support for package \xpackage{amsdtx}'s \cs{autoindex} added.
%   \end{Version}
%   \begin{Version}{2006/06/02 v1.4}
%   \item
%     Bookmark fixed if there are several table of contents,
%     bibliographies, glossaries, \dots
%   \end{Version}
%   \begin{Version}{2007/04/11 v1.5}
%   \item
%     Line ends sanitized.
%   \end{Version}
%   \begin{Version}{2007/11/12 v1.6}
%   \item
%     Bug fix in \cs{listoffigures} (Axel Sommerfeldt).
%   \item
%     Markup added in implementation section.
%   \end{Version}
%   \begin{Version}{2009/12/10 v1.7}
%   \item
%     Use of package \xpackage{rerunfilecheck}.
%   \end{Version}
%   \begin{Version}{2010/02/03 v1.8}
%   \item
%     Fix for bug latex/4096 added.
%   \end{Version}
%   \begin{Version}{2010/03/26 v1.9}
%   \item
%     Definition of \cs{HD@SetMacroIndex} is provided in the \xext{aux} file.
%   \end{Version}
%   \begin{Version}{2011/05/05 v1.10}
%   \item
%     \cs{raisebox} in \cs{HD@target} with height and depth of \texttt{0pt}
%     (request of GL).
%   \end{Version}
%   \begin{Version}{2011/08/19 v1.11}
%   \item
%     Latest comma in option list for hyperref removed to avoid option clash
%     because of \hologo{LaTeX} 2011/06/27 with the new \cs{in@}.
%   \end{Version}
%   \begin{Version}{2016/05/16 v1.12}
%   \item
%     Documentation updates.
%   \end{Version}
%   \begin{Version}{2016/10/02 v1.13}
%   \item
%     Fixes to bookmarknames, also introduce \cs{changehistoryname}
%   \end{Version}
%   \begin{Version}{2018/09/07 v1.14}
%   \item
%     Do not load \xpackage{thumpdf}.
%   \end{Version}
% \end{History}
%
% \PrintIndex
%
% \Finale
\endinput

%        (quote the arguments according to the demands of your shell)
%
% Documentation:
%    (a) If hypdoc.drv is present:
%           latex hypdoc.drv
%    (b) Without hypdoc.drv:
%           latex hypdoc.dtx; ...
%    The class ltxdoc loads the configuration file ltxdoc.cfg
%    if available. Here you can specify further options, e.g.
%    use A4 as paper format:
%       \PassOptionsToClass{a4paper}{article}
%
%    Programm calls to get the documentation (example):
%       pdflatex hypdoc.dtx
%       makeindex -s gind.ist hypdoc.idx
%       pdflatex hypdoc.dtx
%       makeindex -s gind.ist hypdoc.idx
%       pdflatex hypdoc.dtx
%
% Installation:
%    TDS:tex/latex/oberdiek/hypdoc.sty
%    TDS:doc/latex/oberdiek/hypdoc.pdf
%    TDS:source/latex/oberdiek/hypdoc.dtx
%
%<*ignore>
\begingroup
  \catcode123=1 %
  \catcode125=2 %
  \def\x{LaTeX2e}%
\expandafter\endgroup
\ifcase 0\ifx\install y1\fi\expandafter
         \ifx\csname processbatchFile\endcsname\relax\else1\fi
         \ifx\fmtname\x\else 1\fi\relax
\else\csname fi\endcsname
%</ignore>
%<*install>
\input docstrip.tex
\Msg{************************************************************************}
\Msg{* Installation}
\Msg{* Package: hypdoc 2018/09/07 v1.14 Hyper extensions for doc.sty (HO)}
\Msg{************************************************************************}

\keepsilent
\askforoverwritefalse

\let\MetaPrefix\relax
\preamble

This is a generated file.

Project: hypdoc
Version: 2018/09/07 v1.14

Copyright (C) 2002, 2006, 2007, 2009-2011 by
   Heiko Oberdiek <heiko.oberdiek at googlemail.com>

This work may be distributed and/or modified under the
conditions of the LaTeX Project Public License, either
version 1.3c of this license or (at your option) any later
version. This version of this license is in
   http://www.latex-project.org/lppl/lppl-1-3c.txt
and the latest version of this license is in
   http://www.latex-project.org/lppl.txt
and version 1.3 or later is part of all distributions of
LaTeX version 2005/12/01 or later.

This work has the LPPL maintenance status "maintained".

This Current Maintainer of this work is Heiko Oberdiek.

This work consists of the main source file hypdoc.dtx
and the derived files
   hypdoc.sty, hypdoc.pdf, hypdoc.ins, hypdoc.drv.

\endpreamble
\let\MetaPrefix\DoubleperCent

\generate{%
  \file{hypdoc.ins}{\from{hypdoc.dtx}{install}}%
  \file{hypdoc.drv}{\from{hypdoc.dtx}{driver}}%
  \usedir{tex/latex/oberdiek}%
  \file{hypdoc.sty}{\from{hypdoc.dtx}{package}}%
  \nopreamble
  \nopostamble
%  \usedir{source/latex/oberdiek/catalogue}%
%  \file{hypdoc.xml}{\from{hypdoc.dtx}{catalogue}}%
}

\catcode32=13\relax% active space
\let =\space%
\Msg{************************************************************************}
\Msg{*}
\Msg{* To finish the installation you have to move the following}
\Msg{* file into a directory searched by TeX:}
\Msg{*}
\Msg{*     hypdoc.sty}
\Msg{*}
\Msg{* To produce the documentation run the file `hypdoc.drv'}
\Msg{* through LaTeX.}
\Msg{*}
\Msg{* Happy TeXing!}
\Msg{*}
\Msg{************************************************************************}

\endbatchfile
%</install>
%<*ignore>
\fi
%</ignore>
%<*driver>
\NeedsTeXFormat{LaTeX2e}
\ProvidesFile{hypdoc.drv}%
  [2018/09/07 v1.14 Hyper extensions for doc.sty (HO)]%
\documentclass{ltxdoc}
\usepackage{holtxdoc}[2011/11/22]
\begin{document}
  \DocInput{hypdoc.dtx}%
\end{document}
%</driver>
% \fi
%
%
% \CharacterTable
%  {Upper-case    \A\B\C\D\E\F\G\H\I\J\K\L\M\N\O\P\Q\R\S\T\U\V\W\X\Y\Z
%   Lower-case    \a\b\c\d\e\f\g\h\i\j\k\l\m\n\o\p\q\r\s\t\u\v\w\x\y\z
%   Digits        \0\1\2\3\4\5\6\7\8\9
%   Exclamation   \!     Double quote  \"     Hash (number) \#
%   Dollar        \$     Percent       \%     Ampersand     \&
%   Acute accent  \'     Left paren    \(     Right paren   \)
%   Asterisk      \*     Plus          \+     Comma         \,
%   Minus         \-     Point         \.     Solidus       \/
%   Colon         \:     Semicolon     \;     Less than     \<
%   Equals        \=     Greater than  \>     Question mark \?
%   Commercial at \@     Left bracket  \[     Backslash     \\
%   Right bracket \]     Circumflex    \^     Underscore    \_
%   Grave accent  \`     Left brace    \{     Vertical bar  \|
%   Right brace   \}     Tilde         \~}
%
% \GetFileInfo{hypdoc.drv}
%
% \title{The \xpackage{hypdoc} package}
% \date{2018/09/07 v1.14}
% \author{Heiko Oberdiek\thanks
% {Please report any issues at https://github.com/ho-tex/oberdiek/issues}\\
% \xemail{heiko.oberdiek at googlemail.com}}
%
% \maketitle
%
% \begin{abstract}
% This package adds hyper features to the package
% \xpackage{doc} that is used in the documentation
% system of \LaTeXe. Bookmarks are added and references
% are linked as far as possible.
% \end{abstract}
%
% \tableofcontents
%
% \section{Features}
%
% The package adds the following features:
% \begin{description}
% \item[Bookmarks:]
%   For supporting bookmarks package \verb|hyperref| is loaded.
%   Some \TeX\ code inside section titles can cause problems
%   during bookmark processing; these can be warnings, ugly
%   bookmark titles or even errors. With the help of
%   \verb|\pdfstringdefDisableCommands| or \verb|\texorpdfstring|
%   you can manage these things, see \verb|hyperref|'s package
%   documentation.
% \item[Additional bookmarks:]
%   Unnumbered bookmark entries are generated for:
%   \begin{itemize}
%   \item Table of contents
%   \item List of figures
%   \item List of tables
%   \item Index
%   \item Glossary
%   \end{itemize}
%   Option \texttt{numbered} controls, whether these section or
%   chapters should be numbered. The exception is the table
%   of contents if this it would be the first numbered
%   section. Then just a bookmark is added.
% \item[Metadata:]
%   The loaded package \verb|hyperref|, enforced by
%   option \verb|pdfusetitle|, tries to detect \verb|\title|
%   and \verb|\author| and sets the corresponding PDF
%   information fields. In case of problems see the previous
%   item about bookmarks.
%   These enries can be overwritten or new ones added by
%   \verb|\hypersetup| after the packages is loaded, see
%   documentation of package \verb|hyperref|.
% \item[References, links:] Because of package \verb|\hyperref|
%   |\ref| become links, also \verb|\url|, footnote referencing
%   are supported.
% \item[Index:]
%   The main part of this package deals with index links in order
%   to support the kind of index that package \verb|doc| provides.
% \item[Thumbnails:]
%   Package \verb|thumbpdf| may additionally be loaded for the case that the
%   PDF file also should contain thumbnails, see the documentation
%   of this package. But thumbnail embedding is not really
%   necessary, because nowadays AcrobatReader is able to
%   generate and view thumbnails on the fly.
% \end{description}
%
% \section{Usage}
%
% The purpose of this packages is to add PDF features to
% the documentation of LaTeX packages. Thus just load
% the package in the driver file that generates the
% documentation:
% \begin{quote}
%   \verb|\usepackage{hypdoc}|
% \end{quote}
% Package \verb|doc| is loaded by \verb|hypdoc|, thus you
% also can replace a \verb|\usepackage{doc}| with this line.
% At any case, the package must be loaded before macros of
% package \verb|doc| such as \verb|\CodelineIndex| or
% \verb|\PageIndex| are used.
%
% If the class \verb|ltxdoc| is used, you can also use
% the configuration file for this class to add
% package \verb|hypdoc|. Add the following line
% to the file \verb|ltxdoc.cfg| (or create a new one):
% \begin{quote}
%\begin{verbatim}
%\AtEndOfClass{\RequirePackage{hypdoc}}
%\end{verbatim}
% \end{quote}
%
% \subsection{Options}
%
% There is one option \texttt{numbered}. It controls,
% whether the starred versions of section or chapter
% inside \cs{tableofcontents}, \cs{listoffigures}, \dots
% should be replaced by the non-star versions.
%
% \subsection{Simple example}
% Example of a simple driver file \verb|foobar.drv|
% of a package \verb|foobar|. Often the driver file
% is packed in the file \verb|foobar.dtx| and can
% be extracted by docstrip or if an instruction file
% for docstrip is provided by:
% \begin{quote}
% \verb|tex foobar.ins|
% \end{quote}
% Note that although docstrip doesn't rely on \LaTeX,
% sometimes you need to run the \verb|.ins| file through
% \LaTeX. Oddly enough, often the only \LaTeX\ feature
% in use is \verb|\NeedsTeXFormat{LaTeX2e}| in the first line.
% \begin{quote}
%\begin{verbatim}
%\documentclass{ltxdoc}
%\usepackage{hypdoc}
%\RecordChanges
%\EnableCrossrefs
%\CodelineIndex
%\begin{document}
%\title{Title for package \texttt{foobar}}
%\author{Author of package \texttt{foobar}}
%\date{...}
%\maketitle
%\tableofcontents
%\DocInput{foobar.dtx}
%\PrintChanges
%\PrintIndex
%\end{document}
%\end{verbatim}
% \end{quote}
%
% Then compile it, for example with pdf\LaTeX.
% \begin{quote}
%\begin{verbatim}
%pdflatex foobar.drv
%makeindex -s gind.ist foobar.idx
%makeindex -s gglo.ist -o foobar.gls foobar.glo
%pdflatex foobar.drv
%\end{verbatim}
% \end{quote}
% You will need several cycles until the cross references
% and similar things are correct.
%
% \subsection{Example \texttt{source2e.tex}}
%
% The source code of the base of \LaTeX\ is available as
% a bunch of \verb|.dtx| files. \LaTeX\ provides the file
% \verb|source2e.tex| to merge them all together.
%
% Now you can either load package \verb|hypdoc| in a private copy
% of this file (please, respect the LPPL) or use the configuration
% file \verb|ltxdoc.cfg|. Example that also sets A4 paper size:
% \begin{quote}
%\begin{verbatim}
%\PassOptionsToClass{a4paper}{article}
%\AtEndOfClass{\RequirePackage{hypdoc}}
%\end{verbatim}
% \end{quote}
%
% Then \verb|source2e.tex| can be compiled by pdf\TeX, for example:
% \begin{quote}
%\begin{verbatim}
%pdflatex source2e
%makeindex -s gind.ist source2e.idx
%makeindex -s gglo.ist -o source2e.glo source2e.gls
%pdflatex source2e
%makeindex -s gind.ist source2e.idx
%makeindex -s gglo.ist -o source2e.glo source2e.gls
%pdflatex source2e
%\end{verbatim}
% \end{quote}
%
% \StopEventually{
% }
%
% \section{Implementation}
%
%    \begin{macrocode}
%<*package>
\NeedsTeXFormat{LaTeX2e}
\ProvidesPackage{hypdoc}%
  [2018/09/07 v1.14 Hyper extensions for doc.sty (HO)]%
%    \end{macrocode}
%    The package identification is done at the top of the |.dtx| file
%    in order to use only one identification string.
%
%    For unique command names this package uses |HD@| as prefix
%
% \subsection{Options}
%
%    \begin{macro}{\ifHD@numbered}
%    \begin{macrocode}
\newif\ifHD@numbered
%    \end{macrocode}
%    \end{macro}
%    \begin{macrocode}
\DeclareOption{numbered}{\HD@numberedtrue}
\ProcessOptions*\relax
%    \end{macrocode}
%
% \subsection{Package loading}
%
%    \begin{macrocode}
\RequirePackage{atveryend}[2010/03/24]
\RequirePackage{doc}
%    \end{macrocode}
%    \begin{macrocode}
\ifx\numexpr\@undefined
  \RequirePackage{calc}%
\fi
%    \end{macrocode}
%    \begin{macrocode}
\RequirePackage[%
  hyperindex=false,%
  pdfusetitle,%
  colorlinks,%
  pdfpagelabels%
]{hyperref}[2002/05/09]%
\hypersetup{%
  plainpages=false,%
  bookmarksopen,%
  bookmarksnumbered,%
  bookmarksopenlevel=0,%
  pdfstartview={FitBH \hypercalcbp{%
    \paperheight-\topmargin-1in-\headheight-\headsep
  }},%
}
%    \end{macrocode}
%
%    \begin{macrocode}
\newcounter{HD@unique}
%    \end{macrocode}
%    \begin{macro}{\theHD@unique}
%    \begin{macrocode}
\renewcommand{\theHD@unique}{%
  \number\value{HD@unique}%
}
%    \end{macrocode}
%    \end{macro}
%
%    \begin{macrocode}
\g@addto@macro\Hy@UseMaketitleInfos{%
  \@ifundefined{fileinfo}{}{%
    \ifx\@pdfsubject\@empty
      \hypersetup{pdfsubject={\fileinfo}}%
    \fi
  }%
}
%    \end{macrocode}
%
%    \begin{macrocode}
\pdfstringdefDisableCommands{%
  \let\thanks\@gobble
  \let\footnote\@gobble
  \def\cs#1{\textbackslash #1}%
  \let\normalfont\@empty
  \let\scshape\@empty
  \def\and{and }%
}
%    \end{macrocode}
%
% \subsubsection{Patch for bug latex/4096}
%
%    \begin{macrocode}
\begingroup\expandafter\expandafter\expandafter\endgroup
\expandafter\ifx\csname mod@math@codes\endcsname\relax
\else
  \def\HD@ModFix#1{%
    \mathcode`#1=`#1\relax
  }%
  \g@addto@macro\mod@math@codes{%
    \HD@ModFix\-%
    \HD@ModFix\+%
    \HD@ModFix\:%
    \HD@ModFix\=%
    \HD@ModFix\*%
  }%
\fi
%    \end{macrocode}
%
% \subsection{Bookmarks}
%
%    \begin{macro}{\HD@sectionpatch}
%    \begin{macrocode}
\def\HD@sectionpatch{%
  \ifHD@numbered
    \HD@@sectionpatch{section}%
    \HD@@sectionpatch{chapter}%
  \else
    \let\Hy@writebookmark\HD@disable@writebookmark
  \fi
}
%    \end{macrocode}
%    \end{macro}
%    \begin{macro}{\HD@@sectionpatch}
%    \begin{macrocode}
\def\HD@@sectionpatch#1{%
  \expandafter\let\csname HDorg@@#1\expandafter\endcsname
    \csname #1\endcsname
  \@namedef{#1}{%
    \@ifstar{\@nameuse{HDorg@@#1}}{\@nameuse{HDorg@@#1}}%
  }%
}
%    \end{macrocode}
%    \end{macro}
%    \begin{macro}{\HD@disable@writebookmark}
%    \begin{macrocode}
\def\HD@disable@writebookmark#1#2#3#4#5{}
%    \end{macrocode}
%    \end{macro}
%
%    \begin{macro}{\HD@guesstoclevel}
%    A wild guessing of the toclevel. There is the hope, that
%    the macro |#1| starts with a sectioning command.
%    \begin{macrocode}
\def\HD@guesstoclevel#1{%
  \expandafter\HD@@guesstoclevel#1\section\HD@end
}
%    \end{macrocode}
%    \end{macro}
%    \begin{macro}{\HD@@guesstoclevel}
%    \begin{macrocode}
\def\HD@@guesstoclevel#1#2\HD@end{%
  \@ifundefined{toclevel@\expandafter\@gobble\string#1\@empty}{%
    1%
  }{%
    \@nameuse{toclevel@\expandafter\@gobble\string#1\@empty}%
  }%
}
%    \end{macrocode}
%    \end{macro}
%
% \subsection{\cs{tableofcontents}}
%
%    \begin{macro}{\HDorg@tableofcontents}
%    \begin{macrocode}
\let\HDorg@tableofcontents\tableofcontents
%    \end{macrocode}
%    \end{macro}
%    \begin{macro}{\tableofcontents}
%    As first section the table of contents shouldn't
%    go into the toc itself. Also a number looks too funny then.
%    Just a bookmark is enough in this case.
%    \begin{macrocode}
\def\tableofcontents{%
  \begingroup
    \@ifundefined{c@section}{%
      \def\c@section{0}%
    }{}%
  \expandafter\endgroup
  \ifcase\c@section
    \stepcounter{HD@unique}%
    \pdfbookmark[\HD@guesstoclevel{\HDorg@tableofcontents}]%
        {\contentsname}{toc\theHD@unique}%
    \HDorg@tableofcontents
  \else
    \ifHD@numbered
    \else
      \stepcounter{HD@unique}%
      \pdfbookmark[\HD@guesstoclevel{\HDorg@tableofcontents}]%
          {\contentsname}{toc\theHD@unique}%
    \fi
    \begingroup
      \HD@sectionpatch
      \HDorg@tableofcontents
    \endgroup
  \fi
}
%    \end{macrocode}
%    \end{macro}
%
% \subsection{\cs{listoffigures}}
%
%    \begin{macro}{\HDorg@listoffigures}
%    \begin{macrocode}
\let\HDorg@listoffigures\listoffigures
%    \end{macrocode}
%    \end{macro}
%    \begin{macro}{\listoffigures}
%    \begin{macrocode}
\def\listoffigures{%
  \ifHD@numbered
  \else
    \stepcounter{HD@unique}%
    \pdfbookmark[\HD@guesstoclevel{\HDorg@listoffigures}]%
        {\listfigurename}{toc\theHD@unique}%
  \fi
  \begingroup
    \HD@sectionpatch
    \HDorg@listoffigures
  \endgroup
}
%    \end{macrocode}
%    \end{macro}
%
% \subsection{\cs{listoftables}}
%
%    \begin{macro}{\HDorg@listoftables}
%    \begin{macrocode}
\let\HDorg@listoftables\listoftables
%    \end{macrocode}
%    \end{macro}
%    \begin{macro}{\listoftables}
%    \begin{macrocode}
\def\listoftables{%
  \ifHD@numbered
  \else
    \stepcounter{HD@unique}%
    \pdfbookmark[\HD@guesstoclevel{\HDorg@listoftables}]%
        {\listtablename}{toc\theHD@unique}%
  \fi
  \begingroup
    \HD@sectionpatch
    \HDorg@listoftables
  \endgroup
}
%    \end{macrocode}
%    \end{macro}
%
% \subsection{\cs{thebibliography}}
%
%    \begin{macro}{\HDorg@thebibliography}
%    \begin{macrocode}
\let\HDorg@thebibliography\thebibliography
%    \end{macrocode}
%    \end{macro}
%    \begin{macro}{\thebibliography}
%    \begin{macrocode}
\def\thebibliography{%
  \ifHD@numbered
  \else
    \stepcounter{HD@unique}%
    \pdfbookmark[\HD@guesstoclevel{\HDorg@thebibliography}]%
      {\@ifundefined{chapter}{\refname}{\bibname}}{bib\theHD@unique}%
  \fi
  \HD@sectionpatch
  \HDorg@thebibliography
}
%    \end{macrocode}
%    \end{macro}
%
% \subsection{\cs{theindex}}
%
%    \begin{macro}{\HDorg@theindex}
%    \begin{macrocode}
\let\HDorg@theindex\theindex
%    \end{macrocode}
%    \end{macro}
%    \begin{macro}{\theindex}
%    \begin{macrocode}
\def\theindex{%
  \let\HDorg@index@prologue\index@prologue
  \let\HDorg@writebookmark\Hy@writebookmark
  \HD@sectionpatch
  \g@addto@macro\IndexParms{%
    \ifHD@numbered
    \else
      \let\Hy@writebookmark\HDorg@writebookmark
    \fi
    \let\bfseries\HD@bfseries
  }%
  \edef\HD@toclevel@index{\HD@guesstoclevel{\index@prologue}}%
  \begingroup
    \count@=\HD@toclevel@index\relax
    \advance\count@ by 1 %
    \xdef\HD@toclevel@subindex{\the\count@}%
  \endgroup
  \def\index@prologue{%
    \ifHD@numbered
    \else
      \begingroup
        \stepcounter{HD@unique}%
        \let\Hy@writebookmark\HDorg@writebookmark
        \pdfbookmark[\HD@toclevel@index]%
            {\indexname}{index\theHD@unique}%
      \endgroup
    \fi
    \HDorg@index@prologue
  }%
  \HDorg@theindex
}
%    \end{macrocode}
%    \end{macro}
%    \begin{macro}{\HDorg@bfseries}
%    \begin{macrocode}
\let\HDorg@bfseries\bfseries
%    \end{macrocode}
%    \end{macro}
%    \begin{macro}{\HD@bfseries}
%    \begin{macrocode}
\def\HD@bfseries{\futurelet\HD@hfil\HD@@bfseries}%
%    \end{macrocode}
%    \end{macro}
%    \begin{macro}{\HD@@bfseries}
%    \begin{macrocode}
\def\HD@@bfseries{%
  \HDorg@bfseries
  \ifx\HD@hfil\hfil
    \expandafter\HD@@@bfseries
  \fi
}
%    \end{macrocode}
%    \end{macro}
%    \begin{macro}{\HD@@@bfseries}
%    \begin{macrocode}
\def\HD@@@bfseries\hfil#1\hfil{%
  \ifx\\#1\\%
  \else
    \raisebox{\baselineskip}[0pt]{%
      \kern-\HD@margin\relax
      \pdfbookmark[\HD@toclevel@subindex]{#1}{HD.#1}%
      \kern\HD@margin\relax
    }%
  \fi
  \hfil#1\hfil
}%
%    \end{macrocode}
%    \end{macro}
%    \begin{macro}{\HD@margin}
%    \begin{macrocode}
\def\HD@margin{1mm}
%    \end{macrocode}
%    \end{macro}
%
% \subsection{\cs{theglossary}}
%
%    \begin{macro}{\HDorg@theglossary}
%    \begin{macrocode}
\let\HDorg@theglossary\theglossary
%    \end{macrocode}
%    \end{macro}
%    \begin{macro}{\theglossary}
%    \begin{macrocode}
\def\theglossary{%
  \let\HDorg@glossary@prologue\glossary@prologue
  \let\HDorg@writebookmark\Hy@writebookmark
  \HD@sectionpatch
  \def\glossary@prologue{%
    \ifHD@numbered
    \else
      \begingroup
        \let\Hy@writebookmark\HDorg@writebookmark
        \stepcounter{HD@unique}%
        \pdfbookmark[\HD@guesstoclevel{\HDorg@glossary@prologue}]%
            {\changehistoryname}{glossary\theHD@unique}%
      \endgroup
    \fi
    \HDorg@glossary@prologue
  }%
  \HDorg@theglossary
}
%    \end{macrocode}
%    \end{macro}
%    \begin{macro}{\glossaryname}
%    \begin{macrocode}
\providecommand*{\glossaryname}{Change History}
%    \end{macrocode}
%    \end{macro}
%
%    \begin{macro}{\glossaryname}
%    \begin{macrocode}
\providecommand*{\changehistoryname}{\glossaryname}
%    \end{macrocode}
%    \end{macro}
%
% \subsection{Index}
%
%    \begin{macrocode}
\newcounter{HD@hypercount}
%    \end{macrocode}
%    \begin{macro}{\HD@helplength}
%    \begin{macrocode}
\newlength{\HD@helplength}
%    \end{macrocode}
%    \end{macro}
%    \begin{macro}{\ifHD@savedest}
%    \begin{macrocode}
\newif\ifHD@savedest
\HD@savedesttrue
%    \end{macrocode}
%    \end{macro}
%    \begin{macrocode}
\AtBeginDocument{%
  \HD@savedest@init
}
%    \end{macrocode}
%    \begin{macrocode}
\RequirePackage{rerunfilecheck}[2009/12/10]
%    \end{macrocode}
%    \begin{macro}{\HD@savedest@init}
%    \begin{macrocode}
\begingroup
  \catcode`\^^M=\active
\@firstofone{\endgroup
  \def\HD@savedest@init{%
    \ifHD@savedest
      \let\HDorg@tilde~%
      \let~\HD@savedest@entry
      \edef\HDorg@catcodeM{\the\catcode`\^^M}%
      \catcode`\^^M=\active
      \def^^M{}%
      \InputIfFileExists{\jobname.hd}{}{}%
      \let~\HDorg@tilde
      \catcode`\^^M=\HDorg@catcodeM\relax
      \newwrite\HD@savedest@out
      \RerunFileCheck{\jobname.hd}{%
        \immediate\closeout\HD@savedest@out
      }{}%
      \immediate\openout\HD@savedest@out=\jobname.hd\relax
    \else
      \let\HD@savedest@add\@gobble
    \fi
  }%
%    \end{macrocode}
%    \end{macro}
%    \begin{macro}{\HD@savedest@entry}
%    \begin{macrocode}
  \def\HD@savedest@entry#1^^M{%
    \@namedef{HD.#1}{}%
  }%
}
%    \end{macrocode}
%    \end{macro}
%    \begin{macro}{\HD@savedest@add}
%    \begin{macrocode}
\def\HD@savedest@add#1{%
  \immediate\write\HD@savedest@out{\string~#1}%
}
%    \end{macrocode}
%    \end{macro}
%    \begin{macro}{\ifHD@targetraise}
%    \begin{macrocode}
\newif\ifHD@targetraise
\HD@targetraisetrue
%    \end{macrocode}
%    \end{macro}
%    \begin{macro}{\HD@target}
%    \begin{macrocode}
\def\HD@target{%
  \ifHD@targetraise
    \ifhmode
      \settoheight{\HD@helplength}{[H]}%
      \addtolength{\HD@helplength}{0.4ex}%
      \raisebox{\HD@helplength}[0pt][0pt]{%
        \HD@maketarget
      }%
    \else
      \HD@maketarget
    \fi
  \else
    \HD@maketarget
  \fi
}
%    \end{macrocode}
%    \end{macro}
%    \begin{macro}{\HD@maketarget}
%    \begin{macrocode}
\def\HD@maketarget{%
  \stepcounter{HD@hypercount}%
  \def\HD@next{%
    \llap{\hypertarget{HD.\the\c@HD@hypercount}{}\,}%
  }%
  \ifHD@savedest
    \expandafter\ifx\csname HD.\the\c@HD@hypercount\endcsname\@empty
    \else
      \let\HD@next\relax
    \fi
  \fi
  \HD@next
}
%    \end{macrocode}
%    \end{macro}
%
%    \begin{macro}{\CodelineIndex}
%    \begin{macrocode}
\g@addto@macro\CodelineIndex{%
  \let\special@index\HD@codeline@wrindex
}
%    \end{macrocode}
%    \end{macro}
%    \begin{macro}{\PageIndex}
%    \begin{macrocode}
\g@addto@macro\PageIndex{%
  \let\special@index\HD@page@wrindex
}
%    \end{macrocode}
%    \end{macro}
%
%    \begin{macro}{\HD@codeline@wrindex}
%    \begin{macrocode}
\def\HD@codeline@wrindex#1{%
  \begingroup
    \let\HDorg@encapchar\encapchar
    \def\encapchar##1\encapchar##2\@nil{%
      \HDorg@encapchar
      hdclindex{\the\c@HD@hypercount}{##1}%
    }%
    \codeline@wrindex{%
      #1\encapchar\encapchar\@nil
    }%
  \endgroup
}
%    \end{macrocode}
%    \end{macro}
%    \begin{macro}{\HD@page@wrindex}
%    \begin{macrocode}
\def\HD@page@wrindex#1{%
  \begingroup
    \let\HDorg@encapchar\encapchar
    \def\encapchar##1\encapchar##2\@nil{%
      \HDorg@encapchar
      hdpindex{##1}%
    }%
    \HDorg@index{#1\encapchar\encapchar\@nil}%
  \endgroup
}
%    \end{macrocode}
%    \end{macro}
%
%    \begin{macro}{\hdclindex}
%    \begin{macrocode}
\def\hdclindex#1#2#3{%
  \csname\ifx\\#2\\relax\else#2\fi\endcsname{%
    \hyperlink{HD.#1}{#3}%
    \HD@savedest@add{#1}%
  }%
}
%    \end{macrocode}
%    \end{macro}
%    \begin{macro}{\hdpindex}
%    \begin{macrocode}
\def\hdpindex#1#2{%
  \csname\ifx\\#1\\relax\else#1\fi\endcsname{%
    \hyperpage{#2}%
  }%
}
%    \end{macrocode}
%    \end{macro}
%
% \subsection{\cs{PrintDescribeMacro}}
%
%    \begin{macro}{\HDorg@PrintDescribeMacro}
%    \begin{macrocode}
\let\HDorg@PrintDescribeMacro\PrintDescribeMacro
%    \end{macrocode}
%    \end{macro}
%    \begin{macro}{\PrintDescribeMacro}
%    \begin{macrocode}
\renewcommand\PrintDescribeMacro[1]{%
  \begingroup
    \let\HDorg@MacroFont\MacroFont
    \def\MacroFont{%
      \HD@target
      \HDorg@MacroFont
    }%
    \HDorg@PrintDescribeMacro{#1}%
  \endgroup
}
%    \end{macrocode}
%    \end{macro}
%
% \subsection{\cs{PrintMacroName}}
%
%    \begin{macro}{\HDorg@PrintMacroName}
%    \begin{macrocode}
\let\HDorg@PrintMacroName\PrintMacroName
%    \end{macrocode}
%    \end{macro}
%    \begin{macro}{\PrintMacroName}
%    \begin{macrocode}
\renewcommand\PrintMacroName[1]{%
  \begingroup
    \let\HDorg@MacroFont\MacroFont
    \def\MacroFont{%
      \HD@target
      \HDorg@MacroFont
    }%
    \HDorg@PrintMacroName{#1}%
  \endgroup
}
%    \end{macrocode}
%    \end{macro}
%
% \subsection{\cs{theCodelineNo}}
%
%    \begin{macro}{\HDorg@theCodelineNo}
%    \begin{macrocode}
\let\HDorg@theCodelineNo\theCodelineNo
%    \end{macrocode}
%    \end{macro}
%    \begin{macro}{\theCodelineNo}
%    \begin{macrocode}
\renewcommand*\theCodelineNo{%
  \settowidth{\HD@helplength}{\HDorg@theCodelineNo\ }%
  \ifdim\HD@helplength<\MacroIndent
    \addtolength{\HD@helplength}{-\MacroIndent}%
    \llap{%
      \begingroup\HD@target\endgroup
      \hspace*{-\HD@helplength}%
    }%
  \else
    \HD@target
  \fi
  \HDorg@theCodelineNo
}
%    \end{macrocode}
%    \end{macro}
%
%    \begin{macro}{\HDorg@index}
%    \begin{macrocode}
\AtBeginDocument{%
  \let\HDorg@index\index
}
%    \end{macrocode}
%    \end{macro}
%
% \subsection{\cs{SpecialUsageIndex}}
%
%    \begin{macro}{\HDorg@SpecialUsageIndex}
%    \begin{macrocode}
\let\HDorg@SpecialUsageIndex\SpecialUsageIndex
\renewcommand*\SpecialUsageIndex[1]{%
  \@bsphack
  \begingroup
    \HD@target
    \let\index\HDorg@index
    \let\HDorg@encapchar\encapchar
    \edef\encapchar usage{%
       \HDorg@encapchar hdclindex{\the\c@HD@hypercount}{usage}%
    }%
    \HDorg@SpecialUsageIndex{#1}%
  \endgroup
  \@esphack
}
%    \end{macrocode}
%    \end{macro}
%
% \subsection{\cs{autoindex}}
%
%    \begin{macrocode}
\begingroup\expandafter\expandafter\expandafter\endgroup
\expandafter\ifx\csname autoindex\endcsname\relax
\else
%    \end{macrocode}
%    \begin{macro}{\HDorg@autoindex}
%    \begin{macrocode}
  \let\HDorg@autoindex\autoindex
%    \end{macrocode}
%    \end{macro}
%    \begin{macro}{\autoindex}
%    \begin{macrocode}
  \renewcommand*\autoindex[1]{%
    \@bsphack
    \begingroup
      \HD@target
      \let\index\HDorg@index
      \let\HDorg@encapchar\encapchar
      \edef\encapchar usage{%
         \HDorg@encapchar hdclindex{\the\c@HD@hypercount}{usage}%
      }%
      \HDorg@autoindex{#1}%
    \endgroup
    \@esphack
  }%
\fi
%    \end{macrocode}
%    \end{macro}
%
% \subsection{\cs{SpecialEnvIndex}}
%
%    \begin{macro}{\HDorg@SpecialEnvIndex}
%    \begin{macrocode}
\let\HDorg@SpecialEnvIndex\SpecialEnvIndex
%    \end{macrocode}
%    \end{macro}
%    \begin{macro}{\SpecialEnvIndex}
%    \begin{macrocode}
\renewcommand*\SpecialEnvIndex[1]{%
  \@bsphack
  \begingroup
    \HD@target
    \let\HDorg@encapchar\encapchar
    \edef\encapchar usage{%
      \HDorg@encapchar hdclindex{\the\c@HD@hypercount}{usage}%
    }%
    \HDorg@SpecialEnvIndex{#1}%
  \endgroup
  \@esphack
}
%    \end{macrocode}
%    \end{macro}
%
% \subsection{\cs{SortIndex}}
%
%    \begin{macro}{\HDorg@SortIndex}
%    \begin{macrocode}
\let\HDorg@SortIndex\SortIndex
%    \end{macrocode}
%    \end{macro}
%    \begin{macro}{\SortIndex}
%    \begin{macrocode}
\renewcommand*{\SortIndex}[2]{%
  \@bsphack
  \begingroup
    \let\index\HD@page@wrindex
    \HDorg@SortIndex{#1}{#2}%
  \endgroup
  \@esphack
}
%    \end{macrocode}
%    \end{macro}
%
% \subsection{\cs{@wrglossary}}
%
%    \begin{macro}{\HDorg@wrglossary}
%    \begin{macrocode}
\let\HDorg@wrglossary\@wrglossary
%    \end{macrocode}
%    \end{macro}
%    \begin{macro}{\@wrglossary}
%    \begin{macrocode}
\def\@wrglossary#1{%
  \let\HDorg@encapchar\encapchar
  \def\encapchar##1\encapchar##2\@nil{%
    \HDorg@encapchar
    hdpindex{##1}%
  }%
  \HDorg@wrglossary{#1\encapchar\encapchar\@nil}%
}
%    \end{macrocode}
%    \end{macro}
%
% \subsection{\cs{MacroIndent}}
%
%    \cs{MacroIndent} automatically calculated.
%    \begin{macrocode}
\AfterLastShipout{%
  \if@filesw
    \immediate\write\@auxout{%
      \string\providecommand\string\HD@SetMacroIndent[1]{}%
    }%
    \immediate\write\@auxout{%
      \string\HD@SetMacroIndent{\number\value{CodelineNo}}%
    }%
  \fi
}
%    \end{macrocode}
%    \begin{macro}{\HD@SetMacroIndent}
%    \begin{macrocode}
\def\HD@SetMacroIndent#1{%
  \begingroup
    \value{CodelineNo}=#1\relax
    \settowidth{\MacroIndent}{\HDorg@theCodelineNo\ }%
    \global\MacroIndent\MacroIndent
  \endgroup
}
%    \end{macrocode}
%    \end{macro}
%
%    \begin{macrocode}
%</package>
%    \end{macrocode}
%
% \section{Installation}
%
% \subsection{Download}
%
% \paragraph{Package.} This package is available on
% CTAN\footnote{\url{http://ctan.org/pkg/hypdoc}}:
% \begin{description}
% \item[\CTAN{macros/latex/contrib/oberdiek/hypdoc.dtx}] The source file.
% \item[\CTAN{macros/latex/contrib/oberdiek/hypdoc.pdf}] Documentation.
% \end{description}
%
%
% \paragraph{Bundle.} All the packages of the bundle `oberdiek'
% are also available in a TDS compliant ZIP archive. There
% the packages are already unpacked and the documentation files
% are generated. The files and directories obey the TDS standard.
% \begin{description}
% \item[\CTAN{install/macros/latex/contrib/oberdiek.tds.zip}]
% \end{description}
% \emph{TDS} refers to the standard ``A Directory Structure
% for \TeX\ Files'' (\CTAN{tds/tds.pdf}). Directories
% with \xfile{texmf} in their name are usually organized this way.
%
% \subsection{Bundle installation}
%
% \paragraph{Unpacking.} Unpack the \xfile{oberdiek.tds.zip} in the
% TDS tree (also known as \xfile{texmf} tree) of your choice.
% Example (linux):
% \begin{quote}
%   |unzip oberdiek.tds.zip -d ~/texmf|
% \end{quote}
%
% \paragraph{Script installation.}
% Check the directory \xfile{TDS:scripts/oberdiek/} for
% scripts that need further installation steps.
% Package \xpackage{attachfile2} comes with the Perl script
% \xfile{pdfatfi.pl} that should be installed in such a way
% that it can be called as \texttt{pdfatfi}.
% Example (linux):
% \begin{quote}
%   |chmod +x scripts/oberdiek/pdfatfi.pl|\\
%   |cp scripts/oberdiek/pdfatfi.pl /usr/local/bin/|
% \end{quote}
%
% \subsection{Package installation}
%
% \paragraph{Unpacking.} The \xfile{.dtx} file is a self-extracting
% \docstrip\ archive. The files are extracted by running the
% \xfile{.dtx} through \plainTeX:
% \begin{quote}
%   \verb|tex hypdoc.dtx|
% \end{quote}
%
% \paragraph{TDS.} Now the different files must be moved into
% the different directories in your installation TDS tree
% (also known as \xfile{texmf} tree):
% \begin{quote}
% \def\t{^^A
% \begin{tabular}{@{}>{\ttfamily}l@{ $\rightarrow$ }>{\ttfamily}l@{}}
%   hypdoc.sty & tex/latex/oberdiek/hypdoc.sty\\
%   hypdoc.pdf & doc/latex/oberdiek/hypdoc.pdf\\
%   hypdoc.dtx & source/latex/oberdiek/hypdoc.dtx\\
% \end{tabular}^^A
% }^^A
% \sbox0{\t}^^A
% \ifdim\wd0>\linewidth
%   \begingroup
%     \advance\linewidth by\leftmargin
%     \advance\linewidth by\rightmargin
%   \edef\x{\endgroup
%     \def\noexpand\lw{\the\linewidth}^^A
%   }\x
%   \def\lwbox{^^A
%     \leavevmode
%     \hbox to \linewidth{^^A
%       \kern-\leftmargin\relax
%       \hss
%       \usebox0
%       \hss
%       \kern-\rightmargin\relax
%     }^^A
%   }^^A
%   \ifdim\wd0>\lw
%     \sbox0{\small\t}^^A
%     \ifdim\wd0>\linewidth
%       \ifdim\wd0>\lw
%         \sbox0{\footnotesize\t}^^A
%         \ifdim\wd0>\linewidth
%           \ifdim\wd0>\lw
%             \sbox0{\scriptsize\t}^^A
%             \ifdim\wd0>\linewidth
%               \ifdim\wd0>\lw
%                 \sbox0{\tiny\t}^^A
%                 \ifdim\wd0>\linewidth
%                   \lwbox
%                 \else
%                   \usebox0
%                 \fi
%               \else
%                 \lwbox
%               \fi
%             \else
%               \usebox0
%             \fi
%           \else
%             \lwbox
%           \fi
%         \else
%           \usebox0
%         \fi
%       \else
%         \lwbox
%       \fi
%     \else
%       \usebox0
%     \fi
%   \else
%     \lwbox
%   \fi
% \else
%   \usebox0
% \fi
% \end{quote}
% If you have a \xfile{docstrip.cfg} that configures and enables \docstrip's
% TDS installing feature, then some files can already be in the right
% place, see the documentation of \docstrip.
%
% \subsection{Refresh file name databases}
%
% If your \TeX~distribution
% (\teTeX, \mikTeX, \dots) relies on file name databases, you must refresh
% these. For example, \teTeX\ users run \verb|texhash| or
% \verb|mktexlsr|.
%
% \subsection{Some details for the interested}
%
% \paragraph{Attached source.}
%
% The PDF documentation on CTAN also includes the
% \xfile{.dtx} source file. It can be extracted by
% AcrobatReader 6 or higher. Another option is \textsf{pdftk},
% e.g. unpack the file into the current directory:
% \begin{quote}
%   \verb|pdftk hypdoc.pdf unpack_files output .|
% \end{quote}
%
% \paragraph{Unpacking with \LaTeX.}
% The \xfile{.dtx} chooses its action depending on the format:
% \begin{description}
% \item[\plainTeX:] Run \docstrip\ and extract the files.
% \item[\LaTeX:] Generate the documentation.
% \end{description}
% If you insist on using \LaTeX\ for \docstrip\ (really,
% \docstrip\ does not need \LaTeX), then inform the autodetect routine
% about your intention:
% \begin{quote}
%   \verb|latex \let\install=y% \iffalse meta-comment
%
% File: hypdoc.dtx
% Version: 2018/09/07 v1.14
% Info: Hyper extensions for doc.sty
%
% Copyright (C) 2002, 2006, 2007, 2009-2011 by
%    Heiko Oberdiek <heiko.oberdiek at googlemail.com>
%    2016
%    https://github.com/ho-tex/oberdiek/issues
%
% This work may be distributed and/or modified under the
% conditions of the LaTeX Project Public License, either
% version 1.3c of this license or (at your option) any later
% version. This version of this license is in
%    http://www.latex-project.org/lppl/lppl-1-3c.txt
% and the latest version of this license is in
%    http://www.latex-project.org/lppl.txt
% and version 1.3 or later is part of all distributions of
% LaTeX version 2005/12/01 or later.
%
% This work has the LPPL maintenance status "maintained".
%
% This Current Maintainer of this work is Heiko Oberdiek.
%
% This work consists of the main source file hypdoc.dtx
% and the derived files
%    hypdoc.sty, hypdoc.pdf, hypdoc.ins, hypdoc.drv.
%
% Distribution:
%    CTAN:macros/latex/contrib/oberdiek/hypdoc.dtx
%    CTAN:macros/latex/contrib/oberdiek/hypdoc.pdf
%
% Unpacking:
%    (a) If hypdoc.ins is present:
%           tex hypdoc.ins
%    (b) Without hypdoc.ins:
%           tex hypdoc.dtx
%    (c) If you insist on using LaTeX
%           latex \let\install=y\input{hypdoc.dtx}
%        (quote the arguments according to the demands of your shell)
%
% Documentation:
%    (a) If hypdoc.drv is present:
%           latex hypdoc.drv
%    (b) Without hypdoc.drv:
%           latex hypdoc.dtx; ...
%    The class ltxdoc loads the configuration file ltxdoc.cfg
%    if available. Here you can specify further options, e.g.
%    use A4 as paper format:
%       \PassOptionsToClass{a4paper}{article}
%
%    Programm calls to get the documentation (example):
%       pdflatex hypdoc.dtx
%       makeindex -s gind.ist hypdoc.idx
%       pdflatex hypdoc.dtx
%       makeindex -s gind.ist hypdoc.idx
%       pdflatex hypdoc.dtx
%
% Installation:
%    TDS:tex/latex/oberdiek/hypdoc.sty
%    TDS:doc/latex/oberdiek/hypdoc.pdf
%    TDS:source/latex/oberdiek/hypdoc.dtx
%
%<*ignore>
\begingroup
  \catcode123=1 %
  \catcode125=2 %
  \def\x{LaTeX2e}%
\expandafter\endgroup
\ifcase 0\ifx\install y1\fi\expandafter
         \ifx\csname processbatchFile\endcsname\relax\else1\fi
         \ifx\fmtname\x\else 1\fi\relax
\else\csname fi\endcsname
%</ignore>
%<*install>
\input docstrip.tex
\Msg{************************************************************************}
\Msg{* Installation}
\Msg{* Package: hypdoc 2018/09/07 v1.14 Hyper extensions for doc.sty (HO)}
\Msg{************************************************************************}

\keepsilent
\askforoverwritefalse

\let\MetaPrefix\relax
\preamble

This is a generated file.

Project: hypdoc
Version: 2018/09/07 v1.14

Copyright (C) 2002, 2006, 2007, 2009-2011 by
   Heiko Oberdiek <heiko.oberdiek at googlemail.com>

This work may be distributed and/or modified under the
conditions of the LaTeX Project Public License, either
version 1.3c of this license or (at your option) any later
version. This version of this license is in
   http://www.latex-project.org/lppl/lppl-1-3c.txt
and the latest version of this license is in
   http://www.latex-project.org/lppl.txt
and version 1.3 or later is part of all distributions of
LaTeX version 2005/12/01 or later.

This work has the LPPL maintenance status "maintained".

This Current Maintainer of this work is Heiko Oberdiek.

This work consists of the main source file hypdoc.dtx
and the derived files
   hypdoc.sty, hypdoc.pdf, hypdoc.ins, hypdoc.drv.

\endpreamble
\let\MetaPrefix\DoubleperCent

\generate{%
  \file{hypdoc.ins}{\from{hypdoc.dtx}{install}}%
  \file{hypdoc.drv}{\from{hypdoc.dtx}{driver}}%
  \usedir{tex/latex/oberdiek}%
  \file{hypdoc.sty}{\from{hypdoc.dtx}{package}}%
  \nopreamble
  \nopostamble
%  \usedir{source/latex/oberdiek/catalogue}%
%  \file{hypdoc.xml}{\from{hypdoc.dtx}{catalogue}}%
}

\catcode32=13\relax% active space
\let =\space%
\Msg{************************************************************************}
\Msg{*}
\Msg{* To finish the installation you have to move the following}
\Msg{* file into a directory searched by TeX:}
\Msg{*}
\Msg{*     hypdoc.sty}
\Msg{*}
\Msg{* To produce the documentation run the file `hypdoc.drv'}
\Msg{* through LaTeX.}
\Msg{*}
\Msg{* Happy TeXing!}
\Msg{*}
\Msg{************************************************************************}

\endbatchfile
%</install>
%<*ignore>
\fi
%</ignore>
%<*driver>
\NeedsTeXFormat{LaTeX2e}
\ProvidesFile{hypdoc.drv}%
  [2018/09/07 v1.14 Hyper extensions for doc.sty (HO)]%
\documentclass{ltxdoc}
\usepackage{holtxdoc}[2011/11/22]
\begin{document}
  \DocInput{hypdoc.dtx}%
\end{document}
%</driver>
% \fi
%
%
% \CharacterTable
%  {Upper-case    \A\B\C\D\E\F\G\H\I\J\K\L\M\N\O\P\Q\R\S\T\U\V\W\X\Y\Z
%   Lower-case    \a\b\c\d\e\f\g\h\i\j\k\l\m\n\o\p\q\r\s\t\u\v\w\x\y\z
%   Digits        \0\1\2\3\4\5\6\7\8\9
%   Exclamation   \!     Double quote  \"     Hash (number) \#
%   Dollar        \$     Percent       \%     Ampersand     \&
%   Acute accent  \'     Left paren    \(     Right paren   \)
%   Asterisk      \*     Plus          \+     Comma         \,
%   Minus         \-     Point         \.     Solidus       \/
%   Colon         \:     Semicolon     \;     Less than     \<
%   Equals        \=     Greater than  \>     Question mark \?
%   Commercial at \@     Left bracket  \[     Backslash     \\
%   Right bracket \]     Circumflex    \^     Underscore    \_
%   Grave accent  \`     Left brace    \{     Vertical bar  \|
%   Right brace   \}     Tilde         \~}
%
% \GetFileInfo{hypdoc.drv}
%
% \title{The \xpackage{hypdoc} package}
% \date{2018/09/07 v1.14}
% \author{Heiko Oberdiek\thanks
% {Please report any issues at https://github.com/ho-tex/oberdiek/issues}\\
% \xemail{heiko.oberdiek at googlemail.com}}
%
% \maketitle
%
% \begin{abstract}
% This package adds hyper features to the package
% \xpackage{doc} that is used in the documentation
% system of \LaTeXe. Bookmarks are added and references
% are linked as far as possible.
% \end{abstract}
%
% \tableofcontents
%
% \section{Features}
%
% The package adds the following features:
% \begin{description}
% \item[Bookmarks:]
%   For supporting bookmarks package \verb|hyperref| is loaded.
%   Some \TeX\ code inside section titles can cause problems
%   during bookmark processing; these can be warnings, ugly
%   bookmark titles or even errors. With the help of
%   \verb|\pdfstringdefDisableCommands| or \verb|\texorpdfstring|
%   you can manage these things, see \verb|hyperref|'s package
%   documentation.
% \item[Additional bookmarks:]
%   Unnumbered bookmark entries are generated for:
%   \begin{itemize}
%   \item Table of contents
%   \item List of figures
%   \item List of tables
%   \item Index
%   \item Glossary
%   \end{itemize}
%   Option \texttt{numbered} controls, whether these section or
%   chapters should be numbered. The exception is the table
%   of contents if this it would be the first numbered
%   section. Then just a bookmark is added.
% \item[Metadata:]
%   The loaded package \verb|hyperref|, enforced by
%   option \verb|pdfusetitle|, tries to detect \verb|\title|
%   and \verb|\author| and sets the corresponding PDF
%   information fields. In case of problems see the previous
%   item about bookmarks.
%   These enries can be overwritten or new ones added by
%   \verb|\hypersetup| after the packages is loaded, see
%   documentation of package \verb|hyperref|.
% \item[References, links:] Because of package \verb|\hyperref|
%   |\ref| become links, also \verb|\url|, footnote referencing
%   are supported.
% \item[Index:]
%   The main part of this package deals with index links in order
%   to support the kind of index that package \verb|doc| provides.
% \item[Thumbnails:]
%   Package \verb|thumbpdf| may additionally be loaded for the case that the
%   PDF file also should contain thumbnails, see the documentation
%   of this package. But thumbnail embedding is not really
%   necessary, because nowadays AcrobatReader is able to
%   generate and view thumbnails on the fly.
% \end{description}
%
% \section{Usage}
%
% The purpose of this packages is to add PDF features to
% the documentation of LaTeX packages. Thus just load
% the package in the driver file that generates the
% documentation:
% \begin{quote}
%   \verb|\usepackage{hypdoc}|
% \end{quote}
% Package \verb|doc| is loaded by \verb|hypdoc|, thus you
% also can replace a \verb|\usepackage{doc}| with this line.
% At any case, the package must be loaded before macros of
% package \verb|doc| such as \verb|\CodelineIndex| or
% \verb|\PageIndex| are used.
%
% If the class \verb|ltxdoc| is used, you can also use
% the configuration file for this class to add
% package \verb|hypdoc|. Add the following line
% to the file \verb|ltxdoc.cfg| (or create a new one):
% \begin{quote}
%\begin{verbatim}
%\AtEndOfClass{\RequirePackage{hypdoc}}
%\end{verbatim}
% \end{quote}
%
% \subsection{Options}
%
% There is one option \texttt{numbered}. It controls,
% whether the starred versions of section or chapter
% inside \cs{tableofcontents}, \cs{listoffigures}, \dots
% should be replaced by the non-star versions.
%
% \subsection{Simple example}
% Example of a simple driver file \verb|foobar.drv|
% of a package \verb|foobar|. Often the driver file
% is packed in the file \verb|foobar.dtx| and can
% be extracted by docstrip or if an instruction file
% for docstrip is provided by:
% \begin{quote}
% \verb|tex foobar.ins|
% \end{quote}
% Note that although docstrip doesn't rely on \LaTeX,
% sometimes you need to run the \verb|.ins| file through
% \LaTeX. Oddly enough, often the only \LaTeX\ feature
% in use is \verb|\NeedsTeXFormat{LaTeX2e}| in the first line.
% \begin{quote}
%\begin{verbatim}
%\documentclass{ltxdoc}
%\usepackage{hypdoc}
%\RecordChanges
%\EnableCrossrefs
%\CodelineIndex
%\begin{document}
%\title{Title for package \texttt{foobar}}
%\author{Author of package \texttt{foobar}}
%\date{...}
%\maketitle
%\tableofcontents
%\DocInput{foobar.dtx}
%\PrintChanges
%\PrintIndex
%\end{document}
%\end{verbatim}
% \end{quote}
%
% Then compile it, for example with pdf\LaTeX.
% \begin{quote}
%\begin{verbatim}
%pdflatex foobar.drv
%makeindex -s gind.ist foobar.idx
%makeindex -s gglo.ist -o foobar.gls foobar.glo
%pdflatex foobar.drv
%\end{verbatim}
% \end{quote}
% You will need several cycles until the cross references
% and similar things are correct.
%
% \subsection{Example \texttt{source2e.tex}}
%
% The source code of the base of \LaTeX\ is available as
% a bunch of \verb|.dtx| files. \LaTeX\ provides the file
% \verb|source2e.tex| to merge them all together.
%
% Now you can either load package \verb|hypdoc| in a private copy
% of this file (please, respect the LPPL) or use the configuration
% file \verb|ltxdoc.cfg|. Example that also sets A4 paper size:
% \begin{quote}
%\begin{verbatim}
%\PassOptionsToClass{a4paper}{article}
%\AtEndOfClass{\RequirePackage{hypdoc}}
%\end{verbatim}
% \end{quote}
%
% Then \verb|source2e.tex| can be compiled by pdf\TeX, for example:
% \begin{quote}
%\begin{verbatim}
%pdflatex source2e
%makeindex -s gind.ist source2e.idx
%makeindex -s gglo.ist -o source2e.glo source2e.gls
%pdflatex source2e
%makeindex -s gind.ist source2e.idx
%makeindex -s gglo.ist -o source2e.glo source2e.gls
%pdflatex source2e
%\end{verbatim}
% \end{quote}
%
% \StopEventually{
% }
%
% \section{Implementation}
%
%    \begin{macrocode}
%<*package>
\NeedsTeXFormat{LaTeX2e}
\ProvidesPackage{hypdoc}%
  [2018/09/07 v1.14 Hyper extensions for doc.sty (HO)]%
%    \end{macrocode}
%    The package identification is done at the top of the |.dtx| file
%    in order to use only one identification string.
%
%    For unique command names this package uses |HD@| as prefix
%
% \subsection{Options}
%
%    \begin{macro}{\ifHD@numbered}
%    \begin{macrocode}
\newif\ifHD@numbered
%    \end{macrocode}
%    \end{macro}
%    \begin{macrocode}
\DeclareOption{numbered}{\HD@numberedtrue}
\ProcessOptions*\relax
%    \end{macrocode}
%
% \subsection{Package loading}
%
%    \begin{macrocode}
\RequirePackage{atveryend}[2010/03/24]
\RequirePackage{doc}
%    \end{macrocode}
%    \begin{macrocode}
\ifx\numexpr\@undefined
  \RequirePackage{calc}%
\fi
%    \end{macrocode}
%    \begin{macrocode}
\RequirePackage[%
  hyperindex=false,%
  pdfusetitle,%
  colorlinks,%
  pdfpagelabels%
]{hyperref}[2002/05/09]%
\hypersetup{%
  plainpages=false,%
  bookmarksopen,%
  bookmarksnumbered,%
  bookmarksopenlevel=0,%
  pdfstartview={FitBH \hypercalcbp{%
    \paperheight-\topmargin-1in-\headheight-\headsep
  }},%
}
%    \end{macrocode}
%
%    \begin{macrocode}
\newcounter{HD@unique}
%    \end{macrocode}
%    \begin{macro}{\theHD@unique}
%    \begin{macrocode}
\renewcommand{\theHD@unique}{%
  \number\value{HD@unique}%
}
%    \end{macrocode}
%    \end{macro}
%
%    \begin{macrocode}
\g@addto@macro\Hy@UseMaketitleInfos{%
  \@ifundefined{fileinfo}{}{%
    \ifx\@pdfsubject\@empty
      \hypersetup{pdfsubject={\fileinfo}}%
    \fi
  }%
}
%    \end{macrocode}
%
%    \begin{macrocode}
\pdfstringdefDisableCommands{%
  \let\thanks\@gobble
  \let\footnote\@gobble
  \def\cs#1{\textbackslash #1}%
  \let\normalfont\@empty
  \let\scshape\@empty
  \def\and{and }%
}
%    \end{macrocode}
%
% \subsubsection{Patch for bug latex/4096}
%
%    \begin{macrocode}
\begingroup\expandafter\expandafter\expandafter\endgroup
\expandafter\ifx\csname mod@math@codes\endcsname\relax
\else
  \def\HD@ModFix#1{%
    \mathcode`#1=`#1\relax
  }%
  \g@addto@macro\mod@math@codes{%
    \HD@ModFix\-%
    \HD@ModFix\+%
    \HD@ModFix\:%
    \HD@ModFix\=%
    \HD@ModFix\*%
  }%
\fi
%    \end{macrocode}
%
% \subsection{Bookmarks}
%
%    \begin{macro}{\HD@sectionpatch}
%    \begin{macrocode}
\def\HD@sectionpatch{%
  \ifHD@numbered
    \HD@@sectionpatch{section}%
    \HD@@sectionpatch{chapter}%
  \else
    \let\Hy@writebookmark\HD@disable@writebookmark
  \fi
}
%    \end{macrocode}
%    \end{macro}
%    \begin{macro}{\HD@@sectionpatch}
%    \begin{macrocode}
\def\HD@@sectionpatch#1{%
  \expandafter\let\csname HDorg@@#1\expandafter\endcsname
    \csname #1\endcsname
  \@namedef{#1}{%
    \@ifstar{\@nameuse{HDorg@@#1}}{\@nameuse{HDorg@@#1}}%
  }%
}
%    \end{macrocode}
%    \end{macro}
%    \begin{macro}{\HD@disable@writebookmark}
%    \begin{macrocode}
\def\HD@disable@writebookmark#1#2#3#4#5{}
%    \end{macrocode}
%    \end{macro}
%
%    \begin{macro}{\HD@guesstoclevel}
%    A wild guessing of the toclevel. There is the hope, that
%    the macro |#1| starts with a sectioning command.
%    \begin{macrocode}
\def\HD@guesstoclevel#1{%
  \expandafter\HD@@guesstoclevel#1\section\HD@end
}
%    \end{macrocode}
%    \end{macro}
%    \begin{macro}{\HD@@guesstoclevel}
%    \begin{macrocode}
\def\HD@@guesstoclevel#1#2\HD@end{%
  \@ifundefined{toclevel@\expandafter\@gobble\string#1\@empty}{%
    1%
  }{%
    \@nameuse{toclevel@\expandafter\@gobble\string#1\@empty}%
  }%
}
%    \end{macrocode}
%    \end{macro}
%
% \subsection{\cs{tableofcontents}}
%
%    \begin{macro}{\HDorg@tableofcontents}
%    \begin{macrocode}
\let\HDorg@tableofcontents\tableofcontents
%    \end{macrocode}
%    \end{macro}
%    \begin{macro}{\tableofcontents}
%    As first section the table of contents shouldn't
%    go into the toc itself. Also a number looks too funny then.
%    Just a bookmark is enough in this case.
%    \begin{macrocode}
\def\tableofcontents{%
  \begingroup
    \@ifundefined{c@section}{%
      \def\c@section{0}%
    }{}%
  \expandafter\endgroup
  \ifcase\c@section
    \stepcounter{HD@unique}%
    \pdfbookmark[\HD@guesstoclevel{\HDorg@tableofcontents}]%
        {\contentsname}{toc\theHD@unique}%
    \HDorg@tableofcontents
  \else
    \ifHD@numbered
    \else
      \stepcounter{HD@unique}%
      \pdfbookmark[\HD@guesstoclevel{\HDorg@tableofcontents}]%
          {\contentsname}{toc\theHD@unique}%
    \fi
    \begingroup
      \HD@sectionpatch
      \HDorg@tableofcontents
    \endgroup
  \fi
}
%    \end{macrocode}
%    \end{macro}
%
% \subsection{\cs{listoffigures}}
%
%    \begin{macro}{\HDorg@listoffigures}
%    \begin{macrocode}
\let\HDorg@listoffigures\listoffigures
%    \end{macrocode}
%    \end{macro}
%    \begin{macro}{\listoffigures}
%    \begin{macrocode}
\def\listoffigures{%
  \ifHD@numbered
  \else
    \stepcounter{HD@unique}%
    \pdfbookmark[\HD@guesstoclevel{\HDorg@listoffigures}]%
        {\listfigurename}{toc\theHD@unique}%
  \fi
  \begingroup
    \HD@sectionpatch
    \HDorg@listoffigures
  \endgroup
}
%    \end{macrocode}
%    \end{macro}
%
% \subsection{\cs{listoftables}}
%
%    \begin{macro}{\HDorg@listoftables}
%    \begin{macrocode}
\let\HDorg@listoftables\listoftables
%    \end{macrocode}
%    \end{macro}
%    \begin{macro}{\listoftables}
%    \begin{macrocode}
\def\listoftables{%
  \ifHD@numbered
  \else
    \stepcounter{HD@unique}%
    \pdfbookmark[\HD@guesstoclevel{\HDorg@listoftables}]%
        {\listtablename}{toc\theHD@unique}%
  \fi
  \begingroup
    \HD@sectionpatch
    \HDorg@listoftables
  \endgroup
}
%    \end{macrocode}
%    \end{macro}
%
% \subsection{\cs{thebibliography}}
%
%    \begin{macro}{\HDorg@thebibliography}
%    \begin{macrocode}
\let\HDorg@thebibliography\thebibliography
%    \end{macrocode}
%    \end{macro}
%    \begin{macro}{\thebibliography}
%    \begin{macrocode}
\def\thebibliography{%
  \ifHD@numbered
  \else
    \stepcounter{HD@unique}%
    \pdfbookmark[\HD@guesstoclevel{\HDorg@thebibliography}]%
      {\@ifundefined{chapter}{\refname}{\bibname}}{bib\theHD@unique}%
  \fi
  \HD@sectionpatch
  \HDorg@thebibliography
}
%    \end{macrocode}
%    \end{macro}
%
% \subsection{\cs{theindex}}
%
%    \begin{macro}{\HDorg@theindex}
%    \begin{macrocode}
\let\HDorg@theindex\theindex
%    \end{macrocode}
%    \end{macro}
%    \begin{macro}{\theindex}
%    \begin{macrocode}
\def\theindex{%
  \let\HDorg@index@prologue\index@prologue
  \let\HDorg@writebookmark\Hy@writebookmark
  \HD@sectionpatch
  \g@addto@macro\IndexParms{%
    \ifHD@numbered
    \else
      \let\Hy@writebookmark\HDorg@writebookmark
    \fi
    \let\bfseries\HD@bfseries
  }%
  \edef\HD@toclevel@index{\HD@guesstoclevel{\index@prologue}}%
  \begingroup
    \count@=\HD@toclevel@index\relax
    \advance\count@ by 1 %
    \xdef\HD@toclevel@subindex{\the\count@}%
  \endgroup
  \def\index@prologue{%
    \ifHD@numbered
    \else
      \begingroup
        \stepcounter{HD@unique}%
        \let\Hy@writebookmark\HDorg@writebookmark
        \pdfbookmark[\HD@toclevel@index]%
            {\indexname}{index\theHD@unique}%
      \endgroup
    \fi
    \HDorg@index@prologue
  }%
  \HDorg@theindex
}
%    \end{macrocode}
%    \end{macro}
%    \begin{macro}{\HDorg@bfseries}
%    \begin{macrocode}
\let\HDorg@bfseries\bfseries
%    \end{macrocode}
%    \end{macro}
%    \begin{macro}{\HD@bfseries}
%    \begin{macrocode}
\def\HD@bfseries{\futurelet\HD@hfil\HD@@bfseries}%
%    \end{macrocode}
%    \end{macro}
%    \begin{macro}{\HD@@bfseries}
%    \begin{macrocode}
\def\HD@@bfseries{%
  \HDorg@bfseries
  \ifx\HD@hfil\hfil
    \expandafter\HD@@@bfseries
  \fi
}
%    \end{macrocode}
%    \end{macro}
%    \begin{macro}{\HD@@@bfseries}
%    \begin{macrocode}
\def\HD@@@bfseries\hfil#1\hfil{%
  \ifx\\#1\\%
  \else
    \raisebox{\baselineskip}[0pt]{%
      \kern-\HD@margin\relax
      \pdfbookmark[\HD@toclevel@subindex]{#1}{HD.#1}%
      \kern\HD@margin\relax
    }%
  \fi
  \hfil#1\hfil
}%
%    \end{macrocode}
%    \end{macro}
%    \begin{macro}{\HD@margin}
%    \begin{macrocode}
\def\HD@margin{1mm}
%    \end{macrocode}
%    \end{macro}
%
% \subsection{\cs{theglossary}}
%
%    \begin{macro}{\HDorg@theglossary}
%    \begin{macrocode}
\let\HDorg@theglossary\theglossary
%    \end{macrocode}
%    \end{macro}
%    \begin{macro}{\theglossary}
%    \begin{macrocode}
\def\theglossary{%
  \let\HDorg@glossary@prologue\glossary@prologue
  \let\HDorg@writebookmark\Hy@writebookmark
  \HD@sectionpatch
  \def\glossary@prologue{%
    \ifHD@numbered
    \else
      \begingroup
        \let\Hy@writebookmark\HDorg@writebookmark
        \stepcounter{HD@unique}%
        \pdfbookmark[\HD@guesstoclevel{\HDorg@glossary@prologue}]%
            {\changehistoryname}{glossary\theHD@unique}%
      \endgroup
    \fi
    \HDorg@glossary@prologue
  }%
  \HDorg@theglossary
}
%    \end{macrocode}
%    \end{macro}
%    \begin{macro}{\glossaryname}
%    \begin{macrocode}
\providecommand*{\glossaryname}{Change History}
%    \end{macrocode}
%    \end{macro}
%
%    \begin{macro}{\glossaryname}
%    \begin{macrocode}
\providecommand*{\changehistoryname}{\glossaryname}
%    \end{macrocode}
%    \end{macro}
%
% \subsection{Index}
%
%    \begin{macrocode}
\newcounter{HD@hypercount}
%    \end{macrocode}
%    \begin{macro}{\HD@helplength}
%    \begin{macrocode}
\newlength{\HD@helplength}
%    \end{macrocode}
%    \end{macro}
%    \begin{macro}{\ifHD@savedest}
%    \begin{macrocode}
\newif\ifHD@savedest
\HD@savedesttrue
%    \end{macrocode}
%    \end{macro}
%    \begin{macrocode}
\AtBeginDocument{%
  \HD@savedest@init
}
%    \end{macrocode}
%    \begin{macrocode}
\RequirePackage{rerunfilecheck}[2009/12/10]
%    \end{macrocode}
%    \begin{macro}{\HD@savedest@init}
%    \begin{macrocode}
\begingroup
  \catcode`\^^M=\active
\@firstofone{\endgroup
  \def\HD@savedest@init{%
    \ifHD@savedest
      \let\HDorg@tilde~%
      \let~\HD@savedest@entry
      \edef\HDorg@catcodeM{\the\catcode`\^^M}%
      \catcode`\^^M=\active
      \def^^M{}%
      \InputIfFileExists{\jobname.hd}{}{}%
      \let~\HDorg@tilde
      \catcode`\^^M=\HDorg@catcodeM\relax
      \newwrite\HD@savedest@out
      \RerunFileCheck{\jobname.hd}{%
        \immediate\closeout\HD@savedest@out
      }{}%
      \immediate\openout\HD@savedest@out=\jobname.hd\relax
    \else
      \let\HD@savedest@add\@gobble
    \fi
  }%
%    \end{macrocode}
%    \end{macro}
%    \begin{macro}{\HD@savedest@entry}
%    \begin{macrocode}
  \def\HD@savedest@entry#1^^M{%
    \@namedef{HD.#1}{}%
  }%
}
%    \end{macrocode}
%    \end{macro}
%    \begin{macro}{\HD@savedest@add}
%    \begin{macrocode}
\def\HD@savedest@add#1{%
  \immediate\write\HD@savedest@out{\string~#1}%
}
%    \end{macrocode}
%    \end{macro}
%    \begin{macro}{\ifHD@targetraise}
%    \begin{macrocode}
\newif\ifHD@targetraise
\HD@targetraisetrue
%    \end{macrocode}
%    \end{macro}
%    \begin{macro}{\HD@target}
%    \begin{macrocode}
\def\HD@target{%
  \ifHD@targetraise
    \ifhmode
      \settoheight{\HD@helplength}{[H]}%
      \addtolength{\HD@helplength}{0.4ex}%
      \raisebox{\HD@helplength}[0pt][0pt]{%
        \HD@maketarget
      }%
    \else
      \HD@maketarget
    \fi
  \else
    \HD@maketarget
  \fi
}
%    \end{macrocode}
%    \end{macro}
%    \begin{macro}{\HD@maketarget}
%    \begin{macrocode}
\def\HD@maketarget{%
  \stepcounter{HD@hypercount}%
  \def\HD@next{%
    \llap{\hypertarget{HD.\the\c@HD@hypercount}{}\,}%
  }%
  \ifHD@savedest
    \expandafter\ifx\csname HD.\the\c@HD@hypercount\endcsname\@empty
    \else
      \let\HD@next\relax
    \fi
  \fi
  \HD@next
}
%    \end{macrocode}
%    \end{macro}
%
%    \begin{macro}{\CodelineIndex}
%    \begin{macrocode}
\g@addto@macro\CodelineIndex{%
  \let\special@index\HD@codeline@wrindex
}
%    \end{macrocode}
%    \end{macro}
%    \begin{macro}{\PageIndex}
%    \begin{macrocode}
\g@addto@macro\PageIndex{%
  \let\special@index\HD@page@wrindex
}
%    \end{macrocode}
%    \end{macro}
%
%    \begin{macro}{\HD@codeline@wrindex}
%    \begin{macrocode}
\def\HD@codeline@wrindex#1{%
  \begingroup
    \let\HDorg@encapchar\encapchar
    \def\encapchar##1\encapchar##2\@nil{%
      \HDorg@encapchar
      hdclindex{\the\c@HD@hypercount}{##1}%
    }%
    \codeline@wrindex{%
      #1\encapchar\encapchar\@nil
    }%
  \endgroup
}
%    \end{macrocode}
%    \end{macro}
%    \begin{macro}{\HD@page@wrindex}
%    \begin{macrocode}
\def\HD@page@wrindex#1{%
  \begingroup
    \let\HDorg@encapchar\encapchar
    \def\encapchar##1\encapchar##2\@nil{%
      \HDorg@encapchar
      hdpindex{##1}%
    }%
    \HDorg@index{#1\encapchar\encapchar\@nil}%
  \endgroup
}
%    \end{macrocode}
%    \end{macro}
%
%    \begin{macro}{\hdclindex}
%    \begin{macrocode}
\def\hdclindex#1#2#3{%
  \csname\ifx\\#2\\relax\else#2\fi\endcsname{%
    \hyperlink{HD.#1}{#3}%
    \HD@savedest@add{#1}%
  }%
}
%    \end{macrocode}
%    \end{macro}
%    \begin{macro}{\hdpindex}
%    \begin{macrocode}
\def\hdpindex#1#2{%
  \csname\ifx\\#1\\relax\else#1\fi\endcsname{%
    \hyperpage{#2}%
  }%
}
%    \end{macrocode}
%    \end{macro}
%
% \subsection{\cs{PrintDescribeMacro}}
%
%    \begin{macro}{\HDorg@PrintDescribeMacro}
%    \begin{macrocode}
\let\HDorg@PrintDescribeMacro\PrintDescribeMacro
%    \end{macrocode}
%    \end{macro}
%    \begin{macro}{\PrintDescribeMacro}
%    \begin{macrocode}
\renewcommand\PrintDescribeMacro[1]{%
  \begingroup
    \let\HDorg@MacroFont\MacroFont
    \def\MacroFont{%
      \HD@target
      \HDorg@MacroFont
    }%
    \HDorg@PrintDescribeMacro{#1}%
  \endgroup
}
%    \end{macrocode}
%    \end{macro}
%
% \subsection{\cs{PrintMacroName}}
%
%    \begin{macro}{\HDorg@PrintMacroName}
%    \begin{macrocode}
\let\HDorg@PrintMacroName\PrintMacroName
%    \end{macrocode}
%    \end{macro}
%    \begin{macro}{\PrintMacroName}
%    \begin{macrocode}
\renewcommand\PrintMacroName[1]{%
  \begingroup
    \let\HDorg@MacroFont\MacroFont
    \def\MacroFont{%
      \HD@target
      \HDorg@MacroFont
    }%
    \HDorg@PrintMacroName{#1}%
  \endgroup
}
%    \end{macrocode}
%    \end{macro}
%
% \subsection{\cs{theCodelineNo}}
%
%    \begin{macro}{\HDorg@theCodelineNo}
%    \begin{macrocode}
\let\HDorg@theCodelineNo\theCodelineNo
%    \end{macrocode}
%    \end{macro}
%    \begin{macro}{\theCodelineNo}
%    \begin{macrocode}
\renewcommand*\theCodelineNo{%
  \settowidth{\HD@helplength}{\HDorg@theCodelineNo\ }%
  \ifdim\HD@helplength<\MacroIndent
    \addtolength{\HD@helplength}{-\MacroIndent}%
    \llap{%
      \begingroup\HD@target\endgroup
      \hspace*{-\HD@helplength}%
    }%
  \else
    \HD@target
  \fi
  \HDorg@theCodelineNo
}
%    \end{macrocode}
%    \end{macro}
%
%    \begin{macro}{\HDorg@index}
%    \begin{macrocode}
\AtBeginDocument{%
  \let\HDorg@index\index
}
%    \end{macrocode}
%    \end{macro}
%
% \subsection{\cs{SpecialUsageIndex}}
%
%    \begin{macro}{\HDorg@SpecialUsageIndex}
%    \begin{macrocode}
\let\HDorg@SpecialUsageIndex\SpecialUsageIndex
\renewcommand*\SpecialUsageIndex[1]{%
  \@bsphack
  \begingroup
    \HD@target
    \let\index\HDorg@index
    \let\HDorg@encapchar\encapchar
    \edef\encapchar usage{%
       \HDorg@encapchar hdclindex{\the\c@HD@hypercount}{usage}%
    }%
    \HDorg@SpecialUsageIndex{#1}%
  \endgroup
  \@esphack
}
%    \end{macrocode}
%    \end{macro}
%
% \subsection{\cs{autoindex}}
%
%    \begin{macrocode}
\begingroup\expandafter\expandafter\expandafter\endgroup
\expandafter\ifx\csname autoindex\endcsname\relax
\else
%    \end{macrocode}
%    \begin{macro}{\HDorg@autoindex}
%    \begin{macrocode}
  \let\HDorg@autoindex\autoindex
%    \end{macrocode}
%    \end{macro}
%    \begin{macro}{\autoindex}
%    \begin{macrocode}
  \renewcommand*\autoindex[1]{%
    \@bsphack
    \begingroup
      \HD@target
      \let\index\HDorg@index
      \let\HDorg@encapchar\encapchar
      \edef\encapchar usage{%
         \HDorg@encapchar hdclindex{\the\c@HD@hypercount}{usage}%
      }%
      \HDorg@autoindex{#1}%
    \endgroup
    \@esphack
  }%
\fi
%    \end{macrocode}
%    \end{macro}
%
% \subsection{\cs{SpecialEnvIndex}}
%
%    \begin{macro}{\HDorg@SpecialEnvIndex}
%    \begin{macrocode}
\let\HDorg@SpecialEnvIndex\SpecialEnvIndex
%    \end{macrocode}
%    \end{macro}
%    \begin{macro}{\SpecialEnvIndex}
%    \begin{macrocode}
\renewcommand*\SpecialEnvIndex[1]{%
  \@bsphack
  \begingroup
    \HD@target
    \let\HDorg@encapchar\encapchar
    \edef\encapchar usage{%
      \HDorg@encapchar hdclindex{\the\c@HD@hypercount}{usage}%
    }%
    \HDorg@SpecialEnvIndex{#1}%
  \endgroup
  \@esphack
}
%    \end{macrocode}
%    \end{macro}
%
% \subsection{\cs{SortIndex}}
%
%    \begin{macro}{\HDorg@SortIndex}
%    \begin{macrocode}
\let\HDorg@SortIndex\SortIndex
%    \end{macrocode}
%    \end{macro}
%    \begin{macro}{\SortIndex}
%    \begin{macrocode}
\renewcommand*{\SortIndex}[2]{%
  \@bsphack
  \begingroup
    \let\index\HD@page@wrindex
    \HDorg@SortIndex{#1}{#2}%
  \endgroup
  \@esphack
}
%    \end{macrocode}
%    \end{macro}
%
% \subsection{\cs{@wrglossary}}
%
%    \begin{macro}{\HDorg@wrglossary}
%    \begin{macrocode}
\let\HDorg@wrglossary\@wrglossary
%    \end{macrocode}
%    \end{macro}
%    \begin{macro}{\@wrglossary}
%    \begin{macrocode}
\def\@wrglossary#1{%
  \let\HDorg@encapchar\encapchar
  \def\encapchar##1\encapchar##2\@nil{%
    \HDorg@encapchar
    hdpindex{##1}%
  }%
  \HDorg@wrglossary{#1\encapchar\encapchar\@nil}%
}
%    \end{macrocode}
%    \end{macro}
%
% \subsection{\cs{MacroIndent}}
%
%    \cs{MacroIndent} automatically calculated.
%    \begin{macrocode}
\AfterLastShipout{%
  \if@filesw
    \immediate\write\@auxout{%
      \string\providecommand\string\HD@SetMacroIndent[1]{}%
    }%
    \immediate\write\@auxout{%
      \string\HD@SetMacroIndent{\number\value{CodelineNo}}%
    }%
  \fi
}
%    \end{macrocode}
%    \begin{macro}{\HD@SetMacroIndent}
%    \begin{macrocode}
\def\HD@SetMacroIndent#1{%
  \begingroup
    \value{CodelineNo}=#1\relax
    \settowidth{\MacroIndent}{\HDorg@theCodelineNo\ }%
    \global\MacroIndent\MacroIndent
  \endgroup
}
%    \end{macrocode}
%    \end{macro}
%
%    \begin{macrocode}
%</package>
%    \end{macrocode}
%
% \section{Installation}
%
% \subsection{Download}
%
% \paragraph{Package.} This package is available on
% CTAN\footnote{\url{http://ctan.org/pkg/hypdoc}}:
% \begin{description}
% \item[\CTAN{macros/latex/contrib/oberdiek/hypdoc.dtx}] The source file.
% \item[\CTAN{macros/latex/contrib/oberdiek/hypdoc.pdf}] Documentation.
% \end{description}
%
%
% \paragraph{Bundle.} All the packages of the bundle `oberdiek'
% are also available in a TDS compliant ZIP archive. There
% the packages are already unpacked and the documentation files
% are generated. The files and directories obey the TDS standard.
% \begin{description}
% \item[\CTAN{install/macros/latex/contrib/oberdiek.tds.zip}]
% \end{description}
% \emph{TDS} refers to the standard ``A Directory Structure
% for \TeX\ Files'' (\CTAN{tds/tds.pdf}). Directories
% with \xfile{texmf} in their name are usually organized this way.
%
% \subsection{Bundle installation}
%
% \paragraph{Unpacking.} Unpack the \xfile{oberdiek.tds.zip} in the
% TDS tree (also known as \xfile{texmf} tree) of your choice.
% Example (linux):
% \begin{quote}
%   |unzip oberdiek.tds.zip -d ~/texmf|
% \end{quote}
%
% \paragraph{Script installation.}
% Check the directory \xfile{TDS:scripts/oberdiek/} for
% scripts that need further installation steps.
% Package \xpackage{attachfile2} comes with the Perl script
% \xfile{pdfatfi.pl} that should be installed in such a way
% that it can be called as \texttt{pdfatfi}.
% Example (linux):
% \begin{quote}
%   |chmod +x scripts/oberdiek/pdfatfi.pl|\\
%   |cp scripts/oberdiek/pdfatfi.pl /usr/local/bin/|
% \end{quote}
%
% \subsection{Package installation}
%
% \paragraph{Unpacking.} The \xfile{.dtx} file is a self-extracting
% \docstrip\ archive. The files are extracted by running the
% \xfile{.dtx} through \plainTeX:
% \begin{quote}
%   \verb|tex hypdoc.dtx|
% \end{quote}
%
% \paragraph{TDS.} Now the different files must be moved into
% the different directories in your installation TDS tree
% (also known as \xfile{texmf} tree):
% \begin{quote}
% \def\t{^^A
% \begin{tabular}{@{}>{\ttfamily}l@{ $\rightarrow$ }>{\ttfamily}l@{}}
%   hypdoc.sty & tex/latex/oberdiek/hypdoc.sty\\
%   hypdoc.pdf & doc/latex/oberdiek/hypdoc.pdf\\
%   hypdoc.dtx & source/latex/oberdiek/hypdoc.dtx\\
% \end{tabular}^^A
% }^^A
% \sbox0{\t}^^A
% \ifdim\wd0>\linewidth
%   \begingroup
%     \advance\linewidth by\leftmargin
%     \advance\linewidth by\rightmargin
%   \edef\x{\endgroup
%     \def\noexpand\lw{\the\linewidth}^^A
%   }\x
%   \def\lwbox{^^A
%     \leavevmode
%     \hbox to \linewidth{^^A
%       \kern-\leftmargin\relax
%       \hss
%       \usebox0
%       \hss
%       \kern-\rightmargin\relax
%     }^^A
%   }^^A
%   \ifdim\wd0>\lw
%     \sbox0{\small\t}^^A
%     \ifdim\wd0>\linewidth
%       \ifdim\wd0>\lw
%         \sbox0{\footnotesize\t}^^A
%         \ifdim\wd0>\linewidth
%           \ifdim\wd0>\lw
%             \sbox0{\scriptsize\t}^^A
%             \ifdim\wd0>\linewidth
%               \ifdim\wd0>\lw
%                 \sbox0{\tiny\t}^^A
%                 \ifdim\wd0>\linewidth
%                   \lwbox
%                 \else
%                   \usebox0
%                 \fi
%               \else
%                 \lwbox
%               \fi
%             \else
%               \usebox0
%             \fi
%           \else
%             \lwbox
%           \fi
%         \else
%           \usebox0
%         \fi
%       \else
%         \lwbox
%       \fi
%     \else
%       \usebox0
%     \fi
%   \else
%     \lwbox
%   \fi
% \else
%   \usebox0
% \fi
% \end{quote}
% If you have a \xfile{docstrip.cfg} that configures and enables \docstrip's
% TDS installing feature, then some files can already be in the right
% place, see the documentation of \docstrip.
%
% \subsection{Refresh file name databases}
%
% If your \TeX~distribution
% (\teTeX, \mikTeX, \dots) relies on file name databases, you must refresh
% these. For example, \teTeX\ users run \verb|texhash| or
% \verb|mktexlsr|.
%
% \subsection{Some details for the interested}
%
% \paragraph{Attached source.}
%
% The PDF documentation on CTAN also includes the
% \xfile{.dtx} source file. It can be extracted by
% AcrobatReader 6 or higher. Another option is \textsf{pdftk},
% e.g. unpack the file into the current directory:
% \begin{quote}
%   \verb|pdftk hypdoc.pdf unpack_files output .|
% \end{quote}
%
% \paragraph{Unpacking with \LaTeX.}
% The \xfile{.dtx} chooses its action depending on the format:
% \begin{description}
% \item[\plainTeX:] Run \docstrip\ and extract the files.
% \item[\LaTeX:] Generate the documentation.
% \end{description}
% If you insist on using \LaTeX\ for \docstrip\ (really,
% \docstrip\ does not need \LaTeX), then inform the autodetect routine
% about your intention:
% \begin{quote}
%   \verb|latex \let\install=y\input{hypdoc.dtx}|
% \end{quote}
% Do not forget to quote the argument according to the demands
% of your shell.
%
% \paragraph{Generating the documentation.}
% You can use both the \xfile{.dtx} or the \xfile{.drv} to generate
% the documentation. The process can be configured by the
% configuration file \xfile{ltxdoc.cfg}. For instance, put this
% line into this file, if you want to have A4 as paper format:
% \begin{quote}
%   \verb|\PassOptionsToClass{a4paper}{article}|
% \end{quote}
% An example follows how to generate the
% documentation with pdf\LaTeX:
% \begin{quote}
%\begin{verbatim}
%pdflatex hypdoc.dtx
%makeindex -s gind.ist hypdoc.idx
%pdflatex hypdoc.dtx
%makeindex -s gind.ist hypdoc.idx
%pdflatex hypdoc.dtx
%\end{verbatim}
% \end{quote}
%
% \section{Catalogue}
%
% The following XML file can be used as source for the
% \href{http://mirror.ctan.org/help/Catalogue/catalogue.html}{\TeX\ Catalogue}.
% The elements \texttt{caption} and \texttt{description} are imported
% from the original XML file from the Catalogue.
% The name of the XML file in the Catalogue is \xfile{hypdoc.xml}.
%    \begin{macrocode}
%<*catalogue>
<?xml version='1.0' encoding='us-ascii'?>
<!DOCTYPE entry SYSTEM 'catalogue.dtd'>
<entry datestamp='$Date$' modifier='$Author$' id='hypdoc'>
  <name>hypdoc</name>
  <caption>Hyper extensions for doc.sty.</caption>
  <authorref id='auth:oberdiek'/>
  <copyright owner='Heiko Oberdiek' year='2002,2006,2007,2009-2011'/>
  <license type='lppl1.3'/>
  <version number='1.14'/>
  <description>
    This package adds hypertext features to the package
    <xref refid='doc'>doc</xref> that is used in the documentation
    system of <xref refid='latex'>LaTeX2e</xref>. Bookmarks are added
    and references are linked as far as possible.
    <p/>
    The package is part of the <xref refid='oberdiek'>oberdiek</xref>
    bundle.
  </description>
  <documentation details='Package documentation'
      href='ctan:/macros/latex/contrib/oberdiek/hypdoc.pdf'/>
  <ctan file='true' path='/macros/latex/contrib/oberdiek/hypdoc.dtx'/>
  <miktex location='oberdiek'/>
  <texlive location='oberdiek'/>
  <install path='/macros/latex/contrib/oberdiek/oberdiek.tds.zip'/>
</entry>
%</catalogue>
%    \end{macrocode}
%
% \begin{History}
%   \begin{Version}{2002/05/10 v1.0}
%   \item
%     First standalone version.
%   \end{Version}
%   \begin{Version}{2006/02/20 v1.1}
%   \item
%     Option \xoption{numbered} added.
%   \item
%     \cs{HD@savedest@init} fixed (empty \xext{hd} file triggered
%     first page with fl ligature).
%   \item
%     Bookmarks for \cs{listoffigures} and \cs{listoftables}.
%   \item
%     Documentation: features and usage added.
%   \item
%     LPPL 1.3.
%   \end{Version}
%   \begin{Version}{2006/04/27 v1.2}
%   \item
%     \cs{HD@target} fixed (\xfile{multicol.dtx}).
%   \end{Version}
%   \begin{Version}{2006/06/01 v1.3}
%   \item
%     Support for package \xpackage{amsdtx}'s \cs{autoindex} added.
%   \end{Version}
%   \begin{Version}{2006/06/02 v1.4}
%   \item
%     Bookmark fixed if there are several table of contents,
%     bibliographies, glossaries, \dots
%   \end{Version}
%   \begin{Version}{2007/04/11 v1.5}
%   \item
%     Line ends sanitized.
%   \end{Version}
%   \begin{Version}{2007/11/12 v1.6}
%   \item
%     Bug fix in \cs{listoffigures} (Axel Sommerfeldt).
%   \item
%     Markup added in implementation section.
%   \end{Version}
%   \begin{Version}{2009/12/10 v1.7}
%   \item
%     Use of package \xpackage{rerunfilecheck}.
%   \end{Version}
%   \begin{Version}{2010/02/03 v1.8}
%   \item
%     Fix for bug latex/4096 added.
%   \end{Version}
%   \begin{Version}{2010/03/26 v1.9}
%   \item
%     Definition of \cs{HD@SetMacroIndex} is provided in the \xext{aux} file.
%   \end{Version}
%   \begin{Version}{2011/05/05 v1.10}
%   \item
%     \cs{raisebox} in \cs{HD@target} with height and depth of \texttt{0pt}
%     (request of GL).
%   \end{Version}
%   \begin{Version}{2011/08/19 v1.11}
%   \item
%     Latest comma in option list for hyperref removed to avoid option clash
%     because of \hologo{LaTeX} 2011/06/27 with the new \cs{in@}.
%   \end{Version}
%   \begin{Version}{2016/05/16 v1.12}
%   \item
%     Documentation updates.
%   \end{Version}
%   \begin{Version}{2016/10/02 v1.13}
%   \item
%     Fixes to bookmarknames, also introduce \cs{changehistoryname}
%   \end{Version}
%   \begin{Version}{2018/09/07 v1.14}
%   \item
%     Do not load \xpackage{thumpdf}.
%   \end{Version}
% \end{History}
%
% \PrintIndex
%
% \Finale
\endinput
|
% \end{quote}
% Do not forget to quote the argument according to the demands
% of your shell.
%
% \paragraph{Generating the documentation.}
% You can use both the \xfile{.dtx} or the \xfile{.drv} to generate
% the documentation. The process can be configured by the
% configuration file \xfile{ltxdoc.cfg}. For instance, put this
% line into this file, if you want to have A4 as paper format:
% \begin{quote}
%   \verb|\PassOptionsToClass{a4paper}{article}|
% \end{quote}
% An example follows how to generate the
% documentation with pdf\LaTeX:
% \begin{quote}
%\begin{verbatim}
%pdflatex hypdoc.dtx
%makeindex -s gind.ist hypdoc.idx
%pdflatex hypdoc.dtx
%makeindex -s gind.ist hypdoc.idx
%pdflatex hypdoc.dtx
%\end{verbatim}
% \end{quote}
%
% \section{Catalogue}
%
% The following XML file can be used as source for the
% \href{http://mirror.ctan.org/help/Catalogue/catalogue.html}{\TeX\ Catalogue}.
% The elements \texttt{caption} and \texttt{description} are imported
% from the original XML file from the Catalogue.
% The name of the XML file in the Catalogue is \xfile{hypdoc.xml}.
%    \begin{macrocode}
%<*catalogue>
<?xml version='1.0' encoding='us-ascii'?>
<!DOCTYPE entry SYSTEM 'catalogue.dtd'>
<entry datestamp='$Date$' modifier='$Author$' id='hypdoc'>
  <name>hypdoc</name>
  <caption>Hyper extensions for doc.sty.</caption>
  <authorref id='auth:oberdiek'/>
  <copyright owner='Heiko Oberdiek' year='2002,2006,2007,2009-2011'/>
  <license type='lppl1.3'/>
  <version number='1.14'/>
  <description>
    This package adds hypertext features to the package
    <xref refid='doc'>doc</xref> that is used in the documentation
    system of <xref refid='latex'>LaTeX2e</xref>. Bookmarks are added
    and references are linked as far as possible.
    <p/>
    The package is part of the <xref refid='oberdiek'>oberdiek</xref>
    bundle.
  </description>
  <documentation details='Package documentation'
      href='ctan:/macros/latex/contrib/oberdiek/hypdoc.pdf'/>
  <ctan file='true' path='/macros/latex/contrib/oberdiek/hypdoc.dtx'/>
  <miktex location='oberdiek'/>
  <texlive location='oberdiek'/>
  <install path='/macros/latex/contrib/oberdiek/oberdiek.tds.zip'/>
</entry>
%</catalogue>
%    \end{macrocode}
%
% \begin{History}
%   \begin{Version}{2002/05/10 v1.0}
%   \item
%     First standalone version.
%   \end{Version}
%   \begin{Version}{2006/02/20 v1.1}
%   \item
%     Option \xoption{numbered} added.
%   \item
%     \cs{HD@savedest@init} fixed (empty \xext{hd} file triggered
%     first page with fl ligature).
%   \item
%     Bookmarks for \cs{listoffigures} and \cs{listoftables}.
%   \item
%     Documentation: features and usage added.
%   \item
%     LPPL 1.3.
%   \end{Version}
%   \begin{Version}{2006/04/27 v1.2}
%   \item
%     \cs{HD@target} fixed (\xfile{multicol.dtx}).
%   \end{Version}
%   \begin{Version}{2006/06/01 v1.3}
%   \item
%     Support for package \xpackage{amsdtx}'s \cs{autoindex} added.
%   \end{Version}
%   \begin{Version}{2006/06/02 v1.4}
%   \item
%     Bookmark fixed if there are several table of contents,
%     bibliographies, glossaries, \dots
%   \end{Version}
%   \begin{Version}{2007/04/11 v1.5}
%   \item
%     Line ends sanitized.
%   \end{Version}
%   \begin{Version}{2007/11/12 v1.6}
%   \item
%     Bug fix in \cs{listoffigures} (Axel Sommerfeldt).
%   \item
%     Markup added in implementation section.
%   \end{Version}
%   \begin{Version}{2009/12/10 v1.7}
%   \item
%     Use of package \xpackage{rerunfilecheck}.
%   \end{Version}
%   \begin{Version}{2010/02/03 v1.8}
%   \item
%     Fix for bug latex/4096 added.
%   \end{Version}
%   \begin{Version}{2010/03/26 v1.9}
%   \item
%     Definition of \cs{HD@SetMacroIndex} is provided in the \xext{aux} file.
%   \end{Version}
%   \begin{Version}{2011/05/05 v1.10}
%   \item
%     \cs{raisebox} in \cs{HD@target} with height and depth of \texttt{0pt}
%     (request of GL).
%   \end{Version}
%   \begin{Version}{2011/08/19 v1.11}
%   \item
%     Latest comma in option list for hyperref removed to avoid option clash
%     because of \hologo{LaTeX} 2011/06/27 with the new \cs{in@}.
%   \end{Version}
%   \begin{Version}{2016/05/16 v1.12}
%   \item
%     Documentation updates.
%   \end{Version}
%   \begin{Version}{2016/10/02 v1.13}
%   \item
%     Fixes to bookmarknames, also introduce \cs{changehistoryname}
%   \end{Version}
%   \begin{Version}{2018/09/07 v1.14}
%   \item
%     Do not load \xpackage{thumpdf}.
%   \end{Version}
% \end{History}
%
% \PrintIndex
%
% \Finale
\endinput

%        (quote the arguments according to the demands of your shell)
%
% Documentation:
%    (a) If hypdoc.drv is present:
%           latex hypdoc.drv
%    (b) Without hypdoc.drv:
%           latex hypdoc.dtx; ...
%    The class ltxdoc loads the configuration file ltxdoc.cfg
%    if available. Here you can specify further options, e.g.
%    use A4 as paper format:
%       \PassOptionsToClass{a4paper}{article}
%
%    Programm calls to get the documentation (example):
%       pdflatex hypdoc.dtx
%       makeindex -s gind.ist hypdoc.idx
%       pdflatex hypdoc.dtx
%       makeindex -s gind.ist hypdoc.idx
%       pdflatex hypdoc.dtx
%
% Installation:
%    TDS:tex/latex/oberdiek/hypdoc.sty
%    TDS:doc/latex/oberdiek/hypdoc.pdf
%    TDS:source/latex/oberdiek/hypdoc.dtx
%
%<*ignore>
\begingroup
  \catcode123=1 %
  \catcode125=2 %
  \def\x{LaTeX2e}%
\expandafter\endgroup
\ifcase 0\ifx\install y1\fi\expandafter
         \ifx\csname processbatchFile\endcsname\relax\else1\fi
         \ifx\fmtname\x\else 1\fi\relax
\else\csname fi\endcsname
%</ignore>
%<*install>
\input docstrip.tex
\Msg{************************************************************************}
\Msg{* Installation}
\Msg{* Package: hypdoc 2018/09/07 v1.14 Hyper extensions for doc.sty (HO)}
\Msg{************************************************************************}

\keepsilent
\askforoverwritefalse

\let\MetaPrefix\relax
\preamble

This is a generated file.

Project: hypdoc
Version: 2018/09/07 v1.14

Copyright (C) 2002, 2006, 2007, 2009-2011 by
   Heiko Oberdiek <heiko.oberdiek at googlemail.com>

This work may be distributed and/or modified under the
conditions of the LaTeX Project Public License, either
version 1.3c of this license or (at your option) any later
version. This version of this license is in
   http://www.latex-project.org/lppl/lppl-1-3c.txt
and the latest version of this license is in
   http://www.latex-project.org/lppl.txt
and version 1.3 or later is part of all distributions of
LaTeX version 2005/12/01 or later.

This work has the LPPL maintenance status "maintained".

This Current Maintainer of this work is Heiko Oberdiek.

This work consists of the main source file hypdoc.dtx
and the derived files
   hypdoc.sty, hypdoc.pdf, hypdoc.ins, hypdoc.drv.

\endpreamble
\let\MetaPrefix\DoubleperCent

\generate{%
  \file{hypdoc.ins}{\from{hypdoc.dtx}{install}}%
  \file{hypdoc.drv}{\from{hypdoc.dtx}{driver}}%
  \usedir{tex/latex/oberdiek}%
  \file{hypdoc.sty}{\from{hypdoc.dtx}{package}}%
  \nopreamble
  \nopostamble
%  \usedir{source/latex/oberdiek/catalogue}%
%  \file{hypdoc.xml}{\from{hypdoc.dtx}{catalogue}}%
}

\catcode32=13\relax% active space
\let =\space%
\Msg{************************************************************************}
\Msg{*}
\Msg{* To finish the installation you have to move the following}
\Msg{* file into a directory searched by TeX:}
\Msg{*}
\Msg{*     hypdoc.sty}
\Msg{*}
\Msg{* To produce the documentation run the file `hypdoc.drv'}
\Msg{* through LaTeX.}
\Msg{*}
\Msg{* Happy TeXing!}
\Msg{*}
\Msg{************************************************************************}

\endbatchfile
%</install>
%<*ignore>
\fi
%</ignore>
%<*driver>
\NeedsTeXFormat{LaTeX2e}
\ProvidesFile{hypdoc.drv}%
  [2018/09/07 v1.14 Hyper extensions for doc.sty (HO)]%
\documentclass{ltxdoc}
\usepackage{holtxdoc}[2011/11/22]
\begin{document}
  \DocInput{hypdoc.dtx}%
\end{document}
%</driver>
% \fi
%
%
% \CharacterTable
%  {Upper-case    \A\B\C\D\E\F\G\H\I\J\K\L\M\N\O\P\Q\R\S\T\U\V\W\X\Y\Z
%   Lower-case    \a\b\c\d\e\f\g\h\i\j\k\l\m\n\o\p\q\r\s\t\u\v\w\x\y\z
%   Digits        \0\1\2\3\4\5\6\7\8\9
%   Exclamation   \!     Double quote  \"     Hash (number) \#
%   Dollar        \$     Percent       \%     Ampersand     \&
%   Acute accent  \'     Left paren    \(     Right paren   \)
%   Asterisk      \*     Plus          \+     Comma         \,
%   Minus         \-     Point         \.     Solidus       \/
%   Colon         \:     Semicolon     \;     Less than     \<
%   Equals        \=     Greater than  \>     Question mark \?
%   Commercial at \@     Left bracket  \[     Backslash     \\
%   Right bracket \]     Circumflex    \^     Underscore    \_
%   Grave accent  \`     Left brace    \{     Vertical bar  \|
%   Right brace   \}     Tilde         \~}
%
% \GetFileInfo{hypdoc.drv}
%
% \title{The \xpackage{hypdoc} package}
% \date{2018/09/07 v1.14}
% \author{Heiko Oberdiek\thanks
% {Please report any issues at https://github.com/ho-tex/oberdiek/issues}\\
% \xemail{heiko.oberdiek at googlemail.com}}
%
% \maketitle
%
% \begin{abstract}
% This package adds hyper features to the package
% \xpackage{doc} that is used in the documentation
% system of \LaTeXe. Bookmarks are added and references
% are linked as far as possible.
% \end{abstract}
%
% \tableofcontents
%
% \section{Features}
%
% The package adds the following features:
% \begin{description}
% \item[Bookmarks:]
%   For supporting bookmarks package \verb|hyperref| is loaded.
%   Some \TeX\ code inside section titles can cause problems
%   during bookmark processing; these can be warnings, ugly
%   bookmark titles or even errors. With the help of
%   \verb|\pdfstringdefDisableCommands| or \verb|\texorpdfstring|
%   you can manage these things, see \verb|hyperref|'s package
%   documentation.
% \item[Additional bookmarks:]
%   Unnumbered bookmark entries are generated for:
%   \begin{itemize}
%   \item Table of contents
%   \item List of figures
%   \item List of tables
%   \item Index
%   \item Glossary
%   \end{itemize}
%   Option \texttt{numbered} controls, whether these section or
%   chapters should be numbered. The exception is the table
%   of contents if this it would be the first numbered
%   section. Then just a bookmark is added.
% \item[Metadata:]
%   The loaded package \verb|hyperref|, enforced by
%   option \verb|pdfusetitle|, tries to detect \verb|\title|
%   and \verb|\author| and sets the corresponding PDF
%   information fields. In case of problems see the previous
%   item about bookmarks.
%   These enries can be overwritten or new ones added by
%   \verb|\hypersetup| after the packages is loaded, see
%   documentation of package \verb|hyperref|.
% \item[References, links:] Because of package \verb|\hyperref|
%   |\ref| become links, also \verb|\url|, footnote referencing
%   are supported.
% \item[Index:]
%   The main part of this package deals with index links in order
%   to support the kind of index that package \verb|doc| provides.
% \item[Thumbnails:]
%   Package \verb|thumbpdf| may additionally be loaded for the case that the
%   PDF file also should contain thumbnails, see the documentation
%   of this package. But thumbnail embedding is not really
%   necessary, because nowadays AcrobatReader is able to
%   generate and view thumbnails on the fly.
% \end{description}
%
% \section{Usage}
%
% The purpose of this packages is to add PDF features to
% the documentation of LaTeX packages. Thus just load
% the package in the driver file that generates the
% documentation:
% \begin{quote}
%   \verb|\usepackage{hypdoc}|
% \end{quote}
% Package \verb|doc| is loaded by \verb|hypdoc|, thus you
% also can replace a \verb|\usepackage{doc}| with this line.
% At any case, the package must be loaded before macros of
% package \verb|doc| such as \verb|\CodelineIndex| or
% \verb|\PageIndex| are used.
%
% If the class \verb|ltxdoc| is used, you can also use
% the configuration file for this class to add
% package \verb|hypdoc|. Add the following line
% to the file \verb|ltxdoc.cfg| (or create a new one):
% \begin{quote}
%\begin{verbatim}
%\AtEndOfClass{\RequirePackage{hypdoc}}
%\end{verbatim}
% \end{quote}
%
% \subsection{Options}
%
% There is one option \texttt{numbered}. It controls,
% whether the starred versions of section or chapter
% inside \cs{tableofcontents}, \cs{listoffigures}, \dots
% should be replaced by the non-star versions.
%
% \subsection{Simple example}
% Example of a simple driver file \verb|foobar.drv|
% of a package \verb|foobar|. Often the driver file
% is packed in the file \verb|foobar.dtx| and can
% be extracted by docstrip or if an instruction file
% for docstrip is provided by:
% \begin{quote}
% \verb|tex foobar.ins|
% \end{quote}
% Note that although docstrip doesn't rely on \LaTeX,
% sometimes you need to run the \verb|.ins| file through
% \LaTeX. Oddly enough, often the only \LaTeX\ feature
% in use is \verb|\NeedsTeXFormat{LaTeX2e}| in the first line.
% \begin{quote}
%\begin{verbatim}
%\documentclass{ltxdoc}
%\usepackage{hypdoc}
%\RecordChanges
%\EnableCrossrefs
%\CodelineIndex
%\begin{document}
%\title{Title for package \texttt{foobar}}
%\author{Author of package \texttt{foobar}}
%\date{...}
%\maketitle
%\tableofcontents
%\DocInput{foobar.dtx}
%\PrintChanges
%\PrintIndex
%\end{document}
%\end{verbatim}
% \end{quote}
%
% Then compile it, for example with pdf\LaTeX.
% \begin{quote}
%\begin{verbatim}
%pdflatex foobar.drv
%makeindex -s gind.ist foobar.idx
%makeindex -s gglo.ist -o foobar.gls foobar.glo
%pdflatex foobar.drv
%\end{verbatim}
% \end{quote}
% You will need several cycles until the cross references
% and similar things are correct.
%
% \subsection{Example \texttt{source2e.tex}}
%
% The source code of the base of \LaTeX\ is available as
% a bunch of \verb|.dtx| files. \LaTeX\ provides the file
% \verb|source2e.tex| to merge them all together.
%
% Now you can either load package \verb|hypdoc| in a private copy
% of this file (please, respect the LPPL) or use the configuration
% file \verb|ltxdoc.cfg|. Example that also sets A4 paper size:
% \begin{quote}
%\begin{verbatim}
%\PassOptionsToClass{a4paper}{article}
%\AtEndOfClass{\RequirePackage{hypdoc}}
%\end{verbatim}
% \end{quote}
%
% Then \verb|source2e.tex| can be compiled by pdf\TeX, for example:
% \begin{quote}
%\begin{verbatim}
%pdflatex source2e
%makeindex -s gind.ist source2e.idx
%makeindex -s gglo.ist -o source2e.glo source2e.gls
%pdflatex source2e
%makeindex -s gind.ist source2e.idx
%makeindex -s gglo.ist -o source2e.glo source2e.gls
%pdflatex source2e
%\end{verbatim}
% \end{quote}
%
% \StopEventually{
% }
%
% \section{Implementation}
%
%    \begin{macrocode}
%<*package>
\NeedsTeXFormat{LaTeX2e}
\ProvidesPackage{hypdoc}%
  [2018/09/07 v1.14 Hyper extensions for doc.sty (HO)]%
%    \end{macrocode}
%    The package identification is done at the top of the |.dtx| file
%    in order to use only one identification string.
%
%    For unique command names this package uses |HD@| as prefix
%
% \subsection{Options}
%
%    \begin{macro}{\ifHD@numbered}
%    \begin{macrocode}
\newif\ifHD@numbered
%    \end{macrocode}
%    \end{macro}
%    \begin{macrocode}
\DeclareOption{numbered}{\HD@numberedtrue}
\ProcessOptions*\relax
%    \end{macrocode}
%
% \subsection{Package loading}
%
%    \begin{macrocode}
\RequirePackage{atveryend}[2010/03/24]
\RequirePackage{doc}
%    \end{macrocode}
%    \begin{macrocode}
\ifx\numexpr\@undefined
  \RequirePackage{calc}%
\fi
%    \end{macrocode}
%    \begin{macrocode}
\RequirePackage[%
  hyperindex=false,%
  pdfusetitle,%
  colorlinks,%
  pdfpagelabels%
]{hyperref}[2002/05/09]%
\hypersetup{%
  plainpages=false,%
  bookmarksopen,%
  bookmarksnumbered,%
  bookmarksopenlevel=0,%
  pdfstartview={FitBH \hypercalcbp{%
    \paperheight-\topmargin-1in-\headheight-\headsep
  }},%
}
%    \end{macrocode}
%
%    \begin{macrocode}
\newcounter{HD@unique}
%    \end{macrocode}
%    \begin{macro}{\theHD@unique}
%    \begin{macrocode}
\renewcommand{\theHD@unique}{%
  \number\value{HD@unique}%
}
%    \end{macrocode}
%    \end{macro}
%
%    \begin{macrocode}
\g@addto@macro\Hy@UseMaketitleInfos{%
  \@ifundefined{fileinfo}{}{%
    \ifx\@pdfsubject\@empty
      \hypersetup{pdfsubject={\fileinfo}}%
    \fi
  }%
}
%    \end{macrocode}
%
%    \begin{macrocode}
\pdfstringdefDisableCommands{%
  \let\thanks\@gobble
  \let\footnote\@gobble
  \def\cs#1{\textbackslash #1}%
  \let\normalfont\@empty
  \let\scshape\@empty
  \def\and{and }%
}
%    \end{macrocode}
%
% \subsubsection{Patch for bug latex/4096}
%
%    \begin{macrocode}
\begingroup\expandafter\expandafter\expandafter\endgroup
\expandafter\ifx\csname mod@math@codes\endcsname\relax
\else
  \def\HD@ModFix#1{%
    \mathcode`#1=`#1\relax
  }%
  \g@addto@macro\mod@math@codes{%
    \HD@ModFix\-%
    \HD@ModFix\+%
    \HD@ModFix\:%
    \HD@ModFix\=%
    \HD@ModFix\*%
  }%
\fi
%    \end{macrocode}
%
% \subsection{Bookmarks}
%
%    \begin{macro}{\HD@sectionpatch}
%    \begin{macrocode}
\def\HD@sectionpatch{%
  \ifHD@numbered
    \HD@@sectionpatch{section}%
    \HD@@sectionpatch{chapter}%
  \else
    \let\Hy@writebookmark\HD@disable@writebookmark
  \fi
}
%    \end{macrocode}
%    \end{macro}
%    \begin{macro}{\HD@@sectionpatch}
%    \begin{macrocode}
\def\HD@@sectionpatch#1{%
  \expandafter\let\csname HDorg@@#1\expandafter\endcsname
    \csname #1\endcsname
  \@namedef{#1}{%
    \@ifstar{\@nameuse{HDorg@@#1}}{\@nameuse{HDorg@@#1}}%
  }%
}
%    \end{macrocode}
%    \end{macro}
%    \begin{macro}{\HD@disable@writebookmark}
%    \begin{macrocode}
\def\HD@disable@writebookmark#1#2#3#4#5{}
%    \end{macrocode}
%    \end{macro}
%
%    \begin{macro}{\HD@guesstoclevel}
%    A wild guessing of the toclevel. There is the hope, that
%    the macro |#1| starts with a sectioning command.
%    \begin{macrocode}
\def\HD@guesstoclevel#1{%
  \expandafter\HD@@guesstoclevel#1\section\HD@end
}
%    \end{macrocode}
%    \end{macro}
%    \begin{macro}{\HD@@guesstoclevel}
%    \begin{macrocode}
\def\HD@@guesstoclevel#1#2\HD@end{%
  \@ifundefined{toclevel@\expandafter\@gobble\string#1\@empty}{%
    1%
  }{%
    \@nameuse{toclevel@\expandafter\@gobble\string#1\@empty}%
  }%
}
%    \end{macrocode}
%    \end{macro}
%
% \subsection{\cs{tableofcontents}}
%
%    \begin{macro}{\HDorg@tableofcontents}
%    \begin{macrocode}
\let\HDorg@tableofcontents\tableofcontents
%    \end{macrocode}
%    \end{macro}
%    \begin{macro}{\tableofcontents}
%    As first section the table of contents shouldn't
%    go into the toc itself. Also a number looks too funny then.
%    Just a bookmark is enough in this case.
%    \begin{macrocode}
\def\tableofcontents{%
  \begingroup
    \@ifundefined{c@section}{%
      \def\c@section{0}%
    }{}%
  \expandafter\endgroup
  \ifcase\c@section
    \stepcounter{HD@unique}%
    \pdfbookmark[\HD@guesstoclevel{\HDorg@tableofcontents}]%
        {\contentsname}{toc\theHD@unique}%
    \HDorg@tableofcontents
  \else
    \ifHD@numbered
    \else
      \stepcounter{HD@unique}%
      \pdfbookmark[\HD@guesstoclevel{\HDorg@tableofcontents}]%
          {\contentsname}{toc\theHD@unique}%
    \fi
    \begingroup
      \HD@sectionpatch
      \HDorg@tableofcontents
    \endgroup
  \fi
}
%    \end{macrocode}
%    \end{macro}
%
% \subsection{\cs{listoffigures}}
%
%    \begin{macro}{\HDorg@listoffigures}
%    \begin{macrocode}
\let\HDorg@listoffigures\listoffigures
%    \end{macrocode}
%    \end{macro}
%    \begin{macro}{\listoffigures}
%    \begin{macrocode}
\def\listoffigures{%
  \ifHD@numbered
  \else
    \stepcounter{HD@unique}%
    \pdfbookmark[\HD@guesstoclevel{\HDorg@listoffigures}]%
        {\listfigurename}{toc\theHD@unique}%
  \fi
  \begingroup
    \HD@sectionpatch
    \HDorg@listoffigures
  \endgroup
}
%    \end{macrocode}
%    \end{macro}
%
% \subsection{\cs{listoftables}}
%
%    \begin{macro}{\HDorg@listoftables}
%    \begin{macrocode}
\let\HDorg@listoftables\listoftables
%    \end{macrocode}
%    \end{macro}
%    \begin{macro}{\listoftables}
%    \begin{macrocode}
\def\listoftables{%
  \ifHD@numbered
  \else
    \stepcounter{HD@unique}%
    \pdfbookmark[\HD@guesstoclevel{\HDorg@listoftables}]%
        {\listtablename}{toc\theHD@unique}%
  \fi
  \begingroup
    \HD@sectionpatch
    \HDorg@listoftables
  \endgroup
}
%    \end{macrocode}
%    \end{macro}
%
% \subsection{\cs{thebibliography}}
%
%    \begin{macro}{\HDorg@thebibliography}
%    \begin{macrocode}
\let\HDorg@thebibliography\thebibliography
%    \end{macrocode}
%    \end{macro}
%    \begin{macro}{\thebibliography}
%    \begin{macrocode}
\def\thebibliography{%
  \ifHD@numbered
  \else
    \stepcounter{HD@unique}%
    \pdfbookmark[\HD@guesstoclevel{\HDorg@thebibliography}]%
      {\@ifundefined{chapter}{\refname}{\bibname}}{bib\theHD@unique}%
  \fi
  \HD@sectionpatch
  \HDorg@thebibliography
}
%    \end{macrocode}
%    \end{macro}
%
% \subsection{\cs{theindex}}
%
%    \begin{macro}{\HDorg@theindex}
%    \begin{macrocode}
\let\HDorg@theindex\theindex
%    \end{macrocode}
%    \end{macro}
%    \begin{macro}{\theindex}
%    \begin{macrocode}
\def\theindex{%
  \let\HDorg@index@prologue\index@prologue
  \let\HDorg@writebookmark\Hy@writebookmark
  \HD@sectionpatch
  \g@addto@macro\IndexParms{%
    \ifHD@numbered
    \else
      \let\Hy@writebookmark\HDorg@writebookmark
    \fi
    \let\bfseries\HD@bfseries
  }%
  \edef\HD@toclevel@index{\HD@guesstoclevel{\index@prologue}}%
  \begingroup
    \count@=\HD@toclevel@index\relax
    \advance\count@ by 1 %
    \xdef\HD@toclevel@subindex{\the\count@}%
  \endgroup
  \def\index@prologue{%
    \ifHD@numbered
    \else
      \begingroup
        \stepcounter{HD@unique}%
        \let\Hy@writebookmark\HDorg@writebookmark
        \pdfbookmark[\HD@toclevel@index]%
            {\indexname}{index\theHD@unique}%
      \endgroup
    \fi
    \HDorg@index@prologue
  }%
  \HDorg@theindex
}
%    \end{macrocode}
%    \end{macro}
%    \begin{macro}{\HDorg@bfseries}
%    \begin{macrocode}
\let\HDorg@bfseries\bfseries
%    \end{macrocode}
%    \end{macro}
%    \begin{macro}{\HD@bfseries}
%    \begin{macrocode}
\def\HD@bfseries{\futurelet\HD@hfil\HD@@bfseries}%
%    \end{macrocode}
%    \end{macro}
%    \begin{macro}{\HD@@bfseries}
%    \begin{macrocode}
\def\HD@@bfseries{%
  \HDorg@bfseries
  \ifx\HD@hfil\hfil
    \expandafter\HD@@@bfseries
  \fi
}
%    \end{macrocode}
%    \end{macro}
%    \begin{macro}{\HD@@@bfseries}
%    \begin{macrocode}
\def\HD@@@bfseries\hfil#1\hfil{%
  \ifx\\#1\\%
  \else
    \raisebox{\baselineskip}[0pt]{%
      \kern-\HD@margin\relax
      \pdfbookmark[\HD@toclevel@subindex]{#1}{HD.#1}%
      \kern\HD@margin\relax
    }%
  \fi
  \hfil#1\hfil
}%
%    \end{macrocode}
%    \end{macro}
%    \begin{macro}{\HD@margin}
%    \begin{macrocode}
\def\HD@margin{1mm}
%    \end{macrocode}
%    \end{macro}
%
% \subsection{\cs{theglossary}}
%
%    \begin{macro}{\HDorg@theglossary}
%    \begin{macrocode}
\let\HDorg@theglossary\theglossary
%    \end{macrocode}
%    \end{macro}
%    \begin{macro}{\theglossary}
%    \begin{macrocode}
\def\theglossary{%
  \let\HDorg@glossary@prologue\glossary@prologue
  \let\HDorg@writebookmark\Hy@writebookmark
  \HD@sectionpatch
  \def\glossary@prologue{%
    \ifHD@numbered
    \else
      \begingroup
        \let\Hy@writebookmark\HDorg@writebookmark
        \stepcounter{HD@unique}%
        \pdfbookmark[\HD@guesstoclevel{\HDorg@glossary@prologue}]%
            {\changehistoryname}{glossary\theHD@unique}%
      \endgroup
    \fi
    \HDorg@glossary@prologue
  }%
  \HDorg@theglossary
}
%    \end{macrocode}
%    \end{macro}
%    \begin{macro}{\glossaryname}
%    \begin{macrocode}
\providecommand*{\glossaryname}{Change History}
%    \end{macrocode}
%    \end{macro}
%
%    \begin{macro}{\glossaryname}
%    \begin{macrocode}
\providecommand*{\changehistoryname}{\glossaryname}
%    \end{macrocode}
%    \end{macro}
%
% \subsection{Index}
%
%    \begin{macrocode}
\newcounter{HD@hypercount}
%    \end{macrocode}
%    \begin{macro}{\HD@helplength}
%    \begin{macrocode}
\newlength{\HD@helplength}
%    \end{macrocode}
%    \end{macro}
%    \begin{macro}{\ifHD@savedest}
%    \begin{macrocode}
\newif\ifHD@savedest
\HD@savedesttrue
%    \end{macrocode}
%    \end{macro}
%    \begin{macrocode}
\AtBeginDocument{%
  \HD@savedest@init
}
%    \end{macrocode}
%    \begin{macrocode}
\RequirePackage{rerunfilecheck}[2009/12/10]
%    \end{macrocode}
%    \begin{macro}{\HD@savedest@init}
%    \begin{macrocode}
\begingroup
  \catcode`\^^M=\active
\@firstofone{\endgroup
  \def\HD@savedest@init{%
    \ifHD@savedest
      \let\HDorg@tilde~%
      \let~\HD@savedest@entry
      \edef\HDorg@catcodeM{\the\catcode`\^^M}%
      \catcode`\^^M=\active
      \def^^M{}%
      \InputIfFileExists{\jobname.hd}{}{}%
      \let~\HDorg@tilde
      \catcode`\^^M=\HDorg@catcodeM\relax
      \newwrite\HD@savedest@out
      \RerunFileCheck{\jobname.hd}{%
        \immediate\closeout\HD@savedest@out
      }{}%
      \immediate\openout\HD@savedest@out=\jobname.hd\relax
    \else
      \let\HD@savedest@add\@gobble
    \fi
  }%
%    \end{macrocode}
%    \end{macro}
%    \begin{macro}{\HD@savedest@entry}
%    \begin{macrocode}
  \def\HD@savedest@entry#1^^M{%
    \@namedef{HD.#1}{}%
  }%
}
%    \end{macrocode}
%    \end{macro}
%    \begin{macro}{\HD@savedest@add}
%    \begin{macrocode}
\def\HD@savedest@add#1{%
  \immediate\write\HD@savedest@out{\string~#1}%
}
%    \end{macrocode}
%    \end{macro}
%    \begin{macro}{\ifHD@targetraise}
%    \begin{macrocode}
\newif\ifHD@targetraise
\HD@targetraisetrue
%    \end{macrocode}
%    \end{macro}
%    \begin{macro}{\HD@target}
%    \begin{macrocode}
\def\HD@target{%
  \ifHD@targetraise
    \ifhmode
      \settoheight{\HD@helplength}{[H]}%
      \addtolength{\HD@helplength}{0.4ex}%
      \raisebox{\HD@helplength}[0pt][0pt]{%
        \HD@maketarget
      }%
    \else
      \HD@maketarget
    \fi
  \else
    \HD@maketarget
  \fi
}
%    \end{macrocode}
%    \end{macro}
%    \begin{macro}{\HD@maketarget}
%    \begin{macrocode}
\def\HD@maketarget{%
  \stepcounter{HD@hypercount}%
  \def\HD@next{%
    \llap{\hypertarget{HD.\the\c@HD@hypercount}{}\,}%
  }%
  \ifHD@savedest
    \expandafter\ifx\csname HD.\the\c@HD@hypercount\endcsname\@empty
    \else
      \let\HD@next\relax
    \fi
  \fi
  \HD@next
}
%    \end{macrocode}
%    \end{macro}
%
%    \begin{macro}{\CodelineIndex}
%    \begin{macrocode}
\g@addto@macro\CodelineIndex{%
  \let\special@index\HD@codeline@wrindex
}
%    \end{macrocode}
%    \end{macro}
%    \begin{macro}{\PageIndex}
%    \begin{macrocode}
\g@addto@macro\PageIndex{%
  \let\special@index\HD@page@wrindex
}
%    \end{macrocode}
%    \end{macro}
%
%    \begin{macro}{\HD@codeline@wrindex}
%    \begin{macrocode}
\def\HD@codeline@wrindex#1{%
  \begingroup
    \let\HDorg@encapchar\encapchar
    \def\encapchar##1\encapchar##2\@nil{%
      \HDorg@encapchar
      hdclindex{\the\c@HD@hypercount}{##1}%
    }%
    \codeline@wrindex{%
      #1\encapchar\encapchar\@nil
    }%
  \endgroup
}
%    \end{macrocode}
%    \end{macro}
%    \begin{macro}{\HD@page@wrindex}
%    \begin{macrocode}
\def\HD@page@wrindex#1{%
  \begingroup
    \let\HDorg@encapchar\encapchar
    \def\encapchar##1\encapchar##2\@nil{%
      \HDorg@encapchar
      hdpindex{##1}%
    }%
    \HDorg@index{#1\encapchar\encapchar\@nil}%
  \endgroup
}
%    \end{macrocode}
%    \end{macro}
%
%    \begin{macro}{\hdclindex}
%    \begin{macrocode}
\def\hdclindex#1#2#3{%
  \csname\ifx\\#2\\relax\else#2\fi\endcsname{%
    \hyperlink{HD.#1}{#3}%
    \HD@savedest@add{#1}%
  }%
}
%    \end{macrocode}
%    \end{macro}
%    \begin{macro}{\hdpindex}
%    \begin{macrocode}
\def\hdpindex#1#2{%
  \csname\ifx\\#1\\relax\else#1\fi\endcsname{%
    \hyperpage{#2}%
  }%
}
%    \end{macrocode}
%    \end{macro}
%
% \subsection{\cs{PrintDescribeMacro}}
%
%    \begin{macro}{\HDorg@PrintDescribeMacro}
%    \begin{macrocode}
\let\HDorg@PrintDescribeMacro\PrintDescribeMacro
%    \end{macrocode}
%    \end{macro}
%    \begin{macro}{\PrintDescribeMacro}
%    \begin{macrocode}
\renewcommand\PrintDescribeMacro[1]{%
  \begingroup
    \let\HDorg@MacroFont\MacroFont
    \def\MacroFont{%
      \HD@target
      \HDorg@MacroFont
    }%
    \HDorg@PrintDescribeMacro{#1}%
  \endgroup
}
%    \end{macrocode}
%    \end{macro}
%
% \subsection{\cs{PrintMacroName}}
%
%    \begin{macro}{\HDorg@PrintMacroName}
%    \begin{macrocode}
\let\HDorg@PrintMacroName\PrintMacroName
%    \end{macrocode}
%    \end{macro}
%    \begin{macro}{\PrintMacroName}
%    \begin{macrocode}
\renewcommand\PrintMacroName[1]{%
  \begingroup
    \let\HDorg@MacroFont\MacroFont
    \def\MacroFont{%
      \HD@target
      \HDorg@MacroFont
    }%
    \HDorg@PrintMacroName{#1}%
  \endgroup
}
%    \end{macrocode}
%    \end{macro}
%
% \subsection{\cs{theCodelineNo}}
%
%    \begin{macro}{\HDorg@theCodelineNo}
%    \begin{macrocode}
\let\HDorg@theCodelineNo\theCodelineNo
%    \end{macrocode}
%    \end{macro}
%    \begin{macro}{\theCodelineNo}
%    \begin{macrocode}
\renewcommand*\theCodelineNo{%
  \settowidth{\HD@helplength}{\HDorg@theCodelineNo\ }%
  \ifdim\HD@helplength<\MacroIndent
    \addtolength{\HD@helplength}{-\MacroIndent}%
    \llap{%
      \begingroup\HD@target\endgroup
      \hspace*{-\HD@helplength}%
    }%
  \else
    \HD@target
  \fi
  \HDorg@theCodelineNo
}
%    \end{macrocode}
%    \end{macro}
%
%    \begin{macro}{\HDorg@index}
%    \begin{macrocode}
\AtBeginDocument{%
  \let\HDorg@index\index
}
%    \end{macrocode}
%    \end{macro}
%
% \subsection{\cs{SpecialUsageIndex}}
%
%    \begin{macro}{\HDorg@SpecialUsageIndex}
%    \begin{macrocode}
\let\HDorg@SpecialUsageIndex\SpecialUsageIndex
\renewcommand*\SpecialUsageIndex[1]{%
  \@bsphack
  \begingroup
    \HD@target
    \let\index\HDorg@index
    \let\HDorg@encapchar\encapchar
    \edef\encapchar usage{%
       \HDorg@encapchar hdclindex{\the\c@HD@hypercount}{usage}%
    }%
    \HDorg@SpecialUsageIndex{#1}%
  \endgroup
  \@esphack
}
%    \end{macrocode}
%    \end{macro}
%
% \subsection{\cs{autoindex}}
%
%    \begin{macrocode}
\begingroup\expandafter\expandafter\expandafter\endgroup
\expandafter\ifx\csname autoindex\endcsname\relax
\else
%    \end{macrocode}
%    \begin{macro}{\HDorg@autoindex}
%    \begin{macrocode}
  \let\HDorg@autoindex\autoindex
%    \end{macrocode}
%    \end{macro}
%    \begin{macro}{\autoindex}
%    \begin{macrocode}
  \renewcommand*\autoindex[1]{%
    \@bsphack
    \begingroup
      \HD@target
      \let\index\HDorg@index
      \let\HDorg@encapchar\encapchar
      \edef\encapchar usage{%
         \HDorg@encapchar hdclindex{\the\c@HD@hypercount}{usage}%
      }%
      \HDorg@autoindex{#1}%
    \endgroup
    \@esphack
  }%
\fi
%    \end{macrocode}
%    \end{macro}
%
% \subsection{\cs{SpecialEnvIndex}}
%
%    \begin{macro}{\HDorg@SpecialEnvIndex}
%    \begin{macrocode}
\let\HDorg@SpecialEnvIndex\SpecialEnvIndex
%    \end{macrocode}
%    \end{macro}
%    \begin{macro}{\SpecialEnvIndex}
%    \begin{macrocode}
\renewcommand*\SpecialEnvIndex[1]{%
  \@bsphack
  \begingroup
    \HD@target
    \let\HDorg@encapchar\encapchar
    \edef\encapchar usage{%
      \HDorg@encapchar hdclindex{\the\c@HD@hypercount}{usage}%
    }%
    \HDorg@SpecialEnvIndex{#1}%
  \endgroup
  \@esphack
}
%    \end{macrocode}
%    \end{macro}
%
% \subsection{\cs{SortIndex}}
%
%    \begin{macro}{\HDorg@SortIndex}
%    \begin{macrocode}
\let\HDorg@SortIndex\SortIndex
%    \end{macrocode}
%    \end{macro}
%    \begin{macro}{\SortIndex}
%    \begin{macrocode}
\renewcommand*{\SortIndex}[2]{%
  \@bsphack
  \begingroup
    \let\index\HD@page@wrindex
    \HDorg@SortIndex{#1}{#2}%
  \endgroup
  \@esphack
}
%    \end{macrocode}
%    \end{macro}
%
% \subsection{\cs{@wrglossary}}
%
%    \begin{macro}{\HDorg@wrglossary}
%    \begin{macrocode}
\let\HDorg@wrglossary\@wrglossary
%    \end{macrocode}
%    \end{macro}
%    \begin{macro}{\@wrglossary}
%    \begin{macrocode}
\def\@wrglossary#1{%
  \let\HDorg@encapchar\encapchar
  \def\encapchar##1\encapchar##2\@nil{%
    \HDorg@encapchar
    hdpindex{##1}%
  }%
  \HDorg@wrglossary{#1\encapchar\encapchar\@nil}%
}
%    \end{macrocode}
%    \end{macro}
%
% \subsection{\cs{MacroIndent}}
%
%    \cs{MacroIndent} automatically calculated.
%    \begin{macrocode}
\AfterLastShipout{%
  \if@filesw
    \immediate\write\@auxout{%
      \string\providecommand\string\HD@SetMacroIndent[1]{}%
    }%
    \immediate\write\@auxout{%
      \string\HD@SetMacroIndent{\number\value{CodelineNo}}%
    }%
  \fi
}
%    \end{macrocode}
%    \begin{macro}{\HD@SetMacroIndent}
%    \begin{macrocode}
\def\HD@SetMacroIndent#1{%
  \begingroup
    \value{CodelineNo}=#1\relax
    \settowidth{\MacroIndent}{\HDorg@theCodelineNo\ }%
    \global\MacroIndent\MacroIndent
  \endgroup
}
%    \end{macrocode}
%    \end{macro}
%
%    \begin{macrocode}
%</package>
%    \end{macrocode}
%
% \section{Installation}
%
% \subsection{Download}
%
% \paragraph{Package.} This package is available on
% CTAN\footnote{\url{http://ctan.org/pkg/hypdoc}}:
% \begin{description}
% \item[\CTAN{macros/latex/contrib/oberdiek/hypdoc.dtx}] The source file.
% \item[\CTAN{macros/latex/contrib/oberdiek/hypdoc.pdf}] Documentation.
% \end{description}
%
%
% \paragraph{Bundle.} All the packages of the bundle `oberdiek'
% are also available in a TDS compliant ZIP archive. There
% the packages are already unpacked and the documentation files
% are generated. The files and directories obey the TDS standard.
% \begin{description}
% \item[\CTAN{install/macros/latex/contrib/oberdiek.tds.zip}]
% \end{description}
% \emph{TDS} refers to the standard ``A Directory Structure
% for \TeX\ Files'' (\CTAN{tds/tds.pdf}). Directories
% with \xfile{texmf} in their name are usually organized this way.
%
% \subsection{Bundle installation}
%
% \paragraph{Unpacking.} Unpack the \xfile{oberdiek.tds.zip} in the
% TDS tree (also known as \xfile{texmf} tree) of your choice.
% Example (linux):
% \begin{quote}
%   |unzip oberdiek.tds.zip -d ~/texmf|
% \end{quote}
%
% \paragraph{Script installation.}
% Check the directory \xfile{TDS:scripts/oberdiek/} for
% scripts that need further installation steps.
% Package \xpackage{attachfile2} comes with the Perl script
% \xfile{pdfatfi.pl} that should be installed in such a way
% that it can be called as \texttt{pdfatfi}.
% Example (linux):
% \begin{quote}
%   |chmod +x scripts/oberdiek/pdfatfi.pl|\\
%   |cp scripts/oberdiek/pdfatfi.pl /usr/local/bin/|
% \end{quote}
%
% \subsection{Package installation}
%
% \paragraph{Unpacking.} The \xfile{.dtx} file is a self-extracting
% \docstrip\ archive. The files are extracted by running the
% \xfile{.dtx} through \plainTeX:
% \begin{quote}
%   \verb|tex hypdoc.dtx|
% \end{quote}
%
% \paragraph{TDS.} Now the different files must be moved into
% the different directories in your installation TDS tree
% (also known as \xfile{texmf} tree):
% \begin{quote}
% \def\t{^^A
% \begin{tabular}{@{}>{\ttfamily}l@{ $\rightarrow$ }>{\ttfamily}l@{}}
%   hypdoc.sty & tex/latex/oberdiek/hypdoc.sty\\
%   hypdoc.pdf & doc/latex/oberdiek/hypdoc.pdf\\
%   hypdoc.dtx & source/latex/oberdiek/hypdoc.dtx\\
% \end{tabular}^^A
% }^^A
% \sbox0{\t}^^A
% \ifdim\wd0>\linewidth
%   \begingroup
%     \advance\linewidth by\leftmargin
%     \advance\linewidth by\rightmargin
%   \edef\x{\endgroup
%     \def\noexpand\lw{\the\linewidth}^^A
%   }\x
%   \def\lwbox{^^A
%     \leavevmode
%     \hbox to \linewidth{^^A
%       \kern-\leftmargin\relax
%       \hss
%       \usebox0
%       \hss
%       \kern-\rightmargin\relax
%     }^^A
%   }^^A
%   \ifdim\wd0>\lw
%     \sbox0{\small\t}^^A
%     \ifdim\wd0>\linewidth
%       \ifdim\wd0>\lw
%         \sbox0{\footnotesize\t}^^A
%         \ifdim\wd0>\linewidth
%           \ifdim\wd0>\lw
%             \sbox0{\scriptsize\t}^^A
%             \ifdim\wd0>\linewidth
%               \ifdim\wd0>\lw
%                 \sbox0{\tiny\t}^^A
%                 \ifdim\wd0>\linewidth
%                   \lwbox
%                 \else
%                   \usebox0
%                 \fi
%               \else
%                 \lwbox
%               \fi
%             \else
%               \usebox0
%             \fi
%           \else
%             \lwbox
%           \fi
%         \else
%           \usebox0
%         \fi
%       \else
%         \lwbox
%       \fi
%     \else
%       \usebox0
%     \fi
%   \else
%     \lwbox
%   \fi
% \else
%   \usebox0
% \fi
% \end{quote}
% If you have a \xfile{docstrip.cfg} that configures and enables \docstrip's
% TDS installing feature, then some files can already be in the right
% place, see the documentation of \docstrip.
%
% \subsection{Refresh file name databases}
%
% If your \TeX~distribution
% (\teTeX, \mikTeX, \dots) relies on file name databases, you must refresh
% these. For example, \teTeX\ users run \verb|texhash| or
% \verb|mktexlsr|.
%
% \subsection{Some details for the interested}
%
% \paragraph{Attached source.}
%
% The PDF documentation on CTAN also includes the
% \xfile{.dtx} source file. It can be extracted by
% AcrobatReader 6 or higher. Another option is \textsf{pdftk},
% e.g. unpack the file into the current directory:
% \begin{quote}
%   \verb|pdftk hypdoc.pdf unpack_files output .|
% \end{quote}
%
% \paragraph{Unpacking with \LaTeX.}
% The \xfile{.dtx} chooses its action depending on the format:
% \begin{description}
% \item[\plainTeX:] Run \docstrip\ and extract the files.
% \item[\LaTeX:] Generate the documentation.
% \end{description}
% If you insist on using \LaTeX\ for \docstrip\ (really,
% \docstrip\ does not need \LaTeX), then inform the autodetect routine
% about your intention:
% \begin{quote}
%   \verb|latex \let\install=y% \iffalse meta-comment
%
% File: hypdoc.dtx
% Version: 2018/09/07 v1.14
% Info: Hyper extensions for doc.sty
%
% Copyright (C) 2002, 2006, 2007, 2009-2011 by
%    Heiko Oberdiek <heiko.oberdiek at googlemail.com>
%    2016
%    https://github.com/ho-tex/oberdiek/issues
%
% This work may be distributed and/or modified under the
% conditions of the LaTeX Project Public License, either
% version 1.3c of this license or (at your option) any later
% version. This version of this license is in
%    http://www.latex-project.org/lppl/lppl-1-3c.txt
% and the latest version of this license is in
%    http://www.latex-project.org/lppl.txt
% and version 1.3 or later is part of all distributions of
% LaTeX version 2005/12/01 or later.
%
% This work has the LPPL maintenance status "maintained".
%
% This Current Maintainer of this work is Heiko Oberdiek.
%
% This work consists of the main source file hypdoc.dtx
% and the derived files
%    hypdoc.sty, hypdoc.pdf, hypdoc.ins, hypdoc.drv.
%
% Distribution:
%    CTAN:macros/latex/contrib/oberdiek/hypdoc.dtx
%    CTAN:macros/latex/contrib/oberdiek/hypdoc.pdf
%
% Unpacking:
%    (a) If hypdoc.ins is present:
%           tex hypdoc.ins
%    (b) Without hypdoc.ins:
%           tex hypdoc.dtx
%    (c) If you insist on using LaTeX
%           latex \let\install=y% \iffalse meta-comment
%
% File: hypdoc.dtx
% Version: 2018/09/07 v1.14
% Info: Hyper extensions for doc.sty
%
% Copyright (C) 2002, 2006, 2007, 2009-2011 by
%    Heiko Oberdiek <heiko.oberdiek at googlemail.com>
%    2016
%    https://github.com/ho-tex/oberdiek/issues
%
% This work may be distributed and/or modified under the
% conditions of the LaTeX Project Public License, either
% version 1.3c of this license or (at your option) any later
% version. This version of this license is in
%    http://www.latex-project.org/lppl/lppl-1-3c.txt
% and the latest version of this license is in
%    http://www.latex-project.org/lppl.txt
% and version 1.3 or later is part of all distributions of
% LaTeX version 2005/12/01 or later.
%
% This work has the LPPL maintenance status "maintained".
%
% This Current Maintainer of this work is Heiko Oberdiek.
%
% This work consists of the main source file hypdoc.dtx
% and the derived files
%    hypdoc.sty, hypdoc.pdf, hypdoc.ins, hypdoc.drv.
%
% Distribution:
%    CTAN:macros/latex/contrib/oberdiek/hypdoc.dtx
%    CTAN:macros/latex/contrib/oberdiek/hypdoc.pdf
%
% Unpacking:
%    (a) If hypdoc.ins is present:
%           tex hypdoc.ins
%    (b) Without hypdoc.ins:
%           tex hypdoc.dtx
%    (c) If you insist on using LaTeX
%           latex \let\install=y\input{hypdoc.dtx}
%        (quote the arguments according to the demands of your shell)
%
% Documentation:
%    (a) If hypdoc.drv is present:
%           latex hypdoc.drv
%    (b) Without hypdoc.drv:
%           latex hypdoc.dtx; ...
%    The class ltxdoc loads the configuration file ltxdoc.cfg
%    if available. Here you can specify further options, e.g.
%    use A4 as paper format:
%       \PassOptionsToClass{a4paper}{article}
%
%    Programm calls to get the documentation (example):
%       pdflatex hypdoc.dtx
%       makeindex -s gind.ist hypdoc.idx
%       pdflatex hypdoc.dtx
%       makeindex -s gind.ist hypdoc.idx
%       pdflatex hypdoc.dtx
%
% Installation:
%    TDS:tex/latex/oberdiek/hypdoc.sty
%    TDS:doc/latex/oberdiek/hypdoc.pdf
%    TDS:source/latex/oberdiek/hypdoc.dtx
%
%<*ignore>
\begingroup
  \catcode123=1 %
  \catcode125=2 %
  \def\x{LaTeX2e}%
\expandafter\endgroup
\ifcase 0\ifx\install y1\fi\expandafter
         \ifx\csname processbatchFile\endcsname\relax\else1\fi
         \ifx\fmtname\x\else 1\fi\relax
\else\csname fi\endcsname
%</ignore>
%<*install>
\input docstrip.tex
\Msg{************************************************************************}
\Msg{* Installation}
\Msg{* Package: hypdoc 2018/09/07 v1.14 Hyper extensions for doc.sty (HO)}
\Msg{************************************************************************}

\keepsilent
\askforoverwritefalse

\let\MetaPrefix\relax
\preamble

This is a generated file.

Project: hypdoc
Version: 2018/09/07 v1.14

Copyright (C) 2002, 2006, 2007, 2009-2011 by
   Heiko Oberdiek <heiko.oberdiek at googlemail.com>

This work may be distributed and/or modified under the
conditions of the LaTeX Project Public License, either
version 1.3c of this license or (at your option) any later
version. This version of this license is in
   http://www.latex-project.org/lppl/lppl-1-3c.txt
and the latest version of this license is in
   http://www.latex-project.org/lppl.txt
and version 1.3 or later is part of all distributions of
LaTeX version 2005/12/01 or later.

This work has the LPPL maintenance status "maintained".

This Current Maintainer of this work is Heiko Oberdiek.

This work consists of the main source file hypdoc.dtx
and the derived files
   hypdoc.sty, hypdoc.pdf, hypdoc.ins, hypdoc.drv.

\endpreamble
\let\MetaPrefix\DoubleperCent

\generate{%
  \file{hypdoc.ins}{\from{hypdoc.dtx}{install}}%
  \file{hypdoc.drv}{\from{hypdoc.dtx}{driver}}%
  \usedir{tex/latex/oberdiek}%
  \file{hypdoc.sty}{\from{hypdoc.dtx}{package}}%
  \nopreamble
  \nopostamble
%  \usedir{source/latex/oberdiek/catalogue}%
%  \file{hypdoc.xml}{\from{hypdoc.dtx}{catalogue}}%
}

\catcode32=13\relax% active space
\let =\space%
\Msg{************************************************************************}
\Msg{*}
\Msg{* To finish the installation you have to move the following}
\Msg{* file into a directory searched by TeX:}
\Msg{*}
\Msg{*     hypdoc.sty}
\Msg{*}
\Msg{* To produce the documentation run the file `hypdoc.drv'}
\Msg{* through LaTeX.}
\Msg{*}
\Msg{* Happy TeXing!}
\Msg{*}
\Msg{************************************************************************}

\endbatchfile
%</install>
%<*ignore>
\fi
%</ignore>
%<*driver>
\NeedsTeXFormat{LaTeX2e}
\ProvidesFile{hypdoc.drv}%
  [2018/09/07 v1.14 Hyper extensions for doc.sty (HO)]%
\documentclass{ltxdoc}
\usepackage{holtxdoc}[2011/11/22]
\begin{document}
  \DocInput{hypdoc.dtx}%
\end{document}
%</driver>
% \fi
%
%
% \CharacterTable
%  {Upper-case    \A\B\C\D\E\F\G\H\I\J\K\L\M\N\O\P\Q\R\S\T\U\V\W\X\Y\Z
%   Lower-case    \a\b\c\d\e\f\g\h\i\j\k\l\m\n\o\p\q\r\s\t\u\v\w\x\y\z
%   Digits        \0\1\2\3\4\5\6\7\8\9
%   Exclamation   \!     Double quote  \"     Hash (number) \#
%   Dollar        \$     Percent       \%     Ampersand     \&
%   Acute accent  \'     Left paren    \(     Right paren   \)
%   Asterisk      \*     Plus          \+     Comma         \,
%   Minus         \-     Point         \.     Solidus       \/
%   Colon         \:     Semicolon     \;     Less than     \<
%   Equals        \=     Greater than  \>     Question mark \?
%   Commercial at \@     Left bracket  \[     Backslash     \\
%   Right bracket \]     Circumflex    \^     Underscore    \_
%   Grave accent  \`     Left brace    \{     Vertical bar  \|
%   Right brace   \}     Tilde         \~}
%
% \GetFileInfo{hypdoc.drv}
%
% \title{The \xpackage{hypdoc} package}
% \date{2018/09/07 v1.14}
% \author{Heiko Oberdiek\thanks
% {Please report any issues at https://github.com/ho-tex/oberdiek/issues}\\
% \xemail{heiko.oberdiek at googlemail.com}}
%
% \maketitle
%
% \begin{abstract}
% This package adds hyper features to the package
% \xpackage{doc} that is used in the documentation
% system of \LaTeXe. Bookmarks are added and references
% are linked as far as possible.
% \end{abstract}
%
% \tableofcontents
%
% \section{Features}
%
% The package adds the following features:
% \begin{description}
% \item[Bookmarks:]
%   For supporting bookmarks package \verb|hyperref| is loaded.
%   Some \TeX\ code inside section titles can cause problems
%   during bookmark processing; these can be warnings, ugly
%   bookmark titles or even errors. With the help of
%   \verb|\pdfstringdefDisableCommands| or \verb|\texorpdfstring|
%   you can manage these things, see \verb|hyperref|'s package
%   documentation.
% \item[Additional bookmarks:]
%   Unnumbered bookmark entries are generated for:
%   \begin{itemize}
%   \item Table of contents
%   \item List of figures
%   \item List of tables
%   \item Index
%   \item Glossary
%   \end{itemize}
%   Option \texttt{numbered} controls, whether these section or
%   chapters should be numbered. The exception is the table
%   of contents if this it would be the first numbered
%   section. Then just a bookmark is added.
% \item[Metadata:]
%   The loaded package \verb|hyperref|, enforced by
%   option \verb|pdfusetitle|, tries to detect \verb|\title|
%   and \verb|\author| and sets the corresponding PDF
%   information fields. In case of problems see the previous
%   item about bookmarks.
%   These enries can be overwritten or new ones added by
%   \verb|\hypersetup| after the packages is loaded, see
%   documentation of package \verb|hyperref|.
% \item[References, links:] Because of package \verb|\hyperref|
%   |\ref| become links, also \verb|\url|, footnote referencing
%   are supported.
% \item[Index:]
%   The main part of this package deals with index links in order
%   to support the kind of index that package \verb|doc| provides.
% \item[Thumbnails:]
%   Package \verb|thumbpdf| may additionally be loaded for the case that the
%   PDF file also should contain thumbnails, see the documentation
%   of this package. But thumbnail embedding is not really
%   necessary, because nowadays AcrobatReader is able to
%   generate and view thumbnails on the fly.
% \end{description}
%
% \section{Usage}
%
% The purpose of this packages is to add PDF features to
% the documentation of LaTeX packages. Thus just load
% the package in the driver file that generates the
% documentation:
% \begin{quote}
%   \verb|\usepackage{hypdoc}|
% \end{quote}
% Package \verb|doc| is loaded by \verb|hypdoc|, thus you
% also can replace a \verb|\usepackage{doc}| with this line.
% At any case, the package must be loaded before macros of
% package \verb|doc| such as \verb|\CodelineIndex| or
% \verb|\PageIndex| are used.
%
% If the class \verb|ltxdoc| is used, you can also use
% the configuration file for this class to add
% package \verb|hypdoc|. Add the following line
% to the file \verb|ltxdoc.cfg| (or create a new one):
% \begin{quote}
%\begin{verbatim}
%\AtEndOfClass{\RequirePackage{hypdoc}}
%\end{verbatim}
% \end{quote}
%
% \subsection{Options}
%
% There is one option \texttt{numbered}. It controls,
% whether the starred versions of section or chapter
% inside \cs{tableofcontents}, \cs{listoffigures}, \dots
% should be replaced by the non-star versions.
%
% \subsection{Simple example}
% Example of a simple driver file \verb|foobar.drv|
% of a package \verb|foobar|. Often the driver file
% is packed in the file \verb|foobar.dtx| and can
% be extracted by docstrip or if an instruction file
% for docstrip is provided by:
% \begin{quote}
% \verb|tex foobar.ins|
% \end{quote}
% Note that although docstrip doesn't rely on \LaTeX,
% sometimes you need to run the \verb|.ins| file through
% \LaTeX. Oddly enough, often the only \LaTeX\ feature
% in use is \verb|\NeedsTeXFormat{LaTeX2e}| in the first line.
% \begin{quote}
%\begin{verbatim}
%\documentclass{ltxdoc}
%\usepackage{hypdoc}
%\RecordChanges
%\EnableCrossrefs
%\CodelineIndex
%\begin{document}
%\title{Title for package \texttt{foobar}}
%\author{Author of package \texttt{foobar}}
%\date{...}
%\maketitle
%\tableofcontents
%\DocInput{foobar.dtx}
%\PrintChanges
%\PrintIndex
%\end{document}
%\end{verbatim}
% \end{quote}
%
% Then compile it, for example with pdf\LaTeX.
% \begin{quote}
%\begin{verbatim}
%pdflatex foobar.drv
%makeindex -s gind.ist foobar.idx
%makeindex -s gglo.ist -o foobar.gls foobar.glo
%pdflatex foobar.drv
%\end{verbatim}
% \end{quote}
% You will need several cycles until the cross references
% and similar things are correct.
%
% \subsection{Example \texttt{source2e.tex}}
%
% The source code of the base of \LaTeX\ is available as
% a bunch of \verb|.dtx| files. \LaTeX\ provides the file
% \verb|source2e.tex| to merge them all together.
%
% Now you can either load package \verb|hypdoc| in a private copy
% of this file (please, respect the LPPL) or use the configuration
% file \verb|ltxdoc.cfg|. Example that also sets A4 paper size:
% \begin{quote}
%\begin{verbatim}
%\PassOptionsToClass{a4paper}{article}
%\AtEndOfClass{\RequirePackage{hypdoc}}
%\end{verbatim}
% \end{quote}
%
% Then \verb|source2e.tex| can be compiled by pdf\TeX, for example:
% \begin{quote}
%\begin{verbatim}
%pdflatex source2e
%makeindex -s gind.ist source2e.idx
%makeindex -s gglo.ist -o source2e.glo source2e.gls
%pdflatex source2e
%makeindex -s gind.ist source2e.idx
%makeindex -s gglo.ist -o source2e.glo source2e.gls
%pdflatex source2e
%\end{verbatim}
% \end{quote}
%
% \StopEventually{
% }
%
% \section{Implementation}
%
%    \begin{macrocode}
%<*package>
\NeedsTeXFormat{LaTeX2e}
\ProvidesPackage{hypdoc}%
  [2018/09/07 v1.14 Hyper extensions for doc.sty (HO)]%
%    \end{macrocode}
%    The package identification is done at the top of the |.dtx| file
%    in order to use only one identification string.
%
%    For unique command names this package uses |HD@| as prefix
%
% \subsection{Options}
%
%    \begin{macro}{\ifHD@numbered}
%    \begin{macrocode}
\newif\ifHD@numbered
%    \end{macrocode}
%    \end{macro}
%    \begin{macrocode}
\DeclareOption{numbered}{\HD@numberedtrue}
\ProcessOptions*\relax
%    \end{macrocode}
%
% \subsection{Package loading}
%
%    \begin{macrocode}
\RequirePackage{atveryend}[2010/03/24]
\RequirePackage{doc}
%    \end{macrocode}
%    \begin{macrocode}
\ifx\numexpr\@undefined
  \RequirePackage{calc}%
\fi
%    \end{macrocode}
%    \begin{macrocode}
\RequirePackage[%
  hyperindex=false,%
  pdfusetitle,%
  colorlinks,%
  pdfpagelabels%
]{hyperref}[2002/05/09]%
\hypersetup{%
  plainpages=false,%
  bookmarksopen,%
  bookmarksnumbered,%
  bookmarksopenlevel=0,%
  pdfstartview={FitBH \hypercalcbp{%
    \paperheight-\topmargin-1in-\headheight-\headsep
  }},%
}
%    \end{macrocode}
%
%    \begin{macrocode}
\newcounter{HD@unique}
%    \end{macrocode}
%    \begin{macro}{\theHD@unique}
%    \begin{macrocode}
\renewcommand{\theHD@unique}{%
  \number\value{HD@unique}%
}
%    \end{macrocode}
%    \end{macro}
%
%    \begin{macrocode}
\g@addto@macro\Hy@UseMaketitleInfos{%
  \@ifundefined{fileinfo}{}{%
    \ifx\@pdfsubject\@empty
      \hypersetup{pdfsubject={\fileinfo}}%
    \fi
  }%
}
%    \end{macrocode}
%
%    \begin{macrocode}
\pdfstringdefDisableCommands{%
  \let\thanks\@gobble
  \let\footnote\@gobble
  \def\cs#1{\textbackslash #1}%
  \let\normalfont\@empty
  \let\scshape\@empty
  \def\and{and }%
}
%    \end{macrocode}
%
% \subsubsection{Patch for bug latex/4096}
%
%    \begin{macrocode}
\begingroup\expandafter\expandafter\expandafter\endgroup
\expandafter\ifx\csname mod@math@codes\endcsname\relax
\else
  \def\HD@ModFix#1{%
    \mathcode`#1=`#1\relax
  }%
  \g@addto@macro\mod@math@codes{%
    \HD@ModFix\-%
    \HD@ModFix\+%
    \HD@ModFix\:%
    \HD@ModFix\=%
    \HD@ModFix\*%
  }%
\fi
%    \end{macrocode}
%
% \subsection{Bookmarks}
%
%    \begin{macro}{\HD@sectionpatch}
%    \begin{macrocode}
\def\HD@sectionpatch{%
  \ifHD@numbered
    \HD@@sectionpatch{section}%
    \HD@@sectionpatch{chapter}%
  \else
    \let\Hy@writebookmark\HD@disable@writebookmark
  \fi
}
%    \end{macrocode}
%    \end{macro}
%    \begin{macro}{\HD@@sectionpatch}
%    \begin{macrocode}
\def\HD@@sectionpatch#1{%
  \expandafter\let\csname HDorg@@#1\expandafter\endcsname
    \csname #1\endcsname
  \@namedef{#1}{%
    \@ifstar{\@nameuse{HDorg@@#1}}{\@nameuse{HDorg@@#1}}%
  }%
}
%    \end{macrocode}
%    \end{macro}
%    \begin{macro}{\HD@disable@writebookmark}
%    \begin{macrocode}
\def\HD@disable@writebookmark#1#2#3#4#5{}
%    \end{macrocode}
%    \end{macro}
%
%    \begin{macro}{\HD@guesstoclevel}
%    A wild guessing of the toclevel. There is the hope, that
%    the macro |#1| starts with a sectioning command.
%    \begin{macrocode}
\def\HD@guesstoclevel#1{%
  \expandafter\HD@@guesstoclevel#1\section\HD@end
}
%    \end{macrocode}
%    \end{macro}
%    \begin{macro}{\HD@@guesstoclevel}
%    \begin{macrocode}
\def\HD@@guesstoclevel#1#2\HD@end{%
  \@ifundefined{toclevel@\expandafter\@gobble\string#1\@empty}{%
    1%
  }{%
    \@nameuse{toclevel@\expandafter\@gobble\string#1\@empty}%
  }%
}
%    \end{macrocode}
%    \end{macro}
%
% \subsection{\cs{tableofcontents}}
%
%    \begin{macro}{\HDorg@tableofcontents}
%    \begin{macrocode}
\let\HDorg@tableofcontents\tableofcontents
%    \end{macrocode}
%    \end{macro}
%    \begin{macro}{\tableofcontents}
%    As first section the table of contents shouldn't
%    go into the toc itself. Also a number looks too funny then.
%    Just a bookmark is enough in this case.
%    \begin{macrocode}
\def\tableofcontents{%
  \begingroup
    \@ifundefined{c@section}{%
      \def\c@section{0}%
    }{}%
  \expandafter\endgroup
  \ifcase\c@section
    \stepcounter{HD@unique}%
    \pdfbookmark[\HD@guesstoclevel{\HDorg@tableofcontents}]%
        {\contentsname}{toc\theHD@unique}%
    \HDorg@tableofcontents
  \else
    \ifHD@numbered
    \else
      \stepcounter{HD@unique}%
      \pdfbookmark[\HD@guesstoclevel{\HDorg@tableofcontents}]%
          {\contentsname}{toc\theHD@unique}%
    \fi
    \begingroup
      \HD@sectionpatch
      \HDorg@tableofcontents
    \endgroup
  \fi
}
%    \end{macrocode}
%    \end{macro}
%
% \subsection{\cs{listoffigures}}
%
%    \begin{macro}{\HDorg@listoffigures}
%    \begin{macrocode}
\let\HDorg@listoffigures\listoffigures
%    \end{macrocode}
%    \end{macro}
%    \begin{macro}{\listoffigures}
%    \begin{macrocode}
\def\listoffigures{%
  \ifHD@numbered
  \else
    \stepcounter{HD@unique}%
    \pdfbookmark[\HD@guesstoclevel{\HDorg@listoffigures}]%
        {\listfigurename}{toc\theHD@unique}%
  \fi
  \begingroup
    \HD@sectionpatch
    \HDorg@listoffigures
  \endgroup
}
%    \end{macrocode}
%    \end{macro}
%
% \subsection{\cs{listoftables}}
%
%    \begin{macro}{\HDorg@listoftables}
%    \begin{macrocode}
\let\HDorg@listoftables\listoftables
%    \end{macrocode}
%    \end{macro}
%    \begin{macro}{\listoftables}
%    \begin{macrocode}
\def\listoftables{%
  \ifHD@numbered
  \else
    \stepcounter{HD@unique}%
    \pdfbookmark[\HD@guesstoclevel{\HDorg@listoftables}]%
        {\listtablename}{toc\theHD@unique}%
  \fi
  \begingroup
    \HD@sectionpatch
    \HDorg@listoftables
  \endgroup
}
%    \end{macrocode}
%    \end{macro}
%
% \subsection{\cs{thebibliography}}
%
%    \begin{macro}{\HDorg@thebibliography}
%    \begin{macrocode}
\let\HDorg@thebibliography\thebibliography
%    \end{macrocode}
%    \end{macro}
%    \begin{macro}{\thebibliography}
%    \begin{macrocode}
\def\thebibliography{%
  \ifHD@numbered
  \else
    \stepcounter{HD@unique}%
    \pdfbookmark[\HD@guesstoclevel{\HDorg@thebibliography}]%
      {\@ifundefined{chapter}{\refname}{\bibname}}{bib\theHD@unique}%
  \fi
  \HD@sectionpatch
  \HDorg@thebibliography
}
%    \end{macrocode}
%    \end{macro}
%
% \subsection{\cs{theindex}}
%
%    \begin{macro}{\HDorg@theindex}
%    \begin{macrocode}
\let\HDorg@theindex\theindex
%    \end{macrocode}
%    \end{macro}
%    \begin{macro}{\theindex}
%    \begin{macrocode}
\def\theindex{%
  \let\HDorg@index@prologue\index@prologue
  \let\HDorg@writebookmark\Hy@writebookmark
  \HD@sectionpatch
  \g@addto@macro\IndexParms{%
    \ifHD@numbered
    \else
      \let\Hy@writebookmark\HDorg@writebookmark
    \fi
    \let\bfseries\HD@bfseries
  }%
  \edef\HD@toclevel@index{\HD@guesstoclevel{\index@prologue}}%
  \begingroup
    \count@=\HD@toclevel@index\relax
    \advance\count@ by 1 %
    \xdef\HD@toclevel@subindex{\the\count@}%
  \endgroup
  \def\index@prologue{%
    \ifHD@numbered
    \else
      \begingroup
        \stepcounter{HD@unique}%
        \let\Hy@writebookmark\HDorg@writebookmark
        \pdfbookmark[\HD@toclevel@index]%
            {\indexname}{index\theHD@unique}%
      \endgroup
    \fi
    \HDorg@index@prologue
  }%
  \HDorg@theindex
}
%    \end{macrocode}
%    \end{macro}
%    \begin{macro}{\HDorg@bfseries}
%    \begin{macrocode}
\let\HDorg@bfseries\bfseries
%    \end{macrocode}
%    \end{macro}
%    \begin{macro}{\HD@bfseries}
%    \begin{macrocode}
\def\HD@bfseries{\futurelet\HD@hfil\HD@@bfseries}%
%    \end{macrocode}
%    \end{macro}
%    \begin{macro}{\HD@@bfseries}
%    \begin{macrocode}
\def\HD@@bfseries{%
  \HDorg@bfseries
  \ifx\HD@hfil\hfil
    \expandafter\HD@@@bfseries
  \fi
}
%    \end{macrocode}
%    \end{macro}
%    \begin{macro}{\HD@@@bfseries}
%    \begin{macrocode}
\def\HD@@@bfseries\hfil#1\hfil{%
  \ifx\\#1\\%
  \else
    \raisebox{\baselineskip}[0pt]{%
      \kern-\HD@margin\relax
      \pdfbookmark[\HD@toclevel@subindex]{#1}{HD.#1}%
      \kern\HD@margin\relax
    }%
  \fi
  \hfil#1\hfil
}%
%    \end{macrocode}
%    \end{macro}
%    \begin{macro}{\HD@margin}
%    \begin{macrocode}
\def\HD@margin{1mm}
%    \end{macrocode}
%    \end{macro}
%
% \subsection{\cs{theglossary}}
%
%    \begin{macro}{\HDorg@theglossary}
%    \begin{macrocode}
\let\HDorg@theglossary\theglossary
%    \end{macrocode}
%    \end{macro}
%    \begin{macro}{\theglossary}
%    \begin{macrocode}
\def\theglossary{%
  \let\HDorg@glossary@prologue\glossary@prologue
  \let\HDorg@writebookmark\Hy@writebookmark
  \HD@sectionpatch
  \def\glossary@prologue{%
    \ifHD@numbered
    \else
      \begingroup
        \let\Hy@writebookmark\HDorg@writebookmark
        \stepcounter{HD@unique}%
        \pdfbookmark[\HD@guesstoclevel{\HDorg@glossary@prologue}]%
            {\changehistoryname}{glossary\theHD@unique}%
      \endgroup
    \fi
    \HDorg@glossary@prologue
  }%
  \HDorg@theglossary
}
%    \end{macrocode}
%    \end{macro}
%    \begin{macro}{\glossaryname}
%    \begin{macrocode}
\providecommand*{\glossaryname}{Change History}
%    \end{macrocode}
%    \end{macro}
%
%    \begin{macro}{\glossaryname}
%    \begin{macrocode}
\providecommand*{\changehistoryname}{\glossaryname}
%    \end{macrocode}
%    \end{macro}
%
% \subsection{Index}
%
%    \begin{macrocode}
\newcounter{HD@hypercount}
%    \end{macrocode}
%    \begin{macro}{\HD@helplength}
%    \begin{macrocode}
\newlength{\HD@helplength}
%    \end{macrocode}
%    \end{macro}
%    \begin{macro}{\ifHD@savedest}
%    \begin{macrocode}
\newif\ifHD@savedest
\HD@savedesttrue
%    \end{macrocode}
%    \end{macro}
%    \begin{macrocode}
\AtBeginDocument{%
  \HD@savedest@init
}
%    \end{macrocode}
%    \begin{macrocode}
\RequirePackage{rerunfilecheck}[2009/12/10]
%    \end{macrocode}
%    \begin{macro}{\HD@savedest@init}
%    \begin{macrocode}
\begingroup
  \catcode`\^^M=\active
\@firstofone{\endgroup
  \def\HD@savedest@init{%
    \ifHD@savedest
      \let\HDorg@tilde~%
      \let~\HD@savedest@entry
      \edef\HDorg@catcodeM{\the\catcode`\^^M}%
      \catcode`\^^M=\active
      \def^^M{}%
      \InputIfFileExists{\jobname.hd}{}{}%
      \let~\HDorg@tilde
      \catcode`\^^M=\HDorg@catcodeM\relax
      \newwrite\HD@savedest@out
      \RerunFileCheck{\jobname.hd}{%
        \immediate\closeout\HD@savedest@out
      }{}%
      \immediate\openout\HD@savedest@out=\jobname.hd\relax
    \else
      \let\HD@savedest@add\@gobble
    \fi
  }%
%    \end{macrocode}
%    \end{macro}
%    \begin{macro}{\HD@savedest@entry}
%    \begin{macrocode}
  \def\HD@savedest@entry#1^^M{%
    \@namedef{HD.#1}{}%
  }%
}
%    \end{macrocode}
%    \end{macro}
%    \begin{macro}{\HD@savedest@add}
%    \begin{macrocode}
\def\HD@savedest@add#1{%
  \immediate\write\HD@savedest@out{\string~#1}%
}
%    \end{macrocode}
%    \end{macro}
%    \begin{macro}{\ifHD@targetraise}
%    \begin{macrocode}
\newif\ifHD@targetraise
\HD@targetraisetrue
%    \end{macrocode}
%    \end{macro}
%    \begin{macro}{\HD@target}
%    \begin{macrocode}
\def\HD@target{%
  \ifHD@targetraise
    \ifhmode
      \settoheight{\HD@helplength}{[H]}%
      \addtolength{\HD@helplength}{0.4ex}%
      \raisebox{\HD@helplength}[0pt][0pt]{%
        \HD@maketarget
      }%
    \else
      \HD@maketarget
    \fi
  \else
    \HD@maketarget
  \fi
}
%    \end{macrocode}
%    \end{macro}
%    \begin{macro}{\HD@maketarget}
%    \begin{macrocode}
\def\HD@maketarget{%
  \stepcounter{HD@hypercount}%
  \def\HD@next{%
    \llap{\hypertarget{HD.\the\c@HD@hypercount}{}\,}%
  }%
  \ifHD@savedest
    \expandafter\ifx\csname HD.\the\c@HD@hypercount\endcsname\@empty
    \else
      \let\HD@next\relax
    \fi
  \fi
  \HD@next
}
%    \end{macrocode}
%    \end{macro}
%
%    \begin{macro}{\CodelineIndex}
%    \begin{macrocode}
\g@addto@macro\CodelineIndex{%
  \let\special@index\HD@codeline@wrindex
}
%    \end{macrocode}
%    \end{macro}
%    \begin{macro}{\PageIndex}
%    \begin{macrocode}
\g@addto@macro\PageIndex{%
  \let\special@index\HD@page@wrindex
}
%    \end{macrocode}
%    \end{macro}
%
%    \begin{macro}{\HD@codeline@wrindex}
%    \begin{macrocode}
\def\HD@codeline@wrindex#1{%
  \begingroup
    \let\HDorg@encapchar\encapchar
    \def\encapchar##1\encapchar##2\@nil{%
      \HDorg@encapchar
      hdclindex{\the\c@HD@hypercount}{##1}%
    }%
    \codeline@wrindex{%
      #1\encapchar\encapchar\@nil
    }%
  \endgroup
}
%    \end{macrocode}
%    \end{macro}
%    \begin{macro}{\HD@page@wrindex}
%    \begin{macrocode}
\def\HD@page@wrindex#1{%
  \begingroup
    \let\HDorg@encapchar\encapchar
    \def\encapchar##1\encapchar##2\@nil{%
      \HDorg@encapchar
      hdpindex{##1}%
    }%
    \HDorg@index{#1\encapchar\encapchar\@nil}%
  \endgroup
}
%    \end{macrocode}
%    \end{macro}
%
%    \begin{macro}{\hdclindex}
%    \begin{macrocode}
\def\hdclindex#1#2#3{%
  \csname\ifx\\#2\\relax\else#2\fi\endcsname{%
    \hyperlink{HD.#1}{#3}%
    \HD@savedest@add{#1}%
  }%
}
%    \end{macrocode}
%    \end{macro}
%    \begin{macro}{\hdpindex}
%    \begin{macrocode}
\def\hdpindex#1#2{%
  \csname\ifx\\#1\\relax\else#1\fi\endcsname{%
    \hyperpage{#2}%
  }%
}
%    \end{macrocode}
%    \end{macro}
%
% \subsection{\cs{PrintDescribeMacro}}
%
%    \begin{macro}{\HDorg@PrintDescribeMacro}
%    \begin{macrocode}
\let\HDorg@PrintDescribeMacro\PrintDescribeMacro
%    \end{macrocode}
%    \end{macro}
%    \begin{macro}{\PrintDescribeMacro}
%    \begin{macrocode}
\renewcommand\PrintDescribeMacro[1]{%
  \begingroup
    \let\HDorg@MacroFont\MacroFont
    \def\MacroFont{%
      \HD@target
      \HDorg@MacroFont
    }%
    \HDorg@PrintDescribeMacro{#1}%
  \endgroup
}
%    \end{macrocode}
%    \end{macro}
%
% \subsection{\cs{PrintMacroName}}
%
%    \begin{macro}{\HDorg@PrintMacroName}
%    \begin{macrocode}
\let\HDorg@PrintMacroName\PrintMacroName
%    \end{macrocode}
%    \end{macro}
%    \begin{macro}{\PrintMacroName}
%    \begin{macrocode}
\renewcommand\PrintMacroName[1]{%
  \begingroup
    \let\HDorg@MacroFont\MacroFont
    \def\MacroFont{%
      \HD@target
      \HDorg@MacroFont
    }%
    \HDorg@PrintMacroName{#1}%
  \endgroup
}
%    \end{macrocode}
%    \end{macro}
%
% \subsection{\cs{theCodelineNo}}
%
%    \begin{macro}{\HDorg@theCodelineNo}
%    \begin{macrocode}
\let\HDorg@theCodelineNo\theCodelineNo
%    \end{macrocode}
%    \end{macro}
%    \begin{macro}{\theCodelineNo}
%    \begin{macrocode}
\renewcommand*\theCodelineNo{%
  \settowidth{\HD@helplength}{\HDorg@theCodelineNo\ }%
  \ifdim\HD@helplength<\MacroIndent
    \addtolength{\HD@helplength}{-\MacroIndent}%
    \llap{%
      \begingroup\HD@target\endgroup
      \hspace*{-\HD@helplength}%
    }%
  \else
    \HD@target
  \fi
  \HDorg@theCodelineNo
}
%    \end{macrocode}
%    \end{macro}
%
%    \begin{macro}{\HDorg@index}
%    \begin{macrocode}
\AtBeginDocument{%
  \let\HDorg@index\index
}
%    \end{macrocode}
%    \end{macro}
%
% \subsection{\cs{SpecialUsageIndex}}
%
%    \begin{macro}{\HDorg@SpecialUsageIndex}
%    \begin{macrocode}
\let\HDorg@SpecialUsageIndex\SpecialUsageIndex
\renewcommand*\SpecialUsageIndex[1]{%
  \@bsphack
  \begingroup
    \HD@target
    \let\index\HDorg@index
    \let\HDorg@encapchar\encapchar
    \edef\encapchar usage{%
       \HDorg@encapchar hdclindex{\the\c@HD@hypercount}{usage}%
    }%
    \HDorg@SpecialUsageIndex{#1}%
  \endgroup
  \@esphack
}
%    \end{macrocode}
%    \end{macro}
%
% \subsection{\cs{autoindex}}
%
%    \begin{macrocode}
\begingroup\expandafter\expandafter\expandafter\endgroup
\expandafter\ifx\csname autoindex\endcsname\relax
\else
%    \end{macrocode}
%    \begin{macro}{\HDorg@autoindex}
%    \begin{macrocode}
  \let\HDorg@autoindex\autoindex
%    \end{macrocode}
%    \end{macro}
%    \begin{macro}{\autoindex}
%    \begin{macrocode}
  \renewcommand*\autoindex[1]{%
    \@bsphack
    \begingroup
      \HD@target
      \let\index\HDorg@index
      \let\HDorg@encapchar\encapchar
      \edef\encapchar usage{%
         \HDorg@encapchar hdclindex{\the\c@HD@hypercount}{usage}%
      }%
      \HDorg@autoindex{#1}%
    \endgroup
    \@esphack
  }%
\fi
%    \end{macrocode}
%    \end{macro}
%
% \subsection{\cs{SpecialEnvIndex}}
%
%    \begin{macro}{\HDorg@SpecialEnvIndex}
%    \begin{macrocode}
\let\HDorg@SpecialEnvIndex\SpecialEnvIndex
%    \end{macrocode}
%    \end{macro}
%    \begin{macro}{\SpecialEnvIndex}
%    \begin{macrocode}
\renewcommand*\SpecialEnvIndex[1]{%
  \@bsphack
  \begingroup
    \HD@target
    \let\HDorg@encapchar\encapchar
    \edef\encapchar usage{%
      \HDorg@encapchar hdclindex{\the\c@HD@hypercount}{usage}%
    }%
    \HDorg@SpecialEnvIndex{#1}%
  \endgroup
  \@esphack
}
%    \end{macrocode}
%    \end{macro}
%
% \subsection{\cs{SortIndex}}
%
%    \begin{macro}{\HDorg@SortIndex}
%    \begin{macrocode}
\let\HDorg@SortIndex\SortIndex
%    \end{macrocode}
%    \end{macro}
%    \begin{macro}{\SortIndex}
%    \begin{macrocode}
\renewcommand*{\SortIndex}[2]{%
  \@bsphack
  \begingroup
    \let\index\HD@page@wrindex
    \HDorg@SortIndex{#1}{#2}%
  \endgroup
  \@esphack
}
%    \end{macrocode}
%    \end{macro}
%
% \subsection{\cs{@wrglossary}}
%
%    \begin{macro}{\HDorg@wrglossary}
%    \begin{macrocode}
\let\HDorg@wrglossary\@wrglossary
%    \end{macrocode}
%    \end{macro}
%    \begin{macro}{\@wrglossary}
%    \begin{macrocode}
\def\@wrglossary#1{%
  \let\HDorg@encapchar\encapchar
  \def\encapchar##1\encapchar##2\@nil{%
    \HDorg@encapchar
    hdpindex{##1}%
  }%
  \HDorg@wrglossary{#1\encapchar\encapchar\@nil}%
}
%    \end{macrocode}
%    \end{macro}
%
% \subsection{\cs{MacroIndent}}
%
%    \cs{MacroIndent} automatically calculated.
%    \begin{macrocode}
\AfterLastShipout{%
  \if@filesw
    \immediate\write\@auxout{%
      \string\providecommand\string\HD@SetMacroIndent[1]{}%
    }%
    \immediate\write\@auxout{%
      \string\HD@SetMacroIndent{\number\value{CodelineNo}}%
    }%
  \fi
}
%    \end{macrocode}
%    \begin{macro}{\HD@SetMacroIndent}
%    \begin{macrocode}
\def\HD@SetMacroIndent#1{%
  \begingroup
    \value{CodelineNo}=#1\relax
    \settowidth{\MacroIndent}{\HDorg@theCodelineNo\ }%
    \global\MacroIndent\MacroIndent
  \endgroup
}
%    \end{macrocode}
%    \end{macro}
%
%    \begin{macrocode}
%</package>
%    \end{macrocode}
%
% \section{Installation}
%
% \subsection{Download}
%
% \paragraph{Package.} This package is available on
% CTAN\footnote{\url{http://ctan.org/pkg/hypdoc}}:
% \begin{description}
% \item[\CTAN{macros/latex/contrib/oberdiek/hypdoc.dtx}] The source file.
% \item[\CTAN{macros/latex/contrib/oberdiek/hypdoc.pdf}] Documentation.
% \end{description}
%
%
% \paragraph{Bundle.} All the packages of the bundle `oberdiek'
% are also available in a TDS compliant ZIP archive. There
% the packages are already unpacked and the documentation files
% are generated. The files and directories obey the TDS standard.
% \begin{description}
% \item[\CTAN{install/macros/latex/contrib/oberdiek.tds.zip}]
% \end{description}
% \emph{TDS} refers to the standard ``A Directory Structure
% for \TeX\ Files'' (\CTAN{tds/tds.pdf}). Directories
% with \xfile{texmf} in their name are usually organized this way.
%
% \subsection{Bundle installation}
%
% \paragraph{Unpacking.} Unpack the \xfile{oberdiek.tds.zip} in the
% TDS tree (also known as \xfile{texmf} tree) of your choice.
% Example (linux):
% \begin{quote}
%   |unzip oberdiek.tds.zip -d ~/texmf|
% \end{quote}
%
% \paragraph{Script installation.}
% Check the directory \xfile{TDS:scripts/oberdiek/} for
% scripts that need further installation steps.
% Package \xpackage{attachfile2} comes with the Perl script
% \xfile{pdfatfi.pl} that should be installed in such a way
% that it can be called as \texttt{pdfatfi}.
% Example (linux):
% \begin{quote}
%   |chmod +x scripts/oberdiek/pdfatfi.pl|\\
%   |cp scripts/oberdiek/pdfatfi.pl /usr/local/bin/|
% \end{quote}
%
% \subsection{Package installation}
%
% \paragraph{Unpacking.} The \xfile{.dtx} file is a self-extracting
% \docstrip\ archive. The files are extracted by running the
% \xfile{.dtx} through \plainTeX:
% \begin{quote}
%   \verb|tex hypdoc.dtx|
% \end{quote}
%
% \paragraph{TDS.} Now the different files must be moved into
% the different directories in your installation TDS tree
% (also known as \xfile{texmf} tree):
% \begin{quote}
% \def\t{^^A
% \begin{tabular}{@{}>{\ttfamily}l@{ $\rightarrow$ }>{\ttfamily}l@{}}
%   hypdoc.sty & tex/latex/oberdiek/hypdoc.sty\\
%   hypdoc.pdf & doc/latex/oberdiek/hypdoc.pdf\\
%   hypdoc.dtx & source/latex/oberdiek/hypdoc.dtx\\
% \end{tabular}^^A
% }^^A
% \sbox0{\t}^^A
% \ifdim\wd0>\linewidth
%   \begingroup
%     \advance\linewidth by\leftmargin
%     \advance\linewidth by\rightmargin
%   \edef\x{\endgroup
%     \def\noexpand\lw{\the\linewidth}^^A
%   }\x
%   \def\lwbox{^^A
%     \leavevmode
%     \hbox to \linewidth{^^A
%       \kern-\leftmargin\relax
%       \hss
%       \usebox0
%       \hss
%       \kern-\rightmargin\relax
%     }^^A
%   }^^A
%   \ifdim\wd0>\lw
%     \sbox0{\small\t}^^A
%     \ifdim\wd0>\linewidth
%       \ifdim\wd0>\lw
%         \sbox0{\footnotesize\t}^^A
%         \ifdim\wd0>\linewidth
%           \ifdim\wd0>\lw
%             \sbox0{\scriptsize\t}^^A
%             \ifdim\wd0>\linewidth
%               \ifdim\wd0>\lw
%                 \sbox0{\tiny\t}^^A
%                 \ifdim\wd0>\linewidth
%                   \lwbox
%                 \else
%                   \usebox0
%                 \fi
%               \else
%                 \lwbox
%               \fi
%             \else
%               \usebox0
%             \fi
%           \else
%             \lwbox
%           \fi
%         \else
%           \usebox0
%         \fi
%       \else
%         \lwbox
%       \fi
%     \else
%       \usebox0
%     \fi
%   \else
%     \lwbox
%   \fi
% \else
%   \usebox0
% \fi
% \end{quote}
% If you have a \xfile{docstrip.cfg} that configures and enables \docstrip's
% TDS installing feature, then some files can already be in the right
% place, see the documentation of \docstrip.
%
% \subsection{Refresh file name databases}
%
% If your \TeX~distribution
% (\teTeX, \mikTeX, \dots) relies on file name databases, you must refresh
% these. For example, \teTeX\ users run \verb|texhash| or
% \verb|mktexlsr|.
%
% \subsection{Some details for the interested}
%
% \paragraph{Attached source.}
%
% The PDF documentation on CTAN also includes the
% \xfile{.dtx} source file. It can be extracted by
% AcrobatReader 6 or higher. Another option is \textsf{pdftk},
% e.g. unpack the file into the current directory:
% \begin{quote}
%   \verb|pdftk hypdoc.pdf unpack_files output .|
% \end{quote}
%
% \paragraph{Unpacking with \LaTeX.}
% The \xfile{.dtx} chooses its action depending on the format:
% \begin{description}
% \item[\plainTeX:] Run \docstrip\ and extract the files.
% \item[\LaTeX:] Generate the documentation.
% \end{description}
% If you insist on using \LaTeX\ for \docstrip\ (really,
% \docstrip\ does not need \LaTeX), then inform the autodetect routine
% about your intention:
% \begin{quote}
%   \verb|latex \let\install=y\input{hypdoc.dtx}|
% \end{quote}
% Do not forget to quote the argument according to the demands
% of your shell.
%
% \paragraph{Generating the documentation.}
% You can use both the \xfile{.dtx} or the \xfile{.drv} to generate
% the documentation. The process can be configured by the
% configuration file \xfile{ltxdoc.cfg}. For instance, put this
% line into this file, if you want to have A4 as paper format:
% \begin{quote}
%   \verb|\PassOptionsToClass{a4paper}{article}|
% \end{quote}
% An example follows how to generate the
% documentation with pdf\LaTeX:
% \begin{quote}
%\begin{verbatim}
%pdflatex hypdoc.dtx
%makeindex -s gind.ist hypdoc.idx
%pdflatex hypdoc.dtx
%makeindex -s gind.ist hypdoc.idx
%pdflatex hypdoc.dtx
%\end{verbatim}
% \end{quote}
%
% \section{Catalogue}
%
% The following XML file can be used as source for the
% \href{http://mirror.ctan.org/help/Catalogue/catalogue.html}{\TeX\ Catalogue}.
% The elements \texttt{caption} and \texttt{description} are imported
% from the original XML file from the Catalogue.
% The name of the XML file in the Catalogue is \xfile{hypdoc.xml}.
%    \begin{macrocode}
%<*catalogue>
<?xml version='1.0' encoding='us-ascii'?>
<!DOCTYPE entry SYSTEM 'catalogue.dtd'>
<entry datestamp='$Date$' modifier='$Author$' id='hypdoc'>
  <name>hypdoc</name>
  <caption>Hyper extensions for doc.sty.</caption>
  <authorref id='auth:oberdiek'/>
  <copyright owner='Heiko Oberdiek' year='2002,2006,2007,2009-2011'/>
  <license type='lppl1.3'/>
  <version number='1.14'/>
  <description>
    This package adds hypertext features to the package
    <xref refid='doc'>doc</xref> that is used in the documentation
    system of <xref refid='latex'>LaTeX2e</xref>. Bookmarks are added
    and references are linked as far as possible.
    <p/>
    The package is part of the <xref refid='oberdiek'>oberdiek</xref>
    bundle.
  </description>
  <documentation details='Package documentation'
      href='ctan:/macros/latex/contrib/oberdiek/hypdoc.pdf'/>
  <ctan file='true' path='/macros/latex/contrib/oberdiek/hypdoc.dtx'/>
  <miktex location='oberdiek'/>
  <texlive location='oberdiek'/>
  <install path='/macros/latex/contrib/oberdiek/oberdiek.tds.zip'/>
</entry>
%</catalogue>
%    \end{macrocode}
%
% \begin{History}
%   \begin{Version}{2002/05/10 v1.0}
%   \item
%     First standalone version.
%   \end{Version}
%   \begin{Version}{2006/02/20 v1.1}
%   \item
%     Option \xoption{numbered} added.
%   \item
%     \cs{HD@savedest@init} fixed (empty \xext{hd} file triggered
%     first page with fl ligature).
%   \item
%     Bookmarks for \cs{listoffigures} and \cs{listoftables}.
%   \item
%     Documentation: features and usage added.
%   \item
%     LPPL 1.3.
%   \end{Version}
%   \begin{Version}{2006/04/27 v1.2}
%   \item
%     \cs{HD@target} fixed (\xfile{multicol.dtx}).
%   \end{Version}
%   \begin{Version}{2006/06/01 v1.3}
%   \item
%     Support for package \xpackage{amsdtx}'s \cs{autoindex} added.
%   \end{Version}
%   \begin{Version}{2006/06/02 v1.4}
%   \item
%     Bookmark fixed if there are several table of contents,
%     bibliographies, glossaries, \dots
%   \end{Version}
%   \begin{Version}{2007/04/11 v1.5}
%   \item
%     Line ends sanitized.
%   \end{Version}
%   \begin{Version}{2007/11/12 v1.6}
%   \item
%     Bug fix in \cs{listoffigures} (Axel Sommerfeldt).
%   \item
%     Markup added in implementation section.
%   \end{Version}
%   \begin{Version}{2009/12/10 v1.7}
%   \item
%     Use of package \xpackage{rerunfilecheck}.
%   \end{Version}
%   \begin{Version}{2010/02/03 v1.8}
%   \item
%     Fix for bug latex/4096 added.
%   \end{Version}
%   \begin{Version}{2010/03/26 v1.9}
%   \item
%     Definition of \cs{HD@SetMacroIndex} is provided in the \xext{aux} file.
%   \end{Version}
%   \begin{Version}{2011/05/05 v1.10}
%   \item
%     \cs{raisebox} in \cs{HD@target} with height and depth of \texttt{0pt}
%     (request of GL).
%   \end{Version}
%   \begin{Version}{2011/08/19 v1.11}
%   \item
%     Latest comma in option list for hyperref removed to avoid option clash
%     because of \hologo{LaTeX} 2011/06/27 with the new \cs{in@}.
%   \end{Version}
%   \begin{Version}{2016/05/16 v1.12}
%   \item
%     Documentation updates.
%   \end{Version}
%   \begin{Version}{2016/10/02 v1.13}
%   \item
%     Fixes to bookmarknames, also introduce \cs{changehistoryname}
%   \end{Version}
%   \begin{Version}{2018/09/07 v1.14}
%   \item
%     Do not load \xpackage{thumpdf}.
%   \end{Version}
% \end{History}
%
% \PrintIndex
%
% \Finale
\endinput

%        (quote the arguments according to the demands of your shell)
%
% Documentation:
%    (a) If hypdoc.drv is present:
%           latex hypdoc.drv
%    (b) Without hypdoc.drv:
%           latex hypdoc.dtx; ...
%    The class ltxdoc loads the configuration file ltxdoc.cfg
%    if available. Here you can specify further options, e.g.
%    use A4 as paper format:
%       \PassOptionsToClass{a4paper}{article}
%
%    Programm calls to get the documentation (example):
%       pdflatex hypdoc.dtx
%       makeindex -s gind.ist hypdoc.idx
%       pdflatex hypdoc.dtx
%       makeindex -s gind.ist hypdoc.idx
%       pdflatex hypdoc.dtx
%
% Installation:
%    TDS:tex/latex/oberdiek/hypdoc.sty
%    TDS:doc/latex/oberdiek/hypdoc.pdf
%    TDS:source/latex/oberdiek/hypdoc.dtx
%
%<*ignore>
\begingroup
  \catcode123=1 %
  \catcode125=2 %
  \def\x{LaTeX2e}%
\expandafter\endgroup
\ifcase 0\ifx\install y1\fi\expandafter
         \ifx\csname processbatchFile\endcsname\relax\else1\fi
         \ifx\fmtname\x\else 1\fi\relax
\else\csname fi\endcsname
%</ignore>
%<*install>
\input docstrip.tex
\Msg{************************************************************************}
\Msg{* Installation}
\Msg{* Package: hypdoc 2018/09/07 v1.14 Hyper extensions for doc.sty (HO)}
\Msg{************************************************************************}

\keepsilent
\askforoverwritefalse

\let\MetaPrefix\relax
\preamble

This is a generated file.

Project: hypdoc
Version: 2018/09/07 v1.14

Copyright (C) 2002, 2006, 2007, 2009-2011 by
   Heiko Oberdiek <heiko.oberdiek at googlemail.com>

This work may be distributed and/or modified under the
conditions of the LaTeX Project Public License, either
version 1.3c of this license or (at your option) any later
version. This version of this license is in
   http://www.latex-project.org/lppl/lppl-1-3c.txt
and the latest version of this license is in
   http://www.latex-project.org/lppl.txt
and version 1.3 or later is part of all distributions of
LaTeX version 2005/12/01 or later.

This work has the LPPL maintenance status "maintained".

This Current Maintainer of this work is Heiko Oberdiek.

This work consists of the main source file hypdoc.dtx
and the derived files
   hypdoc.sty, hypdoc.pdf, hypdoc.ins, hypdoc.drv.

\endpreamble
\let\MetaPrefix\DoubleperCent

\generate{%
  \file{hypdoc.ins}{\from{hypdoc.dtx}{install}}%
  \file{hypdoc.drv}{\from{hypdoc.dtx}{driver}}%
  \usedir{tex/latex/oberdiek}%
  \file{hypdoc.sty}{\from{hypdoc.dtx}{package}}%
  \nopreamble
  \nopostamble
%  \usedir{source/latex/oberdiek/catalogue}%
%  \file{hypdoc.xml}{\from{hypdoc.dtx}{catalogue}}%
}

\catcode32=13\relax% active space
\let =\space%
\Msg{************************************************************************}
\Msg{*}
\Msg{* To finish the installation you have to move the following}
\Msg{* file into a directory searched by TeX:}
\Msg{*}
\Msg{*     hypdoc.sty}
\Msg{*}
\Msg{* To produce the documentation run the file `hypdoc.drv'}
\Msg{* through LaTeX.}
\Msg{*}
\Msg{* Happy TeXing!}
\Msg{*}
\Msg{************************************************************************}

\endbatchfile
%</install>
%<*ignore>
\fi
%</ignore>
%<*driver>
\NeedsTeXFormat{LaTeX2e}
\ProvidesFile{hypdoc.drv}%
  [2018/09/07 v1.14 Hyper extensions for doc.sty (HO)]%
\documentclass{ltxdoc}
\usepackage{holtxdoc}[2011/11/22]
\begin{document}
  \DocInput{hypdoc.dtx}%
\end{document}
%</driver>
% \fi
%
%
% \CharacterTable
%  {Upper-case    \A\B\C\D\E\F\G\H\I\J\K\L\M\N\O\P\Q\R\S\T\U\V\W\X\Y\Z
%   Lower-case    \a\b\c\d\e\f\g\h\i\j\k\l\m\n\o\p\q\r\s\t\u\v\w\x\y\z
%   Digits        \0\1\2\3\4\5\6\7\8\9
%   Exclamation   \!     Double quote  \"     Hash (number) \#
%   Dollar        \$     Percent       \%     Ampersand     \&
%   Acute accent  \'     Left paren    \(     Right paren   \)
%   Asterisk      \*     Plus          \+     Comma         \,
%   Minus         \-     Point         \.     Solidus       \/
%   Colon         \:     Semicolon     \;     Less than     \<
%   Equals        \=     Greater than  \>     Question mark \?
%   Commercial at \@     Left bracket  \[     Backslash     \\
%   Right bracket \]     Circumflex    \^     Underscore    \_
%   Grave accent  \`     Left brace    \{     Vertical bar  \|
%   Right brace   \}     Tilde         \~}
%
% \GetFileInfo{hypdoc.drv}
%
% \title{The \xpackage{hypdoc} package}
% \date{2018/09/07 v1.14}
% \author{Heiko Oberdiek\thanks
% {Please report any issues at https://github.com/ho-tex/oberdiek/issues}\\
% \xemail{heiko.oberdiek at googlemail.com}}
%
% \maketitle
%
% \begin{abstract}
% This package adds hyper features to the package
% \xpackage{doc} that is used in the documentation
% system of \LaTeXe. Bookmarks are added and references
% are linked as far as possible.
% \end{abstract}
%
% \tableofcontents
%
% \section{Features}
%
% The package adds the following features:
% \begin{description}
% \item[Bookmarks:]
%   For supporting bookmarks package \verb|hyperref| is loaded.
%   Some \TeX\ code inside section titles can cause problems
%   during bookmark processing; these can be warnings, ugly
%   bookmark titles or even errors. With the help of
%   \verb|\pdfstringdefDisableCommands| or \verb|\texorpdfstring|
%   you can manage these things, see \verb|hyperref|'s package
%   documentation.
% \item[Additional bookmarks:]
%   Unnumbered bookmark entries are generated for:
%   \begin{itemize}
%   \item Table of contents
%   \item List of figures
%   \item List of tables
%   \item Index
%   \item Glossary
%   \end{itemize}
%   Option \texttt{numbered} controls, whether these section or
%   chapters should be numbered. The exception is the table
%   of contents if this it would be the first numbered
%   section. Then just a bookmark is added.
% \item[Metadata:]
%   The loaded package \verb|hyperref|, enforced by
%   option \verb|pdfusetitle|, tries to detect \verb|\title|
%   and \verb|\author| and sets the corresponding PDF
%   information fields. In case of problems see the previous
%   item about bookmarks.
%   These enries can be overwritten or new ones added by
%   \verb|\hypersetup| after the packages is loaded, see
%   documentation of package \verb|hyperref|.
% \item[References, links:] Because of package \verb|\hyperref|
%   |\ref| become links, also \verb|\url|, footnote referencing
%   are supported.
% \item[Index:]
%   The main part of this package deals with index links in order
%   to support the kind of index that package \verb|doc| provides.
% \item[Thumbnails:]
%   Package \verb|thumbpdf| may additionally be loaded for the case that the
%   PDF file also should contain thumbnails, see the documentation
%   of this package. But thumbnail embedding is not really
%   necessary, because nowadays AcrobatReader is able to
%   generate and view thumbnails on the fly.
% \end{description}
%
% \section{Usage}
%
% The purpose of this packages is to add PDF features to
% the documentation of LaTeX packages. Thus just load
% the package in the driver file that generates the
% documentation:
% \begin{quote}
%   \verb|\usepackage{hypdoc}|
% \end{quote}
% Package \verb|doc| is loaded by \verb|hypdoc|, thus you
% also can replace a \verb|\usepackage{doc}| with this line.
% At any case, the package must be loaded before macros of
% package \verb|doc| such as \verb|\CodelineIndex| or
% \verb|\PageIndex| are used.
%
% If the class \verb|ltxdoc| is used, you can also use
% the configuration file for this class to add
% package \verb|hypdoc|. Add the following line
% to the file \verb|ltxdoc.cfg| (or create a new one):
% \begin{quote}
%\begin{verbatim}
%\AtEndOfClass{\RequirePackage{hypdoc}}
%\end{verbatim}
% \end{quote}
%
% \subsection{Options}
%
% There is one option \texttt{numbered}. It controls,
% whether the starred versions of section or chapter
% inside \cs{tableofcontents}, \cs{listoffigures}, \dots
% should be replaced by the non-star versions.
%
% \subsection{Simple example}
% Example of a simple driver file \verb|foobar.drv|
% of a package \verb|foobar|. Often the driver file
% is packed in the file \verb|foobar.dtx| and can
% be extracted by docstrip or if an instruction file
% for docstrip is provided by:
% \begin{quote}
% \verb|tex foobar.ins|
% \end{quote}
% Note that although docstrip doesn't rely on \LaTeX,
% sometimes you need to run the \verb|.ins| file through
% \LaTeX. Oddly enough, often the only \LaTeX\ feature
% in use is \verb|\NeedsTeXFormat{LaTeX2e}| in the first line.
% \begin{quote}
%\begin{verbatim}
%\documentclass{ltxdoc}
%\usepackage{hypdoc}
%\RecordChanges
%\EnableCrossrefs
%\CodelineIndex
%\begin{document}
%\title{Title for package \texttt{foobar}}
%\author{Author of package \texttt{foobar}}
%\date{...}
%\maketitle
%\tableofcontents
%\DocInput{foobar.dtx}
%\PrintChanges
%\PrintIndex
%\end{document}
%\end{verbatim}
% \end{quote}
%
% Then compile it, for example with pdf\LaTeX.
% \begin{quote}
%\begin{verbatim}
%pdflatex foobar.drv
%makeindex -s gind.ist foobar.idx
%makeindex -s gglo.ist -o foobar.gls foobar.glo
%pdflatex foobar.drv
%\end{verbatim}
% \end{quote}
% You will need several cycles until the cross references
% and similar things are correct.
%
% \subsection{Example \texttt{source2e.tex}}
%
% The source code of the base of \LaTeX\ is available as
% a bunch of \verb|.dtx| files. \LaTeX\ provides the file
% \verb|source2e.tex| to merge them all together.
%
% Now you can either load package \verb|hypdoc| in a private copy
% of this file (please, respect the LPPL) or use the configuration
% file \verb|ltxdoc.cfg|. Example that also sets A4 paper size:
% \begin{quote}
%\begin{verbatim}
%\PassOptionsToClass{a4paper}{article}
%\AtEndOfClass{\RequirePackage{hypdoc}}
%\end{verbatim}
% \end{quote}
%
% Then \verb|source2e.tex| can be compiled by pdf\TeX, for example:
% \begin{quote}
%\begin{verbatim}
%pdflatex source2e
%makeindex -s gind.ist source2e.idx
%makeindex -s gglo.ist -o source2e.glo source2e.gls
%pdflatex source2e
%makeindex -s gind.ist source2e.idx
%makeindex -s gglo.ist -o source2e.glo source2e.gls
%pdflatex source2e
%\end{verbatim}
% \end{quote}
%
% \StopEventually{
% }
%
% \section{Implementation}
%
%    \begin{macrocode}
%<*package>
\NeedsTeXFormat{LaTeX2e}
\ProvidesPackage{hypdoc}%
  [2018/09/07 v1.14 Hyper extensions for doc.sty (HO)]%
%    \end{macrocode}
%    The package identification is done at the top of the |.dtx| file
%    in order to use only one identification string.
%
%    For unique command names this package uses |HD@| as prefix
%
% \subsection{Options}
%
%    \begin{macro}{\ifHD@numbered}
%    \begin{macrocode}
\newif\ifHD@numbered
%    \end{macrocode}
%    \end{macro}
%    \begin{macrocode}
\DeclareOption{numbered}{\HD@numberedtrue}
\ProcessOptions*\relax
%    \end{macrocode}
%
% \subsection{Package loading}
%
%    \begin{macrocode}
\RequirePackage{atveryend}[2010/03/24]
\RequirePackage{doc}
%    \end{macrocode}
%    \begin{macrocode}
\ifx\numexpr\@undefined
  \RequirePackage{calc}%
\fi
%    \end{macrocode}
%    \begin{macrocode}
\RequirePackage[%
  hyperindex=false,%
  pdfusetitle,%
  colorlinks,%
  pdfpagelabels%
]{hyperref}[2002/05/09]%
\hypersetup{%
  plainpages=false,%
  bookmarksopen,%
  bookmarksnumbered,%
  bookmarksopenlevel=0,%
  pdfstartview={FitBH \hypercalcbp{%
    \paperheight-\topmargin-1in-\headheight-\headsep
  }},%
}
%    \end{macrocode}
%
%    \begin{macrocode}
\newcounter{HD@unique}
%    \end{macrocode}
%    \begin{macro}{\theHD@unique}
%    \begin{macrocode}
\renewcommand{\theHD@unique}{%
  \number\value{HD@unique}%
}
%    \end{macrocode}
%    \end{macro}
%
%    \begin{macrocode}
\g@addto@macro\Hy@UseMaketitleInfos{%
  \@ifundefined{fileinfo}{}{%
    \ifx\@pdfsubject\@empty
      \hypersetup{pdfsubject={\fileinfo}}%
    \fi
  }%
}
%    \end{macrocode}
%
%    \begin{macrocode}
\pdfstringdefDisableCommands{%
  \let\thanks\@gobble
  \let\footnote\@gobble
  \def\cs#1{\textbackslash #1}%
  \let\normalfont\@empty
  \let\scshape\@empty
  \def\and{and }%
}
%    \end{macrocode}
%
% \subsubsection{Patch for bug latex/4096}
%
%    \begin{macrocode}
\begingroup\expandafter\expandafter\expandafter\endgroup
\expandafter\ifx\csname mod@math@codes\endcsname\relax
\else
  \def\HD@ModFix#1{%
    \mathcode`#1=`#1\relax
  }%
  \g@addto@macro\mod@math@codes{%
    \HD@ModFix\-%
    \HD@ModFix\+%
    \HD@ModFix\:%
    \HD@ModFix\=%
    \HD@ModFix\*%
  }%
\fi
%    \end{macrocode}
%
% \subsection{Bookmarks}
%
%    \begin{macro}{\HD@sectionpatch}
%    \begin{macrocode}
\def\HD@sectionpatch{%
  \ifHD@numbered
    \HD@@sectionpatch{section}%
    \HD@@sectionpatch{chapter}%
  \else
    \let\Hy@writebookmark\HD@disable@writebookmark
  \fi
}
%    \end{macrocode}
%    \end{macro}
%    \begin{macro}{\HD@@sectionpatch}
%    \begin{macrocode}
\def\HD@@sectionpatch#1{%
  \expandafter\let\csname HDorg@@#1\expandafter\endcsname
    \csname #1\endcsname
  \@namedef{#1}{%
    \@ifstar{\@nameuse{HDorg@@#1}}{\@nameuse{HDorg@@#1}}%
  }%
}
%    \end{macrocode}
%    \end{macro}
%    \begin{macro}{\HD@disable@writebookmark}
%    \begin{macrocode}
\def\HD@disable@writebookmark#1#2#3#4#5{}
%    \end{macrocode}
%    \end{macro}
%
%    \begin{macro}{\HD@guesstoclevel}
%    A wild guessing of the toclevel. There is the hope, that
%    the macro |#1| starts with a sectioning command.
%    \begin{macrocode}
\def\HD@guesstoclevel#1{%
  \expandafter\HD@@guesstoclevel#1\section\HD@end
}
%    \end{macrocode}
%    \end{macro}
%    \begin{macro}{\HD@@guesstoclevel}
%    \begin{macrocode}
\def\HD@@guesstoclevel#1#2\HD@end{%
  \@ifundefined{toclevel@\expandafter\@gobble\string#1\@empty}{%
    1%
  }{%
    \@nameuse{toclevel@\expandafter\@gobble\string#1\@empty}%
  }%
}
%    \end{macrocode}
%    \end{macro}
%
% \subsection{\cs{tableofcontents}}
%
%    \begin{macro}{\HDorg@tableofcontents}
%    \begin{macrocode}
\let\HDorg@tableofcontents\tableofcontents
%    \end{macrocode}
%    \end{macro}
%    \begin{macro}{\tableofcontents}
%    As first section the table of contents shouldn't
%    go into the toc itself. Also a number looks too funny then.
%    Just a bookmark is enough in this case.
%    \begin{macrocode}
\def\tableofcontents{%
  \begingroup
    \@ifundefined{c@section}{%
      \def\c@section{0}%
    }{}%
  \expandafter\endgroup
  \ifcase\c@section
    \stepcounter{HD@unique}%
    \pdfbookmark[\HD@guesstoclevel{\HDorg@tableofcontents}]%
        {\contentsname}{toc\theHD@unique}%
    \HDorg@tableofcontents
  \else
    \ifHD@numbered
    \else
      \stepcounter{HD@unique}%
      \pdfbookmark[\HD@guesstoclevel{\HDorg@tableofcontents}]%
          {\contentsname}{toc\theHD@unique}%
    \fi
    \begingroup
      \HD@sectionpatch
      \HDorg@tableofcontents
    \endgroup
  \fi
}
%    \end{macrocode}
%    \end{macro}
%
% \subsection{\cs{listoffigures}}
%
%    \begin{macro}{\HDorg@listoffigures}
%    \begin{macrocode}
\let\HDorg@listoffigures\listoffigures
%    \end{macrocode}
%    \end{macro}
%    \begin{macro}{\listoffigures}
%    \begin{macrocode}
\def\listoffigures{%
  \ifHD@numbered
  \else
    \stepcounter{HD@unique}%
    \pdfbookmark[\HD@guesstoclevel{\HDorg@listoffigures}]%
        {\listfigurename}{toc\theHD@unique}%
  \fi
  \begingroup
    \HD@sectionpatch
    \HDorg@listoffigures
  \endgroup
}
%    \end{macrocode}
%    \end{macro}
%
% \subsection{\cs{listoftables}}
%
%    \begin{macro}{\HDorg@listoftables}
%    \begin{macrocode}
\let\HDorg@listoftables\listoftables
%    \end{macrocode}
%    \end{macro}
%    \begin{macro}{\listoftables}
%    \begin{macrocode}
\def\listoftables{%
  \ifHD@numbered
  \else
    \stepcounter{HD@unique}%
    \pdfbookmark[\HD@guesstoclevel{\HDorg@listoftables}]%
        {\listtablename}{toc\theHD@unique}%
  \fi
  \begingroup
    \HD@sectionpatch
    \HDorg@listoftables
  \endgroup
}
%    \end{macrocode}
%    \end{macro}
%
% \subsection{\cs{thebibliography}}
%
%    \begin{macro}{\HDorg@thebibliography}
%    \begin{macrocode}
\let\HDorg@thebibliography\thebibliography
%    \end{macrocode}
%    \end{macro}
%    \begin{macro}{\thebibliography}
%    \begin{macrocode}
\def\thebibliography{%
  \ifHD@numbered
  \else
    \stepcounter{HD@unique}%
    \pdfbookmark[\HD@guesstoclevel{\HDorg@thebibliography}]%
      {\@ifundefined{chapter}{\refname}{\bibname}}{bib\theHD@unique}%
  \fi
  \HD@sectionpatch
  \HDorg@thebibliography
}
%    \end{macrocode}
%    \end{macro}
%
% \subsection{\cs{theindex}}
%
%    \begin{macro}{\HDorg@theindex}
%    \begin{macrocode}
\let\HDorg@theindex\theindex
%    \end{macrocode}
%    \end{macro}
%    \begin{macro}{\theindex}
%    \begin{macrocode}
\def\theindex{%
  \let\HDorg@index@prologue\index@prologue
  \let\HDorg@writebookmark\Hy@writebookmark
  \HD@sectionpatch
  \g@addto@macro\IndexParms{%
    \ifHD@numbered
    \else
      \let\Hy@writebookmark\HDorg@writebookmark
    \fi
    \let\bfseries\HD@bfseries
  }%
  \edef\HD@toclevel@index{\HD@guesstoclevel{\index@prologue}}%
  \begingroup
    \count@=\HD@toclevel@index\relax
    \advance\count@ by 1 %
    \xdef\HD@toclevel@subindex{\the\count@}%
  \endgroup
  \def\index@prologue{%
    \ifHD@numbered
    \else
      \begingroup
        \stepcounter{HD@unique}%
        \let\Hy@writebookmark\HDorg@writebookmark
        \pdfbookmark[\HD@toclevel@index]%
            {\indexname}{index\theHD@unique}%
      \endgroup
    \fi
    \HDorg@index@prologue
  }%
  \HDorg@theindex
}
%    \end{macrocode}
%    \end{macro}
%    \begin{macro}{\HDorg@bfseries}
%    \begin{macrocode}
\let\HDorg@bfseries\bfseries
%    \end{macrocode}
%    \end{macro}
%    \begin{macro}{\HD@bfseries}
%    \begin{macrocode}
\def\HD@bfseries{\futurelet\HD@hfil\HD@@bfseries}%
%    \end{macrocode}
%    \end{macro}
%    \begin{macro}{\HD@@bfseries}
%    \begin{macrocode}
\def\HD@@bfseries{%
  \HDorg@bfseries
  \ifx\HD@hfil\hfil
    \expandafter\HD@@@bfseries
  \fi
}
%    \end{macrocode}
%    \end{macro}
%    \begin{macro}{\HD@@@bfseries}
%    \begin{macrocode}
\def\HD@@@bfseries\hfil#1\hfil{%
  \ifx\\#1\\%
  \else
    \raisebox{\baselineskip}[0pt]{%
      \kern-\HD@margin\relax
      \pdfbookmark[\HD@toclevel@subindex]{#1}{HD.#1}%
      \kern\HD@margin\relax
    }%
  \fi
  \hfil#1\hfil
}%
%    \end{macrocode}
%    \end{macro}
%    \begin{macro}{\HD@margin}
%    \begin{macrocode}
\def\HD@margin{1mm}
%    \end{macrocode}
%    \end{macro}
%
% \subsection{\cs{theglossary}}
%
%    \begin{macro}{\HDorg@theglossary}
%    \begin{macrocode}
\let\HDorg@theglossary\theglossary
%    \end{macrocode}
%    \end{macro}
%    \begin{macro}{\theglossary}
%    \begin{macrocode}
\def\theglossary{%
  \let\HDorg@glossary@prologue\glossary@prologue
  \let\HDorg@writebookmark\Hy@writebookmark
  \HD@sectionpatch
  \def\glossary@prologue{%
    \ifHD@numbered
    \else
      \begingroup
        \let\Hy@writebookmark\HDorg@writebookmark
        \stepcounter{HD@unique}%
        \pdfbookmark[\HD@guesstoclevel{\HDorg@glossary@prologue}]%
            {\changehistoryname}{glossary\theHD@unique}%
      \endgroup
    \fi
    \HDorg@glossary@prologue
  }%
  \HDorg@theglossary
}
%    \end{macrocode}
%    \end{macro}
%    \begin{macro}{\glossaryname}
%    \begin{macrocode}
\providecommand*{\glossaryname}{Change History}
%    \end{macrocode}
%    \end{macro}
%
%    \begin{macro}{\glossaryname}
%    \begin{macrocode}
\providecommand*{\changehistoryname}{\glossaryname}
%    \end{macrocode}
%    \end{macro}
%
% \subsection{Index}
%
%    \begin{macrocode}
\newcounter{HD@hypercount}
%    \end{macrocode}
%    \begin{macro}{\HD@helplength}
%    \begin{macrocode}
\newlength{\HD@helplength}
%    \end{macrocode}
%    \end{macro}
%    \begin{macro}{\ifHD@savedest}
%    \begin{macrocode}
\newif\ifHD@savedest
\HD@savedesttrue
%    \end{macrocode}
%    \end{macro}
%    \begin{macrocode}
\AtBeginDocument{%
  \HD@savedest@init
}
%    \end{macrocode}
%    \begin{macrocode}
\RequirePackage{rerunfilecheck}[2009/12/10]
%    \end{macrocode}
%    \begin{macro}{\HD@savedest@init}
%    \begin{macrocode}
\begingroup
  \catcode`\^^M=\active
\@firstofone{\endgroup
  \def\HD@savedest@init{%
    \ifHD@savedest
      \let\HDorg@tilde~%
      \let~\HD@savedest@entry
      \edef\HDorg@catcodeM{\the\catcode`\^^M}%
      \catcode`\^^M=\active
      \def^^M{}%
      \InputIfFileExists{\jobname.hd}{}{}%
      \let~\HDorg@tilde
      \catcode`\^^M=\HDorg@catcodeM\relax
      \newwrite\HD@savedest@out
      \RerunFileCheck{\jobname.hd}{%
        \immediate\closeout\HD@savedest@out
      }{}%
      \immediate\openout\HD@savedest@out=\jobname.hd\relax
    \else
      \let\HD@savedest@add\@gobble
    \fi
  }%
%    \end{macrocode}
%    \end{macro}
%    \begin{macro}{\HD@savedest@entry}
%    \begin{macrocode}
  \def\HD@savedest@entry#1^^M{%
    \@namedef{HD.#1}{}%
  }%
}
%    \end{macrocode}
%    \end{macro}
%    \begin{macro}{\HD@savedest@add}
%    \begin{macrocode}
\def\HD@savedest@add#1{%
  \immediate\write\HD@savedest@out{\string~#1}%
}
%    \end{macrocode}
%    \end{macro}
%    \begin{macro}{\ifHD@targetraise}
%    \begin{macrocode}
\newif\ifHD@targetraise
\HD@targetraisetrue
%    \end{macrocode}
%    \end{macro}
%    \begin{macro}{\HD@target}
%    \begin{macrocode}
\def\HD@target{%
  \ifHD@targetraise
    \ifhmode
      \settoheight{\HD@helplength}{[H]}%
      \addtolength{\HD@helplength}{0.4ex}%
      \raisebox{\HD@helplength}[0pt][0pt]{%
        \HD@maketarget
      }%
    \else
      \HD@maketarget
    \fi
  \else
    \HD@maketarget
  \fi
}
%    \end{macrocode}
%    \end{macro}
%    \begin{macro}{\HD@maketarget}
%    \begin{macrocode}
\def\HD@maketarget{%
  \stepcounter{HD@hypercount}%
  \def\HD@next{%
    \llap{\hypertarget{HD.\the\c@HD@hypercount}{}\,}%
  }%
  \ifHD@savedest
    \expandafter\ifx\csname HD.\the\c@HD@hypercount\endcsname\@empty
    \else
      \let\HD@next\relax
    \fi
  \fi
  \HD@next
}
%    \end{macrocode}
%    \end{macro}
%
%    \begin{macro}{\CodelineIndex}
%    \begin{macrocode}
\g@addto@macro\CodelineIndex{%
  \let\special@index\HD@codeline@wrindex
}
%    \end{macrocode}
%    \end{macro}
%    \begin{macro}{\PageIndex}
%    \begin{macrocode}
\g@addto@macro\PageIndex{%
  \let\special@index\HD@page@wrindex
}
%    \end{macrocode}
%    \end{macro}
%
%    \begin{macro}{\HD@codeline@wrindex}
%    \begin{macrocode}
\def\HD@codeline@wrindex#1{%
  \begingroup
    \let\HDorg@encapchar\encapchar
    \def\encapchar##1\encapchar##2\@nil{%
      \HDorg@encapchar
      hdclindex{\the\c@HD@hypercount}{##1}%
    }%
    \codeline@wrindex{%
      #1\encapchar\encapchar\@nil
    }%
  \endgroup
}
%    \end{macrocode}
%    \end{macro}
%    \begin{macro}{\HD@page@wrindex}
%    \begin{macrocode}
\def\HD@page@wrindex#1{%
  \begingroup
    \let\HDorg@encapchar\encapchar
    \def\encapchar##1\encapchar##2\@nil{%
      \HDorg@encapchar
      hdpindex{##1}%
    }%
    \HDorg@index{#1\encapchar\encapchar\@nil}%
  \endgroup
}
%    \end{macrocode}
%    \end{macro}
%
%    \begin{macro}{\hdclindex}
%    \begin{macrocode}
\def\hdclindex#1#2#3{%
  \csname\ifx\\#2\\relax\else#2\fi\endcsname{%
    \hyperlink{HD.#1}{#3}%
    \HD@savedest@add{#1}%
  }%
}
%    \end{macrocode}
%    \end{macro}
%    \begin{macro}{\hdpindex}
%    \begin{macrocode}
\def\hdpindex#1#2{%
  \csname\ifx\\#1\\relax\else#1\fi\endcsname{%
    \hyperpage{#2}%
  }%
}
%    \end{macrocode}
%    \end{macro}
%
% \subsection{\cs{PrintDescribeMacro}}
%
%    \begin{macro}{\HDorg@PrintDescribeMacro}
%    \begin{macrocode}
\let\HDorg@PrintDescribeMacro\PrintDescribeMacro
%    \end{macrocode}
%    \end{macro}
%    \begin{macro}{\PrintDescribeMacro}
%    \begin{macrocode}
\renewcommand\PrintDescribeMacro[1]{%
  \begingroup
    \let\HDorg@MacroFont\MacroFont
    \def\MacroFont{%
      \HD@target
      \HDorg@MacroFont
    }%
    \HDorg@PrintDescribeMacro{#1}%
  \endgroup
}
%    \end{macrocode}
%    \end{macro}
%
% \subsection{\cs{PrintMacroName}}
%
%    \begin{macro}{\HDorg@PrintMacroName}
%    \begin{macrocode}
\let\HDorg@PrintMacroName\PrintMacroName
%    \end{macrocode}
%    \end{macro}
%    \begin{macro}{\PrintMacroName}
%    \begin{macrocode}
\renewcommand\PrintMacroName[1]{%
  \begingroup
    \let\HDorg@MacroFont\MacroFont
    \def\MacroFont{%
      \HD@target
      \HDorg@MacroFont
    }%
    \HDorg@PrintMacroName{#1}%
  \endgroup
}
%    \end{macrocode}
%    \end{macro}
%
% \subsection{\cs{theCodelineNo}}
%
%    \begin{macro}{\HDorg@theCodelineNo}
%    \begin{macrocode}
\let\HDorg@theCodelineNo\theCodelineNo
%    \end{macrocode}
%    \end{macro}
%    \begin{macro}{\theCodelineNo}
%    \begin{macrocode}
\renewcommand*\theCodelineNo{%
  \settowidth{\HD@helplength}{\HDorg@theCodelineNo\ }%
  \ifdim\HD@helplength<\MacroIndent
    \addtolength{\HD@helplength}{-\MacroIndent}%
    \llap{%
      \begingroup\HD@target\endgroup
      \hspace*{-\HD@helplength}%
    }%
  \else
    \HD@target
  \fi
  \HDorg@theCodelineNo
}
%    \end{macrocode}
%    \end{macro}
%
%    \begin{macro}{\HDorg@index}
%    \begin{macrocode}
\AtBeginDocument{%
  \let\HDorg@index\index
}
%    \end{macrocode}
%    \end{macro}
%
% \subsection{\cs{SpecialUsageIndex}}
%
%    \begin{macro}{\HDorg@SpecialUsageIndex}
%    \begin{macrocode}
\let\HDorg@SpecialUsageIndex\SpecialUsageIndex
\renewcommand*\SpecialUsageIndex[1]{%
  \@bsphack
  \begingroup
    \HD@target
    \let\index\HDorg@index
    \let\HDorg@encapchar\encapchar
    \edef\encapchar usage{%
       \HDorg@encapchar hdclindex{\the\c@HD@hypercount}{usage}%
    }%
    \HDorg@SpecialUsageIndex{#1}%
  \endgroup
  \@esphack
}
%    \end{macrocode}
%    \end{macro}
%
% \subsection{\cs{autoindex}}
%
%    \begin{macrocode}
\begingroup\expandafter\expandafter\expandafter\endgroup
\expandafter\ifx\csname autoindex\endcsname\relax
\else
%    \end{macrocode}
%    \begin{macro}{\HDorg@autoindex}
%    \begin{macrocode}
  \let\HDorg@autoindex\autoindex
%    \end{macrocode}
%    \end{macro}
%    \begin{macro}{\autoindex}
%    \begin{macrocode}
  \renewcommand*\autoindex[1]{%
    \@bsphack
    \begingroup
      \HD@target
      \let\index\HDorg@index
      \let\HDorg@encapchar\encapchar
      \edef\encapchar usage{%
         \HDorg@encapchar hdclindex{\the\c@HD@hypercount}{usage}%
      }%
      \HDorg@autoindex{#1}%
    \endgroup
    \@esphack
  }%
\fi
%    \end{macrocode}
%    \end{macro}
%
% \subsection{\cs{SpecialEnvIndex}}
%
%    \begin{macro}{\HDorg@SpecialEnvIndex}
%    \begin{macrocode}
\let\HDorg@SpecialEnvIndex\SpecialEnvIndex
%    \end{macrocode}
%    \end{macro}
%    \begin{macro}{\SpecialEnvIndex}
%    \begin{macrocode}
\renewcommand*\SpecialEnvIndex[1]{%
  \@bsphack
  \begingroup
    \HD@target
    \let\HDorg@encapchar\encapchar
    \edef\encapchar usage{%
      \HDorg@encapchar hdclindex{\the\c@HD@hypercount}{usage}%
    }%
    \HDorg@SpecialEnvIndex{#1}%
  \endgroup
  \@esphack
}
%    \end{macrocode}
%    \end{macro}
%
% \subsection{\cs{SortIndex}}
%
%    \begin{macro}{\HDorg@SortIndex}
%    \begin{macrocode}
\let\HDorg@SortIndex\SortIndex
%    \end{macrocode}
%    \end{macro}
%    \begin{macro}{\SortIndex}
%    \begin{macrocode}
\renewcommand*{\SortIndex}[2]{%
  \@bsphack
  \begingroup
    \let\index\HD@page@wrindex
    \HDorg@SortIndex{#1}{#2}%
  \endgroup
  \@esphack
}
%    \end{macrocode}
%    \end{macro}
%
% \subsection{\cs{@wrglossary}}
%
%    \begin{macro}{\HDorg@wrglossary}
%    \begin{macrocode}
\let\HDorg@wrglossary\@wrglossary
%    \end{macrocode}
%    \end{macro}
%    \begin{macro}{\@wrglossary}
%    \begin{macrocode}
\def\@wrglossary#1{%
  \let\HDorg@encapchar\encapchar
  \def\encapchar##1\encapchar##2\@nil{%
    \HDorg@encapchar
    hdpindex{##1}%
  }%
  \HDorg@wrglossary{#1\encapchar\encapchar\@nil}%
}
%    \end{macrocode}
%    \end{macro}
%
% \subsection{\cs{MacroIndent}}
%
%    \cs{MacroIndent} automatically calculated.
%    \begin{macrocode}
\AfterLastShipout{%
  \if@filesw
    \immediate\write\@auxout{%
      \string\providecommand\string\HD@SetMacroIndent[1]{}%
    }%
    \immediate\write\@auxout{%
      \string\HD@SetMacroIndent{\number\value{CodelineNo}}%
    }%
  \fi
}
%    \end{macrocode}
%    \begin{macro}{\HD@SetMacroIndent}
%    \begin{macrocode}
\def\HD@SetMacroIndent#1{%
  \begingroup
    \value{CodelineNo}=#1\relax
    \settowidth{\MacroIndent}{\HDorg@theCodelineNo\ }%
    \global\MacroIndent\MacroIndent
  \endgroup
}
%    \end{macrocode}
%    \end{macro}
%
%    \begin{macrocode}
%</package>
%    \end{macrocode}
%
% \section{Installation}
%
% \subsection{Download}
%
% \paragraph{Package.} This package is available on
% CTAN\footnote{\url{http://ctan.org/pkg/hypdoc}}:
% \begin{description}
% \item[\CTAN{macros/latex/contrib/oberdiek/hypdoc.dtx}] The source file.
% \item[\CTAN{macros/latex/contrib/oberdiek/hypdoc.pdf}] Documentation.
% \end{description}
%
%
% \paragraph{Bundle.} All the packages of the bundle `oberdiek'
% are also available in a TDS compliant ZIP archive. There
% the packages are already unpacked and the documentation files
% are generated. The files and directories obey the TDS standard.
% \begin{description}
% \item[\CTAN{install/macros/latex/contrib/oberdiek.tds.zip}]
% \end{description}
% \emph{TDS} refers to the standard ``A Directory Structure
% for \TeX\ Files'' (\CTAN{tds/tds.pdf}). Directories
% with \xfile{texmf} in their name are usually organized this way.
%
% \subsection{Bundle installation}
%
% \paragraph{Unpacking.} Unpack the \xfile{oberdiek.tds.zip} in the
% TDS tree (also known as \xfile{texmf} tree) of your choice.
% Example (linux):
% \begin{quote}
%   |unzip oberdiek.tds.zip -d ~/texmf|
% \end{quote}
%
% \paragraph{Script installation.}
% Check the directory \xfile{TDS:scripts/oberdiek/} for
% scripts that need further installation steps.
% Package \xpackage{attachfile2} comes with the Perl script
% \xfile{pdfatfi.pl} that should be installed in such a way
% that it can be called as \texttt{pdfatfi}.
% Example (linux):
% \begin{quote}
%   |chmod +x scripts/oberdiek/pdfatfi.pl|\\
%   |cp scripts/oberdiek/pdfatfi.pl /usr/local/bin/|
% \end{quote}
%
% \subsection{Package installation}
%
% \paragraph{Unpacking.} The \xfile{.dtx} file is a self-extracting
% \docstrip\ archive. The files are extracted by running the
% \xfile{.dtx} through \plainTeX:
% \begin{quote}
%   \verb|tex hypdoc.dtx|
% \end{quote}
%
% \paragraph{TDS.} Now the different files must be moved into
% the different directories in your installation TDS tree
% (also known as \xfile{texmf} tree):
% \begin{quote}
% \def\t{^^A
% \begin{tabular}{@{}>{\ttfamily}l@{ $\rightarrow$ }>{\ttfamily}l@{}}
%   hypdoc.sty & tex/latex/oberdiek/hypdoc.sty\\
%   hypdoc.pdf & doc/latex/oberdiek/hypdoc.pdf\\
%   hypdoc.dtx & source/latex/oberdiek/hypdoc.dtx\\
% \end{tabular}^^A
% }^^A
% \sbox0{\t}^^A
% \ifdim\wd0>\linewidth
%   \begingroup
%     \advance\linewidth by\leftmargin
%     \advance\linewidth by\rightmargin
%   \edef\x{\endgroup
%     \def\noexpand\lw{\the\linewidth}^^A
%   }\x
%   \def\lwbox{^^A
%     \leavevmode
%     \hbox to \linewidth{^^A
%       \kern-\leftmargin\relax
%       \hss
%       \usebox0
%       \hss
%       \kern-\rightmargin\relax
%     }^^A
%   }^^A
%   \ifdim\wd0>\lw
%     \sbox0{\small\t}^^A
%     \ifdim\wd0>\linewidth
%       \ifdim\wd0>\lw
%         \sbox0{\footnotesize\t}^^A
%         \ifdim\wd0>\linewidth
%           \ifdim\wd0>\lw
%             \sbox0{\scriptsize\t}^^A
%             \ifdim\wd0>\linewidth
%               \ifdim\wd0>\lw
%                 \sbox0{\tiny\t}^^A
%                 \ifdim\wd0>\linewidth
%                   \lwbox
%                 \else
%                   \usebox0
%                 \fi
%               \else
%                 \lwbox
%               \fi
%             \else
%               \usebox0
%             \fi
%           \else
%             \lwbox
%           \fi
%         \else
%           \usebox0
%         \fi
%       \else
%         \lwbox
%       \fi
%     \else
%       \usebox0
%     \fi
%   \else
%     \lwbox
%   \fi
% \else
%   \usebox0
% \fi
% \end{quote}
% If you have a \xfile{docstrip.cfg} that configures and enables \docstrip's
% TDS installing feature, then some files can already be in the right
% place, see the documentation of \docstrip.
%
% \subsection{Refresh file name databases}
%
% If your \TeX~distribution
% (\teTeX, \mikTeX, \dots) relies on file name databases, you must refresh
% these. For example, \teTeX\ users run \verb|texhash| or
% \verb|mktexlsr|.
%
% \subsection{Some details for the interested}
%
% \paragraph{Attached source.}
%
% The PDF documentation on CTAN also includes the
% \xfile{.dtx} source file. It can be extracted by
% AcrobatReader 6 or higher. Another option is \textsf{pdftk},
% e.g. unpack the file into the current directory:
% \begin{quote}
%   \verb|pdftk hypdoc.pdf unpack_files output .|
% \end{quote}
%
% \paragraph{Unpacking with \LaTeX.}
% The \xfile{.dtx} chooses its action depending on the format:
% \begin{description}
% \item[\plainTeX:] Run \docstrip\ and extract the files.
% \item[\LaTeX:] Generate the documentation.
% \end{description}
% If you insist on using \LaTeX\ for \docstrip\ (really,
% \docstrip\ does not need \LaTeX), then inform the autodetect routine
% about your intention:
% \begin{quote}
%   \verb|latex \let\install=y% \iffalse meta-comment
%
% File: hypdoc.dtx
% Version: 2018/09/07 v1.14
% Info: Hyper extensions for doc.sty
%
% Copyright (C) 2002, 2006, 2007, 2009-2011 by
%    Heiko Oberdiek <heiko.oberdiek at googlemail.com>
%    2016
%    https://github.com/ho-tex/oberdiek/issues
%
% This work may be distributed and/or modified under the
% conditions of the LaTeX Project Public License, either
% version 1.3c of this license or (at your option) any later
% version. This version of this license is in
%    http://www.latex-project.org/lppl/lppl-1-3c.txt
% and the latest version of this license is in
%    http://www.latex-project.org/lppl.txt
% and version 1.3 or later is part of all distributions of
% LaTeX version 2005/12/01 or later.
%
% This work has the LPPL maintenance status "maintained".
%
% This Current Maintainer of this work is Heiko Oberdiek.
%
% This work consists of the main source file hypdoc.dtx
% and the derived files
%    hypdoc.sty, hypdoc.pdf, hypdoc.ins, hypdoc.drv.
%
% Distribution:
%    CTAN:macros/latex/contrib/oberdiek/hypdoc.dtx
%    CTAN:macros/latex/contrib/oberdiek/hypdoc.pdf
%
% Unpacking:
%    (a) If hypdoc.ins is present:
%           tex hypdoc.ins
%    (b) Without hypdoc.ins:
%           tex hypdoc.dtx
%    (c) If you insist on using LaTeX
%           latex \let\install=y\input{hypdoc.dtx}
%        (quote the arguments according to the demands of your shell)
%
% Documentation:
%    (a) If hypdoc.drv is present:
%           latex hypdoc.drv
%    (b) Without hypdoc.drv:
%           latex hypdoc.dtx; ...
%    The class ltxdoc loads the configuration file ltxdoc.cfg
%    if available. Here you can specify further options, e.g.
%    use A4 as paper format:
%       \PassOptionsToClass{a4paper}{article}
%
%    Programm calls to get the documentation (example):
%       pdflatex hypdoc.dtx
%       makeindex -s gind.ist hypdoc.idx
%       pdflatex hypdoc.dtx
%       makeindex -s gind.ist hypdoc.idx
%       pdflatex hypdoc.dtx
%
% Installation:
%    TDS:tex/latex/oberdiek/hypdoc.sty
%    TDS:doc/latex/oberdiek/hypdoc.pdf
%    TDS:source/latex/oberdiek/hypdoc.dtx
%
%<*ignore>
\begingroup
  \catcode123=1 %
  \catcode125=2 %
  \def\x{LaTeX2e}%
\expandafter\endgroup
\ifcase 0\ifx\install y1\fi\expandafter
         \ifx\csname processbatchFile\endcsname\relax\else1\fi
         \ifx\fmtname\x\else 1\fi\relax
\else\csname fi\endcsname
%</ignore>
%<*install>
\input docstrip.tex
\Msg{************************************************************************}
\Msg{* Installation}
\Msg{* Package: hypdoc 2018/09/07 v1.14 Hyper extensions for doc.sty (HO)}
\Msg{************************************************************************}

\keepsilent
\askforoverwritefalse

\let\MetaPrefix\relax
\preamble

This is a generated file.

Project: hypdoc
Version: 2018/09/07 v1.14

Copyright (C) 2002, 2006, 2007, 2009-2011 by
   Heiko Oberdiek <heiko.oberdiek at googlemail.com>

This work may be distributed and/or modified under the
conditions of the LaTeX Project Public License, either
version 1.3c of this license or (at your option) any later
version. This version of this license is in
   http://www.latex-project.org/lppl/lppl-1-3c.txt
and the latest version of this license is in
   http://www.latex-project.org/lppl.txt
and version 1.3 or later is part of all distributions of
LaTeX version 2005/12/01 or later.

This work has the LPPL maintenance status "maintained".

This Current Maintainer of this work is Heiko Oberdiek.

This work consists of the main source file hypdoc.dtx
and the derived files
   hypdoc.sty, hypdoc.pdf, hypdoc.ins, hypdoc.drv.

\endpreamble
\let\MetaPrefix\DoubleperCent

\generate{%
  \file{hypdoc.ins}{\from{hypdoc.dtx}{install}}%
  \file{hypdoc.drv}{\from{hypdoc.dtx}{driver}}%
  \usedir{tex/latex/oberdiek}%
  \file{hypdoc.sty}{\from{hypdoc.dtx}{package}}%
  \nopreamble
  \nopostamble
%  \usedir{source/latex/oberdiek/catalogue}%
%  \file{hypdoc.xml}{\from{hypdoc.dtx}{catalogue}}%
}

\catcode32=13\relax% active space
\let =\space%
\Msg{************************************************************************}
\Msg{*}
\Msg{* To finish the installation you have to move the following}
\Msg{* file into a directory searched by TeX:}
\Msg{*}
\Msg{*     hypdoc.sty}
\Msg{*}
\Msg{* To produce the documentation run the file `hypdoc.drv'}
\Msg{* through LaTeX.}
\Msg{*}
\Msg{* Happy TeXing!}
\Msg{*}
\Msg{************************************************************************}

\endbatchfile
%</install>
%<*ignore>
\fi
%</ignore>
%<*driver>
\NeedsTeXFormat{LaTeX2e}
\ProvidesFile{hypdoc.drv}%
  [2018/09/07 v1.14 Hyper extensions for doc.sty (HO)]%
\documentclass{ltxdoc}
\usepackage{holtxdoc}[2011/11/22]
\begin{document}
  \DocInput{hypdoc.dtx}%
\end{document}
%</driver>
% \fi
%
%
% \CharacterTable
%  {Upper-case    \A\B\C\D\E\F\G\H\I\J\K\L\M\N\O\P\Q\R\S\T\U\V\W\X\Y\Z
%   Lower-case    \a\b\c\d\e\f\g\h\i\j\k\l\m\n\o\p\q\r\s\t\u\v\w\x\y\z
%   Digits        \0\1\2\3\4\5\6\7\8\9
%   Exclamation   \!     Double quote  \"     Hash (number) \#
%   Dollar        \$     Percent       \%     Ampersand     \&
%   Acute accent  \'     Left paren    \(     Right paren   \)
%   Asterisk      \*     Plus          \+     Comma         \,
%   Minus         \-     Point         \.     Solidus       \/
%   Colon         \:     Semicolon     \;     Less than     \<
%   Equals        \=     Greater than  \>     Question mark \?
%   Commercial at \@     Left bracket  \[     Backslash     \\
%   Right bracket \]     Circumflex    \^     Underscore    \_
%   Grave accent  \`     Left brace    \{     Vertical bar  \|
%   Right brace   \}     Tilde         \~}
%
% \GetFileInfo{hypdoc.drv}
%
% \title{The \xpackage{hypdoc} package}
% \date{2018/09/07 v1.14}
% \author{Heiko Oberdiek\thanks
% {Please report any issues at https://github.com/ho-tex/oberdiek/issues}\\
% \xemail{heiko.oberdiek at googlemail.com}}
%
% \maketitle
%
% \begin{abstract}
% This package adds hyper features to the package
% \xpackage{doc} that is used in the documentation
% system of \LaTeXe. Bookmarks are added and references
% are linked as far as possible.
% \end{abstract}
%
% \tableofcontents
%
% \section{Features}
%
% The package adds the following features:
% \begin{description}
% \item[Bookmarks:]
%   For supporting bookmarks package \verb|hyperref| is loaded.
%   Some \TeX\ code inside section titles can cause problems
%   during bookmark processing; these can be warnings, ugly
%   bookmark titles or even errors. With the help of
%   \verb|\pdfstringdefDisableCommands| or \verb|\texorpdfstring|
%   you can manage these things, see \verb|hyperref|'s package
%   documentation.
% \item[Additional bookmarks:]
%   Unnumbered bookmark entries are generated for:
%   \begin{itemize}
%   \item Table of contents
%   \item List of figures
%   \item List of tables
%   \item Index
%   \item Glossary
%   \end{itemize}
%   Option \texttt{numbered} controls, whether these section or
%   chapters should be numbered. The exception is the table
%   of contents if this it would be the first numbered
%   section. Then just a bookmark is added.
% \item[Metadata:]
%   The loaded package \verb|hyperref|, enforced by
%   option \verb|pdfusetitle|, tries to detect \verb|\title|
%   and \verb|\author| and sets the corresponding PDF
%   information fields. In case of problems see the previous
%   item about bookmarks.
%   These enries can be overwritten or new ones added by
%   \verb|\hypersetup| after the packages is loaded, see
%   documentation of package \verb|hyperref|.
% \item[References, links:] Because of package \verb|\hyperref|
%   |\ref| become links, also \verb|\url|, footnote referencing
%   are supported.
% \item[Index:]
%   The main part of this package deals with index links in order
%   to support the kind of index that package \verb|doc| provides.
% \item[Thumbnails:]
%   Package \verb|thumbpdf| may additionally be loaded for the case that the
%   PDF file also should contain thumbnails, see the documentation
%   of this package. But thumbnail embedding is not really
%   necessary, because nowadays AcrobatReader is able to
%   generate and view thumbnails on the fly.
% \end{description}
%
% \section{Usage}
%
% The purpose of this packages is to add PDF features to
% the documentation of LaTeX packages. Thus just load
% the package in the driver file that generates the
% documentation:
% \begin{quote}
%   \verb|\usepackage{hypdoc}|
% \end{quote}
% Package \verb|doc| is loaded by \verb|hypdoc|, thus you
% also can replace a \verb|\usepackage{doc}| with this line.
% At any case, the package must be loaded before macros of
% package \verb|doc| such as \verb|\CodelineIndex| or
% \verb|\PageIndex| are used.
%
% If the class \verb|ltxdoc| is used, you can also use
% the configuration file for this class to add
% package \verb|hypdoc|. Add the following line
% to the file \verb|ltxdoc.cfg| (or create a new one):
% \begin{quote}
%\begin{verbatim}
%\AtEndOfClass{\RequirePackage{hypdoc}}
%\end{verbatim}
% \end{quote}
%
% \subsection{Options}
%
% There is one option \texttt{numbered}. It controls,
% whether the starred versions of section or chapter
% inside \cs{tableofcontents}, \cs{listoffigures}, \dots
% should be replaced by the non-star versions.
%
% \subsection{Simple example}
% Example of a simple driver file \verb|foobar.drv|
% of a package \verb|foobar|. Often the driver file
% is packed in the file \verb|foobar.dtx| and can
% be extracted by docstrip or if an instruction file
% for docstrip is provided by:
% \begin{quote}
% \verb|tex foobar.ins|
% \end{quote}
% Note that although docstrip doesn't rely on \LaTeX,
% sometimes you need to run the \verb|.ins| file through
% \LaTeX. Oddly enough, often the only \LaTeX\ feature
% in use is \verb|\NeedsTeXFormat{LaTeX2e}| in the first line.
% \begin{quote}
%\begin{verbatim}
%\documentclass{ltxdoc}
%\usepackage{hypdoc}
%\RecordChanges
%\EnableCrossrefs
%\CodelineIndex
%\begin{document}
%\title{Title for package \texttt{foobar}}
%\author{Author of package \texttt{foobar}}
%\date{...}
%\maketitle
%\tableofcontents
%\DocInput{foobar.dtx}
%\PrintChanges
%\PrintIndex
%\end{document}
%\end{verbatim}
% \end{quote}
%
% Then compile it, for example with pdf\LaTeX.
% \begin{quote}
%\begin{verbatim}
%pdflatex foobar.drv
%makeindex -s gind.ist foobar.idx
%makeindex -s gglo.ist -o foobar.gls foobar.glo
%pdflatex foobar.drv
%\end{verbatim}
% \end{quote}
% You will need several cycles until the cross references
% and similar things are correct.
%
% \subsection{Example \texttt{source2e.tex}}
%
% The source code of the base of \LaTeX\ is available as
% a bunch of \verb|.dtx| files. \LaTeX\ provides the file
% \verb|source2e.tex| to merge them all together.
%
% Now you can either load package \verb|hypdoc| in a private copy
% of this file (please, respect the LPPL) or use the configuration
% file \verb|ltxdoc.cfg|. Example that also sets A4 paper size:
% \begin{quote}
%\begin{verbatim}
%\PassOptionsToClass{a4paper}{article}
%\AtEndOfClass{\RequirePackage{hypdoc}}
%\end{verbatim}
% \end{quote}
%
% Then \verb|source2e.tex| can be compiled by pdf\TeX, for example:
% \begin{quote}
%\begin{verbatim}
%pdflatex source2e
%makeindex -s gind.ist source2e.idx
%makeindex -s gglo.ist -o source2e.glo source2e.gls
%pdflatex source2e
%makeindex -s gind.ist source2e.idx
%makeindex -s gglo.ist -o source2e.glo source2e.gls
%pdflatex source2e
%\end{verbatim}
% \end{quote}
%
% \StopEventually{
% }
%
% \section{Implementation}
%
%    \begin{macrocode}
%<*package>
\NeedsTeXFormat{LaTeX2e}
\ProvidesPackage{hypdoc}%
  [2018/09/07 v1.14 Hyper extensions for doc.sty (HO)]%
%    \end{macrocode}
%    The package identification is done at the top of the |.dtx| file
%    in order to use only one identification string.
%
%    For unique command names this package uses |HD@| as prefix
%
% \subsection{Options}
%
%    \begin{macro}{\ifHD@numbered}
%    \begin{macrocode}
\newif\ifHD@numbered
%    \end{macrocode}
%    \end{macro}
%    \begin{macrocode}
\DeclareOption{numbered}{\HD@numberedtrue}
\ProcessOptions*\relax
%    \end{macrocode}
%
% \subsection{Package loading}
%
%    \begin{macrocode}
\RequirePackage{atveryend}[2010/03/24]
\RequirePackage{doc}
%    \end{macrocode}
%    \begin{macrocode}
\ifx\numexpr\@undefined
  \RequirePackage{calc}%
\fi
%    \end{macrocode}
%    \begin{macrocode}
\RequirePackage[%
  hyperindex=false,%
  pdfusetitle,%
  colorlinks,%
  pdfpagelabels%
]{hyperref}[2002/05/09]%
\hypersetup{%
  plainpages=false,%
  bookmarksopen,%
  bookmarksnumbered,%
  bookmarksopenlevel=0,%
  pdfstartview={FitBH \hypercalcbp{%
    \paperheight-\topmargin-1in-\headheight-\headsep
  }},%
}
%    \end{macrocode}
%
%    \begin{macrocode}
\newcounter{HD@unique}
%    \end{macrocode}
%    \begin{macro}{\theHD@unique}
%    \begin{macrocode}
\renewcommand{\theHD@unique}{%
  \number\value{HD@unique}%
}
%    \end{macrocode}
%    \end{macro}
%
%    \begin{macrocode}
\g@addto@macro\Hy@UseMaketitleInfos{%
  \@ifundefined{fileinfo}{}{%
    \ifx\@pdfsubject\@empty
      \hypersetup{pdfsubject={\fileinfo}}%
    \fi
  }%
}
%    \end{macrocode}
%
%    \begin{macrocode}
\pdfstringdefDisableCommands{%
  \let\thanks\@gobble
  \let\footnote\@gobble
  \def\cs#1{\textbackslash #1}%
  \let\normalfont\@empty
  \let\scshape\@empty
  \def\and{and }%
}
%    \end{macrocode}
%
% \subsubsection{Patch for bug latex/4096}
%
%    \begin{macrocode}
\begingroup\expandafter\expandafter\expandafter\endgroup
\expandafter\ifx\csname mod@math@codes\endcsname\relax
\else
  \def\HD@ModFix#1{%
    \mathcode`#1=`#1\relax
  }%
  \g@addto@macro\mod@math@codes{%
    \HD@ModFix\-%
    \HD@ModFix\+%
    \HD@ModFix\:%
    \HD@ModFix\=%
    \HD@ModFix\*%
  }%
\fi
%    \end{macrocode}
%
% \subsection{Bookmarks}
%
%    \begin{macro}{\HD@sectionpatch}
%    \begin{macrocode}
\def\HD@sectionpatch{%
  \ifHD@numbered
    \HD@@sectionpatch{section}%
    \HD@@sectionpatch{chapter}%
  \else
    \let\Hy@writebookmark\HD@disable@writebookmark
  \fi
}
%    \end{macrocode}
%    \end{macro}
%    \begin{macro}{\HD@@sectionpatch}
%    \begin{macrocode}
\def\HD@@sectionpatch#1{%
  \expandafter\let\csname HDorg@@#1\expandafter\endcsname
    \csname #1\endcsname
  \@namedef{#1}{%
    \@ifstar{\@nameuse{HDorg@@#1}}{\@nameuse{HDorg@@#1}}%
  }%
}
%    \end{macrocode}
%    \end{macro}
%    \begin{macro}{\HD@disable@writebookmark}
%    \begin{macrocode}
\def\HD@disable@writebookmark#1#2#3#4#5{}
%    \end{macrocode}
%    \end{macro}
%
%    \begin{macro}{\HD@guesstoclevel}
%    A wild guessing of the toclevel. There is the hope, that
%    the macro |#1| starts with a sectioning command.
%    \begin{macrocode}
\def\HD@guesstoclevel#1{%
  \expandafter\HD@@guesstoclevel#1\section\HD@end
}
%    \end{macrocode}
%    \end{macro}
%    \begin{macro}{\HD@@guesstoclevel}
%    \begin{macrocode}
\def\HD@@guesstoclevel#1#2\HD@end{%
  \@ifundefined{toclevel@\expandafter\@gobble\string#1\@empty}{%
    1%
  }{%
    \@nameuse{toclevel@\expandafter\@gobble\string#1\@empty}%
  }%
}
%    \end{macrocode}
%    \end{macro}
%
% \subsection{\cs{tableofcontents}}
%
%    \begin{macro}{\HDorg@tableofcontents}
%    \begin{macrocode}
\let\HDorg@tableofcontents\tableofcontents
%    \end{macrocode}
%    \end{macro}
%    \begin{macro}{\tableofcontents}
%    As first section the table of contents shouldn't
%    go into the toc itself. Also a number looks too funny then.
%    Just a bookmark is enough in this case.
%    \begin{macrocode}
\def\tableofcontents{%
  \begingroup
    \@ifundefined{c@section}{%
      \def\c@section{0}%
    }{}%
  \expandafter\endgroup
  \ifcase\c@section
    \stepcounter{HD@unique}%
    \pdfbookmark[\HD@guesstoclevel{\HDorg@tableofcontents}]%
        {\contentsname}{toc\theHD@unique}%
    \HDorg@tableofcontents
  \else
    \ifHD@numbered
    \else
      \stepcounter{HD@unique}%
      \pdfbookmark[\HD@guesstoclevel{\HDorg@tableofcontents}]%
          {\contentsname}{toc\theHD@unique}%
    \fi
    \begingroup
      \HD@sectionpatch
      \HDorg@tableofcontents
    \endgroup
  \fi
}
%    \end{macrocode}
%    \end{macro}
%
% \subsection{\cs{listoffigures}}
%
%    \begin{macro}{\HDorg@listoffigures}
%    \begin{macrocode}
\let\HDorg@listoffigures\listoffigures
%    \end{macrocode}
%    \end{macro}
%    \begin{macro}{\listoffigures}
%    \begin{macrocode}
\def\listoffigures{%
  \ifHD@numbered
  \else
    \stepcounter{HD@unique}%
    \pdfbookmark[\HD@guesstoclevel{\HDorg@listoffigures}]%
        {\listfigurename}{toc\theHD@unique}%
  \fi
  \begingroup
    \HD@sectionpatch
    \HDorg@listoffigures
  \endgroup
}
%    \end{macrocode}
%    \end{macro}
%
% \subsection{\cs{listoftables}}
%
%    \begin{macro}{\HDorg@listoftables}
%    \begin{macrocode}
\let\HDorg@listoftables\listoftables
%    \end{macrocode}
%    \end{macro}
%    \begin{macro}{\listoftables}
%    \begin{macrocode}
\def\listoftables{%
  \ifHD@numbered
  \else
    \stepcounter{HD@unique}%
    \pdfbookmark[\HD@guesstoclevel{\HDorg@listoftables}]%
        {\listtablename}{toc\theHD@unique}%
  \fi
  \begingroup
    \HD@sectionpatch
    \HDorg@listoftables
  \endgroup
}
%    \end{macrocode}
%    \end{macro}
%
% \subsection{\cs{thebibliography}}
%
%    \begin{macro}{\HDorg@thebibliography}
%    \begin{macrocode}
\let\HDorg@thebibliography\thebibliography
%    \end{macrocode}
%    \end{macro}
%    \begin{macro}{\thebibliography}
%    \begin{macrocode}
\def\thebibliography{%
  \ifHD@numbered
  \else
    \stepcounter{HD@unique}%
    \pdfbookmark[\HD@guesstoclevel{\HDorg@thebibliography}]%
      {\@ifundefined{chapter}{\refname}{\bibname}}{bib\theHD@unique}%
  \fi
  \HD@sectionpatch
  \HDorg@thebibliography
}
%    \end{macrocode}
%    \end{macro}
%
% \subsection{\cs{theindex}}
%
%    \begin{macro}{\HDorg@theindex}
%    \begin{macrocode}
\let\HDorg@theindex\theindex
%    \end{macrocode}
%    \end{macro}
%    \begin{macro}{\theindex}
%    \begin{macrocode}
\def\theindex{%
  \let\HDorg@index@prologue\index@prologue
  \let\HDorg@writebookmark\Hy@writebookmark
  \HD@sectionpatch
  \g@addto@macro\IndexParms{%
    \ifHD@numbered
    \else
      \let\Hy@writebookmark\HDorg@writebookmark
    \fi
    \let\bfseries\HD@bfseries
  }%
  \edef\HD@toclevel@index{\HD@guesstoclevel{\index@prologue}}%
  \begingroup
    \count@=\HD@toclevel@index\relax
    \advance\count@ by 1 %
    \xdef\HD@toclevel@subindex{\the\count@}%
  \endgroup
  \def\index@prologue{%
    \ifHD@numbered
    \else
      \begingroup
        \stepcounter{HD@unique}%
        \let\Hy@writebookmark\HDorg@writebookmark
        \pdfbookmark[\HD@toclevel@index]%
            {\indexname}{index\theHD@unique}%
      \endgroup
    \fi
    \HDorg@index@prologue
  }%
  \HDorg@theindex
}
%    \end{macrocode}
%    \end{macro}
%    \begin{macro}{\HDorg@bfseries}
%    \begin{macrocode}
\let\HDorg@bfseries\bfseries
%    \end{macrocode}
%    \end{macro}
%    \begin{macro}{\HD@bfseries}
%    \begin{macrocode}
\def\HD@bfseries{\futurelet\HD@hfil\HD@@bfseries}%
%    \end{macrocode}
%    \end{macro}
%    \begin{macro}{\HD@@bfseries}
%    \begin{macrocode}
\def\HD@@bfseries{%
  \HDorg@bfseries
  \ifx\HD@hfil\hfil
    \expandafter\HD@@@bfseries
  \fi
}
%    \end{macrocode}
%    \end{macro}
%    \begin{macro}{\HD@@@bfseries}
%    \begin{macrocode}
\def\HD@@@bfseries\hfil#1\hfil{%
  \ifx\\#1\\%
  \else
    \raisebox{\baselineskip}[0pt]{%
      \kern-\HD@margin\relax
      \pdfbookmark[\HD@toclevel@subindex]{#1}{HD.#1}%
      \kern\HD@margin\relax
    }%
  \fi
  \hfil#1\hfil
}%
%    \end{macrocode}
%    \end{macro}
%    \begin{macro}{\HD@margin}
%    \begin{macrocode}
\def\HD@margin{1mm}
%    \end{macrocode}
%    \end{macro}
%
% \subsection{\cs{theglossary}}
%
%    \begin{macro}{\HDorg@theglossary}
%    \begin{macrocode}
\let\HDorg@theglossary\theglossary
%    \end{macrocode}
%    \end{macro}
%    \begin{macro}{\theglossary}
%    \begin{macrocode}
\def\theglossary{%
  \let\HDorg@glossary@prologue\glossary@prologue
  \let\HDorg@writebookmark\Hy@writebookmark
  \HD@sectionpatch
  \def\glossary@prologue{%
    \ifHD@numbered
    \else
      \begingroup
        \let\Hy@writebookmark\HDorg@writebookmark
        \stepcounter{HD@unique}%
        \pdfbookmark[\HD@guesstoclevel{\HDorg@glossary@prologue}]%
            {\changehistoryname}{glossary\theHD@unique}%
      \endgroup
    \fi
    \HDorg@glossary@prologue
  }%
  \HDorg@theglossary
}
%    \end{macrocode}
%    \end{macro}
%    \begin{macro}{\glossaryname}
%    \begin{macrocode}
\providecommand*{\glossaryname}{Change History}
%    \end{macrocode}
%    \end{macro}
%
%    \begin{macro}{\glossaryname}
%    \begin{macrocode}
\providecommand*{\changehistoryname}{\glossaryname}
%    \end{macrocode}
%    \end{macro}
%
% \subsection{Index}
%
%    \begin{macrocode}
\newcounter{HD@hypercount}
%    \end{macrocode}
%    \begin{macro}{\HD@helplength}
%    \begin{macrocode}
\newlength{\HD@helplength}
%    \end{macrocode}
%    \end{macro}
%    \begin{macro}{\ifHD@savedest}
%    \begin{macrocode}
\newif\ifHD@savedest
\HD@savedesttrue
%    \end{macrocode}
%    \end{macro}
%    \begin{macrocode}
\AtBeginDocument{%
  \HD@savedest@init
}
%    \end{macrocode}
%    \begin{macrocode}
\RequirePackage{rerunfilecheck}[2009/12/10]
%    \end{macrocode}
%    \begin{macro}{\HD@savedest@init}
%    \begin{macrocode}
\begingroup
  \catcode`\^^M=\active
\@firstofone{\endgroup
  \def\HD@savedest@init{%
    \ifHD@savedest
      \let\HDorg@tilde~%
      \let~\HD@savedest@entry
      \edef\HDorg@catcodeM{\the\catcode`\^^M}%
      \catcode`\^^M=\active
      \def^^M{}%
      \InputIfFileExists{\jobname.hd}{}{}%
      \let~\HDorg@tilde
      \catcode`\^^M=\HDorg@catcodeM\relax
      \newwrite\HD@savedest@out
      \RerunFileCheck{\jobname.hd}{%
        \immediate\closeout\HD@savedest@out
      }{}%
      \immediate\openout\HD@savedest@out=\jobname.hd\relax
    \else
      \let\HD@savedest@add\@gobble
    \fi
  }%
%    \end{macrocode}
%    \end{macro}
%    \begin{macro}{\HD@savedest@entry}
%    \begin{macrocode}
  \def\HD@savedest@entry#1^^M{%
    \@namedef{HD.#1}{}%
  }%
}
%    \end{macrocode}
%    \end{macro}
%    \begin{macro}{\HD@savedest@add}
%    \begin{macrocode}
\def\HD@savedest@add#1{%
  \immediate\write\HD@savedest@out{\string~#1}%
}
%    \end{macrocode}
%    \end{macro}
%    \begin{macro}{\ifHD@targetraise}
%    \begin{macrocode}
\newif\ifHD@targetraise
\HD@targetraisetrue
%    \end{macrocode}
%    \end{macro}
%    \begin{macro}{\HD@target}
%    \begin{macrocode}
\def\HD@target{%
  \ifHD@targetraise
    \ifhmode
      \settoheight{\HD@helplength}{[H]}%
      \addtolength{\HD@helplength}{0.4ex}%
      \raisebox{\HD@helplength}[0pt][0pt]{%
        \HD@maketarget
      }%
    \else
      \HD@maketarget
    \fi
  \else
    \HD@maketarget
  \fi
}
%    \end{macrocode}
%    \end{macro}
%    \begin{macro}{\HD@maketarget}
%    \begin{macrocode}
\def\HD@maketarget{%
  \stepcounter{HD@hypercount}%
  \def\HD@next{%
    \llap{\hypertarget{HD.\the\c@HD@hypercount}{}\,}%
  }%
  \ifHD@savedest
    \expandafter\ifx\csname HD.\the\c@HD@hypercount\endcsname\@empty
    \else
      \let\HD@next\relax
    \fi
  \fi
  \HD@next
}
%    \end{macrocode}
%    \end{macro}
%
%    \begin{macro}{\CodelineIndex}
%    \begin{macrocode}
\g@addto@macro\CodelineIndex{%
  \let\special@index\HD@codeline@wrindex
}
%    \end{macrocode}
%    \end{macro}
%    \begin{macro}{\PageIndex}
%    \begin{macrocode}
\g@addto@macro\PageIndex{%
  \let\special@index\HD@page@wrindex
}
%    \end{macrocode}
%    \end{macro}
%
%    \begin{macro}{\HD@codeline@wrindex}
%    \begin{macrocode}
\def\HD@codeline@wrindex#1{%
  \begingroup
    \let\HDorg@encapchar\encapchar
    \def\encapchar##1\encapchar##2\@nil{%
      \HDorg@encapchar
      hdclindex{\the\c@HD@hypercount}{##1}%
    }%
    \codeline@wrindex{%
      #1\encapchar\encapchar\@nil
    }%
  \endgroup
}
%    \end{macrocode}
%    \end{macro}
%    \begin{macro}{\HD@page@wrindex}
%    \begin{macrocode}
\def\HD@page@wrindex#1{%
  \begingroup
    \let\HDorg@encapchar\encapchar
    \def\encapchar##1\encapchar##2\@nil{%
      \HDorg@encapchar
      hdpindex{##1}%
    }%
    \HDorg@index{#1\encapchar\encapchar\@nil}%
  \endgroup
}
%    \end{macrocode}
%    \end{macro}
%
%    \begin{macro}{\hdclindex}
%    \begin{macrocode}
\def\hdclindex#1#2#3{%
  \csname\ifx\\#2\\relax\else#2\fi\endcsname{%
    \hyperlink{HD.#1}{#3}%
    \HD@savedest@add{#1}%
  }%
}
%    \end{macrocode}
%    \end{macro}
%    \begin{macro}{\hdpindex}
%    \begin{macrocode}
\def\hdpindex#1#2{%
  \csname\ifx\\#1\\relax\else#1\fi\endcsname{%
    \hyperpage{#2}%
  }%
}
%    \end{macrocode}
%    \end{macro}
%
% \subsection{\cs{PrintDescribeMacro}}
%
%    \begin{macro}{\HDorg@PrintDescribeMacro}
%    \begin{macrocode}
\let\HDorg@PrintDescribeMacro\PrintDescribeMacro
%    \end{macrocode}
%    \end{macro}
%    \begin{macro}{\PrintDescribeMacro}
%    \begin{macrocode}
\renewcommand\PrintDescribeMacro[1]{%
  \begingroup
    \let\HDorg@MacroFont\MacroFont
    \def\MacroFont{%
      \HD@target
      \HDorg@MacroFont
    }%
    \HDorg@PrintDescribeMacro{#1}%
  \endgroup
}
%    \end{macrocode}
%    \end{macro}
%
% \subsection{\cs{PrintMacroName}}
%
%    \begin{macro}{\HDorg@PrintMacroName}
%    \begin{macrocode}
\let\HDorg@PrintMacroName\PrintMacroName
%    \end{macrocode}
%    \end{macro}
%    \begin{macro}{\PrintMacroName}
%    \begin{macrocode}
\renewcommand\PrintMacroName[1]{%
  \begingroup
    \let\HDorg@MacroFont\MacroFont
    \def\MacroFont{%
      \HD@target
      \HDorg@MacroFont
    }%
    \HDorg@PrintMacroName{#1}%
  \endgroup
}
%    \end{macrocode}
%    \end{macro}
%
% \subsection{\cs{theCodelineNo}}
%
%    \begin{macro}{\HDorg@theCodelineNo}
%    \begin{macrocode}
\let\HDorg@theCodelineNo\theCodelineNo
%    \end{macrocode}
%    \end{macro}
%    \begin{macro}{\theCodelineNo}
%    \begin{macrocode}
\renewcommand*\theCodelineNo{%
  \settowidth{\HD@helplength}{\HDorg@theCodelineNo\ }%
  \ifdim\HD@helplength<\MacroIndent
    \addtolength{\HD@helplength}{-\MacroIndent}%
    \llap{%
      \begingroup\HD@target\endgroup
      \hspace*{-\HD@helplength}%
    }%
  \else
    \HD@target
  \fi
  \HDorg@theCodelineNo
}
%    \end{macrocode}
%    \end{macro}
%
%    \begin{macro}{\HDorg@index}
%    \begin{macrocode}
\AtBeginDocument{%
  \let\HDorg@index\index
}
%    \end{macrocode}
%    \end{macro}
%
% \subsection{\cs{SpecialUsageIndex}}
%
%    \begin{macro}{\HDorg@SpecialUsageIndex}
%    \begin{macrocode}
\let\HDorg@SpecialUsageIndex\SpecialUsageIndex
\renewcommand*\SpecialUsageIndex[1]{%
  \@bsphack
  \begingroup
    \HD@target
    \let\index\HDorg@index
    \let\HDorg@encapchar\encapchar
    \edef\encapchar usage{%
       \HDorg@encapchar hdclindex{\the\c@HD@hypercount}{usage}%
    }%
    \HDorg@SpecialUsageIndex{#1}%
  \endgroup
  \@esphack
}
%    \end{macrocode}
%    \end{macro}
%
% \subsection{\cs{autoindex}}
%
%    \begin{macrocode}
\begingroup\expandafter\expandafter\expandafter\endgroup
\expandafter\ifx\csname autoindex\endcsname\relax
\else
%    \end{macrocode}
%    \begin{macro}{\HDorg@autoindex}
%    \begin{macrocode}
  \let\HDorg@autoindex\autoindex
%    \end{macrocode}
%    \end{macro}
%    \begin{macro}{\autoindex}
%    \begin{macrocode}
  \renewcommand*\autoindex[1]{%
    \@bsphack
    \begingroup
      \HD@target
      \let\index\HDorg@index
      \let\HDorg@encapchar\encapchar
      \edef\encapchar usage{%
         \HDorg@encapchar hdclindex{\the\c@HD@hypercount}{usage}%
      }%
      \HDorg@autoindex{#1}%
    \endgroup
    \@esphack
  }%
\fi
%    \end{macrocode}
%    \end{macro}
%
% \subsection{\cs{SpecialEnvIndex}}
%
%    \begin{macro}{\HDorg@SpecialEnvIndex}
%    \begin{macrocode}
\let\HDorg@SpecialEnvIndex\SpecialEnvIndex
%    \end{macrocode}
%    \end{macro}
%    \begin{macro}{\SpecialEnvIndex}
%    \begin{macrocode}
\renewcommand*\SpecialEnvIndex[1]{%
  \@bsphack
  \begingroup
    \HD@target
    \let\HDorg@encapchar\encapchar
    \edef\encapchar usage{%
      \HDorg@encapchar hdclindex{\the\c@HD@hypercount}{usage}%
    }%
    \HDorg@SpecialEnvIndex{#1}%
  \endgroup
  \@esphack
}
%    \end{macrocode}
%    \end{macro}
%
% \subsection{\cs{SortIndex}}
%
%    \begin{macro}{\HDorg@SortIndex}
%    \begin{macrocode}
\let\HDorg@SortIndex\SortIndex
%    \end{macrocode}
%    \end{macro}
%    \begin{macro}{\SortIndex}
%    \begin{macrocode}
\renewcommand*{\SortIndex}[2]{%
  \@bsphack
  \begingroup
    \let\index\HD@page@wrindex
    \HDorg@SortIndex{#1}{#2}%
  \endgroup
  \@esphack
}
%    \end{macrocode}
%    \end{macro}
%
% \subsection{\cs{@wrglossary}}
%
%    \begin{macro}{\HDorg@wrglossary}
%    \begin{macrocode}
\let\HDorg@wrglossary\@wrglossary
%    \end{macrocode}
%    \end{macro}
%    \begin{macro}{\@wrglossary}
%    \begin{macrocode}
\def\@wrglossary#1{%
  \let\HDorg@encapchar\encapchar
  \def\encapchar##1\encapchar##2\@nil{%
    \HDorg@encapchar
    hdpindex{##1}%
  }%
  \HDorg@wrglossary{#1\encapchar\encapchar\@nil}%
}
%    \end{macrocode}
%    \end{macro}
%
% \subsection{\cs{MacroIndent}}
%
%    \cs{MacroIndent} automatically calculated.
%    \begin{macrocode}
\AfterLastShipout{%
  \if@filesw
    \immediate\write\@auxout{%
      \string\providecommand\string\HD@SetMacroIndent[1]{}%
    }%
    \immediate\write\@auxout{%
      \string\HD@SetMacroIndent{\number\value{CodelineNo}}%
    }%
  \fi
}
%    \end{macrocode}
%    \begin{macro}{\HD@SetMacroIndent}
%    \begin{macrocode}
\def\HD@SetMacroIndent#1{%
  \begingroup
    \value{CodelineNo}=#1\relax
    \settowidth{\MacroIndent}{\HDorg@theCodelineNo\ }%
    \global\MacroIndent\MacroIndent
  \endgroup
}
%    \end{macrocode}
%    \end{macro}
%
%    \begin{macrocode}
%</package>
%    \end{macrocode}
%
% \section{Installation}
%
% \subsection{Download}
%
% \paragraph{Package.} This package is available on
% CTAN\footnote{\url{http://ctan.org/pkg/hypdoc}}:
% \begin{description}
% \item[\CTAN{macros/latex/contrib/oberdiek/hypdoc.dtx}] The source file.
% \item[\CTAN{macros/latex/contrib/oberdiek/hypdoc.pdf}] Documentation.
% \end{description}
%
%
% \paragraph{Bundle.} All the packages of the bundle `oberdiek'
% are also available in a TDS compliant ZIP archive. There
% the packages are already unpacked and the documentation files
% are generated. The files and directories obey the TDS standard.
% \begin{description}
% \item[\CTAN{install/macros/latex/contrib/oberdiek.tds.zip}]
% \end{description}
% \emph{TDS} refers to the standard ``A Directory Structure
% for \TeX\ Files'' (\CTAN{tds/tds.pdf}). Directories
% with \xfile{texmf} in their name are usually organized this way.
%
% \subsection{Bundle installation}
%
% \paragraph{Unpacking.} Unpack the \xfile{oberdiek.tds.zip} in the
% TDS tree (also known as \xfile{texmf} tree) of your choice.
% Example (linux):
% \begin{quote}
%   |unzip oberdiek.tds.zip -d ~/texmf|
% \end{quote}
%
% \paragraph{Script installation.}
% Check the directory \xfile{TDS:scripts/oberdiek/} for
% scripts that need further installation steps.
% Package \xpackage{attachfile2} comes with the Perl script
% \xfile{pdfatfi.pl} that should be installed in such a way
% that it can be called as \texttt{pdfatfi}.
% Example (linux):
% \begin{quote}
%   |chmod +x scripts/oberdiek/pdfatfi.pl|\\
%   |cp scripts/oberdiek/pdfatfi.pl /usr/local/bin/|
% \end{quote}
%
% \subsection{Package installation}
%
% \paragraph{Unpacking.} The \xfile{.dtx} file is a self-extracting
% \docstrip\ archive. The files are extracted by running the
% \xfile{.dtx} through \plainTeX:
% \begin{quote}
%   \verb|tex hypdoc.dtx|
% \end{quote}
%
% \paragraph{TDS.} Now the different files must be moved into
% the different directories in your installation TDS tree
% (also known as \xfile{texmf} tree):
% \begin{quote}
% \def\t{^^A
% \begin{tabular}{@{}>{\ttfamily}l@{ $\rightarrow$ }>{\ttfamily}l@{}}
%   hypdoc.sty & tex/latex/oberdiek/hypdoc.sty\\
%   hypdoc.pdf & doc/latex/oberdiek/hypdoc.pdf\\
%   hypdoc.dtx & source/latex/oberdiek/hypdoc.dtx\\
% \end{tabular}^^A
% }^^A
% \sbox0{\t}^^A
% \ifdim\wd0>\linewidth
%   \begingroup
%     \advance\linewidth by\leftmargin
%     \advance\linewidth by\rightmargin
%   \edef\x{\endgroup
%     \def\noexpand\lw{\the\linewidth}^^A
%   }\x
%   \def\lwbox{^^A
%     \leavevmode
%     \hbox to \linewidth{^^A
%       \kern-\leftmargin\relax
%       \hss
%       \usebox0
%       \hss
%       \kern-\rightmargin\relax
%     }^^A
%   }^^A
%   \ifdim\wd0>\lw
%     \sbox0{\small\t}^^A
%     \ifdim\wd0>\linewidth
%       \ifdim\wd0>\lw
%         \sbox0{\footnotesize\t}^^A
%         \ifdim\wd0>\linewidth
%           \ifdim\wd0>\lw
%             \sbox0{\scriptsize\t}^^A
%             \ifdim\wd0>\linewidth
%               \ifdim\wd0>\lw
%                 \sbox0{\tiny\t}^^A
%                 \ifdim\wd0>\linewidth
%                   \lwbox
%                 \else
%                   \usebox0
%                 \fi
%               \else
%                 \lwbox
%               \fi
%             \else
%               \usebox0
%             \fi
%           \else
%             \lwbox
%           \fi
%         \else
%           \usebox0
%         \fi
%       \else
%         \lwbox
%       \fi
%     \else
%       \usebox0
%     \fi
%   \else
%     \lwbox
%   \fi
% \else
%   \usebox0
% \fi
% \end{quote}
% If you have a \xfile{docstrip.cfg} that configures and enables \docstrip's
% TDS installing feature, then some files can already be in the right
% place, see the documentation of \docstrip.
%
% \subsection{Refresh file name databases}
%
% If your \TeX~distribution
% (\teTeX, \mikTeX, \dots) relies on file name databases, you must refresh
% these. For example, \teTeX\ users run \verb|texhash| or
% \verb|mktexlsr|.
%
% \subsection{Some details for the interested}
%
% \paragraph{Attached source.}
%
% The PDF documentation on CTAN also includes the
% \xfile{.dtx} source file. It can be extracted by
% AcrobatReader 6 or higher. Another option is \textsf{pdftk},
% e.g. unpack the file into the current directory:
% \begin{quote}
%   \verb|pdftk hypdoc.pdf unpack_files output .|
% \end{quote}
%
% \paragraph{Unpacking with \LaTeX.}
% The \xfile{.dtx} chooses its action depending on the format:
% \begin{description}
% \item[\plainTeX:] Run \docstrip\ and extract the files.
% \item[\LaTeX:] Generate the documentation.
% \end{description}
% If you insist on using \LaTeX\ for \docstrip\ (really,
% \docstrip\ does not need \LaTeX), then inform the autodetect routine
% about your intention:
% \begin{quote}
%   \verb|latex \let\install=y\input{hypdoc.dtx}|
% \end{quote}
% Do not forget to quote the argument according to the demands
% of your shell.
%
% \paragraph{Generating the documentation.}
% You can use both the \xfile{.dtx} or the \xfile{.drv} to generate
% the documentation. The process can be configured by the
% configuration file \xfile{ltxdoc.cfg}. For instance, put this
% line into this file, if you want to have A4 as paper format:
% \begin{quote}
%   \verb|\PassOptionsToClass{a4paper}{article}|
% \end{quote}
% An example follows how to generate the
% documentation with pdf\LaTeX:
% \begin{quote}
%\begin{verbatim}
%pdflatex hypdoc.dtx
%makeindex -s gind.ist hypdoc.idx
%pdflatex hypdoc.dtx
%makeindex -s gind.ist hypdoc.idx
%pdflatex hypdoc.dtx
%\end{verbatim}
% \end{quote}
%
% \section{Catalogue}
%
% The following XML file can be used as source for the
% \href{http://mirror.ctan.org/help/Catalogue/catalogue.html}{\TeX\ Catalogue}.
% The elements \texttt{caption} and \texttt{description} are imported
% from the original XML file from the Catalogue.
% The name of the XML file in the Catalogue is \xfile{hypdoc.xml}.
%    \begin{macrocode}
%<*catalogue>
<?xml version='1.0' encoding='us-ascii'?>
<!DOCTYPE entry SYSTEM 'catalogue.dtd'>
<entry datestamp='$Date$' modifier='$Author$' id='hypdoc'>
  <name>hypdoc</name>
  <caption>Hyper extensions for doc.sty.</caption>
  <authorref id='auth:oberdiek'/>
  <copyright owner='Heiko Oberdiek' year='2002,2006,2007,2009-2011'/>
  <license type='lppl1.3'/>
  <version number='1.14'/>
  <description>
    This package adds hypertext features to the package
    <xref refid='doc'>doc</xref> that is used in the documentation
    system of <xref refid='latex'>LaTeX2e</xref>. Bookmarks are added
    and references are linked as far as possible.
    <p/>
    The package is part of the <xref refid='oberdiek'>oberdiek</xref>
    bundle.
  </description>
  <documentation details='Package documentation'
      href='ctan:/macros/latex/contrib/oberdiek/hypdoc.pdf'/>
  <ctan file='true' path='/macros/latex/contrib/oberdiek/hypdoc.dtx'/>
  <miktex location='oberdiek'/>
  <texlive location='oberdiek'/>
  <install path='/macros/latex/contrib/oberdiek/oberdiek.tds.zip'/>
</entry>
%</catalogue>
%    \end{macrocode}
%
% \begin{History}
%   \begin{Version}{2002/05/10 v1.0}
%   \item
%     First standalone version.
%   \end{Version}
%   \begin{Version}{2006/02/20 v1.1}
%   \item
%     Option \xoption{numbered} added.
%   \item
%     \cs{HD@savedest@init} fixed (empty \xext{hd} file triggered
%     first page with fl ligature).
%   \item
%     Bookmarks for \cs{listoffigures} and \cs{listoftables}.
%   \item
%     Documentation: features and usage added.
%   \item
%     LPPL 1.3.
%   \end{Version}
%   \begin{Version}{2006/04/27 v1.2}
%   \item
%     \cs{HD@target} fixed (\xfile{multicol.dtx}).
%   \end{Version}
%   \begin{Version}{2006/06/01 v1.3}
%   \item
%     Support for package \xpackage{amsdtx}'s \cs{autoindex} added.
%   \end{Version}
%   \begin{Version}{2006/06/02 v1.4}
%   \item
%     Bookmark fixed if there are several table of contents,
%     bibliographies, glossaries, \dots
%   \end{Version}
%   \begin{Version}{2007/04/11 v1.5}
%   \item
%     Line ends sanitized.
%   \end{Version}
%   \begin{Version}{2007/11/12 v1.6}
%   \item
%     Bug fix in \cs{listoffigures} (Axel Sommerfeldt).
%   \item
%     Markup added in implementation section.
%   \end{Version}
%   \begin{Version}{2009/12/10 v1.7}
%   \item
%     Use of package \xpackage{rerunfilecheck}.
%   \end{Version}
%   \begin{Version}{2010/02/03 v1.8}
%   \item
%     Fix for bug latex/4096 added.
%   \end{Version}
%   \begin{Version}{2010/03/26 v1.9}
%   \item
%     Definition of \cs{HD@SetMacroIndex} is provided in the \xext{aux} file.
%   \end{Version}
%   \begin{Version}{2011/05/05 v1.10}
%   \item
%     \cs{raisebox} in \cs{HD@target} with height and depth of \texttt{0pt}
%     (request of GL).
%   \end{Version}
%   \begin{Version}{2011/08/19 v1.11}
%   \item
%     Latest comma in option list for hyperref removed to avoid option clash
%     because of \hologo{LaTeX} 2011/06/27 with the new \cs{in@}.
%   \end{Version}
%   \begin{Version}{2016/05/16 v1.12}
%   \item
%     Documentation updates.
%   \end{Version}
%   \begin{Version}{2016/10/02 v1.13}
%   \item
%     Fixes to bookmarknames, also introduce \cs{changehistoryname}
%   \end{Version}
%   \begin{Version}{2018/09/07 v1.14}
%   \item
%     Do not load \xpackage{thumpdf}.
%   \end{Version}
% \end{History}
%
% \PrintIndex
%
% \Finale
\endinput
|
% \end{quote}
% Do not forget to quote the argument according to the demands
% of your shell.
%
% \paragraph{Generating the documentation.}
% You can use both the \xfile{.dtx} or the \xfile{.drv} to generate
% the documentation. The process can be configured by the
% configuration file \xfile{ltxdoc.cfg}. For instance, put this
% line into this file, if you want to have A4 as paper format:
% \begin{quote}
%   \verb|\PassOptionsToClass{a4paper}{article}|
% \end{quote}
% An example follows how to generate the
% documentation with pdf\LaTeX:
% \begin{quote}
%\begin{verbatim}
%pdflatex hypdoc.dtx
%makeindex -s gind.ist hypdoc.idx
%pdflatex hypdoc.dtx
%makeindex -s gind.ist hypdoc.idx
%pdflatex hypdoc.dtx
%\end{verbatim}
% \end{quote}
%
% \section{Catalogue}
%
% The following XML file can be used as source for the
% \href{http://mirror.ctan.org/help/Catalogue/catalogue.html}{\TeX\ Catalogue}.
% The elements \texttt{caption} and \texttt{description} are imported
% from the original XML file from the Catalogue.
% The name of the XML file in the Catalogue is \xfile{hypdoc.xml}.
%    \begin{macrocode}
%<*catalogue>
<?xml version='1.0' encoding='us-ascii'?>
<!DOCTYPE entry SYSTEM 'catalogue.dtd'>
<entry datestamp='$Date$' modifier='$Author$' id='hypdoc'>
  <name>hypdoc</name>
  <caption>Hyper extensions for doc.sty.</caption>
  <authorref id='auth:oberdiek'/>
  <copyright owner='Heiko Oberdiek' year='2002,2006,2007,2009-2011'/>
  <license type='lppl1.3'/>
  <version number='1.14'/>
  <description>
    This package adds hypertext features to the package
    <xref refid='doc'>doc</xref> that is used in the documentation
    system of <xref refid='latex'>LaTeX2e</xref>. Bookmarks are added
    and references are linked as far as possible.
    <p/>
    The package is part of the <xref refid='oberdiek'>oberdiek</xref>
    bundle.
  </description>
  <documentation details='Package documentation'
      href='ctan:/macros/latex/contrib/oberdiek/hypdoc.pdf'/>
  <ctan file='true' path='/macros/latex/contrib/oberdiek/hypdoc.dtx'/>
  <miktex location='oberdiek'/>
  <texlive location='oberdiek'/>
  <install path='/macros/latex/contrib/oberdiek/oberdiek.tds.zip'/>
</entry>
%</catalogue>
%    \end{macrocode}
%
% \begin{History}
%   \begin{Version}{2002/05/10 v1.0}
%   \item
%     First standalone version.
%   \end{Version}
%   \begin{Version}{2006/02/20 v1.1}
%   \item
%     Option \xoption{numbered} added.
%   \item
%     \cs{HD@savedest@init} fixed (empty \xext{hd} file triggered
%     first page with fl ligature).
%   \item
%     Bookmarks for \cs{listoffigures} and \cs{listoftables}.
%   \item
%     Documentation: features and usage added.
%   \item
%     LPPL 1.3.
%   \end{Version}
%   \begin{Version}{2006/04/27 v1.2}
%   \item
%     \cs{HD@target} fixed (\xfile{multicol.dtx}).
%   \end{Version}
%   \begin{Version}{2006/06/01 v1.3}
%   \item
%     Support for package \xpackage{amsdtx}'s \cs{autoindex} added.
%   \end{Version}
%   \begin{Version}{2006/06/02 v1.4}
%   \item
%     Bookmark fixed if there are several table of contents,
%     bibliographies, glossaries, \dots
%   \end{Version}
%   \begin{Version}{2007/04/11 v1.5}
%   \item
%     Line ends sanitized.
%   \end{Version}
%   \begin{Version}{2007/11/12 v1.6}
%   \item
%     Bug fix in \cs{listoffigures} (Axel Sommerfeldt).
%   \item
%     Markup added in implementation section.
%   \end{Version}
%   \begin{Version}{2009/12/10 v1.7}
%   \item
%     Use of package \xpackage{rerunfilecheck}.
%   \end{Version}
%   \begin{Version}{2010/02/03 v1.8}
%   \item
%     Fix for bug latex/4096 added.
%   \end{Version}
%   \begin{Version}{2010/03/26 v1.9}
%   \item
%     Definition of \cs{HD@SetMacroIndex} is provided in the \xext{aux} file.
%   \end{Version}
%   \begin{Version}{2011/05/05 v1.10}
%   \item
%     \cs{raisebox} in \cs{HD@target} with height and depth of \texttt{0pt}
%     (request of GL).
%   \end{Version}
%   \begin{Version}{2011/08/19 v1.11}
%   \item
%     Latest comma in option list for hyperref removed to avoid option clash
%     because of \hologo{LaTeX} 2011/06/27 with the new \cs{in@}.
%   \end{Version}
%   \begin{Version}{2016/05/16 v1.12}
%   \item
%     Documentation updates.
%   \end{Version}
%   \begin{Version}{2016/10/02 v1.13}
%   \item
%     Fixes to bookmarknames, also introduce \cs{changehistoryname}
%   \end{Version}
%   \begin{Version}{2018/09/07 v1.14}
%   \item
%     Do not load \xpackage{thumpdf}.
%   \end{Version}
% \end{History}
%
% \PrintIndex
%
% \Finale
\endinput
|
% \end{quote}
% Do not forget to quote the argument according to the demands
% of your shell.
%
% \paragraph{Generating the documentation.}
% You can use both the \xfile{.dtx} or the \xfile{.drv} to generate
% the documentation. The process can be configured by the
% configuration file \xfile{ltxdoc.cfg}. For instance, put this
% line into this file, if you want to have A4 as paper format:
% \begin{quote}
%   \verb|\PassOptionsToClass{a4paper}{article}|
% \end{quote}
% An example follows how to generate the
% documentation with pdf\LaTeX:
% \begin{quote}
%\begin{verbatim}
%pdflatex hypdoc.dtx
%makeindex -s gind.ist hypdoc.idx
%pdflatex hypdoc.dtx
%makeindex -s gind.ist hypdoc.idx
%pdflatex hypdoc.dtx
%\end{verbatim}
% \end{quote}
%
% \section{Catalogue}
%
% The following XML file can be used as source for the
% \href{http://mirror.ctan.org/help/Catalogue/catalogue.html}{\TeX\ Catalogue}.
% The elements \texttt{caption} and \texttt{description} are imported
% from the original XML file from the Catalogue.
% The name of the XML file in the Catalogue is \xfile{hypdoc.xml}.
%    \begin{macrocode}
%<*catalogue>
<?xml version='1.0' encoding='us-ascii'?>
<!DOCTYPE entry SYSTEM 'catalogue.dtd'>
<entry datestamp='$Date$' modifier='$Author$' id='hypdoc'>
  <name>hypdoc</name>
  <caption>Hyper extensions for doc.sty.</caption>
  <authorref id='auth:oberdiek'/>
  <copyright owner='Heiko Oberdiek' year='2002,2006,2007,2009-2011'/>
  <license type='lppl1.3'/>
  <version number='1.14'/>
  <description>
    This package adds hypertext features to the package
    <xref refid='doc'>doc</xref> that is used in the documentation
    system of <xref refid='latex'>LaTeX2e</xref>. Bookmarks are added
    and references are linked as far as possible.
    <p/>
    The package is part of the <xref refid='oberdiek'>oberdiek</xref>
    bundle.
  </description>
  <documentation details='Package documentation'
      href='ctan:/macros/latex/contrib/oberdiek/hypdoc.pdf'/>
  <ctan file='true' path='/macros/latex/contrib/oberdiek/hypdoc.dtx'/>
  <miktex location='oberdiek'/>
  <texlive location='oberdiek'/>
  <install path='/macros/latex/contrib/oberdiek/oberdiek.tds.zip'/>
</entry>
%</catalogue>
%    \end{macrocode}
%
% \begin{History}
%   \begin{Version}{2002/05/10 v1.0}
%   \item
%     First standalone version.
%   \end{Version}
%   \begin{Version}{2006/02/20 v1.1}
%   \item
%     Option \xoption{numbered} added.
%   \item
%     \cs{HD@savedest@init} fixed (empty \xext{hd} file triggered
%     first page with fl ligature).
%   \item
%     Bookmarks for \cs{listoffigures} and \cs{listoftables}.
%   \item
%     Documentation: features and usage added.
%   \item
%     LPPL 1.3.
%   \end{Version}
%   \begin{Version}{2006/04/27 v1.2}
%   \item
%     \cs{HD@target} fixed (\xfile{multicol.dtx}).
%   \end{Version}
%   \begin{Version}{2006/06/01 v1.3}
%   \item
%     Support for package \xpackage{amsdtx}'s \cs{autoindex} added.
%   \end{Version}
%   \begin{Version}{2006/06/02 v1.4}
%   \item
%     Bookmark fixed if there are several table of contents,
%     bibliographies, glossaries, \dots
%   \end{Version}
%   \begin{Version}{2007/04/11 v1.5}
%   \item
%     Line ends sanitized.
%   \end{Version}
%   \begin{Version}{2007/11/12 v1.6}
%   \item
%     Bug fix in \cs{listoffigures} (Axel Sommerfeldt).
%   \item
%     Markup added in implementation section.
%   \end{Version}
%   \begin{Version}{2009/12/10 v1.7}
%   \item
%     Use of package \xpackage{rerunfilecheck}.
%   \end{Version}
%   \begin{Version}{2010/02/03 v1.8}
%   \item
%     Fix for bug latex/4096 added.
%   \end{Version}
%   \begin{Version}{2010/03/26 v1.9}
%   \item
%     Definition of \cs{HD@SetMacroIndex} is provided in the \xext{aux} file.
%   \end{Version}
%   \begin{Version}{2011/05/05 v1.10}
%   \item
%     \cs{raisebox} in \cs{HD@target} with height and depth of \texttt{0pt}
%     (request of GL).
%   \end{Version}
%   \begin{Version}{2011/08/19 v1.11}
%   \item
%     Latest comma in option list for hyperref removed to avoid option clash
%     because of \hologo{LaTeX} 2011/06/27 with the new \cs{in@}.
%   \end{Version}
%   \begin{Version}{2016/05/16 v1.12}
%   \item
%     Documentation updates.
%   \end{Version}
%   \begin{Version}{2016/10/02 v1.13}
%   \item
%     Fixes to bookmarknames, also introduce \cs{changehistoryname}
%   \end{Version}
%   \begin{Version}{2018/09/07 v1.14}
%   \item
%     Do not load \xpackage{thumpdf}.
%   \end{Version}
% \end{History}
%
% \PrintIndex
%
% \Finale
\endinput

%        (quote the arguments according to the demands of your shell)
%
% Documentation:
%    (a) If hypdoc.drv is present:
%           latex hypdoc.drv
%    (b) Without hypdoc.drv:
%           latex hypdoc.dtx; ...
%    The class ltxdoc loads the configuration file ltxdoc.cfg
%    if available. Here you can specify further options, e.g.
%    use A4 as paper format:
%       \PassOptionsToClass{a4paper}{article}
%
%    Programm calls to get the documentation (example):
%       pdflatex hypdoc.dtx
%       makeindex -s gind.ist hypdoc.idx
%       pdflatex hypdoc.dtx
%       makeindex -s gind.ist hypdoc.idx
%       pdflatex hypdoc.dtx
%
% Installation:
%    TDS:tex/latex/oberdiek/hypdoc.sty
%    TDS:doc/latex/oberdiek/hypdoc.pdf
%    TDS:source/latex/oberdiek/hypdoc.dtx
%
%<*ignore>
\begingroup
  \catcode123=1 %
  \catcode125=2 %
  \def\x{LaTeX2e}%
\expandafter\endgroup
\ifcase 0\ifx\install y1\fi\expandafter
         \ifx\csname processbatchFile\endcsname\relax\else1\fi
         \ifx\fmtname\x\else 1\fi\relax
\else\csname fi\endcsname
%</ignore>
%<*install>
\input docstrip.tex
\Msg{************************************************************************}
\Msg{* Installation}
\Msg{* Package: hypdoc 2018/09/07 v1.14 Hyper extensions for doc.sty (HO)}
\Msg{************************************************************************}

\keepsilent
\askforoverwritefalse

\let\MetaPrefix\relax
\preamble

This is a generated file.

Project: hypdoc
Version: 2018/09/07 v1.14

Copyright (C) 2002, 2006, 2007, 2009-2011 by
   Heiko Oberdiek <heiko.oberdiek at googlemail.com>

This work may be distributed and/or modified under the
conditions of the LaTeX Project Public License, either
version 1.3c of this license or (at your option) any later
version. This version of this license is in
   http://www.latex-project.org/lppl/lppl-1-3c.txt
and the latest version of this license is in
   http://www.latex-project.org/lppl.txt
and version 1.3 or later is part of all distributions of
LaTeX version 2005/12/01 or later.

This work has the LPPL maintenance status "maintained".

This Current Maintainer of this work is Heiko Oberdiek.

This work consists of the main source file hypdoc.dtx
and the derived files
   hypdoc.sty, hypdoc.pdf, hypdoc.ins, hypdoc.drv.

\endpreamble
\let\MetaPrefix\DoubleperCent

\generate{%
  \file{hypdoc.ins}{\from{hypdoc.dtx}{install}}%
  \file{hypdoc.drv}{\from{hypdoc.dtx}{driver}}%
  \usedir{tex/latex/oberdiek}%
  \file{hypdoc.sty}{\from{hypdoc.dtx}{package}}%
  \nopreamble
  \nopostamble
%  \usedir{source/latex/oberdiek/catalogue}%
%  \file{hypdoc.xml}{\from{hypdoc.dtx}{catalogue}}%
}

\catcode32=13\relax% active space
\let =\space%
\Msg{************************************************************************}
\Msg{*}
\Msg{* To finish the installation you have to move the following}
\Msg{* file into a directory searched by TeX:}
\Msg{*}
\Msg{*     hypdoc.sty}
\Msg{*}
\Msg{* To produce the documentation run the file `hypdoc.drv'}
\Msg{* through LaTeX.}
\Msg{*}
\Msg{* Happy TeXing!}
\Msg{*}
\Msg{************************************************************************}

\endbatchfile
%</install>
%<*ignore>
\fi
%</ignore>
%<*driver>
\NeedsTeXFormat{LaTeX2e}
\ProvidesFile{hypdoc.drv}%
  [2018/09/07 v1.14 Hyper extensions for doc.sty (HO)]%
\documentclass{ltxdoc}
\usepackage{holtxdoc}[2011/11/22]
\begin{document}
  \DocInput{hypdoc.dtx}%
\end{document}
%</driver>
% \fi
%
%
% \CharacterTable
%  {Upper-case    \A\B\C\D\E\F\G\H\I\J\K\L\M\N\O\P\Q\R\S\T\U\V\W\X\Y\Z
%   Lower-case    \a\b\c\d\e\f\g\h\i\j\k\l\m\n\o\p\q\r\s\t\u\v\w\x\y\z
%   Digits        \0\1\2\3\4\5\6\7\8\9
%   Exclamation   \!     Double quote  \"     Hash (number) \#
%   Dollar        \$     Percent       \%     Ampersand     \&
%   Acute accent  \'     Left paren    \(     Right paren   \)
%   Asterisk      \*     Plus          \+     Comma         \,
%   Minus         \-     Point         \.     Solidus       \/
%   Colon         \:     Semicolon     \;     Less than     \<
%   Equals        \=     Greater than  \>     Question mark \?
%   Commercial at \@     Left bracket  \[     Backslash     \\
%   Right bracket \]     Circumflex    \^     Underscore    \_
%   Grave accent  \`     Left brace    \{     Vertical bar  \|
%   Right brace   \}     Tilde         \~}
%
% \GetFileInfo{hypdoc.drv}
%
% \title{The \xpackage{hypdoc} package}
% \date{2018/09/07 v1.14}
% \author{Heiko Oberdiek\thanks
% {Please report any issues at https://github.com/ho-tex/oberdiek/issues}\\
% \xemail{heiko.oberdiek at googlemail.com}}
%
% \maketitle
%
% \begin{abstract}
% This package adds hyper features to the package
% \xpackage{doc} that is used in the documentation
% system of \LaTeXe. Bookmarks are added and references
% are linked as far as possible.
% \end{abstract}
%
% \tableofcontents
%
% \section{Features}
%
% The package adds the following features:
% \begin{description}
% \item[Bookmarks:]
%   For supporting bookmarks package \verb|hyperref| is loaded.
%   Some \TeX\ code inside section titles can cause problems
%   during bookmark processing; these can be warnings, ugly
%   bookmark titles or even errors. With the help of
%   \verb|\pdfstringdefDisableCommands| or \verb|\texorpdfstring|
%   you can manage these things, see \verb|hyperref|'s package
%   documentation.
% \item[Additional bookmarks:]
%   Unnumbered bookmark entries are generated for:
%   \begin{itemize}
%   \item Table of contents
%   \item List of figures
%   \item List of tables
%   \item Index
%   \item Glossary
%   \end{itemize}
%   Option \texttt{numbered} controls, whether these section or
%   chapters should be numbered. The exception is the table
%   of contents if this it would be the first numbered
%   section. Then just a bookmark is added.
% \item[Metadata:]
%   The loaded package \verb|hyperref|, enforced by
%   option \verb|pdfusetitle|, tries to detect \verb|\title|
%   and \verb|\author| and sets the corresponding PDF
%   information fields. In case of problems see the previous
%   item about bookmarks.
%   These enries can be overwritten or new ones added by
%   \verb|\hypersetup| after the packages is loaded, see
%   documentation of package \verb|hyperref|.
% \item[References, links:] Because of package \verb|\hyperref|
%   |\ref| become links, also \verb|\url|, footnote referencing
%   are supported.
% \item[Index:]
%   The main part of this package deals with index links in order
%   to support the kind of index that package \verb|doc| provides.
% \item[Thumbnails:]
%   Package \verb|thumbpdf| may additionally be loaded for the case that the
%   PDF file also should contain thumbnails, see the documentation
%   of this package. But thumbnail embedding is not really
%   necessary, because nowadays AcrobatReader is able to
%   generate and view thumbnails on the fly.
% \end{description}
%
% \section{Usage}
%
% The purpose of this packages is to add PDF features to
% the documentation of LaTeX packages. Thus just load
% the package in the driver file that generates the
% documentation:
% \begin{quote}
%   \verb|\usepackage{hypdoc}|
% \end{quote}
% Package \verb|doc| is loaded by \verb|hypdoc|, thus you
% also can replace a \verb|\usepackage{doc}| with this line.
% At any case, the package must be loaded before macros of
% package \verb|doc| such as \verb|\CodelineIndex| or
% \verb|\PageIndex| are used.
%
% If the class \verb|ltxdoc| is used, you can also use
% the configuration file for this class to add
% package \verb|hypdoc|. Add the following line
% to the file \verb|ltxdoc.cfg| (or create a new one):
% \begin{quote}
%\begin{verbatim}
%\AtEndOfClass{\RequirePackage{hypdoc}}
%\end{verbatim}
% \end{quote}
%
% \subsection{Options}
%
% There is one option \texttt{numbered}. It controls,
% whether the starred versions of section or chapter
% inside \cs{tableofcontents}, \cs{listoffigures}, \dots
% should be replaced by the non-star versions.
%
% \subsection{Simple example}
% Example of a simple driver file \verb|foobar.drv|
% of a package \verb|foobar|. Often the driver file
% is packed in the file \verb|foobar.dtx| and can
% be extracted by docstrip or if an instruction file
% for docstrip is provided by:
% \begin{quote}
% \verb|tex foobar.ins|
% \end{quote}
% Note that although docstrip doesn't rely on \LaTeX,
% sometimes you need to run the \verb|.ins| file through
% \LaTeX. Oddly enough, often the only \LaTeX\ feature
% in use is \verb|\NeedsTeXFormat{LaTeX2e}| in the first line.
% \begin{quote}
%\begin{verbatim}
%\documentclass{ltxdoc}
%\usepackage{hypdoc}
%\RecordChanges
%\EnableCrossrefs
%\CodelineIndex
%\begin{document}
%\title{Title for package \texttt{foobar}}
%\author{Author of package \texttt{foobar}}
%\date{...}
%\maketitle
%\tableofcontents
%\DocInput{foobar.dtx}
%\PrintChanges
%\PrintIndex
%\end{document}
%\end{verbatim}
% \end{quote}
%
% Then compile it, for example with pdf\LaTeX.
% \begin{quote}
%\begin{verbatim}
%pdflatex foobar.drv
%makeindex -s gind.ist foobar.idx
%makeindex -s gglo.ist -o foobar.gls foobar.glo
%pdflatex foobar.drv
%\end{verbatim}
% \end{quote}
% You will need several cycles until the cross references
% and similar things are correct.
%
% \subsection{Example \texttt{source2e.tex}}
%
% The source code of the base of \LaTeX\ is available as
% a bunch of \verb|.dtx| files. \LaTeX\ provides the file
% \verb|source2e.tex| to merge them all together.
%
% Now you can either load package \verb|hypdoc| in a private copy
% of this file (please, respect the LPPL) or use the configuration
% file \verb|ltxdoc.cfg|. Example that also sets A4 paper size:
% \begin{quote}
%\begin{verbatim}
%\PassOptionsToClass{a4paper}{article}
%\AtEndOfClass{\RequirePackage{hypdoc}}
%\end{verbatim}
% \end{quote}
%
% Then \verb|source2e.tex| can be compiled by pdf\TeX, for example:
% \begin{quote}
%\begin{verbatim}
%pdflatex source2e
%makeindex -s gind.ist source2e.idx
%makeindex -s gglo.ist -o source2e.glo source2e.gls
%pdflatex source2e
%makeindex -s gind.ist source2e.idx
%makeindex -s gglo.ist -o source2e.glo source2e.gls
%pdflatex source2e
%\end{verbatim}
% \end{quote}
%
% \StopEventually{
% }
%
% \section{Implementation}
%
%    \begin{macrocode}
%<*package>
\NeedsTeXFormat{LaTeX2e}
\ProvidesPackage{hypdoc}%
  [2018/09/07 v1.14 Hyper extensions for doc.sty (HO)]%
%    \end{macrocode}
%    The package identification is done at the top of the |.dtx| file
%    in order to use only one identification string.
%
%    For unique command names this package uses |HD@| as prefix
%
% \subsection{Options}
%
%    \begin{macro}{\ifHD@numbered}
%    \begin{macrocode}
\newif\ifHD@numbered
%    \end{macrocode}
%    \end{macro}
%    \begin{macrocode}
\DeclareOption{numbered}{\HD@numberedtrue}
\ProcessOptions*\relax
%    \end{macrocode}
%
% \subsection{Package loading}
%
%    \begin{macrocode}
\RequirePackage{atveryend}[2010/03/24]
\RequirePackage{doc}
%    \end{macrocode}
%    \begin{macrocode}
\ifx\numexpr\@undefined
  \RequirePackage{calc}%
\fi
%    \end{macrocode}
%    \begin{macrocode}
\RequirePackage[%
  hyperindex=false,%
  pdfusetitle,%
  colorlinks,%
  pdfpagelabels%
]{hyperref}[2002/05/09]%
\hypersetup{%
  plainpages=false,%
  bookmarksopen,%
  bookmarksnumbered,%
  bookmarksopenlevel=0,%
  pdfstartview={FitBH \hypercalcbp{%
    \paperheight-\topmargin-1in-\headheight-\headsep
  }},%
}
%    \end{macrocode}
%
%    \begin{macrocode}
\newcounter{HD@unique}
%    \end{macrocode}
%    \begin{macro}{\theHD@unique}
%    \begin{macrocode}
\renewcommand{\theHD@unique}{%
  \number\value{HD@unique}%
}
%    \end{macrocode}
%    \end{macro}
%
%    \begin{macrocode}
\g@addto@macro\Hy@UseMaketitleInfos{%
  \@ifundefined{fileinfo}{}{%
    \ifx\@pdfsubject\@empty
      \hypersetup{pdfsubject={\fileinfo}}%
    \fi
  }%
}
%    \end{macrocode}
%
%    \begin{macrocode}
\pdfstringdefDisableCommands{%
  \let\thanks\@gobble
  \let\footnote\@gobble
  \def\cs#1{\textbackslash #1}%
  \let\normalfont\@empty
  \let\scshape\@empty
  \def\and{and }%
}
%    \end{macrocode}
%
% \subsubsection{Patch for bug latex/4096}
%
%    \begin{macrocode}
\begingroup\expandafter\expandafter\expandafter\endgroup
\expandafter\ifx\csname mod@math@codes\endcsname\relax
\else
  \def\HD@ModFix#1{%
    \mathcode`#1=`#1\relax
  }%
  \g@addto@macro\mod@math@codes{%
    \HD@ModFix\-%
    \HD@ModFix\+%
    \HD@ModFix\:%
    \HD@ModFix\=%
    \HD@ModFix\*%
  }%
\fi
%    \end{macrocode}
%
% \subsection{Bookmarks}
%
%    \begin{macro}{\HD@sectionpatch}
%    \begin{macrocode}
\def\HD@sectionpatch{%
  \ifHD@numbered
    \HD@@sectionpatch{section}%
    \HD@@sectionpatch{chapter}%
  \else
    \let\Hy@writebookmark\HD@disable@writebookmark
  \fi
}
%    \end{macrocode}
%    \end{macro}
%    \begin{macro}{\HD@@sectionpatch}
%    \begin{macrocode}
\def\HD@@sectionpatch#1{%
  \expandafter\let\csname HDorg@@#1\expandafter\endcsname
    \csname #1\endcsname
  \@namedef{#1}{%
    \@ifstar{\@nameuse{HDorg@@#1}}{\@nameuse{HDorg@@#1}}%
  }%
}
%    \end{macrocode}
%    \end{macro}
%    \begin{macro}{\HD@disable@writebookmark}
%    \begin{macrocode}
\def\HD@disable@writebookmark#1#2#3#4#5{}
%    \end{macrocode}
%    \end{macro}
%
%    \begin{macro}{\HD@guesstoclevel}
%    A wild guessing of the toclevel. There is the hope, that
%    the macro |#1| starts with a sectioning command.
%    \begin{macrocode}
\def\HD@guesstoclevel#1{%
  \expandafter\HD@@guesstoclevel#1\section\HD@end
}
%    \end{macrocode}
%    \end{macro}
%    \begin{macro}{\HD@@guesstoclevel}
%    \begin{macrocode}
\def\HD@@guesstoclevel#1#2\HD@end{%
  \@ifundefined{toclevel@\expandafter\@gobble\string#1\@empty}{%
    1%
  }{%
    \@nameuse{toclevel@\expandafter\@gobble\string#1\@empty}%
  }%
}
%    \end{macrocode}
%    \end{macro}
%
% \subsection{\cs{tableofcontents}}
%
%    \begin{macro}{\HDorg@tableofcontents}
%    \begin{macrocode}
\let\HDorg@tableofcontents\tableofcontents
%    \end{macrocode}
%    \end{macro}
%    \begin{macro}{\tableofcontents}
%    As first section the table of contents shouldn't
%    go into the toc itself. Also a number looks too funny then.
%    Just a bookmark is enough in this case.
%    \begin{macrocode}
\def\tableofcontents{%
  \begingroup
    \@ifundefined{c@section}{%
      \def\c@section{0}%
    }{}%
  \expandafter\endgroup
  \ifcase\c@section
    \stepcounter{HD@unique}%
    \pdfbookmark[\HD@guesstoclevel{\HDorg@tableofcontents}]%
        {\contentsname}{toc\theHD@unique}%
    \HDorg@tableofcontents
  \else
    \ifHD@numbered
    \else
      \stepcounter{HD@unique}%
      \pdfbookmark[\HD@guesstoclevel{\HDorg@tableofcontents}]%
          {\contentsname}{toc\theHD@unique}%
    \fi
    \begingroup
      \HD@sectionpatch
      \HDorg@tableofcontents
    \endgroup
  \fi
}
%    \end{macrocode}
%    \end{macro}
%
% \subsection{\cs{listoffigures}}
%
%    \begin{macro}{\HDorg@listoffigures}
%    \begin{macrocode}
\let\HDorg@listoffigures\listoffigures
%    \end{macrocode}
%    \end{macro}
%    \begin{macro}{\listoffigures}
%    \begin{macrocode}
\def\listoffigures{%
  \ifHD@numbered
  \else
    \stepcounter{HD@unique}%
    \pdfbookmark[\HD@guesstoclevel{\HDorg@listoffigures}]%
        {\listfigurename}{toc\theHD@unique}%
  \fi
  \begingroup
    \HD@sectionpatch
    \HDorg@listoffigures
  \endgroup
}
%    \end{macrocode}
%    \end{macro}
%
% \subsection{\cs{listoftables}}
%
%    \begin{macro}{\HDorg@listoftables}
%    \begin{macrocode}
\let\HDorg@listoftables\listoftables
%    \end{macrocode}
%    \end{macro}
%    \begin{macro}{\listoftables}
%    \begin{macrocode}
\def\listoftables{%
  \ifHD@numbered
  \else
    \stepcounter{HD@unique}%
    \pdfbookmark[\HD@guesstoclevel{\HDorg@listoftables}]%
        {\listtablename}{toc\theHD@unique}%
  \fi
  \begingroup
    \HD@sectionpatch
    \HDorg@listoftables
  \endgroup
}
%    \end{macrocode}
%    \end{macro}
%
% \subsection{\cs{thebibliography}}
%
%    \begin{macro}{\HDorg@thebibliography}
%    \begin{macrocode}
\let\HDorg@thebibliography\thebibliography
%    \end{macrocode}
%    \end{macro}
%    \begin{macro}{\thebibliography}
%    \begin{macrocode}
\def\thebibliography{%
  \ifHD@numbered
  \else
    \stepcounter{HD@unique}%
    \pdfbookmark[\HD@guesstoclevel{\HDorg@thebibliography}]%
      {\@ifundefined{chapter}{\refname}{\bibname}}{bib\theHD@unique}%
  \fi
  \HD@sectionpatch
  \HDorg@thebibliography
}
%    \end{macrocode}
%    \end{macro}
%
% \subsection{\cs{theindex}}
%
%    \begin{macro}{\HDorg@theindex}
%    \begin{macrocode}
\let\HDorg@theindex\theindex
%    \end{macrocode}
%    \end{macro}
%    \begin{macro}{\theindex}
%    \begin{macrocode}
\def\theindex{%
  \let\HDorg@index@prologue\index@prologue
  \let\HDorg@writebookmark\Hy@writebookmark
  \HD@sectionpatch
  \g@addto@macro\IndexParms{%
    \ifHD@numbered
    \else
      \let\Hy@writebookmark\HDorg@writebookmark
    \fi
    \let\bfseries\HD@bfseries
  }%
  \edef\HD@toclevel@index{\HD@guesstoclevel{\index@prologue}}%
  \begingroup
    \count@=\HD@toclevel@index\relax
    \advance\count@ by 1 %
    \xdef\HD@toclevel@subindex{\the\count@}%
  \endgroup
  \def\index@prologue{%
    \ifHD@numbered
    \else
      \begingroup
        \stepcounter{HD@unique}%
        \let\Hy@writebookmark\HDorg@writebookmark
        \pdfbookmark[\HD@toclevel@index]%
            {\indexname}{index\theHD@unique}%
      \endgroup
    \fi
    \HDorg@index@prologue
  }%
  \HDorg@theindex
}
%    \end{macrocode}
%    \end{macro}
%    \begin{macro}{\HDorg@bfseries}
%    \begin{macrocode}
\let\HDorg@bfseries\bfseries
%    \end{macrocode}
%    \end{macro}
%    \begin{macro}{\HD@bfseries}
%    \begin{macrocode}
\def\HD@bfseries{\futurelet\HD@hfil\HD@@bfseries}%
%    \end{macrocode}
%    \end{macro}
%    \begin{macro}{\HD@@bfseries}
%    \begin{macrocode}
\def\HD@@bfseries{%
  \HDorg@bfseries
  \ifx\HD@hfil\hfil
    \expandafter\HD@@@bfseries
  \fi
}
%    \end{macrocode}
%    \end{macro}
%    \begin{macro}{\HD@@@bfseries}
%    \begin{macrocode}
\def\HD@@@bfseries\hfil#1\hfil{%
  \ifx\\#1\\%
  \else
    \raisebox{\baselineskip}[0pt]{%
      \kern-\HD@margin\relax
      \pdfbookmark[\HD@toclevel@subindex]{#1}{HD.#1}%
      \kern\HD@margin\relax
    }%
  \fi
  \hfil#1\hfil
}%
%    \end{macrocode}
%    \end{macro}
%    \begin{macro}{\HD@margin}
%    \begin{macrocode}
\def\HD@margin{1mm}
%    \end{macrocode}
%    \end{macro}
%
% \subsection{\cs{theglossary}}
%
%    \begin{macro}{\HDorg@theglossary}
%    \begin{macrocode}
\let\HDorg@theglossary\theglossary
%    \end{macrocode}
%    \end{macro}
%    \begin{macro}{\theglossary}
%    \begin{macrocode}
\def\theglossary{%
  \let\HDorg@glossary@prologue\glossary@prologue
  \let\HDorg@writebookmark\Hy@writebookmark
  \HD@sectionpatch
  \def\glossary@prologue{%
    \ifHD@numbered
    \else
      \begingroup
        \let\Hy@writebookmark\HDorg@writebookmark
        \stepcounter{HD@unique}%
        \pdfbookmark[\HD@guesstoclevel{\HDorg@glossary@prologue}]%
            {\changehistoryname}{glossary\theHD@unique}%
      \endgroup
    \fi
    \HDorg@glossary@prologue
  }%
  \HDorg@theglossary
}
%    \end{macrocode}
%    \end{macro}
%    \begin{macro}{\glossaryname}
%    \begin{macrocode}
\providecommand*{\glossaryname}{Change History}
%    \end{macrocode}
%    \end{macro}
%
%    \begin{macro}{\glossaryname}
%    \begin{macrocode}
\providecommand*{\changehistoryname}{\glossaryname}
%    \end{macrocode}
%    \end{macro}
%
% \subsection{Index}
%
%    \begin{macrocode}
\newcounter{HD@hypercount}
%    \end{macrocode}
%    \begin{macro}{\HD@helplength}
%    \begin{macrocode}
\newlength{\HD@helplength}
%    \end{macrocode}
%    \end{macro}
%    \begin{macro}{\ifHD@savedest}
%    \begin{macrocode}
\newif\ifHD@savedest
\HD@savedesttrue
%    \end{macrocode}
%    \end{macro}
%    \begin{macrocode}
\AtBeginDocument{%
  \HD@savedest@init
}
%    \end{macrocode}
%    \begin{macrocode}
\RequirePackage{rerunfilecheck}[2009/12/10]
%    \end{macrocode}
%    \begin{macro}{\HD@savedest@init}
%    \begin{macrocode}
\begingroup
  \catcode`\^^M=\active
\@firstofone{\endgroup
  \def\HD@savedest@init{%
    \ifHD@savedest
      \let\HDorg@tilde~%
      \let~\HD@savedest@entry
      \edef\HDorg@catcodeM{\the\catcode`\^^M}%
      \catcode`\^^M=\active
      \def^^M{}%
      \InputIfFileExists{\jobname.hd}{}{}%
      \let~\HDorg@tilde
      \catcode`\^^M=\HDorg@catcodeM\relax
      \newwrite\HD@savedest@out
      \RerunFileCheck{\jobname.hd}{%
        \immediate\closeout\HD@savedest@out
      }{}%
      \immediate\openout\HD@savedest@out=\jobname.hd\relax
    \else
      \let\HD@savedest@add\@gobble
    \fi
  }%
%    \end{macrocode}
%    \end{macro}
%    \begin{macro}{\HD@savedest@entry}
%    \begin{macrocode}
  \def\HD@savedest@entry#1^^M{%
    \@namedef{HD.#1}{}%
  }%
}
%    \end{macrocode}
%    \end{macro}
%    \begin{macro}{\HD@savedest@add}
%    \begin{macrocode}
\def\HD@savedest@add#1{%
  \immediate\write\HD@savedest@out{\string~#1}%
}
%    \end{macrocode}
%    \end{macro}
%    \begin{macro}{\ifHD@targetraise}
%    \begin{macrocode}
\newif\ifHD@targetraise
\HD@targetraisetrue
%    \end{macrocode}
%    \end{macro}
%    \begin{macro}{\HD@target}
%    \begin{macrocode}
\def\HD@target{%
  \ifHD@targetraise
    \ifhmode
      \settoheight{\HD@helplength}{[H]}%
      \addtolength{\HD@helplength}{0.4ex}%
      \raisebox{\HD@helplength}[0pt][0pt]{%
        \HD@maketarget
      }%
    \else
      \HD@maketarget
    \fi
  \else
    \HD@maketarget
  \fi
}
%    \end{macrocode}
%    \end{macro}
%    \begin{macro}{\HD@maketarget}
%    \begin{macrocode}
\def\HD@maketarget{%
  \stepcounter{HD@hypercount}%
  \def\HD@next{%
    \llap{\hypertarget{HD.\the\c@HD@hypercount}{}\,}%
  }%
  \ifHD@savedest
    \expandafter\ifx\csname HD.\the\c@HD@hypercount\endcsname\@empty
    \else
      \let\HD@next\relax
    \fi
  \fi
  \HD@next
}
%    \end{macrocode}
%    \end{macro}
%
%    \begin{macro}{\CodelineIndex}
%    \begin{macrocode}
\g@addto@macro\CodelineIndex{%
  \let\special@index\HD@codeline@wrindex
}
%    \end{macrocode}
%    \end{macro}
%    \begin{macro}{\PageIndex}
%    \begin{macrocode}
\g@addto@macro\PageIndex{%
  \let\special@index\HD@page@wrindex
}
%    \end{macrocode}
%    \end{macro}
%
%    \begin{macro}{\HD@codeline@wrindex}
%    \begin{macrocode}
\def\HD@codeline@wrindex#1{%
  \begingroup
    \let\HDorg@encapchar\encapchar
    \def\encapchar##1\encapchar##2\@nil{%
      \HDorg@encapchar
      hdclindex{\the\c@HD@hypercount}{##1}%
    }%
    \codeline@wrindex{%
      #1\encapchar\encapchar\@nil
    }%
  \endgroup
}
%    \end{macrocode}
%    \end{macro}
%    \begin{macro}{\HD@page@wrindex}
%    \begin{macrocode}
\def\HD@page@wrindex#1{%
  \begingroup
    \let\HDorg@encapchar\encapchar
    \def\encapchar##1\encapchar##2\@nil{%
      \HDorg@encapchar
      hdpindex{##1}%
    }%
    \HDorg@index{#1\encapchar\encapchar\@nil}%
  \endgroup
}
%    \end{macrocode}
%    \end{macro}
%
%    \begin{macro}{\hdclindex}
%    \begin{macrocode}
\def\hdclindex#1#2#3{%
  \csname\ifx\\#2\\relax\else#2\fi\endcsname{%
    \hyperlink{HD.#1}{#3}%
    \HD@savedest@add{#1}%
  }%
}
%    \end{macrocode}
%    \end{macro}
%    \begin{macro}{\hdpindex}
%    \begin{macrocode}
\def\hdpindex#1#2{%
  \csname\ifx\\#1\\relax\else#1\fi\endcsname{%
    \hyperpage{#2}%
  }%
}
%    \end{macrocode}
%    \end{macro}
%
% \subsection{\cs{PrintDescribeMacro}}
%
%    \begin{macro}{\HDorg@PrintDescribeMacro}
%    \begin{macrocode}
\let\HDorg@PrintDescribeMacro\PrintDescribeMacro
%    \end{macrocode}
%    \end{macro}
%    \begin{macro}{\PrintDescribeMacro}
%    \begin{macrocode}
\renewcommand\PrintDescribeMacro[1]{%
  \begingroup
    \let\HDorg@MacroFont\MacroFont
    \def\MacroFont{%
      \HD@target
      \HDorg@MacroFont
    }%
    \HDorg@PrintDescribeMacro{#1}%
  \endgroup
}
%    \end{macrocode}
%    \end{macro}
%
% \subsection{\cs{PrintMacroName}}
%
%    \begin{macro}{\HDorg@PrintMacroName}
%    \begin{macrocode}
\let\HDorg@PrintMacroName\PrintMacroName
%    \end{macrocode}
%    \end{macro}
%    \begin{macro}{\PrintMacroName}
%    \begin{macrocode}
\renewcommand\PrintMacroName[1]{%
  \begingroup
    \let\HDorg@MacroFont\MacroFont
    \def\MacroFont{%
      \HD@target
      \HDorg@MacroFont
    }%
    \HDorg@PrintMacroName{#1}%
  \endgroup
}
%    \end{macrocode}
%    \end{macro}
%
% \subsection{\cs{theCodelineNo}}
%
%    \begin{macro}{\HDorg@theCodelineNo}
%    \begin{macrocode}
\let\HDorg@theCodelineNo\theCodelineNo
%    \end{macrocode}
%    \end{macro}
%    \begin{macro}{\theCodelineNo}
%    \begin{macrocode}
\renewcommand*\theCodelineNo{%
  \settowidth{\HD@helplength}{\HDorg@theCodelineNo\ }%
  \ifdim\HD@helplength<\MacroIndent
    \addtolength{\HD@helplength}{-\MacroIndent}%
    \llap{%
      \begingroup\HD@target\endgroup
      \hspace*{-\HD@helplength}%
    }%
  \else
    \HD@target
  \fi
  \HDorg@theCodelineNo
}
%    \end{macrocode}
%    \end{macro}
%
%    \begin{macro}{\HDorg@index}
%    \begin{macrocode}
\AtBeginDocument{%
  \let\HDorg@index\index
}
%    \end{macrocode}
%    \end{macro}
%
% \subsection{\cs{SpecialUsageIndex}}
%
%    \begin{macro}{\HDorg@SpecialUsageIndex}
%    \begin{macrocode}
\let\HDorg@SpecialUsageIndex\SpecialUsageIndex
\renewcommand*\SpecialUsageIndex[1]{%
  \@bsphack
  \begingroup
    \HD@target
    \let\index\HDorg@index
    \let\HDorg@encapchar\encapchar
    \edef\encapchar usage{%
       \HDorg@encapchar hdclindex{\the\c@HD@hypercount}{usage}%
    }%
    \HDorg@SpecialUsageIndex{#1}%
  \endgroup
  \@esphack
}
%    \end{macrocode}
%    \end{macro}
%
% \subsection{\cs{autoindex}}
%
%    \begin{macrocode}
\begingroup\expandafter\expandafter\expandafter\endgroup
\expandafter\ifx\csname autoindex\endcsname\relax
\else
%    \end{macrocode}
%    \begin{macro}{\HDorg@autoindex}
%    \begin{macrocode}
  \let\HDorg@autoindex\autoindex
%    \end{macrocode}
%    \end{macro}
%    \begin{macro}{\autoindex}
%    \begin{macrocode}
  \renewcommand*\autoindex[1]{%
    \@bsphack
    \begingroup
      \HD@target
      \let\index\HDorg@index
      \let\HDorg@encapchar\encapchar
      \edef\encapchar usage{%
         \HDorg@encapchar hdclindex{\the\c@HD@hypercount}{usage}%
      }%
      \HDorg@autoindex{#1}%
    \endgroup
    \@esphack
  }%
\fi
%    \end{macrocode}
%    \end{macro}
%
% \subsection{\cs{SpecialEnvIndex}}
%
%    \begin{macro}{\HDorg@SpecialEnvIndex}
%    \begin{macrocode}
\let\HDorg@SpecialEnvIndex\SpecialEnvIndex
%    \end{macrocode}
%    \end{macro}
%    \begin{macro}{\SpecialEnvIndex}
%    \begin{macrocode}
\renewcommand*\SpecialEnvIndex[1]{%
  \@bsphack
  \begingroup
    \HD@target
    \let\HDorg@encapchar\encapchar
    \edef\encapchar usage{%
      \HDorg@encapchar hdclindex{\the\c@HD@hypercount}{usage}%
    }%
    \HDorg@SpecialEnvIndex{#1}%
  \endgroup
  \@esphack
}
%    \end{macrocode}
%    \end{macro}
%
% \subsection{\cs{SortIndex}}
%
%    \begin{macro}{\HDorg@SortIndex}
%    \begin{macrocode}
\let\HDorg@SortIndex\SortIndex
%    \end{macrocode}
%    \end{macro}
%    \begin{macro}{\SortIndex}
%    \begin{macrocode}
\renewcommand*{\SortIndex}[2]{%
  \@bsphack
  \begingroup
    \let\index\HD@page@wrindex
    \HDorg@SortIndex{#1}{#2}%
  \endgroup
  \@esphack
}
%    \end{macrocode}
%    \end{macro}
%
% \subsection{\cs{@wrglossary}}
%
%    \begin{macro}{\HDorg@wrglossary}
%    \begin{macrocode}
\let\HDorg@wrglossary\@wrglossary
%    \end{macrocode}
%    \end{macro}
%    \begin{macro}{\@wrglossary}
%    \begin{macrocode}
\def\@wrglossary#1{%
  \let\HDorg@encapchar\encapchar
  \def\encapchar##1\encapchar##2\@nil{%
    \HDorg@encapchar
    hdpindex{##1}%
  }%
  \HDorg@wrglossary{#1\encapchar\encapchar\@nil}%
}
%    \end{macrocode}
%    \end{macro}
%
% \subsection{\cs{MacroIndent}}
%
%    \cs{MacroIndent} automatically calculated.
%    \begin{macrocode}
\AfterLastShipout{%
  \if@filesw
    \immediate\write\@auxout{%
      \string\providecommand\string\HD@SetMacroIndent[1]{}%
    }%
    \immediate\write\@auxout{%
      \string\HD@SetMacroIndent{\number\value{CodelineNo}}%
    }%
  \fi
}
%    \end{macrocode}
%    \begin{macro}{\HD@SetMacroIndent}
%    \begin{macrocode}
\def\HD@SetMacroIndent#1{%
  \begingroup
    \value{CodelineNo}=#1\relax
    \settowidth{\MacroIndent}{\HDorg@theCodelineNo\ }%
    \global\MacroIndent\MacroIndent
  \endgroup
}
%    \end{macrocode}
%    \end{macro}
%
%    \begin{macrocode}
%</package>
%    \end{macrocode}
%
% \section{Installation}
%
% \subsection{Download}
%
% \paragraph{Package.} This package is available on
% CTAN\footnote{\url{http://ctan.org/pkg/hypdoc}}:
% \begin{description}
% \item[\CTAN{macros/latex/contrib/oberdiek/hypdoc.dtx}] The source file.
% \item[\CTAN{macros/latex/contrib/oberdiek/hypdoc.pdf}] Documentation.
% \end{description}
%
%
% \paragraph{Bundle.} All the packages of the bundle `oberdiek'
% are also available in a TDS compliant ZIP archive. There
% the packages are already unpacked and the documentation files
% are generated. The files and directories obey the TDS standard.
% \begin{description}
% \item[\CTAN{install/macros/latex/contrib/oberdiek.tds.zip}]
% \end{description}
% \emph{TDS} refers to the standard ``A Directory Structure
% for \TeX\ Files'' (\CTAN{tds/tds.pdf}). Directories
% with \xfile{texmf} in their name are usually organized this way.
%
% \subsection{Bundle installation}
%
% \paragraph{Unpacking.} Unpack the \xfile{oberdiek.tds.zip} in the
% TDS tree (also known as \xfile{texmf} tree) of your choice.
% Example (linux):
% \begin{quote}
%   |unzip oberdiek.tds.zip -d ~/texmf|
% \end{quote}
%
% \paragraph{Script installation.}
% Check the directory \xfile{TDS:scripts/oberdiek/} for
% scripts that need further installation steps.
% Package \xpackage{attachfile2} comes with the Perl script
% \xfile{pdfatfi.pl} that should be installed in such a way
% that it can be called as \texttt{pdfatfi}.
% Example (linux):
% \begin{quote}
%   |chmod +x scripts/oberdiek/pdfatfi.pl|\\
%   |cp scripts/oberdiek/pdfatfi.pl /usr/local/bin/|
% \end{quote}
%
% \subsection{Package installation}
%
% \paragraph{Unpacking.} The \xfile{.dtx} file is a self-extracting
% \docstrip\ archive. The files are extracted by running the
% \xfile{.dtx} through \plainTeX:
% \begin{quote}
%   \verb|tex hypdoc.dtx|
% \end{quote}
%
% \paragraph{TDS.} Now the different files must be moved into
% the different directories in your installation TDS tree
% (also known as \xfile{texmf} tree):
% \begin{quote}
% \def\t{^^A
% \begin{tabular}{@{}>{\ttfamily}l@{ $\rightarrow$ }>{\ttfamily}l@{}}
%   hypdoc.sty & tex/latex/oberdiek/hypdoc.sty\\
%   hypdoc.pdf & doc/latex/oberdiek/hypdoc.pdf\\
%   hypdoc.dtx & source/latex/oberdiek/hypdoc.dtx\\
% \end{tabular}^^A
% }^^A
% \sbox0{\t}^^A
% \ifdim\wd0>\linewidth
%   \begingroup
%     \advance\linewidth by\leftmargin
%     \advance\linewidth by\rightmargin
%   \edef\x{\endgroup
%     \def\noexpand\lw{\the\linewidth}^^A
%   }\x
%   \def\lwbox{^^A
%     \leavevmode
%     \hbox to \linewidth{^^A
%       \kern-\leftmargin\relax
%       \hss
%       \usebox0
%       \hss
%       \kern-\rightmargin\relax
%     }^^A
%   }^^A
%   \ifdim\wd0>\lw
%     \sbox0{\small\t}^^A
%     \ifdim\wd0>\linewidth
%       \ifdim\wd0>\lw
%         \sbox0{\footnotesize\t}^^A
%         \ifdim\wd0>\linewidth
%           \ifdim\wd0>\lw
%             \sbox0{\scriptsize\t}^^A
%             \ifdim\wd0>\linewidth
%               \ifdim\wd0>\lw
%                 \sbox0{\tiny\t}^^A
%                 \ifdim\wd0>\linewidth
%                   \lwbox
%                 \else
%                   \usebox0
%                 \fi
%               \else
%                 \lwbox
%               \fi
%             \else
%               \usebox0
%             \fi
%           \else
%             \lwbox
%           \fi
%         \else
%           \usebox0
%         \fi
%       \else
%         \lwbox
%       \fi
%     \else
%       \usebox0
%     \fi
%   \else
%     \lwbox
%   \fi
% \else
%   \usebox0
% \fi
% \end{quote}
% If you have a \xfile{docstrip.cfg} that configures and enables \docstrip's
% TDS installing feature, then some files can already be in the right
% place, see the documentation of \docstrip.
%
% \subsection{Refresh file name databases}
%
% If your \TeX~distribution
% (\teTeX, \mikTeX, \dots) relies on file name databases, you must refresh
% these. For example, \teTeX\ users run \verb|texhash| or
% \verb|mktexlsr|.
%
% \subsection{Some details for the interested}
%
% \paragraph{Attached source.}
%
% The PDF documentation on CTAN also includes the
% \xfile{.dtx} source file. It can be extracted by
% AcrobatReader 6 or higher. Another option is \textsf{pdftk},
% e.g. unpack the file into the current directory:
% \begin{quote}
%   \verb|pdftk hypdoc.pdf unpack_files output .|
% \end{quote}
%
% \paragraph{Unpacking with \LaTeX.}
% The \xfile{.dtx} chooses its action depending on the format:
% \begin{description}
% \item[\plainTeX:] Run \docstrip\ and extract the files.
% \item[\LaTeX:] Generate the documentation.
% \end{description}
% If you insist on using \LaTeX\ for \docstrip\ (really,
% \docstrip\ does not need \LaTeX), then inform the autodetect routine
% about your intention:
% \begin{quote}
%   \verb|latex \let\install=y% \iffalse meta-comment
%
% File: hypdoc.dtx
% Version: 2018/09/07 v1.14
% Info: Hyper extensions for doc.sty
%
% Copyright (C) 2002, 2006, 2007, 2009-2011 by
%    Heiko Oberdiek <heiko.oberdiek at googlemail.com>
%    2016
%    https://github.com/ho-tex/oberdiek/issues
%
% This work may be distributed and/or modified under the
% conditions of the LaTeX Project Public License, either
% version 1.3c of this license or (at your option) any later
% version. This version of this license is in
%    http://www.latex-project.org/lppl/lppl-1-3c.txt
% and the latest version of this license is in
%    http://www.latex-project.org/lppl.txt
% and version 1.3 or later is part of all distributions of
% LaTeX version 2005/12/01 or later.
%
% This work has the LPPL maintenance status "maintained".
%
% This Current Maintainer of this work is Heiko Oberdiek.
%
% This work consists of the main source file hypdoc.dtx
% and the derived files
%    hypdoc.sty, hypdoc.pdf, hypdoc.ins, hypdoc.drv.
%
% Distribution:
%    CTAN:macros/latex/contrib/oberdiek/hypdoc.dtx
%    CTAN:macros/latex/contrib/oberdiek/hypdoc.pdf
%
% Unpacking:
%    (a) If hypdoc.ins is present:
%           tex hypdoc.ins
%    (b) Without hypdoc.ins:
%           tex hypdoc.dtx
%    (c) If you insist on using LaTeX
%           latex \let\install=y% \iffalse meta-comment
%
% File: hypdoc.dtx
% Version: 2018/09/07 v1.14
% Info: Hyper extensions for doc.sty
%
% Copyright (C) 2002, 2006, 2007, 2009-2011 by
%    Heiko Oberdiek <heiko.oberdiek at googlemail.com>
%    2016
%    https://github.com/ho-tex/oberdiek/issues
%
% This work may be distributed and/or modified under the
% conditions of the LaTeX Project Public License, either
% version 1.3c of this license or (at your option) any later
% version. This version of this license is in
%    http://www.latex-project.org/lppl/lppl-1-3c.txt
% and the latest version of this license is in
%    http://www.latex-project.org/lppl.txt
% and version 1.3 or later is part of all distributions of
% LaTeX version 2005/12/01 or later.
%
% This work has the LPPL maintenance status "maintained".
%
% This Current Maintainer of this work is Heiko Oberdiek.
%
% This work consists of the main source file hypdoc.dtx
% and the derived files
%    hypdoc.sty, hypdoc.pdf, hypdoc.ins, hypdoc.drv.
%
% Distribution:
%    CTAN:macros/latex/contrib/oberdiek/hypdoc.dtx
%    CTAN:macros/latex/contrib/oberdiek/hypdoc.pdf
%
% Unpacking:
%    (a) If hypdoc.ins is present:
%           tex hypdoc.ins
%    (b) Without hypdoc.ins:
%           tex hypdoc.dtx
%    (c) If you insist on using LaTeX
%           latex \let\install=y% \iffalse meta-comment
%
% File: hypdoc.dtx
% Version: 2018/09/07 v1.14
% Info: Hyper extensions for doc.sty
%
% Copyright (C) 2002, 2006, 2007, 2009-2011 by
%    Heiko Oberdiek <heiko.oberdiek at googlemail.com>
%    2016
%    https://github.com/ho-tex/oberdiek/issues
%
% This work may be distributed and/or modified under the
% conditions of the LaTeX Project Public License, either
% version 1.3c of this license or (at your option) any later
% version. This version of this license is in
%    http://www.latex-project.org/lppl/lppl-1-3c.txt
% and the latest version of this license is in
%    http://www.latex-project.org/lppl.txt
% and version 1.3 or later is part of all distributions of
% LaTeX version 2005/12/01 or later.
%
% This work has the LPPL maintenance status "maintained".
%
% This Current Maintainer of this work is Heiko Oberdiek.
%
% This work consists of the main source file hypdoc.dtx
% and the derived files
%    hypdoc.sty, hypdoc.pdf, hypdoc.ins, hypdoc.drv.
%
% Distribution:
%    CTAN:macros/latex/contrib/oberdiek/hypdoc.dtx
%    CTAN:macros/latex/contrib/oberdiek/hypdoc.pdf
%
% Unpacking:
%    (a) If hypdoc.ins is present:
%           tex hypdoc.ins
%    (b) Without hypdoc.ins:
%           tex hypdoc.dtx
%    (c) If you insist on using LaTeX
%           latex \let\install=y\input{hypdoc.dtx}
%        (quote the arguments according to the demands of your shell)
%
% Documentation:
%    (a) If hypdoc.drv is present:
%           latex hypdoc.drv
%    (b) Without hypdoc.drv:
%           latex hypdoc.dtx; ...
%    The class ltxdoc loads the configuration file ltxdoc.cfg
%    if available. Here you can specify further options, e.g.
%    use A4 as paper format:
%       \PassOptionsToClass{a4paper}{article}
%
%    Programm calls to get the documentation (example):
%       pdflatex hypdoc.dtx
%       makeindex -s gind.ist hypdoc.idx
%       pdflatex hypdoc.dtx
%       makeindex -s gind.ist hypdoc.idx
%       pdflatex hypdoc.dtx
%
% Installation:
%    TDS:tex/latex/oberdiek/hypdoc.sty
%    TDS:doc/latex/oberdiek/hypdoc.pdf
%    TDS:source/latex/oberdiek/hypdoc.dtx
%
%<*ignore>
\begingroup
  \catcode123=1 %
  \catcode125=2 %
  \def\x{LaTeX2e}%
\expandafter\endgroup
\ifcase 0\ifx\install y1\fi\expandafter
         \ifx\csname processbatchFile\endcsname\relax\else1\fi
         \ifx\fmtname\x\else 1\fi\relax
\else\csname fi\endcsname
%</ignore>
%<*install>
\input docstrip.tex
\Msg{************************************************************************}
\Msg{* Installation}
\Msg{* Package: hypdoc 2018/09/07 v1.14 Hyper extensions for doc.sty (HO)}
\Msg{************************************************************************}

\keepsilent
\askforoverwritefalse

\let\MetaPrefix\relax
\preamble

This is a generated file.

Project: hypdoc
Version: 2018/09/07 v1.14

Copyright (C) 2002, 2006, 2007, 2009-2011 by
   Heiko Oberdiek <heiko.oberdiek at googlemail.com>

This work may be distributed and/or modified under the
conditions of the LaTeX Project Public License, either
version 1.3c of this license or (at your option) any later
version. This version of this license is in
   http://www.latex-project.org/lppl/lppl-1-3c.txt
and the latest version of this license is in
   http://www.latex-project.org/lppl.txt
and version 1.3 or later is part of all distributions of
LaTeX version 2005/12/01 or later.

This work has the LPPL maintenance status "maintained".

This Current Maintainer of this work is Heiko Oberdiek.

This work consists of the main source file hypdoc.dtx
and the derived files
   hypdoc.sty, hypdoc.pdf, hypdoc.ins, hypdoc.drv.

\endpreamble
\let\MetaPrefix\DoubleperCent

\generate{%
  \file{hypdoc.ins}{\from{hypdoc.dtx}{install}}%
  \file{hypdoc.drv}{\from{hypdoc.dtx}{driver}}%
  \usedir{tex/latex/oberdiek}%
  \file{hypdoc.sty}{\from{hypdoc.dtx}{package}}%
  \nopreamble
  \nopostamble
%  \usedir{source/latex/oberdiek/catalogue}%
%  \file{hypdoc.xml}{\from{hypdoc.dtx}{catalogue}}%
}

\catcode32=13\relax% active space
\let =\space%
\Msg{************************************************************************}
\Msg{*}
\Msg{* To finish the installation you have to move the following}
\Msg{* file into a directory searched by TeX:}
\Msg{*}
\Msg{*     hypdoc.sty}
\Msg{*}
\Msg{* To produce the documentation run the file `hypdoc.drv'}
\Msg{* through LaTeX.}
\Msg{*}
\Msg{* Happy TeXing!}
\Msg{*}
\Msg{************************************************************************}

\endbatchfile
%</install>
%<*ignore>
\fi
%</ignore>
%<*driver>
\NeedsTeXFormat{LaTeX2e}
\ProvidesFile{hypdoc.drv}%
  [2018/09/07 v1.14 Hyper extensions for doc.sty (HO)]%
\documentclass{ltxdoc}
\usepackage{holtxdoc}[2011/11/22]
\begin{document}
  \DocInput{hypdoc.dtx}%
\end{document}
%</driver>
% \fi
%
%
% \CharacterTable
%  {Upper-case    \A\B\C\D\E\F\G\H\I\J\K\L\M\N\O\P\Q\R\S\T\U\V\W\X\Y\Z
%   Lower-case    \a\b\c\d\e\f\g\h\i\j\k\l\m\n\o\p\q\r\s\t\u\v\w\x\y\z
%   Digits        \0\1\2\3\4\5\6\7\8\9
%   Exclamation   \!     Double quote  \"     Hash (number) \#
%   Dollar        \$     Percent       \%     Ampersand     \&
%   Acute accent  \'     Left paren    \(     Right paren   \)
%   Asterisk      \*     Plus          \+     Comma         \,
%   Minus         \-     Point         \.     Solidus       \/
%   Colon         \:     Semicolon     \;     Less than     \<
%   Equals        \=     Greater than  \>     Question mark \?
%   Commercial at \@     Left bracket  \[     Backslash     \\
%   Right bracket \]     Circumflex    \^     Underscore    \_
%   Grave accent  \`     Left brace    \{     Vertical bar  \|
%   Right brace   \}     Tilde         \~}
%
% \GetFileInfo{hypdoc.drv}
%
% \title{The \xpackage{hypdoc} package}
% \date{2018/09/07 v1.14}
% \author{Heiko Oberdiek\thanks
% {Please report any issues at https://github.com/ho-tex/oberdiek/issues}\\
% \xemail{heiko.oberdiek at googlemail.com}}
%
% \maketitle
%
% \begin{abstract}
% This package adds hyper features to the package
% \xpackage{doc} that is used in the documentation
% system of \LaTeXe. Bookmarks are added and references
% are linked as far as possible.
% \end{abstract}
%
% \tableofcontents
%
% \section{Features}
%
% The package adds the following features:
% \begin{description}
% \item[Bookmarks:]
%   For supporting bookmarks package \verb|hyperref| is loaded.
%   Some \TeX\ code inside section titles can cause problems
%   during bookmark processing; these can be warnings, ugly
%   bookmark titles or even errors. With the help of
%   \verb|\pdfstringdefDisableCommands| or \verb|\texorpdfstring|
%   you can manage these things, see \verb|hyperref|'s package
%   documentation.
% \item[Additional bookmarks:]
%   Unnumbered bookmark entries are generated for:
%   \begin{itemize}
%   \item Table of contents
%   \item List of figures
%   \item List of tables
%   \item Index
%   \item Glossary
%   \end{itemize}
%   Option \texttt{numbered} controls, whether these section or
%   chapters should be numbered. The exception is the table
%   of contents if this it would be the first numbered
%   section. Then just a bookmark is added.
% \item[Metadata:]
%   The loaded package \verb|hyperref|, enforced by
%   option \verb|pdfusetitle|, tries to detect \verb|\title|
%   and \verb|\author| and sets the corresponding PDF
%   information fields. In case of problems see the previous
%   item about bookmarks.
%   These enries can be overwritten or new ones added by
%   \verb|\hypersetup| after the packages is loaded, see
%   documentation of package \verb|hyperref|.
% \item[References, links:] Because of package \verb|\hyperref|
%   |\ref| become links, also \verb|\url|, footnote referencing
%   are supported.
% \item[Index:]
%   The main part of this package deals with index links in order
%   to support the kind of index that package \verb|doc| provides.
% \item[Thumbnails:]
%   Package \verb|thumbpdf| may additionally be loaded for the case that the
%   PDF file also should contain thumbnails, see the documentation
%   of this package. But thumbnail embedding is not really
%   necessary, because nowadays AcrobatReader is able to
%   generate and view thumbnails on the fly.
% \end{description}
%
% \section{Usage}
%
% The purpose of this packages is to add PDF features to
% the documentation of LaTeX packages. Thus just load
% the package in the driver file that generates the
% documentation:
% \begin{quote}
%   \verb|\usepackage{hypdoc}|
% \end{quote}
% Package \verb|doc| is loaded by \verb|hypdoc|, thus you
% also can replace a \verb|\usepackage{doc}| with this line.
% At any case, the package must be loaded before macros of
% package \verb|doc| such as \verb|\CodelineIndex| or
% \verb|\PageIndex| are used.
%
% If the class \verb|ltxdoc| is used, you can also use
% the configuration file for this class to add
% package \verb|hypdoc|. Add the following line
% to the file \verb|ltxdoc.cfg| (or create a new one):
% \begin{quote}
%\begin{verbatim}
%\AtEndOfClass{\RequirePackage{hypdoc}}
%\end{verbatim}
% \end{quote}
%
% \subsection{Options}
%
% There is one option \texttt{numbered}. It controls,
% whether the starred versions of section or chapter
% inside \cs{tableofcontents}, \cs{listoffigures}, \dots
% should be replaced by the non-star versions.
%
% \subsection{Simple example}
% Example of a simple driver file \verb|foobar.drv|
% of a package \verb|foobar|. Often the driver file
% is packed in the file \verb|foobar.dtx| and can
% be extracted by docstrip or if an instruction file
% for docstrip is provided by:
% \begin{quote}
% \verb|tex foobar.ins|
% \end{quote}
% Note that although docstrip doesn't rely on \LaTeX,
% sometimes you need to run the \verb|.ins| file through
% \LaTeX. Oddly enough, often the only \LaTeX\ feature
% in use is \verb|\NeedsTeXFormat{LaTeX2e}| in the first line.
% \begin{quote}
%\begin{verbatim}
%\documentclass{ltxdoc}
%\usepackage{hypdoc}
%\RecordChanges
%\EnableCrossrefs
%\CodelineIndex
%\begin{document}
%\title{Title for package \texttt{foobar}}
%\author{Author of package \texttt{foobar}}
%\date{...}
%\maketitle
%\tableofcontents
%\DocInput{foobar.dtx}
%\PrintChanges
%\PrintIndex
%\end{document}
%\end{verbatim}
% \end{quote}
%
% Then compile it, for example with pdf\LaTeX.
% \begin{quote}
%\begin{verbatim}
%pdflatex foobar.drv
%makeindex -s gind.ist foobar.idx
%makeindex -s gglo.ist -o foobar.gls foobar.glo
%pdflatex foobar.drv
%\end{verbatim}
% \end{quote}
% You will need several cycles until the cross references
% and similar things are correct.
%
% \subsection{Example \texttt{source2e.tex}}
%
% The source code of the base of \LaTeX\ is available as
% a bunch of \verb|.dtx| files. \LaTeX\ provides the file
% \verb|source2e.tex| to merge them all together.
%
% Now you can either load package \verb|hypdoc| in a private copy
% of this file (please, respect the LPPL) or use the configuration
% file \verb|ltxdoc.cfg|. Example that also sets A4 paper size:
% \begin{quote}
%\begin{verbatim}
%\PassOptionsToClass{a4paper}{article}
%\AtEndOfClass{\RequirePackage{hypdoc}}
%\end{verbatim}
% \end{quote}
%
% Then \verb|source2e.tex| can be compiled by pdf\TeX, for example:
% \begin{quote}
%\begin{verbatim}
%pdflatex source2e
%makeindex -s gind.ist source2e.idx
%makeindex -s gglo.ist -o source2e.glo source2e.gls
%pdflatex source2e
%makeindex -s gind.ist source2e.idx
%makeindex -s gglo.ist -o source2e.glo source2e.gls
%pdflatex source2e
%\end{verbatim}
% \end{quote}
%
% \StopEventually{
% }
%
% \section{Implementation}
%
%    \begin{macrocode}
%<*package>
\NeedsTeXFormat{LaTeX2e}
\ProvidesPackage{hypdoc}%
  [2018/09/07 v1.14 Hyper extensions for doc.sty (HO)]%
%    \end{macrocode}
%    The package identification is done at the top of the |.dtx| file
%    in order to use only one identification string.
%
%    For unique command names this package uses |HD@| as prefix
%
% \subsection{Options}
%
%    \begin{macro}{\ifHD@numbered}
%    \begin{macrocode}
\newif\ifHD@numbered
%    \end{macrocode}
%    \end{macro}
%    \begin{macrocode}
\DeclareOption{numbered}{\HD@numberedtrue}
\ProcessOptions*\relax
%    \end{macrocode}
%
% \subsection{Package loading}
%
%    \begin{macrocode}
\RequirePackage{atveryend}[2010/03/24]
\RequirePackage{doc}
%    \end{macrocode}
%    \begin{macrocode}
\ifx\numexpr\@undefined
  \RequirePackage{calc}%
\fi
%    \end{macrocode}
%    \begin{macrocode}
\RequirePackage[%
  hyperindex=false,%
  pdfusetitle,%
  colorlinks,%
  pdfpagelabels%
]{hyperref}[2002/05/09]%
\hypersetup{%
  plainpages=false,%
  bookmarksopen,%
  bookmarksnumbered,%
  bookmarksopenlevel=0,%
  pdfstartview={FitBH \hypercalcbp{%
    \paperheight-\topmargin-1in-\headheight-\headsep
  }},%
}
%    \end{macrocode}
%
%    \begin{macrocode}
\newcounter{HD@unique}
%    \end{macrocode}
%    \begin{macro}{\theHD@unique}
%    \begin{macrocode}
\renewcommand{\theHD@unique}{%
  \number\value{HD@unique}%
}
%    \end{macrocode}
%    \end{macro}
%
%    \begin{macrocode}
\g@addto@macro\Hy@UseMaketitleInfos{%
  \@ifundefined{fileinfo}{}{%
    \ifx\@pdfsubject\@empty
      \hypersetup{pdfsubject={\fileinfo}}%
    \fi
  }%
}
%    \end{macrocode}
%
%    \begin{macrocode}
\pdfstringdefDisableCommands{%
  \let\thanks\@gobble
  \let\footnote\@gobble
  \def\cs#1{\textbackslash #1}%
  \let\normalfont\@empty
  \let\scshape\@empty
  \def\and{and }%
}
%    \end{macrocode}
%
% \subsubsection{Patch for bug latex/4096}
%
%    \begin{macrocode}
\begingroup\expandafter\expandafter\expandafter\endgroup
\expandafter\ifx\csname mod@math@codes\endcsname\relax
\else
  \def\HD@ModFix#1{%
    \mathcode`#1=`#1\relax
  }%
  \g@addto@macro\mod@math@codes{%
    \HD@ModFix\-%
    \HD@ModFix\+%
    \HD@ModFix\:%
    \HD@ModFix\=%
    \HD@ModFix\*%
  }%
\fi
%    \end{macrocode}
%
% \subsection{Bookmarks}
%
%    \begin{macro}{\HD@sectionpatch}
%    \begin{macrocode}
\def\HD@sectionpatch{%
  \ifHD@numbered
    \HD@@sectionpatch{section}%
    \HD@@sectionpatch{chapter}%
  \else
    \let\Hy@writebookmark\HD@disable@writebookmark
  \fi
}
%    \end{macrocode}
%    \end{macro}
%    \begin{macro}{\HD@@sectionpatch}
%    \begin{macrocode}
\def\HD@@sectionpatch#1{%
  \expandafter\let\csname HDorg@@#1\expandafter\endcsname
    \csname #1\endcsname
  \@namedef{#1}{%
    \@ifstar{\@nameuse{HDorg@@#1}}{\@nameuse{HDorg@@#1}}%
  }%
}
%    \end{macrocode}
%    \end{macro}
%    \begin{macro}{\HD@disable@writebookmark}
%    \begin{macrocode}
\def\HD@disable@writebookmark#1#2#3#4#5{}
%    \end{macrocode}
%    \end{macro}
%
%    \begin{macro}{\HD@guesstoclevel}
%    A wild guessing of the toclevel. There is the hope, that
%    the macro |#1| starts with a sectioning command.
%    \begin{macrocode}
\def\HD@guesstoclevel#1{%
  \expandafter\HD@@guesstoclevel#1\section\HD@end
}
%    \end{macrocode}
%    \end{macro}
%    \begin{macro}{\HD@@guesstoclevel}
%    \begin{macrocode}
\def\HD@@guesstoclevel#1#2\HD@end{%
  \@ifundefined{toclevel@\expandafter\@gobble\string#1\@empty}{%
    1%
  }{%
    \@nameuse{toclevel@\expandafter\@gobble\string#1\@empty}%
  }%
}
%    \end{macrocode}
%    \end{macro}
%
% \subsection{\cs{tableofcontents}}
%
%    \begin{macro}{\HDorg@tableofcontents}
%    \begin{macrocode}
\let\HDorg@tableofcontents\tableofcontents
%    \end{macrocode}
%    \end{macro}
%    \begin{macro}{\tableofcontents}
%    As first section the table of contents shouldn't
%    go into the toc itself. Also a number looks too funny then.
%    Just a bookmark is enough in this case.
%    \begin{macrocode}
\def\tableofcontents{%
  \begingroup
    \@ifundefined{c@section}{%
      \def\c@section{0}%
    }{}%
  \expandafter\endgroup
  \ifcase\c@section
    \stepcounter{HD@unique}%
    \pdfbookmark[\HD@guesstoclevel{\HDorg@tableofcontents}]%
        {\contentsname}{toc\theHD@unique}%
    \HDorg@tableofcontents
  \else
    \ifHD@numbered
    \else
      \stepcounter{HD@unique}%
      \pdfbookmark[\HD@guesstoclevel{\HDorg@tableofcontents}]%
          {\contentsname}{toc\theHD@unique}%
    \fi
    \begingroup
      \HD@sectionpatch
      \HDorg@tableofcontents
    \endgroup
  \fi
}
%    \end{macrocode}
%    \end{macro}
%
% \subsection{\cs{listoffigures}}
%
%    \begin{macro}{\HDorg@listoffigures}
%    \begin{macrocode}
\let\HDorg@listoffigures\listoffigures
%    \end{macrocode}
%    \end{macro}
%    \begin{macro}{\listoffigures}
%    \begin{macrocode}
\def\listoffigures{%
  \ifHD@numbered
  \else
    \stepcounter{HD@unique}%
    \pdfbookmark[\HD@guesstoclevel{\HDorg@listoffigures}]%
        {\listfigurename}{toc\theHD@unique}%
  \fi
  \begingroup
    \HD@sectionpatch
    \HDorg@listoffigures
  \endgroup
}
%    \end{macrocode}
%    \end{macro}
%
% \subsection{\cs{listoftables}}
%
%    \begin{macro}{\HDorg@listoftables}
%    \begin{macrocode}
\let\HDorg@listoftables\listoftables
%    \end{macrocode}
%    \end{macro}
%    \begin{macro}{\listoftables}
%    \begin{macrocode}
\def\listoftables{%
  \ifHD@numbered
  \else
    \stepcounter{HD@unique}%
    \pdfbookmark[\HD@guesstoclevel{\HDorg@listoftables}]%
        {\listtablename}{toc\theHD@unique}%
  \fi
  \begingroup
    \HD@sectionpatch
    \HDorg@listoftables
  \endgroup
}
%    \end{macrocode}
%    \end{macro}
%
% \subsection{\cs{thebibliography}}
%
%    \begin{macro}{\HDorg@thebibliography}
%    \begin{macrocode}
\let\HDorg@thebibliography\thebibliography
%    \end{macrocode}
%    \end{macro}
%    \begin{macro}{\thebibliography}
%    \begin{macrocode}
\def\thebibliography{%
  \ifHD@numbered
  \else
    \stepcounter{HD@unique}%
    \pdfbookmark[\HD@guesstoclevel{\HDorg@thebibliography}]%
      {\@ifundefined{chapter}{\refname}{\bibname}}{bib\theHD@unique}%
  \fi
  \HD@sectionpatch
  \HDorg@thebibliography
}
%    \end{macrocode}
%    \end{macro}
%
% \subsection{\cs{theindex}}
%
%    \begin{macro}{\HDorg@theindex}
%    \begin{macrocode}
\let\HDorg@theindex\theindex
%    \end{macrocode}
%    \end{macro}
%    \begin{macro}{\theindex}
%    \begin{macrocode}
\def\theindex{%
  \let\HDorg@index@prologue\index@prologue
  \let\HDorg@writebookmark\Hy@writebookmark
  \HD@sectionpatch
  \g@addto@macro\IndexParms{%
    \ifHD@numbered
    \else
      \let\Hy@writebookmark\HDorg@writebookmark
    \fi
    \let\bfseries\HD@bfseries
  }%
  \edef\HD@toclevel@index{\HD@guesstoclevel{\index@prologue}}%
  \begingroup
    \count@=\HD@toclevel@index\relax
    \advance\count@ by 1 %
    \xdef\HD@toclevel@subindex{\the\count@}%
  \endgroup
  \def\index@prologue{%
    \ifHD@numbered
    \else
      \begingroup
        \stepcounter{HD@unique}%
        \let\Hy@writebookmark\HDorg@writebookmark
        \pdfbookmark[\HD@toclevel@index]%
            {\indexname}{index\theHD@unique}%
      \endgroup
    \fi
    \HDorg@index@prologue
  }%
  \HDorg@theindex
}
%    \end{macrocode}
%    \end{macro}
%    \begin{macro}{\HDorg@bfseries}
%    \begin{macrocode}
\let\HDorg@bfseries\bfseries
%    \end{macrocode}
%    \end{macro}
%    \begin{macro}{\HD@bfseries}
%    \begin{macrocode}
\def\HD@bfseries{\futurelet\HD@hfil\HD@@bfseries}%
%    \end{macrocode}
%    \end{macro}
%    \begin{macro}{\HD@@bfseries}
%    \begin{macrocode}
\def\HD@@bfseries{%
  \HDorg@bfseries
  \ifx\HD@hfil\hfil
    \expandafter\HD@@@bfseries
  \fi
}
%    \end{macrocode}
%    \end{macro}
%    \begin{macro}{\HD@@@bfseries}
%    \begin{macrocode}
\def\HD@@@bfseries\hfil#1\hfil{%
  \ifx\\#1\\%
  \else
    \raisebox{\baselineskip}[0pt]{%
      \kern-\HD@margin\relax
      \pdfbookmark[\HD@toclevel@subindex]{#1}{HD.#1}%
      \kern\HD@margin\relax
    }%
  \fi
  \hfil#1\hfil
}%
%    \end{macrocode}
%    \end{macro}
%    \begin{macro}{\HD@margin}
%    \begin{macrocode}
\def\HD@margin{1mm}
%    \end{macrocode}
%    \end{macro}
%
% \subsection{\cs{theglossary}}
%
%    \begin{macro}{\HDorg@theglossary}
%    \begin{macrocode}
\let\HDorg@theglossary\theglossary
%    \end{macrocode}
%    \end{macro}
%    \begin{macro}{\theglossary}
%    \begin{macrocode}
\def\theglossary{%
  \let\HDorg@glossary@prologue\glossary@prologue
  \let\HDorg@writebookmark\Hy@writebookmark
  \HD@sectionpatch
  \def\glossary@prologue{%
    \ifHD@numbered
    \else
      \begingroup
        \let\Hy@writebookmark\HDorg@writebookmark
        \stepcounter{HD@unique}%
        \pdfbookmark[\HD@guesstoclevel{\HDorg@glossary@prologue}]%
            {\changehistoryname}{glossary\theHD@unique}%
      \endgroup
    \fi
    \HDorg@glossary@prologue
  }%
  \HDorg@theglossary
}
%    \end{macrocode}
%    \end{macro}
%    \begin{macro}{\glossaryname}
%    \begin{macrocode}
\providecommand*{\glossaryname}{Change History}
%    \end{macrocode}
%    \end{macro}
%
%    \begin{macro}{\glossaryname}
%    \begin{macrocode}
\providecommand*{\changehistoryname}{\glossaryname}
%    \end{macrocode}
%    \end{macro}
%
% \subsection{Index}
%
%    \begin{macrocode}
\newcounter{HD@hypercount}
%    \end{macrocode}
%    \begin{macro}{\HD@helplength}
%    \begin{macrocode}
\newlength{\HD@helplength}
%    \end{macrocode}
%    \end{macro}
%    \begin{macro}{\ifHD@savedest}
%    \begin{macrocode}
\newif\ifHD@savedest
\HD@savedesttrue
%    \end{macrocode}
%    \end{macro}
%    \begin{macrocode}
\AtBeginDocument{%
  \HD@savedest@init
}
%    \end{macrocode}
%    \begin{macrocode}
\RequirePackage{rerunfilecheck}[2009/12/10]
%    \end{macrocode}
%    \begin{macro}{\HD@savedest@init}
%    \begin{macrocode}
\begingroup
  \catcode`\^^M=\active
\@firstofone{\endgroup
  \def\HD@savedest@init{%
    \ifHD@savedest
      \let\HDorg@tilde~%
      \let~\HD@savedest@entry
      \edef\HDorg@catcodeM{\the\catcode`\^^M}%
      \catcode`\^^M=\active
      \def^^M{}%
      \InputIfFileExists{\jobname.hd}{}{}%
      \let~\HDorg@tilde
      \catcode`\^^M=\HDorg@catcodeM\relax
      \newwrite\HD@savedest@out
      \RerunFileCheck{\jobname.hd}{%
        \immediate\closeout\HD@savedest@out
      }{}%
      \immediate\openout\HD@savedest@out=\jobname.hd\relax
    \else
      \let\HD@savedest@add\@gobble
    \fi
  }%
%    \end{macrocode}
%    \end{macro}
%    \begin{macro}{\HD@savedest@entry}
%    \begin{macrocode}
  \def\HD@savedest@entry#1^^M{%
    \@namedef{HD.#1}{}%
  }%
}
%    \end{macrocode}
%    \end{macro}
%    \begin{macro}{\HD@savedest@add}
%    \begin{macrocode}
\def\HD@savedest@add#1{%
  \immediate\write\HD@savedest@out{\string~#1}%
}
%    \end{macrocode}
%    \end{macro}
%    \begin{macro}{\ifHD@targetraise}
%    \begin{macrocode}
\newif\ifHD@targetraise
\HD@targetraisetrue
%    \end{macrocode}
%    \end{macro}
%    \begin{macro}{\HD@target}
%    \begin{macrocode}
\def\HD@target{%
  \ifHD@targetraise
    \ifhmode
      \settoheight{\HD@helplength}{[H]}%
      \addtolength{\HD@helplength}{0.4ex}%
      \raisebox{\HD@helplength}[0pt][0pt]{%
        \HD@maketarget
      }%
    \else
      \HD@maketarget
    \fi
  \else
    \HD@maketarget
  \fi
}
%    \end{macrocode}
%    \end{macro}
%    \begin{macro}{\HD@maketarget}
%    \begin{macrocode}
\def\HD@maketarget{%
  \stepcounter{HD@hypercount}%
  \def\HD@next{%
    \llap{\hypertarget{HD.\the\c@HD@hypercount}{}\,}%
  }%
  \ifHD@savedest
    \expandafter\ifx\csname HD.\the\c@HD@hypercount\endcsname\@empty
    \else
      \let\HD@next\relax
    \fi
  \fi
  \HD@next
}
%    \end{macrocode}
%    \end{macro}
%
%    \begin{macro}{\CodelineIndex}
%    \begin{macrocode}
\g@addto@macro\CodelineIndex{%
  \let\special@index\HD@codeline@wrindex
}
%    \end{macrocode}
%    \end{macro}
%    \begin{macro}{\PageIndex}
%    \begin{macrocode}
\g@addto@macro\PageIndex{%
  \let\special@index\HD@page@wrindex
}
%    \end{macrocode}
%    \end{macro}
%
%    \begin{macro}{\HD@codeline@wrindex}
%    \begin{macrocode}
\def\HD@codeline@wrindex#1{%
  \begingroup
    \let\HDorg@encapchar\encapchar
    \def\encapchar##1\encapchar##2\@nil{%
      \HDorg@encapchar
      hdclindex{\the\c@HD@hypercount}{##1}%
    }%
    \codeline@wrindex{%
      #1\encapchar\encapchar\@nil
    }%
  \endgroup
}
%    \end{macrocode}
%    \end{macro}
%    \begin{macro}{\HD@page@wrindex}
%    \begin{macrocode}
\def\HD@page@wrindex#1{%
  \begingroup
    \let\HDorg@encapchar\encapchar
    \def\encapchar##1\encapchar##2\@nil{%
      \HDorg@encapchar
      hdpindex{##1}%
    }%
    \HDorg@index{#1\encapchar\encapchar\@nil}%
  \endgroup
}
%    \end{macrocode}
%    \end{macro}
%
%    \begin{macro}{\hdclindex}
%    \begin{macrocode}
\def\hdclindex#1#2#3{%
  \csname\ifx\\#2\\relax\else#2\fi\endcsname{%
    \hyperlink{HD.#1}{#3}%
    \HD@savedest@add{#1}%
  }%
}
%    \end{macrocode}
%    \end{macro}
%    \begin{macro}{\hdpindex}
%    \begin{macrocode}
\def\hdpindex#1#2{%
  \csname\ifx\\#1\\relax\else#1\fi\endcsname{%
    \hyperpage{#2}%
  }%
}
%    \end{macrocode}
%    \end{macro}
%
% \subsection{\cs{PrintDescribeMacro}}
%
%    \begin{macro}{\HDorg@PrintDescribeMacro}
%    \begin{macrocode}
\let\HDorg@PrintDescribeMacro\PrintDescribeMacro
%    \end{macrocode}
%    \end{macro}
%    \begin{macro}{\PrintDescribeMacro}
%    \begin{macrocode}
\renewcommand\PrintDescribeMacro[1]{%
  \begingroup
    \let\HDorg@MacroFont\MacroFont
    \def\MacroFont{%
      \HD@target
      \HDorg@MacroFont
    }%
    \HDorg@PrintDescribeMacro{#1}%
  \endgroup
}
%    \end{macrocode}
%    \end{macro}
%
% \subsection{\cs{PrintMacroName}}
%
%    \begin{macro}{\HDorg@PrintMacroName}
%    \begin{macrocode}
\let\HDorg@PrintMacroName\PrintMacroName
%    \end{macrocode}
%    \end{macro}
%    \begin{macro}{\PrintMacroName}
%    \begin{macrocode}
\renewcommand\PrintMacroName[1]{%
  \begingroup
    \let\HDorg@MacroFont\MacroFont
    \def\MacroFont{%
      \HD@target
      \HDorg@MacroFont
    }%
    \HDorg@PrintMacroName{#1}%
  \endgroup
}
%    \end{macrocode}
%    \end{macro}
%
% \subsection{\cs{theCodelineNo}}
%
%    \begin{macro}{\HDorg@theCodelineNo}
%    \begin{macrocode}
\let\HDorg@theCodelineNo\theCodelineNo
%    \end{macrocode}
%    \end{macro}
%    \begin{macro}{\theCodelineNo}
%    \begin{macrocode}
\renewcommand*\theCodelineNo{%
  \settowidth{\HD@helplength}{\HDorg@theCodelineNo\ }%
  \ifdim\HD@helplength<\MacroIndent
    \addtolength{\HD@helplength}{-\MacroIndent}%
    \llap{%
      \begingroup\HD@target\endgroup
      \hspace*{-\HD@helplength}%
    }%
  \else
    \HD@target
  \fi
  \HDorg@theCodelineNo
}
%    \end{macrocode}
%    \end{macro}
%
%    \begin{macro}{\HDorg@index}
%    \begin{macrocode}
\AtBeginDocument{%
  \let\HDorg@index\index
}
%    \end{macrocode}
%    \end{macro}
%
% \subsection{\cs{SpecialUsageIndex}}
%
%    \begin{macro}{\HDorg@SpecialUsageIndex}
%    \begin{macrocode}
\let\HDorg@SpecialUsageIndex\SpecialUsageIndex
\renewcommand*\SpecialUsageIndex[1]{%
  \@bsphack
  \begingroup
    \HD@target
    \let\index\HDorg@index
    \let\HDorg@encapchar\encapchar
    \edef\encapchar usage{%
       \HDorg@encapchar hdclindex{\the\c@HD@hypercount}{usage}%
    }%
    \HDorg@SpecialUsageIndex{#1}%
  \endgroup
  \@esphack
}
%    \end{macrocode}
%    \end{macro}
%
% \subsection{\cs{autoindex}}
%
%    \begin{macrocode}
\begingroup\expandafter\expandafter\expandafter\endgroup
\expandafter\ifx\csname autoindex\endcsname\relax
\else
%    \end{macrocode}
%    \begin{macro}{\HDorg@autoindex}
%    \begin{macrocode}
  \let\HDorg@autoindex\autoindex
%    \end{macrocode}
%    \end{macro}
%    \begin{macro}{\autoindex}
%    \begin{macrocode}
  \renewcommand*\autoindex[1]{%
    \@bsphack
    \begingroup
      \HD@target
      \let\index\HDorg@index
      \let\HDorg@encapchar\encapchar
      \edef\encapchar usage{%
         \HDorg@encapchar hdclindex{\the\c@HD@hypercount}{usage}%
      }%
      \HDorg@autoindex{#1}%
    \endgroup
    \@esphack
  }%
\fi
%    \end{macrocode}
%    \end{macro}
%
% \subsection{\cs{SpecialEnvIndex}}
%
%    \begin{macro}{\HDorg@SpecialEnvIndex}
%    \begin{macrocode}
\let\HDorg@SpecialEnvIndex\SpecialEnvIndex
%    \end{macrocode}
%    \end{macro}
%    \begin{macro}{\SpecialEnvIndex}
%    \begin{macrocode}
\renewcommand*\SpecialEnvIndex[1]{%
  \@bsphack
  \begingroup
    \HD@target
    \let\HDorg@encapchar\encapchar
    \edef\encapchar usage{%
      \HDorg@encapchar hdclindex{\the\c@HD@hypercount}{usage}%
    }%
    \HDorg@SpecialEnvIndex{#1}%
  \endgroup
  \@esphack
}
%    \end{macrocode}
%    \end{macro}
%
% \subsection{\cs{SortIndex}}
%
%    \begin{macro}{\HDorg@SortIndex}
%    \begin{macrocode}
\let\HDorg@SortIndex\SortIndex
%    \end{macrocode}
%    \end{macro}
%    \begin{macro}{\SortIndex}
%    \begin{macrocode}
\renewcommand*{\SortIndex}[2]{%
  \@bsphack
  \begingroup
    \let\index\HD@page@wrindex
    \HDorg@SortIndex{#1}{#2}%
  \endgroup
  \@esphack
}
%    \end{macrocode}
%    \end{macro}
%
% \subsection{\cs{@wrglossary}}
%
%    \begin{macro}{\HDorg@wrglossary}
%    \begin{macrocode}
\let\HDorg@wrglossary\@wrglossary
%    \end{macrocode}
%    \end{macro}
%    \begin{macro}{\@wrglossary}
%    \begin{macrocode}
\def\@wrglossary#1{%
  \let\HDorg@encapchar\encapchar
  \def\encapchar##1\encapchar##2\@nil{%
    \HDorg@encapchar
    hdpindex{##1}%
  }%
  \HDorg@wrglossary{#1\encapchar\encapchar\@nil}%
}
%    \end{macrocode}
%    \end{macro}
%
% \subsection{\cs{MacroIndent}}
%
%    \cs{MacroIndent} automatically calculated.
%    \begin{macrocode}
\AfterLastShipout{%
  \if@filesw
    \immediate\write\@auxout{%
      \string\providecommand\string\HD@SetMacroIndent[1]{}%
    }%
    \immediate\write\@auxout{%
      \string\HD@SetMacroIndent{\number\value{CodelineNo}}%
    }%
  \fi
}
%    \end{macrocode}
%    \begin{macro}{\HD@SetMacroIndent}
%    \begin{macrocode}
\def\HD@SetMacroIndent#1{%
  \begingroup
    \value{CodelineNo}=#1\relax
    \settowidth{\MacroIndent}{\HDorg@theCodelineNo\ }%
    \global\MacroIndent\MacroIndent
  \endgroup
}
%    \end{macrocode}
%    \end{macro}
%
%    \begin{macrocode}
%</package>
%    \end{macrocode}
%
% \section{Installation}
%
% \subsection{Download}
%
% \paragraph{Package.} This package is available on
% CTAN\footnote{\url{http://ctan.org/pkg/hypdoc}}:
% \begin{description}
% \item[\CTAN{macros/latex/contrib/oberdiek/hypdoc.dtx}] The source file.
% \item[\CTAN{macros/latex/contrib/oberdiek/hypdoc.pdf}] Documentation.
% \end{description}
%
%
% \paragraph{Bundle.} All the packages of the bundle `oberdiek'
% are also available in a TDS compliant ZIP archive. There
% the packages are already unpacked and the documentation files
% are generated. The files and directories obey the TDS standard.
% \begin{description}
% \item[\CTAN{install/macros/latex/contrib/oberdiek.tds.zip}]
% \end{description}
% \emph{TDS} refers to the standard ``A Directory Structure
% for \TeX\ Files'' (\CTAN{tds/tds.pdf}). Directories
% with \xfile{texmf} in their name are usually organized this way.
%
% \subsection{Bundle installation}
%
% \paragraph{Unpacking.} Unpack the \xfile{oberdiek.tds.zip} in the
% TDS tree (also known as \xfile{texmf} tree) of your choice.
% Example (linux):
% \begin{quote}
%   |unzip oberdiek.tds.zip -d ~/texmf|
% \end{quote}
%
% \paragraph{Script installation.}
% Check the directory \xfile{TDS:scripts/oberdiek/} for
% scripts that need further installation steps.
% Package \xpackage{attachfile2} comes with the Perl script
% \xfile{pdfatfi.pl} that should be installed in such a way
% that it can be called as \texttt{pdfatfi}.
% Example (linux):
% \begin{quote}
%   |chmod +x scripts/oberdiek/pdfatfi.pl|\\
%   |cp scripts/oberdiek/pdfatfi.pl /usr/local/bin/|
% \end{quote}
%
% \subsection{Package installation}
%
% \paragraph{Unpacking.} The \xfile{.dtx} file is a self-extracting
% \docstrip\ archive. The files are extracted by running the
% \xfile{.dtx} through \plainTeX:
% \begin{quote}
%   \verb|tex hypdoc.dtx|
% \end{quote}
%
% \paragraph{TDS.} Now the different files must be moved into
% the different directories in your installation TDS tree
% (also known as \xfile{texmf} tree):
% \begin{quote}
% \def\t{^^A
% \begin{tabular}{@{}>{\ttfamily}l@{ $\rightarrow$ }>{\ttfamily}l@{}}
%   hypdoc.sty & tex/latex/oberdiek/hypdoc.sty\\
%   hypdoc.pdf & doc/latex/oberdiek/hypdoc.pdf\\
%   hypdoc.dtx & source/latex/oberdiek/hypdoc.dtx\\
% \end{tabular}^^A
% }^^A
% \sbox0{\t}^^A
% \ifdim\wd0>\linewidth
%   \begingroup
%     \advance\linewidth by\leftmargin
%     \advance\linewidth by\rightmargin
%   \edef\x{\endgroup
%     \def\noexpand\lw{\the\linewidth}^^A
%   }\x
%   \def\lwbox{^^A
%     \leavevmode
%     \hbox to \linewidth{^^A
%       \kern-\leftmargin\relax
%       \hss
%       \usebox0
%       \hss
%       \kern-\rightmargin\relax
%     }^^A
%   }^^A
%   \ifdim\wd0>\lw
%     \sbox0{\small\t}^^A
%     \ifdim\wd0>\linewidth
%       \ifdim\wd0>\lw
%         \sbox0{\footnotesize\t}^^A
%         \ifdim\wd0>\linewidth
%           \ifdim\wd0>\lw
%             \sbox0{\scriptsize\t}^^A
%             \ifdim\wd0>\linewidth
%               \ifdim\wd0>\lw
%                 \sbox0{\tiny\t}^^A
%                 \ifdim\wd0>\linewidth
%                   \lwbox
%                 \else
%                   \usebox0
%                 \fi
%               \else
%                 \lwbox
%               \fi
%             \else
%               \usebox0
%             \fi
%           \else
%             \lwbox
%           \fi
%         \else
%           \usebox0
%         \fi
%       \else
%         \lwbox
%       \fi
%     \else
%       \usebox0
%     \fi
%   \else
%     \lwbox
%   \fi
% \else
%   \usebox0
% \fi
% \end{quote}
% If you have a \xfile{docstrip.cfg} that configures and enables \docstrip's
% TDS installing feature, then some files can already be in the right
% place, see the documentation of \docstrip.
%
% \subsection{Refresh file name databases}
%
% If your \TeX~distribution
% (\teTeX, \mikTeX, \dots) relies on file name databases, you must refresh
% these. For example, \teTeX\ users run \verb|texhash| or
% \verb|mktexlsr|.
%
% \subsection{Some details for the interested}
%
% \paragraph{Attached source.}
%
% The PDF documentation on CTAN also includes the
% \xfile{.dtx} source file. It can be extracted by
% AcrobatReader 6 or higher. Another option is \textsf{pdftk},
% e.g. unpack the file into the current directory:
% \begin{quote}
%   \verb|pdftk hypdoc.pdf unpack_files output .|
% \end{quote}
%
% \paragraph{Unpacking with \LaTeX.}
% The \xfile{.dtx} chooses its action depending on the format:
% \begin{description}
% \item[\plainTeX:] Run \docstrip\ and extract the files.
% \item[\LaTeX:] Generate the documentation.
% \end{description}
% If you insist on using \LaTeX\ for \docstrip\ (really,
% \docstrip\ does not need \LaTeX), then inform the autodetect routine
% about your intention:
% \begin{quote}
%   \verb|latex \let\install=y\input{hypdoc.dtx}|
% \end{quote}
% Do not forget to quote the argument according to the demands
% of your shell.
%
% \paragraph{Generating the documentation.}
% You can use both the \xfile{.dtx} or the \xfile{.drv} to generate
% the documentation. The process can be configured by the
% configuration file \xfile{ltxdoc.cfg}. For instance, put this
% line into this file, if you want to have A4 as paper format:
% \begin{quote}
%   \verb|\PassOptionsToClass{a4paper}{article}|
% \end{quote}
% An example follows how to generate the
% documentation with pdf\LaTeX:
% \begin{quote}
%\begin{verbatim}
%pdflatex hypdoc.dtx
%makeindex -s gind.ist hypdoc.idx
%pdflatex hypdoc.dtx
%makeindex -s gind.ist hypdoc.idx
%pdflatex hypdoc.dtx
%\end{verbatim}
% \end{quote}
%
% \section{Catalogue}
%
% The following XML file can be used as source for the
% \href{http://mirror.ctan.org/help/Catalogue/catalogue.html}{\TeX\ Catalogue}.
% The elements \texttt{caption} and \texttt{description} are imported
% from the original XML file from the Catalogue.
% The name of the XML file in the Catalogue is \xfile{hypdoc.xml}.
%    \begin{macrocode}
%<*catalogue>
<?xml version='1.0' encoding='us-ascii'?>
<!DOCTYPE entry SYSTEM 'catalogue.dtd'>
<entry datestamp='$Date$' modifier='$Author$' id='hypdoc'>
  <name>hypdoc</name>
  <caption>Hyper extensions for doc.sty.</caption>
  <authorref id='auth:oberdiek'/>
  <copyright owner='Heiko Oberdiek' year='2002,2006,2007,2009-2011'/>
  <license type='lppl1.3'/>
  <version number='1.14'/>
  <description>
    This package adds hypertext features to the package
    <xref refid='doc'>doc</xref> that is used in the documentation
    system of <xref refid='latex'>LaTeX2e</xref>. Bookmarks are added
    and references are linked as far as possible.
    <p/>
    The package is part of the <xref refid='oberdiek'>oberdiek</xref>
    bundle.
  </description>
  <documentation details='Package documentation'
      href='ctan:/macros/latex/contrib/oberdiek/hypdoc.pdf'/>
  <ctan file='true' path='/macros/latex/contrib/oberdiek/hypdoc.dtx'/>
  <miktex location='oberdiek'/>
  <texlive location='oberdiek'/>
  <install path='/macros/latex/contrib/oberdiek/oberdiek.tds.zip'/>
</entry>
%</catalogue>
%    \end{macrocode}
%
% \begin{History}
%   \begin{Version}{2002/05/10 v1.0}
%   \item
%     First standalone version.
%   \end{Version}
%   \begin{Version}{2006/02/20 v1.1}
%   \item
%     Option \xoption{numbered} added.
%   \item
%     \cs{HD@savedest@init} fixed (empty \xext{hd} file triggered
%     first page with fl ligature).
%   \item
%     Bookmarks for \cs{listoffigures} and \cs{listoftables}.
%   \item
%     Documentation: features and usage added.
%   \item
%     LPPL 1.3.
%   \end{Version}
%   \begin{Version}{2006/04/27 v1.2}
%   \item
%     \cs{HD@target} fixed (\xfile{multicol.dtx}).
%   \end{Version}
%   \begin{Version}{2006/06/01 v1.3}
%   \item
%     Support for package \xpackage{amsdtx}'s \cs{autoindex} added.
%   \end{Version}
%   \begin{Version}{2006/06/02 v1.4}
%   \item
%     Bookmark fixed if there are several table of contents,
%     bibliographies, glossaries, \dots
%   \end{Version}
%   \begin{Version}{2007/04/11 v1.5}
%   \item
%     Line ends sanitized.
%   \end{Version}
%   \begin{Version}{2007/11/12 v1.6}
%   \item
%     Bug fix in \cs{listoffigures} (Axel Sommerfeldt).
%   \item
%     Markup added in implementation section.
%   \end{Version}
%   \begin{Version}{2009/12/10 v1.7}
%   \item
%     Use of package \xpackage{rerunfilecheck}.
%   \end{Version}
%   \begin{Version}{2010/02/03 v1.8}
%   \item
%     Fix for bug latex/4096 added.
%   \end{Version}
%   \begin{Version}{2010/03/26 v1.9}
%   \item
%     Definition of \cs{HD@SetMacroIndex} is provided in the \xext{aux} file.
%   \end{Version}
%   \begin{Version}{2011/05/05 v1.10}
%   \item
%     \cs{raisebox} in \cs{HD@target} with height and depth of \texttt{0pt}
%     (request of GL).
%   \end{Version}
%   \begin{Version}{2011/08/19 v1.11}
%   \item
%     Latest comma in option list for hyperref removed to avoid option clash
%     because of \hologo{LaTeX} 2011/06/27 with the new \cs{in@}.
%   \end{Version}
%   \begin{Version}{2016/05/16 v1.12}
%   \item
%     Documentation updates.
%   \end{Version}
%   \begin{Version}{2016/10/02 v1.13}
%   \item
%     Fixes to bookmarknames, also introduce \cs{changehistoryname}
%   \end{Version}
%   \begin{Version}{2018/09/07 v1.14}
%   \item
%     Do not load \xpackage{thumpdf}.
%   \end{Version}
% \end{History}
%
% \PrintIndex
%
% \Finale
\endinput

%        (quote the arguments according to the demands of your shell)
%
% Documentation:
%    (a) If hypdoc.drv is present:
%           latex hypdoc.drv
%    (b) Without hypdoc.drv:
%           latex hypdoc.dtx; ...
%    The class ltxdoc loads the configuration file ltxdoc.cfg
%    if available. Here you can specify further options, e.g.
%    use A4 as paper format:
%       \PassOptionsToClass{a4paper}{article}
%
%    Programm calls to get the documentation (example):
%       pdflatex hypdoc.dtx
%       makeindex -s gind.ist hypdoc.idx
%       pdflatex hypdoc.dtx
%       makeindex -s gind.ist hypdoc.idx
%       pdflatex hypdoc.dtx
%
% Installation:
%    TDS:tex/latex/oberdiek/hypdoc.sty
%    TDS:doc/latex/oberdiek/hypdoc.pdf
%    TDS:source/latex/oberdiek/hypdoc.dtx
%
%<*ignore>
\begingroup
  \catcode123=1 %
  \catcode125=2 %
  \def\x{LaTeX2e}%
\expandafter\endgroup
\ifcase 0\ifx\install y1\fi\expandafter
         \ifx\csname processbatchFile\endcsname\relax\else1\fi
         \ifx\fmtname\x\else 1\fi\relax
\else\csname fi\endcsname
%</ignore>
%<*install>
\input docstrip.tex
\Msg{************************************************************************}
\Msg{* Installation}
\Msg{* Package: hypdoc 2018/09/07 v1.14 Hyper extensions for doc.sty (HO)}
\Msg{************************************************************************}

\keepsilent
\askforoverwritefalse

\let\MetaPrefix\relax
\preamble

This is a generated file.

Project: hypdoc
Version: 2018/09/07 v1.14

Copyright (C) 2002, 2006, 2007, 2009-2011 by
   Heiko Oberdiek <heiko.oberdiek at googlemail.com>

This work may be distributed and/or modified under the
conditions of the LaTeX Project Public License, either
version 1.3c of this license or (at your option) any later
version. This version of this license is in
   http://www.latex-project.org/lppl/lppl-1-3c.txt
and the latest version of this license is in
   http://www.latex-project.org/lppl.txt
and version 1.3 or later is part of all distributions of
LaTeX version 2005/12/01 or later.

This work has the LPPL maintenance status "maintained".

This Current Maintainer of this work is Heiko Oberdiek.

This work consists of the main source file hypdoc.dtx
and the derived files
   hypdoc.sty, hypdoc.pdf, hypdoc.ins, hypdoc.drv.

\endpreamble
\let\MetaPrefix\DoubleperCent

\generate{%
  \file{hypdoc.ins}{\from{hypdoc.dtx}{install}}%
  \file{hypdoc.drv}{\from{hypdoc.dtx}{driver}}%
  \usedir{tex/latex/oberdiek}%
  \file{hypdoc.sty}{\from{hypdoc.dtx}{package}}%
  \nopreamble
  \nopostamble
%  \usedir{source/latex/oberdiek/catalogue}%
%  \file{hypdoc.xml}{\from{hypdoc.dtx}{catalogue}}%
}

\catcode32=13\relax% active space
\let =\space%
\Msg{************************************************************************}
\Msg{*}
\Msg{* To finish the installation you have to move the following}
\Msg{* file into a directory searched by TeX:}
\Msg{*}
\Msg{*     hypdoc.sty}
\Msg{*}
\Msg{* To produce the documentation run the file `hypdoc.drv'}
\Msg{* through LaTeX.}
\Msg{*}
\Msg{* Happy TeXing!}
\Msg{*}
\Msg{************************************************************************}

\endbatchfile
%</install>
%<*ignore>
\fi
%</ignore>
%<*driver>
\NeedsTeXFormat{LaTeX2e}
\ProvidesFile{hypdoc.drv}%
  [2018/09/07 v1.14 Hyper extensions for doc.sty (HO)]%
\documentclass{ltxdoc}
\usepackage{holtxdoc}[2011/11/22]
\begin{document}
  \DocInput{hypdoc.dtx}%
\end{document}
%</driver>
% \fi
%
%
% \CharacterTable
%  {Upper-case    \A\B\C\D\E\F\G\H\I\J\K\L\M\N\O\P\Q\R\S\T\U\V\W\X\Y\Z
%   Lower-case    \a\b\c\d\e\f\g\h\i\j\k\l\m\n\o\p\q\r\s\t\u\v\w\x\y\z
%   Digits        \0\1\2\3\4\5\6\7\8\9
%   Exclamation   \!     Double quote  \"     Hash (number) \#
%   Dollar        \$     Percent       \%     Ampersand     \&
%   Acute accent  \'     Left paren    \(     Right paren   \)
%   Asterisk      \*     Plus          \+     Comma         \,
%   Minus         \-     Point         \.     Solidus       \/
%   Colon         \:     Semicolon     \;     Less than     \<
%   Equals        \=     Greater than  \>     Question mark \?
%   Commercial at \@     Left bracket  \[     Backslash     \\
%   Right bracket \]     Circumflex    \^     Underscore    \_
%   Grave accent  \`     Left brace    \{     Vertical bar  \|
%   Right brace   \}     Tilde         \~}
%
% \GetFileInfo{hypdoc.drv}
%
% \title{The \xpackage{hypdoc} package}
% \date{2018/09/07 v1.14}
% \author{Heiko Oberdiek\thanks
% {Please report any issues at https://github.com/ho-tex/oberdiek/issues}\\
% \xemail{heiko.oberdiek at googlemail.com}}
%
% \maketitle
%
% \begin{abstract}
% This package adds hyper features to the package
% \xpackage{doc} that is used in the documentation
% system of \LaTeXe. Bookmarks are added and references
% are linked as far as possible.
% \end{abstract}
%
% \tableofcontents
%
% \section{Features}
%
% The package adds the following features:
% \begin{description}
% \item[Bookmarks:]
%   For supporting bookmarks package \verb|hyperref| is loaded.
%   Some \TeX\ code inside section titles can cause problems
%   during bookmark processing; these can be warnings, ugly
%   bookmark titles or even errors. With the help of
%   \verb|\pdfstringdefDisableCommands| or \verb|\texorpdfstring|
%   you can manage these things, see \verb|hyperref|'s package
%   documentation.
% \item[Additional bookmarks:]
%   Unnumbered bookmark entries are generated for:
%   \begin{itemize}
%   \item Table of contents
%   \item List of figures
%   \item List of tables
%   \item Index
%   \item Glossary
%   \end{itemize}
%   Option \texttt{numbered} controls, whether these section or
%   chapters should be numbered. The exception is the table
%   of contents if this it would be the first numbered
%   section. Then just a bookmark is added.
% \item[Metadata:]
%   The loaded package \verb|hyperref|, enforced by
%   option \verb|pdfusetitle|, tries to detect \verb|\title|
%   and \verb|\author| and sets the corresponding PDF
%   information fields. In case of problems see the previous
%   item about bookmarks.
%   These enries can be overwritten or new ones added by
%   \verb|\hypersetup| after the packages is loaded, see
%   documentation of package \verb|hyperref|.
% \item[References, links:] Because of package \verb|\hyperref|
%   |\ref| become links, also \verb|\url|, footnote referencing
%   are supported.
% \item[Index:]
%   The main part of this package deals with index links in order
%   to support the kind of index that package \verb|doc| provides.
% \item[Thumbnails:]
%   Package \verb|thumbpdf| may additionally be loaded for the case that the
%   PDF file also should contain thumbnails, see the documentation
%   of this package. But thumbnail embedding is not really
%   necessary, because nowadays AcrobatReader is able to
%   generate and view thumbnails on the fly.
% \end{description}
%
% \section{Usage}
%
% The purpose of this packages is to add PDF features to
% the documentation of LaTeX packages. Thus just load
% the package in the driver file that generates the
% documentation:
% \begin{quote}
%   \verb|\usepackage{hypdoc}|
% \end{quote}
% Package \verb|doc| is loaded by \verb|hypdoc|, thus you
% also can replace a \verb|\usepackage{doc}| with this line.
% At any case, the package must be loaded before macros of
% package \verb|doc| such as \verb|\CodelineIndex| or
% \verb|\PageIndex| are used.
%
% If the class \verb|ltxdoc| is used, you can also use
% the configuration file for this class to add
% package \verb|hypdoc|. Add the following line
% to the file \verb|ltxdoc.cfg| (or create a new one):
% \begin{quote}
%\begin{verbatim}
%\AtEndOfClass{\RequirePackage{hypdoc}}
%\end{verbatim}
% \end{quote}
%
% \subsection{Options}
%
% There is one option \texttt{numbered}. It controls,
% whether the starred versions of section or chapter
% inside \cs{tableofcontents}, \cs{listoffigures}, \dots
% should be replaced by the non-star versions.
%
% \subsection{Simple example}
% Example of a simple driver file \verb|foobar.drv|
% of a package \verb|foobar|. Often the driver file
% is packed in the file \verb|foobar.dtx| and can
% be extracted by docstrip or if an instruction file
% for docstrip is provided by:
% \begin{quote}
% \verb|tex foobar.ins|
% \end{quote}
% Note that although docstrip doesn't rely on \LaTeX,
% sometimes you need to run the \verb|.ins| file through
% \LaTeX. Oddly enough, often the only \LaTeX\ feature
% in use is \verb|\NeedsTeXFormat{LaTeX2e}| in the first line.
% \begin{quote}
%\begin{verbatim}
%\documentclass{ltxdoc}
%\usepackage{hypdoc}
%\RecordChanges
%\EnableCrossrefs
%\CodelineIndex
%\begin{document}
%\title{Title for package \texttt{foobar}}
%\author{Author of package \texttt{foobar}}
%\date{...}
%\maketitle
%\tableofcontents
%\DocInput{foobar.dtx}
%\PrintChanges
%\PrintIndex
%\end{document}
%\end{verbatim}
% \end{quote}
%
% Then compile it, for example with pdf\LaTeX.
% \begin{quote}
%\begin{verbatim}
%pdflatex foobar.drv
%makeindex -s gind.ist foobar.idx
%makeindex -s gglo.ist -o foobar.gls foobar.glo
%pdflatex foobar.drv
%\end{verbatim}
% \end{quote}
% You will need several cycles until the cross references
% and similar things are correct.
%
% \subsection{Example \texttt{source2e.tex}}
%
% The source code of the base of \LaTeX\ is available as
% a bunch of \verb|.dtx| files. \LaTeX\ provides the file
% \verb|source2e.tex| to merge them all together.
%
% Now you can either load package \verb|hypdoc| in a private copy
% of this file (please, respect the LPPL) or use the configuration
% file \verb|ltxdoc.cfg|. Example that also sets A4 paper size:
% \begin{quote}
%\begin{verbatim}
%\PassOptionsToClass{a4paper}{article}
%\AtEndOfClass{\RequirePackage{hypdoc}}
%\end{verbatim}
% \end{quote}
%
% Then \verb|source2e.tex| can be compiled by pdf\TeX, for example:
% \begin{quote}
%\begin{verbatim}
%pdflatex source2e
%makeindex -s gind.ist source2e.idx
%makeindex -s gglo.ist -o source2e.glo source2e.gls
%pdflatex source2e
%makeindex -s gind.ist source2e.idx
%makeindex -s gglo.ist -o source2e.glo source2e.gls
%pdflatex source2e
%\end{verbatim}
% \end{quote}
%
% \StopEventually{
% }
%
% \section{Implementation}
%
%    \begin{macrocode}
%<*package>
\NeedsTeXFormat{LaTeX2e}
\ProvidesPackage{hypdoc}%
  [2018/09/07 v1.14 Hyper extensions for doc.sty (HO)]%
%    \end{macrocode}
%    The package identification is done at the top of the |.dtx| file
%    in order to use only one identification string.
%
%    For unique command names this package uses |HD@| as prefix
%
% \subsection{Options}
%
%    \begin{macro}{\ifHD@numbered}
%    \begin{macrocode}
\newif\ifHD@numbered
%    \end{macrocode}
%    \end{macro}
%    \begin{macrocode}
\DeclareOption{numbered}{\HD@numberedtrue}
\ProcessOptions*\relax
%    \end{macrocode}
%
% \subsection{Package loading}
%
%    \begin{macrocode}
\RequirePackage{atveryend}[2010/03/24]
\RequirePackage{doc}
%    \end{macrocode}
%    \begin{macrocode}
\ifx\numexpr\@undefined
  \RequirePackage{calc}%
\fi
%    \end{macrocode}
%    \begin{macrocode}
\RequirePackage[%
  hyperindex=false,%
  pdfusetitle,%
  colorlinks,%
  pdfpagelabels%
]{hyperref}[2002/05/09]%
\hypersetup{%
  plainpages=false,%
  bookmarksopen,%
  bookmarksnumbered,%
  bookmarksopenlevel=0,%
  pdfstartview={FitBH \hypercalcbp{%
    \paperheight-\topmargin-1in-\headheight-\headsep
  }},%
}
%    \end{macrocode}
%
%    \begin{macrocode}
\newcounter{HD@unique}
%    \end{macrocode}
%    \begin{macro}{\theHD@unique}
%    \begin{macrocode}
\renewcommand{\theHD@unique}{%
  \number\value{HD@unique}%
}
%    \end{macrocode}
%    \end{macro}
%
%    \begin{macrocode}
\g@addto@macro\Hy@UseMaketitleInfos{%
  \@ifundefined{fileinfo}{}{%
    \ifx\@pdfsubject\@empty
      \hypersetup{pdfsubject={\fileinfo}}%
    \fi
  }%
}
%    \end{macrocode}
%
%    \begin{macrocode}
\pdfstringdefDisableCommands{%
  \let\thanks\@gobble
  \let\footnote\@gobble
  \def\cs#1{\textbackslash #1}%
  \let\normalfont\@empty
  \let\scshape\@empty
  \def\and{and }%
}
%    \end{macrocode}
%
% \subsubsection{Patch for bug latex/4096}
%
%    \begin{macrocode}
\begingroup\expandafter\expandafter\expandafter\endgroup
\expandafter\ifx\csname mod@math@codes\endcsname\relax
\else
  \def\HD@ModFix#1{%
    \mathcode`#1=`#1\relax
  }%
  \g@addto@macro\mod@math@codes{%
    \HD@ModFix\-%
    \HD@ModFix\+%
    \HD@ModFix\:%
    \HD@ModFix\=%
    \HD@ModFix\*%
  }%
\fi
%    \end{macrocode}
%
% \subsection{Bookmarks}
%
%    \begin{macro}{\HD@sectionpatch}
%    \begin{macrocode}
\def\HD@sectionpatch{%
  \ifHD@numbered
    \HD@@sectionpatch{section}%
    \HD@@sectionpatch{chapter}%
  \else
    \let\Hy@writebookmark\HD@disable@writebookmark
  \fi
}
%    \end{macrocode}
%    \end{macro}
%    \begin{macro}{\HD@@sectionpatch}
%    \begin{macrocode}
\def\HD@@sectionpatch#1{%
  \expandafter\let\csname HDorg@@#1\expandafter\endcsname
    \csname #1\endcsname
  \@namedef{#1}{%
    \@ifstar{\@nameuse{HDorg@@#1}}{\@nameuse{HDorg@@#1}}%
  }%
}
%    \end{macrocode}
%    \end{macro}
%    \begin{macro}{\HD@disable@writebookmark}
%    \begin{macrocode}
\def\HD@disable@writebookmark#1#2#3#4#5{}
%    \end{macrocode}
%    \end{macro}
%
%    \begin{macro}{\HD@guesstoclevel}
%    A wild guessing of the toclevel. There is the hope, that
%    the macro |#1| starts with a sectioning command.
%    \begin{macrocode}
\def\HD@guesstoclevel#1{%
  \expandafter\HD@@guesstoclevel#1\section\HD@end
}
%    \end{macrocode}
%    \end{macro}
%    \begin{macro}{\HD@@guesstoclevel}
%    \begin{macrocode}
\def\HD@@guesstoclevel#1#2\HD@end{%
  \@ifundefined{toclevel@\expandafter\@gobble\string#1\@empty}{%
    1%
  }{%
    \@nameuse{toclevel@\expandafter\@gobble\string#1\@empty}%
  }%
}
%    \end{macrocode}
%    \end{macro}
%
% \subsection{\cs{tableofcontents}}
%
%    \begin{macro}{\HDorg@tableofcontents}
%    \begin{macrocode}
\let\HDorg@tableofcontents\tableofcontents
%    \end{macrocode}
%    \end{macro}
%    \begin{macro}{\tableofcontents}
%    As first section the table of contents shouldn't
%    go into the toc itself. Also a number looks too funny then.
%    Just a bookmark is enough in this case.
%    \begin{macrocode}
\def\tableofcontents{%
  \begingroup
    \@ifundefined{c@section}{%
      \def\c@section{0}%
    }{}%
  \expandafter\endgroup
  \ifcase\c@section
    \stepcounter{HD@unique}%
    \pdfbookmark[\HD@guesstoclevel{\HDorg@tableofcontents}]%
        {\contentsname}{toc\theHD@unique}%
    \HDorg@tableofcontents
  \else
    \ifHD@numbered
    \else
      \stepcounter{HD@unique}%
      \pdfbookmark[\HD@guesstoclevel{\HDorg@tableofcontents}]%
          {\contentsname}{toc\theHD@unique}%
    \fi
    \begingroup
      \HD@sectionpatch
      \HDorg@tableofcontents
    \endgroup
  \fi
}
%    \end{macrocode}
%    \end{macro}
%
% \subsection{\cs{listoffigures}}
%
%    \begin{macro}{\HDorg@listoffigures}
%    \begin{macrocode}
\let\HDorg@listoffigures\listoffigures
%    \end{macrocode}
%    \end{macro}
%    \begin{macro}{\listoffigures}
%    \begin{macrocode}
\def\listoffigures{%
  \ifHD@numbered
  \else
    \stepcounter{HD@unique}%
    \pdfbookmark[\HD@guesstoclevel{\HDorg@listoffigures}]%
        {\listfigurename}{toc\theHD@unique}%
  \fi
  \begingroup
    \HD@sectionpatch
    \HDorg@listoffigures
  \endgroup
}
%    \end{macrocode}
%    \end{macro}
%
% \subsection{\cs{listoftables}}
%
%    \begin{macro}{\HDorg@listoftables}
%    \begin{macrocode}
\let\HDorg@listoftables\listoftables
%    \end{macrocode}
%    \end{macro}
%    \begin{macro}{\listoftables}
%    \begin{macrocode}
\def\listoftables{%
  \ifHD@numbered
  \else
    \stepcounter{HD@unique}%
    \pdfbookmark[\HD@guesstoclevel{\HDorg@listoftables}]%
        {\listtablename}{toc\theHD@unique}%
  \fi
  \begingroup
    \HD@sectionpatch
    \HDorg@listoftables
  \endgroup
}
%    \end{macrocode}
%    \end{macro}
%
% \subsection{\cs{thebibliography}}
%
%    \begin{macro}{\HDorg@thebibliography}
%    \begin{macrocode}
\let\HDorg@thebibliography\thebibliography
%    \end{macrocode}
%    \end{macro}
%    \begin{macro}{\thebibliography}
%    \begin{macrocode}
\def\thebibliography{%
  \ifHD@numbered
  \else
    \stepcounter{HD@unique}%
    \pdfbookmark[\HD@guesstoclevel{\HDorg@thebibliography}]%
      {\@ifundefined{chapter}{\refname}{\bibname}}{bib\theHD@unique}%
  \fi
  \HD@sectionpatch
  \HDorg@thebibliography
}
%    \end{macrocode}
%    \end{macro}
%
% \subsection{\cs{theindex}}
%
%    \begin{macro}{\HDorg@theindex}
%    \begin{macrocode}
\let\HDorg@theindex\theindex
%    \end{macrocode}
%    \end{macro}
%    \begin{macro}{\theindex}
%    \begin{macrocode}
\def\theindex{%
  \let\HDorg@index@prologue\index@prologue
  \let\HDorg@writebookmark\Hy@writebookmark
  \HD@sectionpatch
  \g@addto@macro\IndexParms{%
    \ifHD@numbered
    \else
      \let\Hy@writebookmark\HDorg@writebookmark
    \fi
    \let\bfseries\HD@bfseries
  }%
  \edef\HD@toclevel@index{\HD@guesstoclevel{\index@prologue}}%
  \begingroup
    \count@=\HD@toclevel@index\relax
    \advance\count@ by 1 %
    \xdef\HD@toclevel@subindex{\the\count@}%
  \endgroup
  \def\index@prologue{%
    \ifHD@numbered
    \else
      \begingroup
        \stepcounter{HD@unique}%
        \let\Hy@writebookmark\HDorg@writebookmark
        \pdfbookmark[\HD@toclevel@index]%
            {\indexname}{index\theHD@unique}%
      \endgroup
    \fi
    \HDorg@index@prologue
  }%
  \HDorg@theindex
}
%    \end{macrocode}
%    \end{macro}
%    \begin{macro}{\HDorg@bfseries}
%    \begin{macrocode}
\let\HDorg@bfseries\bfseries
%    \end{macrocode}
%    \end{macro}
%    \begin{macro}{\HD@bfseries}
%    \begin{macrocode}
\def\HD@bfseries{\futurelet\HD@hfil\HD@@bfseries}%
%    \end{macrocode}
%    \end{macro}
%    \begin{macro}{\HD@@bfseries}
%    \begin{macrocode}
\def\HD@@bfseries{%
  \HDorg@bfseries
  \ifx\HD@hfil\hfil
    \expandafter\HD@@@bfseries
  \fi
}
%    \end{macrocode}
%    \end{macro}
%    \begin{macro}{\HD@@@bfseries}
%    \begin{macrocode}
\def\HD@@@bfseries\hfil#1\hfil{%
  \ifx\\#1\\%
  \else
    \raisebox{\baselineskip}[0pt]{%
      \kern-\HD@margin\relax
      \pdfbookmark[\HD@toclevel@subindex]{#1}{HD.#1}%
      \kern\HD@margin\relax
    }%
  \fi
  \hfil#1\hfil
}%
%    \end{macrocode}
%    \end{macro}
%    \begin{macro}{\HD@margin}
%    \begin{macrocode}
\def\HD@margin{1mm}
%    \end{macrocode}
%    \end{macro}
%
% \subsection{\cs{theglossary}}
%
%    \begin{macro}{\HDorg@theglossary}
%    \begin{macrocode}
\let\HDorg@theglossary\theglossary
%    \end{macrocode}
%    \end{macro}
%    \begin{macro}{\theglossary}
%    \begin{macrocode}
\def\theglossary{%
  \let\HDorg@glossary@prologue\glossary@prologue
  \let\HDorg@writebookmark\Hy@writebookmark
  \HD@sectionpatch
  \def\glossary@prologue{%
    \ifHD@numbered
    \else
      \begingroup
        \let\Hy@writebookmark\HDorg@writebookmark
        \stepcounter{HD@unique}%
        \pdfbookmark[\HD@guesstoclevel{\HDorg@glossary@prologue}]%
            {\changehistoryname}{glossary\theHD@unique}%
      \endgroup
    \fi
    \HDorg@glossary@prologue
  }%
  \HDorg@theglossary
}
%    \end{macrocode}
%    \end{macro}
%    \begin{macro}{\glossaryname}
%    \begin{macrocode}
\providecommand*{\glossaryname}{Change History}
%    \end{macrocode}
%    \end{macro}
%
%    \begin{macro}{\glossaryname}
%    \begin{macrocode}
\providecommand*{\changehistoryname}{\glossaryname}
%    \end{macrocode}
%    \end{macro}
%
% \subsection{Index}
%
%    \begin{macrocode}
\newcounter{HD@hypercount}
%    \end{macrocode}
%    \begin{macro}{\HD@helplength}
%    \begin{macrocode}
\newlength{\HD@helplength}
%    \end{macrocode}
%    \end{macro}
%    \begin{macro}{\ifHD@savedest}
%    \begin{macrocode}
\newif\ifHD@savedest
\HD@savedesttrue
%    \end{macrocode}
%    \end{macro}
%    \begin{macrocode}
\AtBeginDocument{%
  \HD@savedest@init
}
%    \end{macrocode}
%    \begin{macrocode}
\RequirePackage{rerunfilecheck}[2009/12/10]
%    \end{macrocode}
%    \begin{macro}{\HD@savedest@init}
%    \begin{macrocode}
\begingroup
  \catcode`\^^M=\active
\@firstofone{\endgroup
  \def\HD@savedest@init{%
    \ifHD@savedest
      \let\HDorg@tilde~%
      \let~\HD@savedest@entry
      \edef\HDorg@catcodeM{\the\catcode`\^^M}%
      \catcode`\^^M=\active
      \def^^M{}%
      \InputIfFileExists{\jobname.hd}{}{}%
      \let~\HDorg@tilde
      \catcode`\^^M=\HDorg@catcodeM\relax
      \newwrite\HD@savedest@out
      \RerunFileCheck{\jobname.hd}{%
        \immediate\closeout\HD@savedest@out
      }{}%
      \immediate\openout\HD@savedest@out=\jobname.hd\relax
    \else
      \let\HD@savedest@add\@gobble
    \fi
  }%
%    \end{macrocode}
%    \end{macro}
%    \begin{macro}{\HD@savedest@entry}
%    \begin{macrocode}
  \def\HD@savedest@entry#1^^M{%
    \@namedef{HD.#1}{}%
  }%
}
%    \end{macrocode}
%    \end{macro}
%    \begin{macro}{\HD@savedest@add}
%    \begin{macrocode}
\def\HD@savedest@add#1{%
  \immediate\write\HD@savedest@out{\string~#1}%
}
%    \end{macrocode}
%    \end{macro}
%    \begin{macro}{\ifHD@targetraise}
%    \begin{macrocode}
\newif\ifHD@targetraise
\HD@targetraisetrue
%    \end{macrocode}
%    \end{macro}
%    \begin{macro}{\HD@target}
%    \begin{macrocode}
\def\HD@target{%
  \ifHD@targetraise
    \ifhmode
      \settoheight{\HD@helplength}{[H]}%
      \addtolength{\HD@helplength}{0.4ex}%
      \raisebox{\HD@helplength}[0pt][0pt]{%
        \HD@maketarget
      }%
    \else
      \HD@maketarget
    \fi
  \else
    \HD@maketarget
  \fi
}
%    \end{macrocode}
%    \end{macro}
%    \begin{macro}{\HD@maketarget}
%    \begin{macrocode}
\def\HD@maketarget{%
  \stepcounter{HD@hypercount}%
  \def\HD@next{%
    \llap{\hypertarget{HD.\the\c@HD@hypercount}{}\,}%
  }%
  \ifHD@savedest
    \expandafter\ifx\csname HD.\the\c@HD@hypercount\endcsname\@empty
    \else
      \let\HD@next\relax
    \fi
  \fi
  \HD@next
}
%    \end{macrocode}
%    \end{macro}
%
%    \begin{macro}{\CodelineIndex}
%    \begin{macrocode}
\g@addto@macro\CodelineIndex{%
  \let\special@index\HD@codeline@wrindex
}
%    \end{macrocode}
%    \end{macro}
%    \begin{macro}{\PageIndex}
%    \begin{macrocode}
\g@addto@macro\PageIndex{%
  \let\special@index\HD@page@wrindex
}
%    \end{macrocode}
%    \end{macro}
%
%    \begin{macro}{\HD@codeline@wrindex}
%    \begin{macrocode}
\def\HD@codeline@wrindex#1{%
  \begingroup
    \let\HDorg@encapchar\encapchar
    \def\encapchar##1\encapchar##2\@nil{%
      \HDorg@encapchar
      hdclindex{\the\c@HD@hypercount}{##1}%
    }%
    \codeline@wrindex{%
      #1\encapchar\encapchar\@nil
    }%
  \endgroup
}
%    \end{macrocode}
%    \end{macro}
%    \begin{macro}{\HD@page@wrindex}
%    \begin{macrocode}
\def\HD@page@wrindex#1{%
  \begingroup
    \let\HDorg@encapchar\encapchar
    \def\encapchar##1\encapchar##2\@nil{%
      \HDorg@encapchar
      hdpindex{##1}%
    }%
    \HDorg@index{#1\encapchar\encapchar\@nil}%
  \endgroup
}
%    \end{macrocode}
%    \end{macro}
%
%    \begin{macro}{\hdclindex}
%    \begin{macrocode}
\def\hdclindex#1#2#3{%
  \csname\ifx\\#2\\relax\else#2\fi\endcsname{%
    \hyperlink{HD.#1}{#3}%
    \HD@savedest@add{#1}%
  }%
}
%    \end{macrocode}
%    \end{macro}
%    \begin{macro}{\hdpindex}
%    \begin{macrocode}
\def\hdpindex#1#2{%
  \csname\ifx\\#1\\relax\else#1\fi\endcsname{%
    \hyperpage{#2}%
  }%
}
%    \end{macrocode}
%    \end{macro}
%
% \subsection{\cs{PrintDescribeMacro}}
%
%    \begin{macro}{\HDorg@PrintDescribeMacro}
%    \begin{macrocode}
\let\HDorg@PrintDescribeMacro\PrintDescribeMacro
%    \end{macrocode}
%    \end{macro}
%    \begin{macro}{\PrintDescribeMacro}
%    \begin{macrocode}
\renewcommand\PrintDescribeMacro[1]{%
  \begingroup
    \let\HDorg@MacroFont\MacroFont
    \def\MacroFont{%
      \HD@target
      \HDorg@MacroFont
    }%
    \HDorg@PrintDescribeMacro{#1}%
  \endgroup
}
%    \end{macrocode}
%    \end{macro}
%
% \subsection{\cs{PrintMacroName}}
%
%    \begin{macro}{\HDorg@PrintMacroName}
%    \begin{macrocode}
\let\HDorg@PrintMacroName\PrintMacroName
%    \end{macrocode}
%    \end{macro}
%    \begin{macro}{\PrintMacroName}
%    \begin{macrocode}
\renewcommand\PrintMacroName[1]{%
  \begingroup
    \let\HDorg@MacroFont\MacroFont
    \def\MacroFont{%
      \HD@target
      \HDorg@MacroFont
    }%
    \HDorg@PrintMacroName{#1}%
  \endgroup
}
%    \end{macrocode}
%    \end{macro}
%
% \subsection{\cs{theCodelineNo}}
%
%    \begin{macro}{\HDorg@theCodelineNo}
%    \begin{macrocode}
\let\HDorg@theCodelineNo\theCodelineNo
%    \end{macrocode}
%    \end{macro}
%    \begin{macro}{\theCodelineNo}
%    \begin{macrocode}
\renewcommand*\theCodelineNo{%
  \settowidth{\HD@helplength}{\HDorg@theCodelineNo\ }%
  \ifdim\HD@helplength<\MacroIndent
    \addtolength{\HD@helplength}{-\MacroIndent}%
    \llap{%
      \begingroup\HD@target\endgroup
      \hspace*{-\HD@helplength}%
    }%
  \else
    \HD@target
  \fi
  \HDorg@theCodelineNo
}
%    \end{macrocode}
%    \end{macro}
%
%    \begin{macro}{\HDorg@index}
%    \begin{macrocode}
\AtBeginDocument{%
  \let\HDorg@index\index
}
%    \end{macrocode}
%    \end{macro}
%
% \subsection{\cs{SpecialUsageIndex}}
%
%    \begin{macro}{\HDorg@SpecialUsageIndex}
%    \begin{macrocode}
\let\HDorg@SpecialUsageIndex\SpecialUsageIndex
\renewcommand*\SpecialUsageIndex[1]{%
  \@bsphack
  \begingroup
    \HD@target
    \let\index\HDorg@index
    \let\HDorg@encapchar\encapchar
    \edef\encapchar usage{%
       \HDorg@encapchar hdclindex{\the\c@HD@hypercount}{usage}%
    }%
    \HDorg@SpecialUsageIndex{#1}%
  \endgroup
  \@esphack
}
%    \end{macrocode}
%    \end{macro}
%
% \subsection{\cs{autoindex}}
%
%    \begin{macrocode}
\begingroup\expandafter\expandafter\expandafter\endgroup
\expandafter\ifx\csname autoindex\endcsname\relax
\else
%    \end{macrocode}
%    \begin{macro}{\HDorg@autoindex}
%    \begin{macrocode}
  \let\HDorg@autoindex\autoindex
%    \end{macrocode}
%    \end{macro}
%    \begin{macro}{\autoindex}
%    \begin{macrocode}
  \renewcommand*\autoindex[1]{%
    \@bsphack
    \begingroup
      \HD@target
      \let\index\HDorg@index
      \let\HDorg@encapchar\encapchar
      \edef\encapchar usage{%
         \HDorg@encapchar hdclindex{\the\c@HD@hypercount}{usage}%
      }%
      \HDorg@autoindex{#1}%
    \endgroup
    \@esphack
  }%
\fi
%    \end{macrocode}
%    \end{macro}
%
% \subsection{\cs{SpecialEnvIndex}}
%
%    \begin{macro}{\HDorg@SpecialEnvIndex}
%    \begin{macrocode}
\let\HDorg@SpecialEnvIndex\SpecialEnvIndex
%    \end{macrocode}
%    \end{macro}
%    \begin{macro}{\SpecialEnvIndex}
%    \begin{macrocode}
\renewcommand*\SpecialEnvIndex[1]{%
  \@bsphack
  \begingroup
    \HD@target
    \let\HDorg@encapchar\encapchar
    \edef\encapchar usage{%
      \HDorg@encapchar hdclindex{\the\c@HD@hypercount}{usage}%
    }%
    \HDorg@SpecialEnvIndex{#1}%
  \endgroup
  \@esphack
}
%    \end{macrocode}
%    \end{macro}
%
% \subsection{\cs{SortIndex}}
%
%    \begin{macro}{\HDorg@SortIndex}
%    \begin{macrocode}
\let\HDorg@SortIndex\SortIndex
%    \end{macrocode}
%    \end{macro}
%    \begin{macro}{\SortIndex}
%    \begin{macrocode}
\renewcommand*{\SortIndex}[2]{%
  \@bsphack
  \begingroup
    \let\index\HD@page@wrindex
    \HDorg@SortIndex{#1}{#2}%
  \endgroup
  \@esphack
}
%    \end{macrocode}
%    \end{macro}
%
% \subsection{\cs{@wrglossary}}
%
%    \begin{macro}{\HDorg@wrglossary}
%    \begin{macrocode}
\let\HDorg@wrglossary\@wrglossary
%    \end{macrocode}
%    \end{macro}
%    \begin{macro}{\@wrglossary}
%    \begin{macrocode}
\def\@wrglossary#1{%
  \let\HDorg@encapchar\encapchar
  \def\encapchar##1\encapchar##2\@nil{%
    \HDorg@encapchar
    hdpindex{##1}%
  }%
  \HDorg@wrglossary{#1\encapchar\encapchar\@nil}%
}
%    \end{macrocode}
%    \end{macro}
%
% \subsection{\cs{MacroIndent}}
%
%    \cs{MacroIndent} automatically calculated.
%    \begin{macrocode}
\AfterLastShipout{%
  \if@filesw
    \immediate\write\@auxout{%
      \string\providecommand\string\HD@SetMacroIndent[1]{}%
    }%
    \immediate\write\@auxout{%
      \string\HD@SetMacroIndent{\number\value{CodelineNo}}%
    }%
  \fi
}
%    \end{macrocode}
%    \begin{macro}{\HD@SetMacroIndent}
%    \begin{macrocode}
\def\HD@SetMacroIndent#1{%
  \begingroup
    \value{CodelineNo}=#1\relax
    \settowidth{\MacroIndent}{\HDorg@theCodelineNo\ }%
    \global\MacroIndent\MacroIndent
  \endgroup
}
%    \end{macrocode}
%    \end{macro}
%
%    \begin{macrocode}
%</package>
%    \end{macrocode}
%
% \section{Installation}
%
% \subsection{Download}
%
% \paragraph{Package.} This package is available on
% CTAN\footnote{\url{http://ctan.org/pkg/hypdoc}}:
% \begin{description}
% \item[\CTAN{macros/latex/contrib/oberdiek/hypdoc.dtx}] The source file.
% \item[\CTAN{macros/latex/contrib/oberdiek/hypdoc.pdf}] Documentation.
% \end{description}
%
%
% \paragraph{Bundle.} All the packages of the bundle `oberdiek'
% are also available in a TDS compliant ZIP archive. There
% the packages are already unpacked and the documentation files
% are generated. The files and directories obey the TDS standard.
% \begin{description}
% \item[\CTAN{install/macros/latex/contrib/oberdiek.tds.zip}]
% \end{description}
% \emph{TDS} refers to the standard ``A Directory Structure
% for \TeX\ Files'' (\CTAN{tds/tds.pdf}). Directories
% with \xfile{texmf} in their name are usually organized this way.
%
% \subsection{Bundle installation}
%
% \paragraph{Unpacking.} Unpack the \xfile{oberdiek.tds.zip} in the
% TDS tree (also known as \xfile{texmf} tree) of your choice.
% Example (linux):
% \begin{quote}
%   |unzip oberdiek.tds.zip -d ~/texmf|
% \end{quote}
%
% \paragraph{Script installation.}
% Check the directory \xfile{TDS:scripts/oberdiek/} for
% scripts that need further installation steps.
% Package \xpackage{attachfile2} comes with the Perl script
% \xfile{pdfatfi.pl} that should be installed in such a way
% that it can be called as \texttt{pdfatfi}.
% Example (linux):
% \begin{quote}
%   |chmod +x scripts/oberdiek/pdfatfi.pl|\\
%   |cp scripts/oberdiek/pdfatfi.pl /usr/local/bin/|
% \end{quote}
%
% \subsection{Package installation}
%
% \paragraph{Unpacking.} The \xfile{.dtx} file is a self-extracting
% \docstrip\ archive. The files are extracted by running the
% \xfile{.dtx} through \plainTeX:
% \begin{quote}
%   \verb|tex hypdoc.dtx|
% \end{quote}
%
% \paragraph{TDS.} Now the different files must be moved into
% the different directories in your installation TDS tree
% (also known as \xfile{texmf} tree):
% \begin{quote}
% \def\t{^^A
% \begin{tabular}{@{}>{\ttfamily}l@{ $\rightarrow$ }>{\ttfamily}l@{}}
%   hypdoc.sty & tex/latex/oberdiek/hypdoc.sty\\
%   hypdoc.pdf & doc/latex/oberdiek/hypdoc.pdf\\
%   hypdoc.dtx & source/latex/oberdiek/hypdoc.dtx\\
% \end{tabular}^^A
% }^^A
% \sbox0{\t}^^A
% \ifdim\wd0>\linewidth
%   \begingroup
%     \advance\linewidth by\leftmargin
%     \advance\linewidth by\rightmargin
%   \edef\x{\endgroup
%     \def\noexpand\lw{\the\linewidth}^^A
%   }\x
%   \def\lwbox{^^A
%     \leavevmode
%     \hbox to \linewidth{^^A
%       \kern-\leftmargin\relax
%       \hss
%       \usebox0
%       \hss
%       \kern-\rightmargin\relax
%     }^^A
%   }^^A
%   \ifdim\wd0>\lw
%     \sbox0{\small\t}^^A
%     \ifdim\wd0>\linewidth
%       \ifdim\wd0>\lw
%         \sbox0{\footnotesize\t}^^A
%         \ifdim\wd0>\linewidth
%           \ifdim\wd0>\lw
%             \sbox0{\scriptsize\t}^^A
%             \ifdim\wd0>\linewidth
%               \ifdim\wd0>\lw
%                 \sbox0{\tiny\t}^^A
%                 \ifdim\wd0>\linewidth
%                   \lwbox
%                 \else
%                   \usebox0
%                 \fi
%               \else
%                 \lwbox
%               \fi
%             \else
%               \usebox0
%             \fi
%           \else
%             \lwbox
%           \fi
%         \else
%           \usebox0
%         \fi
%       \else
%         \lwbox
%       \fi
%     \else
%       \usebox0
%     \fi
%   \else
%     \lwbox
%   \fi
% \else
%   \usebox0
% \fi
% \end{quote}
% If you have a \xfile{docstrip.cfg} that configures and enables \docstrip's
% TDS installing feature, then some files can already be in the right
% place, see the documentation of \docstrip.
%
% \subsection{Refresh file name databases}
%
% If your \TeX~distribution
% (\teTeX, \mikTeX, \dots) relies on file name databases, you must refresh
% these. For example, \teTeX\ users run \verb|texhash| or
% \verb|mktexlsr|.
%
% \subsection{Some details for the interested}
%
% \paragraph{Attached source.}
%
% The PDF documentation on CTAN also includes the
% \xfile{.dtx} source file. It can be extracted by
% AcrobatReader 6 or higher. Another option is \textsf{pdftk},
% e.g. unpack the file into the current directory:
% \begin{quote}
%   \verb|pdftk hypdoc.pdf unpack_files output .|
% \end{quote}
%
% \paragraph{Unpacking with \LaTeX.}
% The \xfile{.dtx} chooses its action depending on the format:
% \begin{description}
% \item[\plainTeX:] Run \docstrip\ and extract the files.
% \item[\LaTeX:] Generate the documentation.
% \end{description}
% If you insist on using \LaTeX\ for \docstrip\ (really,
% \docstrip\ does not need \LaTeX), then inform the autodetect routine
% about your intention:
% \begin{quote}
%   \verb|latex \let\install=y% \iffalse meta-comment
%
% File: hypdoc.dtx
% Version: 2018/09/07 v1.14
% Info: Hyper extensions for doc.sty
%
% Copyright (C) 2002, 2006, 2007, 2009-2011 by
%    Heiko Oberdiek <heiko.oberdiek at googlemail.com>
%    2016
%    https://github.com/ho-tex/oberdiek/issues
%
% This work may be distributed and/or modified under the
% conditions of the LaTeX Project Public License, either
% version 1.3c of this license or (at your option) any later
% version. This version of this license is in
%    http://www.latex-project.org/lppl/lppl-1-3c.txt
% and the latest version of this license is in
%    http://www.latex-project.org/lppl.txt
% and version 1.3 or later is part of all distributions of
% LaTeX version 2005/12/01 or later.
%
% This work has the LPPL maintenance status "maintained".
%
% This Current Maintainer of this work is Heiko Oberdiek.
%
% This work consists of the main source file hypdoc.dtx
% and the derived files
%    hypdoc.sty, hypdoc.pdf, hypdoc.ins, hypdoc.drv.
%
% Distribution:
%    CTAN:macros/latex/contrib/oberdiek/hypdoc.dtx
%    CTAN:macros/latex/contrib/oberdiek/hypdoc.pdf
%
% Unpacking:
%    (a) If hypdoc.ins is present:
%           tex hypdoc.ins
%    (b) Without hypdoc.ins:
%           tex hypdoc.dtx
%    (c) If you insist on using LaTeX
%           latex \let\install=y\input{hypdoc.dtx}
%        (quote the arguments according to the demands of your shell)
%
% Documentation:
%    (a) If hypdoc.drv is present:
%           latex hypdoc.drv
%    (b) Without hypdoc.drv:
%           latex hypdoc.dtx; ...
%    The class ltxdoc loads the configuration file ltxdoc.cfg
%    if available. Here you can specify further options, e.g.
%    use A4 as paper format:
%       \PassOptionsToClass{a4paper}{article}
%
%    Programm calls to get the documentation (example):
%       pdflatex hypdoc.dtx
%       makeindex -s gind.ist hypdoc.idx
%       pdflatex hypdoc.dtx
%       makeindex -s gind.ist hypdoc.idx
%       pdflatex hypdoc.dtx
%
% Installation:
%    TDS:tex/latex/oberdiek/hypdoc.sty
%    TDS:doc/latex/oberdiek/hypdoc.pdf
%    TDS:source/latex/oberdiek/hypdoc.dtx
%
%<*ignore>
\begingroup
  \catcode123=1 %
  \catcode125=2 %
  \def\x{LaTeX2e}%
\expandafter\endgroup
\ifcase 0\ifx\install y1\fi\expandafter
         \ifx\csname processbatchFile\endcsname\relax\else1\fi
         \ifx\fmtname\x\else 1\fi\relax
\else\csname fi\endcsname
%</ignore>
%<*install>
\input docstrip.tex
\Msg{************************************************************************}
\Msg{* Installation}
\Msg{* Package: hypdoc 2018/09/07 v1.14 Hyper extensions for doc.sty (HO)}
\Msg{************************************************************************}

\keepsilent
\askforoverwritefalse

\let\MetaPrefix\relax
\preamble

This is a generated file.

Project: hypdoc
Version: 2018/09/07 v1.14

Copyright (C) 2002, 2006, 2007, 2009-2011 by
   Heiko Oberdiek <heiko.oberdiek at googlemail.com>

This work may be distributed and/or modified under the
conditions of the LaTeX Project Public License, either
version 1.3c of this license or (at your option) any later
version. This version of this license is in
   http://www.latex-project.org/lppl/lppl-1-3c.txt
and the latest version of this license is in
   http://www.latex-project.org/lppl.txt
and version 1.3 or later is part of all distributions of
LaTeX version 2005/12/01 or later.

This work has the LPPL maintenance status "maintained".

This Current Maintainer of this work is Heiko Oberdiek.

This work consists of the main source file hypdoc.dtx
and the derived files
   hypdoc.sty, hypdoc.pdf, hypdoc.ins, hypdoc.drv.

\endpreamble
\let\MetaPrefix\DoubleperCent

\generate{%
  \file{hypdoc.ins}{\from{hypdoc.dtx}{install}}%
  \file{hypdoc.drv}{\from{hypdoc.dtx}{driver}}%
  \usedir{tex/latex/oberdiek}%
  \file{hypdoc.sty}{\from{hypdoc.dtx}{package}}%
  \nopreamble
  \nopostamble
%  \usedir{source/latex/oberdiek/catalogue}%
%  \file{hypdoc.xml}{\from{hypdoc.dtx}{catalogue}}%
}

\catcode32=13\relax% active space
\let =\space%
\Msg{************************************************************************}
\Msg{*}
\Msg{* To finish the installation you have to move the following}
\Msg{* file into a directory searched by TeX:}
\Msg{*}
\Msg{*     hypdoc.sty}
\Msg{*}
\Msg{* To produce the documentation run the file `hypdoc.drv'}
\Msg{* through LaTeX.}
\Msg{*}
\Msg{* Happy TeXing!}
\Msg{*}
\Msg{************************************************************************}

\endbatchfile
%</install>
%<*ignore>
\fi
%</ignore>
%<*driver>
\NeedsTeXFormat{LaTeX2e}
\ProvidesFile{hypdoc.drv}%
  [2018/09/07 v1.14 Hyper extensions for doc.sty (HO)]%
\documentclass{ltxdoc}
\usepackage{holtxdoc}[2011/11/22]
\begin{document}
  \DocInput{hypdoc.dtx}%
\end{document}
%</driver>
% \fi
%
%
% \CharacterTable
%  {Upper-case    \A\B\C\D\E\F\G\H\I\J\K\L\M\N\O\P\Q\R\S\T\U\V\W\X\Y\Z
%   Lower-case    \a\b\c\d\e\f\g\h\i\j\k\l\m\n\o\p\q\r\s\t\u\v\w\x\y\z
%   Digits        \0\1\2\3\4\5\6\7\8\9
%   Exclamation   \!     Double quote  \"     Hash (number) \#
%   Dollar        \$     Percent       \%     Ampersand     \&
%   Acute accent  \'     Left paren    \(     Right paren   \)
%   Asterisk      \*     Plus          \+     Comma         \,
%   Minus         \-     Point         \.     Solidus       \/
%   Colon         \:     Semicolon     \;     Less than     \<
%   Equals        \=     Greater than  \>     Question mark \?
%   Commercial at \@     Left bracket  \[     Backslash     \\
%   Right bracket \]     Circumflex    \^     Underscore    \_
%   Grave accent  \`     Left brace    \{     Vertical bar  \|
%   Right brace   \}     Tilde         \~}
%
% \GetFileInfo{hypdoc.drv}
%
% \title{The \xpackage{hypdoc} package}
% \date{2018/09/07 v1.14}
% \author{Heiko Oberdiek\thanks
% {Please report any issues at https://github.com/ho-tex/oberdiek/issues}\\
% \xemail{heiko.oberdiek at googlemail.com}}
%
% \maketitle
%
% \begin{abstract}
% This package adds hyper features to the package
% \xpackage{doc} that is used in the documentation
% system of \LaTeXe. Bookmarks are added and references
% are linked as far as possible.
% \end{abstract}
%
% \tableofcontents
%
% \section{Features}
%
% The package adds the following features:
% \begin{description}
% \item[Bookmarks:]
%   For supporting bookmarks package \verb|hyperref| is loaded.
%   Some \TeX\ code inside section titles can cause problems
%   during bookmark processing; these can be warnings, ugly
%   bookmark titles or even errors. With the help of
%   \verb|\pdfstringdefDisableCommands| or \verb|\texorpdfstring|
%   you can manage these things, see \verb|hyperref|'s package
%   documentation.
% \item[Additional bookmarks:]
%   Unnumbered bookmark entries are generated for:
%   \begin{itemize}
%   \item Table of contents
%   \item List of figures
%   \item List of tables
%   \item Index
%   \item Glossary
%   \end{itemize}
%   Option \texttt{numbered} controls, whether these section or
%   chapters should be numbered. The exception is the table
%   of contents if this it would be the first numbered
%   section. Then just a bookmark is added.
% \item[Metadata:]
%   The loaded package \verb|hyperref|, enforced by
%   option \verb|pdfusetitle|, tries to detect \verb|\title|
%   and \verb|\author| and sets the corresponding PDF
%   information fields. In case of problems see the previous
%   item about bookmarks.
%   These enries can be overwritten or new ones added by
%   \verb|\hypersetup| after the packages is loaded, see
%   documentation of package \verb|hyperref|.
% \item[References, links:] Because of package \verb|\hyperref|
%   |\ref| become links, also \verb|\url|, footnote referencing
%   are supported.
% \item[Index:]
%   The main part of this package deals with index links in order
%   to support the kind of index that package \verb|doc| provides.
% \item[Thumbnails:]
%   Package \verb|thumbpdf| may additionally be loaded for the case that the
%   PDF file also should contain thumbnails, see the documentation
%   of this package. But thumbnail embedding is not really
%   necessary, because nowadays AcrobatReader is able to
%   generate and view thumbnails on the fly.
% \end{description}
%
% \section{Usage}
%
% The purpose of this packages is to add PDF features to
% the documentation of LaTeX packages. Thus just load
% the package in the driver file that generates the
% documentation:
% \begin{quote}
%   \verb|\usepackage{hypdoc}|
% \end{quote}
% Package \verb|doc| is loaded by \verb|hypdoc|, thus you
% also can replace a \verb|\usepackage{doc}| with this line.
% At any case, the package must be loaded before macros of
% package \verb|doc| such as \verb|\CodelineIndex| or
% \verb|\PageIndex| are used.
%
% If the class \verb|ltxdoc| is used, you can also use
% the configuration file for this class to add
% package \verb|hypdoc|. Add the following line
% to the file \verb|ltxdoc.cfg| (or create a new one):
% \begin{quote}
%\begin{verbatim}
%\AtEndOfClass{\RequirePackage{hypdoc}}
%\end{verbatim}
% \end{quote}
%
% \subsection{Options}
%
% There is one option \texttt{numbered}. It controls,
% whether the starred versions of section or chapter
% inside \cs{tableofcontents}, \cs{listoffigures}, \dots
% should be replaced by the non-star versions.
%
% \subsection{Simple example}
% Example of a simple driver file \verb|foobar.drv|
% of a package \verb|foobar|. Often the driver file
% is packed in the file \verb|foobar.dtx| and can
% be extracted by docstrip or if an instruction file
% for docstrip is provided by:
% \begin{quote}
% \verb|tex foobar.ins|
% \end{quote}
% Note that although docstrip doesn't rely on \LaTeX,
% sometimes you need to run the \verb|.ins| file through
% \LaTeX. Oddly enough, often the only \LaTeX\ feature
% in use is \verb|\NeedsTeXFormat{LaTeX2e}| in the first line.
% \begin{quote}
%\begin{verbatim}
%\documentclass{ltxdoc}
%\usepackage{hypdoc}
%\RecordChanges
%\EnableCrossrefs
%\CodelineIndex
%\begin{document}
%\title{Title for package \texttt{foobar}}
%\author{Author of package \texttt{foobar}}
%\date{...}
%\maketitle
%\tableofcontents
%\DocInput{foobar.dtx}
%\PrintChanges
%\PrintIndex
%\end{document}
%\end{verbatim}
% \end{quote}
%
% Then compile it, for example with pdf\LaTeX.
% \begin{quote}
%\begin{verbatim}
%pdflatex foobar.drv
%makeindex -s gind.ist foobar.idx
%makeindex -s gglo.ist -o foobar.gls foobar.glo
%pdflatex foobar.drv
%\end{verbatim}
% \end{quote}
% You will need several cycles until the cross references
% and similar things are correct.
%
% \subsection{Example \texttt{source2e.tex}}
%
% The source code of the base of \LaTeX\ is available as
% a bunch of \verb|.dtx| files. \LaTeX\ provides the file
% \verb|source2e.tex| to merge them all together.
%
% Now you can either load package \verb|hypdoc| in a private copy
% of this file (please, respect the LPPL) or use the configuration
% file \verb|ltxdoc.cfg|. Example that also sets A4 paper size:
% \begin{quote}
%\begin{verbatim}
%\PassOptionsToClass{a4paper}{article}
%\AtEndOfClass{\RequirePackage{hypdoc}}
%\end{verbatim}
% \end{quote}
%
% Then \verb|source2e.tex| can be compiled by pdf\TeX, for example:
% \begin{quote}
%\begin{verbatim}
%pdflatex source2e
%makeindex -s gind.ist source2e.idx
%makeindex -s gglo.ist -o source2e.glo source2e.gls
%pdflatex source2e
%makeindex -s gind.ist source2e.idx
%makeindex -s gglo.ist -o source2e.glo source2e.gls
%pdflatex source2e
%\end{verbatim}
% \end{quote}
%
% \StopEventually{
% }
%
% \section{Implementation}
%
%    \begin{macrocode}
%<*package>
\NeedsTeXFormat{LaTeX2e}
\ProvidesPackage{hypdoc}%
  [2018/09/07 v1.14 Hyper extensions for doc.sty (HO)]%
%    \end{macrocode}
%    The package identification is done at the top of the |.dtx| file
%    in order to use only one identification string.
%
%    For unique command names this package uses |HD@| as prefix
%
% \subsection{Options}
%
%    \begin{macro}{\ifHD@numbered}
%    \begin{macrocode}
\newif\ifHD@numbered
%    \end{macrocode}
%    \end{macro}
%    \begin{macrocode}
\DeclareOption{numbered}{\HD@numberedtrue}
\ProcessOptions*\relax
%    \end{macrocode}
%
% \subsection{Package loading}
%
%    \begin{macrocode}
\RequirePackage{atveryend}[2010/03/24]
\RequirePackage{doc}
%    \end{macrocode}
%    \begin{macrocode}
\ifx\numexpr\@undefined
  \RequirePackage{calc}%
\fi
%    \end{macrocode}
%    \begin{macrocode}
\RequirePackage[%
  hyperindex=false,%
  pdfusetitle,%
  colorlinks,%
  pdfpagelabels%
]{hyperref}[2002/05/09]%
\hypersetup{%
  plainpages=false,%
  bookmarksopen,%
  bookmarksnumbered,%
  bookmarksopenlevel=0,%
  pdfstartview={FitBH \hypercalcbp{%
    \paperheight-\topmargin-1in-\headheight-\headsep
  }},%
}
%    \end{macrocode}
%
%    \begin{macrocode}
\newcounter{HD@unique}
%    \end{macrocode}
%    \begin{macro}{\theHD@unique}
%    \begin{macrocode}
\renewcommand{\theHD@unique}{%
  \number\value{HD@unique}%
}
%    \end{macrocode}
%    \end{macro}
%
%    \begin{macrocode}
\g@addto@macro\Hy@UseMaketitleInfos{%
  \@ifundefined{fileinfo}{}{%
    \ifx\@pdfsubject\@empty
      \hypersetup{pdfsubject={\fileinfo}}%
    \fi
  }%
}
%    \end{macrocode}
%
%    \begin{macrocode}
\pdfstringdefDisableCommands{%
  \let\thanks\@gobble
  \let\footnote\@gobble
  \def\cs#1{\textbackslash #1}%
  \let\normalfont\@empty
  \let\scshape\@empty
  \def\and{and }%
}
%    \end{macrocode}
%
% \subsubsection{Patch for bug latex/4096}
%
%    \begin{macrocode}
\begingroup\expandafter\expandafter\expandafter\endgroup
\expandafter\ifx\csname mod@math@codes\endcsname\relax
\else
  \def\HD@ModFix#1{%
    \mathcode`#1=`#1\relax
  }%
  \g@addto@macro\mod@math@codes{%
    \HD@ModFix\-%
    \HD@ModFix\+%
    \HD@ModFix\:%
    \HD@ModFix\=%
    \HD@ModFix\*%
  }%
\fi
%    \end{macrocode}
%
% \subsection{Bookmarks}
%
%    \begin{macro}{\HD@sectionpatch}
%    \begin{macrocode}
\def\HD@sectionpatch{%
  \ifHD@numbered
    \HD@@sectionpatch{section}%
    \HD@@sectionpatch{chapter}%
  \else
    \let\Hy@writebookmark\HD@disable@writebookmark
  \fi
}
%    \end{macrocode}
%    \end{macro}
%    \begin{macro}{\HD@@sectionpatch}
%    \begin{macrocode}
\def\HD@@sectionpatch#1{%
  \expandafter\let\csname HDorg@@#1\expandafter\endcsname
    \csname #1\endcsname
  \@namedef{#1}{%
    \@ifstar{\@nameuse{HDorg@@#1}}{\@nameuse{HDorg@@#1}}%
  }%
}
%    \end{macrocode}
%    \end{macro}
%    \begin{macro}{\HD@disable@writebookmark}
%    \begin{macrocode}
\def\HD@disable@writebookmark#1#2#3#4#5{}
%    \end{macrocode}
%    \end{macro}
%
%    \begin{macro}{\HD@guesstoclevel}
%    A wild guessing of the toclevel. There is the hope, that
%    the macro |#1| starts with a sectioning command.
%    \begin{macrocode}
\def\HD@guesstoclevel#1{%
  \expandafter\HD@@guesstoclevel#1\section\HD@end
}
%    \end{macrocode}
%    \end{macro}
%    \begin{macro}{\HD@@guesstoclevel}
%    \begin{macrocode}
\def\HD@@guesstoclevel#1#2\HD@end{%
  \@ifundefined{toclevel@\expandafter\@gobble\string#1\@empty}{%
    1%
  }{%
    \@nameuse{toclevel@\expandafter\@gobble\string#1\@empty}%
  }%
}
%    \end{macrocode}
%    \end{macro}
%
% \subsection{\cs{tableofcontents}}
%
%    \begin{macro}{\HDorg@tableofcontents}
%    \begin{macrocode}
\let\HDorg@tableofcontents\tableofcontents
%    \end{macrocode}
%    \end{macro}
%    \begin{macro}{\tableofcontents}
%    As first section the table of contents shouldn't
%    go into the toc itself. Also a number looks too funny then.
%    Just a bookmark is enough in this case.
%    \begin{macrocode}
\def\tableofcontents{%
  \begingroup
    \@ifundefined{c@section}{%
      \def\c@section{0}%
    }{}%
  \expandafter\endgroup
  \ifcase\c@section
    \stepcounter{HD@unique}%
    \pdfbookmark[\HD@guesstoclevel{\HDorg@tableofcontents}]%
        {\contentsname}{toc\theHD@unique}%
    \HDorg@tableofcontents
  \else
    \ifHD@numbered
    \else
      \stepcounter{HD@unique}%
      \pdfbookmark[\HD@guesstoclevel{\HDorg@tableofcontents}]%
          {\contentsname}{toc\theHD@unique}%
    \fi
    \begingroup
      \HD@sectionpatch
      \HDorg@tableofcontents
    \endgroup
  \fi
}
%    \end{macrocode}
%    \end{macro}
%
% \subsection{\cs{listoffigures}}
%
%    \begin{macro}{\HDorg@listoffigures}
%    \begin{macrocode}
\let\HDorg@listoffigures\listoffigures
%    \end{macrocode}
%    \end{macro}
%    \begin{macro}{\listoffigures}
%    \begin{macrocode}
\def\listoffigures{%
  \ifHD@numbered
  \else
    \stepcounter{HD@unique}%
    \pdfbookmark[\HD@guesstoclevel{\HDorg@listoffigures}]%
        {\listfigurename}{toc\theHD@unique}%
  \fi
  \begingroup
    \HD@sectionpatch
    \HDorg@listoffigures
  \endgroup
}
%    \end{macrocode}
%    \end{macro}
%
% \subsection{\cs{listoftables}}
%
%    \begin{macro}{\HDorg@listoftables}
%    \begin{macrocode}
\let\HDorg@listoftables\listoftables
%    \end{macrocode}
%    \end{macro}
%    \begin{macro}{\listoftables}
%    \begin{macrocode}
\def\listoftables{%
  \ifHD@numbered
  \else
    \stepcounter{HD@unique}%
    \pdfbookmark[\HD@guesstoclevel{\HDorg@listoftables}]%
        {\listtablename}{toc\theHD@unique}%
  \fi
  \begingroup
    \HD@sectionpatch
    \HDorg@listoftables
  \endgroup
}
%    \end{macrocode}
%    \end{macro}
%
% \subsection{\cs{thebibliography}}
%
%    \begin{macro}{\HDorg@thebibliography}
%    \begin{macrocode}
\let\HDorg@thebibliography\thebibliography
%    \end{macrocode}
%    \end{macro}
%    \begin{macro}{\thebibliography}
%    \begin{macrocode}
\def\thebibliography{%
  \ifHD@numbered
  \else
    \stepcounter{HD@unique}%
    \pdfbookmark[\HD@guesstoclevel{\HDorg@thebibliography}]%
      {\@ifundefined{chapter}{\refname}{\bibname}}{bib\theHD@unique}%
  \fi
  \HD@sectionpatch
  \HDorg@thebibliography
}
%    \end{macrocode}
%    \end{macro}
%
% \subsection{\cs{theindex}}
%
%    \begin{macro}{\HDorg@theindex}
%    \begin{macrocode}
\let\HDorg@theindex\theindex
%    \end{macrocode}
%    \end{macro}
%    \begin{macro}{\theindex}
%    \begin{macrocode}
\def\theindex{%
  \let\HDorg@index@prologue\index@prologue
  \let\HDorg@writebookmark\Hy@writebookmark
  \HD@sectionpatch
  \g@addto@macro\IndexParms{%
    \ifHD@numbered
    \else
      \let\Hy@writebookmark\HDorg@writebookmark
    \fi
    \let\bfseries\HD@bfseries
  }%
  \edef\HD@toclevel@index{\HD@guesstoclevel{\index@prologue}}%
  \begingroup
    \count@=\HD@toclevel@index\relax
    \advance\count@ by 1 %
    \xdef\HD@toclevel@subindex{\the\count@}%
  \endgroup
  \def\index@prologue{%
    \ifHD@numbered
    \else
      \begingroup
        \stepcounter{HD@unique}%
        \let\Hy@writebookmark\HDorg@writebookmark
        \pdfbookmark[\HD@toclevel@index]%
            {\indexname}{index\theHD@unique}%
      \endgroup
    \fi
    \HDorg@index@prologue
  }%
  \HDorg@theindex
}
%    \end{macrocode}
%    \end{macro}
%    \begin{macro}{\HDorg@bfseries}
%    \begin{macrocode}
\let\HDorg@bfseries\bfseries
%    \end{macrocode}
%    \end{macro}
%    \begin{macro}{\HD@bfseries}
%    \begin{macrocode}
\def\HD@bfseries{\futurelet\HD@hfil\HD@@bfseries}%
%    \end{macrocode}
%    \end{macro}
%    \begin{macro}{\HD@@bfseries}
%    \begin{macrocode}
\def\HD@@bfseries{%
  \HDorg@bfseries
  \ifx\HD@hfil\hfil
    \expandafter\HD@@@bfseries
  \fi
}
%    \end{macrocode}
%    \end{macro}
%    \begin{macro}{\HD@@@bfseries}
%    \begin{macrocode}
\def\HD@@@bfseries\hfil#1\hfil{%
  \ifx\\#1\\%
  \else
    \raisebox{\baselineskip}[0pt]{%
      \kern-\HD@margin\relax
      \pdfbookmark[\HD@toclevel@subindex]{#1}{HD.#1}%
      \kern\HD@margin\relax
    }%
  \fi
  \hfil#1\hfil
}%
%    \end{macrocode}
%    \end{macro}
%    \begin{macro}{\HD@margin}
%    \begin{macrocode}
\def\HD@margin{1mm}
%    \end{macrocode}
%    \end{macro}
%
% \subsection{\cs{theglossary}}
%
%    \begin{macro}{\HDorg@theglossary}
%    \begin{macrocode}
\let\HDorg@theglossary\theglossary
%    \end{macrocode}
%    \end{macro}
%    \begin{macro}{\theglossary}
%    \begin{macrocode}
\def\theglossary{%
  \let\HDorg@glossary@prologue\glossary@prologue
  \let\HDorg@writebookmark\Hy@writebookmark
  \HD@sectionpatch
  \def\glossary@prologue{%
    \ifHD@numbered
    \else
      \begingroup
        \let\Hy@writebookmark\HDorg@writebookmark
        \stepcounter{HD@unique}%
        \pdfbookmark[\HD@guesstoclevel{\HDorg@glossary@prologue}]%
            {\changehistoryname}{glossary\theHD@unique}%
      \endgroup
    \fi
    \HDorg@glossary@prologue
  }%
  \HDorg@theglossary
}
%    \end{macrocode}
%    \end{macro}
%    \begin{macro}{\glossaryname}
%    \begin{macrocode}
\providecommand*{\glossaryname}{Change History}
%    \end{macrocode}
%    \end{macro}
%
%    \begin{macro}{\glossaryname}
%    \begin{macrocode}
\providecommand*{\changehistoryname}{\glossaryname}
%    \end{macrocode}
%    \end{macro}
%
% \subsection{Index}
%
%    \begin{macrocode}
\newcounter{HD@hypercount}
%    \end{macrocode}
%    \begin{macro}{\HD@helplength}
%    \begin{macrocode}
\newlength{\HD@helplength}
%    \end{macrocode}
%    \end{macro}
%    \begin{macro}{\ifHD@savedest}
%    \begin{macrocode}
\newif\ifHD@savedest
\HD@savedesttrue
%    \end{macrocode}
%    \end{macro}
%    \begin{macrocode}
\AtBeginDocument{%
  \HD@savedest@init
}
%    \end{macrocode}
%    \begin{macrocode}
\RequirePackage{rerunfilecheck}[2009/12/10]
%    \end{macrocode}
%    \begin{macro}{\HD@savedest@init}
%    \begin{macrocode}
\begingroup
  \catcode`\^^M=\active
\@firstofone{\endgroup
  \def\HD@savedest@init{%
    \ifHD@savedest
      \let\HDorg@tilde~%
      \let~\HD@savedest@entry
      \edef\HDorg@catcodeM{\the\catcode`\^^M}%
      \catcode`\^^M=\active
      \def^^M{}%
      \InputIfFileExists{\jobname.hd}{}{}%
      \let~\HDorg@tilde
      \catcode`\^^M=\HDorg@catcodeM\relax
      \newwrite\HD@savedest@out
      \RerunFileCheck{\jobname.hd}{%
        \immediate\closeout\HD@savedest@out
      }{}%
      \immediate\openout\HD@savedest@out=\jobname.hd\relax
    \else
      \let\HD@savedest@add\@gobble
    \fi
  }%
%    \end{macrocode}
%    \end{macro}
%    \begin{macro}{\HD@savedest@entry}
%    \begin{macrocode}
  \def\HD@savedest@entry#1^^M{%
    \@namedef{HD.#1}{}%
  }%
}
%    \end{macrocode}
%    \end{macro}
%    \begin{macro}{\HD@savedest@add}
%    \begin{macrocode}
\def\HD@savedest@add#1{%
  \immediate\write\HD@savedest@out{\string~#1}%
}
%    \end{macrocode}
%    \end{macro}
%    \begin{macro}{\ifHD@targetraise}
%    \begin{macrocode}
\newif\ifHD@targetraise
\HD@targetraisetrue
%    \end{macrocode}
%    \end{macro}
%    \begin{macro}{\HD@target}
%    \begin{macrocode}
\def\HD@target{%
  \ifHD@targetraise
    \ifhmode
      \settoheight{\HD@helplength}{[H]}%
      \addtolength{\HD@helplength}{0.4ex}%
      \raisebox{\HD@helplength}[0pt][0pt]{%
        \HD@maketarget
      }%
    \else
      \HD@maketarget
    \fi
  \else
    \HD@maketarget
  \fi
}
%    \end{macrocode}
%    \end{macro}
%    \begin{macro}{\HD@maketarget}
%    \begin{macrocode}
\def\HD@maketarget{%
  \stepcounter{HD@hypercount}%
  \def\HD@next{%
    \llap{\hypertarget{HD.\the\c@HD@hypercount}{}\,}%
  }%
  \ifHD@savedest
    \expandafter\ifx\csname HD.\the\c@HD@hypercount\endcsname\@empty
    \else
      \let\HD@next\relax
    \fi
  \fi
  \HD@next
}
%    \end{macrocode}
%    \end{macro}
%
%    \begin{macro}{\CodelineIndex}
%    \begin{macrocode}
\g@addto@macro\CodelineIndex{%
  \let\special@index\HD@codeline@wrindex
}
%    \end{macrocode}
%    \end{macro}
%    \begin{macro}{\PageIndex}
%    \begin{macrocode}
\g@addto@macro\PageIndex{%
  \let\special@index\HD@page@wrindex
}
%    \end{macrocode}
%    \end{macro}
%
%    \begin{macro}{\HD@codeline@wrindex}
%    \begin{macrocode}
\def\HD@codeline@wrindex#1{%
  \begingroup
    \let\HDorg@encapchar\encapchar
    \def\encapchar##1\encapchar##2\@nil{%
      \HDorg@encapchar
      hdclindex{\the\c@HD@hypercount}{##1}%
    }%
    \codeline@wrindex{%
      #1\encapchar\encapchar\@nil
    }%
  \endgroup
}
%    \end{macrocode}
%    \end{macro}
%    \begin{macro}{\HD@page@wrindex}
%    \begin{macrocode}
\def\HD@page@wrindex#1{%
  \begingroup
    \let\HDorg@encapchar\encapchar
    \def\encapchar##1\encapchar##2\@nil{%
      \HDorg@encapchar
      hdpindex{##1}%
    }%
    \HDorg@index{#1\encapchar\encapchar\@nil}%
  \endgroup
}
%    \end{macrocode}
%    \end{macro}
%
%    \begin{macro}{\hdclindex}
%    \begin{macrocode}
\def\hdclindex#1#2#3{%
  \csname\ifx\\#2\\relax\else#2\fi\endcsname{%
    \hyperlink{HD.#1}{#3}%
    \HD@savedest@add{#1}%
  }%
}
%    \end{macrocode}
%    \end{macro}
%    \begin{macro}{\hdpindex}
%    \begin{macrocode}
\def\hdpindex#1#2{%
  \csname\ifx\\#1\\relax\else#1\fi\endcsname{%
    \hyperpage{#2}%
  }%
}
%    \end{macrocode}
%    \end{macro}
%
% \subsection{\cs{PrintDescribeMacro}}
%
%    \begin{macro}{\HDorg@PrintDescribeMacro}
%    \begin{macrocode}
\let\HDorg@PrintDescribeMacro\PrintDescribeMacro
%    \end{macrocode}
%    \end{macro}
%    \begin{macro}{\PrintDescribeMacro}
%    \begin{macrocode}
\renewcommand\PrintDescribeMacro[1]{%
  \begingroup
    \let\HDorg@MacroFont\MacroFont
    \def\MacroFont{%
      \HD@target
      \HDorg@MacroFont
    }%
    \HDorg@PrintDescribeMacro{#1}%
  \endgroup
}
%    \end{macrocode}
%    \end{macro}
%
% \subsection{\cs{PrintMacroName}}
%
%    \begin{macro}{\HDorg@PrintMacroName}
%    \begin{macrocode}
\let\HDorg@PrintMacroName\PrintMacroName
%    \end{macrocode}
%    \end{macro}
%    \begin{macro}{\PrintMacroName}
%    \begin{macrocode}
\renewcommand\PrintMacroName[1]{%
  \begingroup
    \let\HDorg@MacroFont\MacroFont
    \def\MacroFont{%
      \HD@target
      \HDorg@MacroFont
    }%
    \HDorg@PrintMacroName{#1}%
  \endgroup
}
%    \end{macrocode}
%    \end{macro}
%
% \subsection{\cs{theCodelineNo}}
%
%    \begin{macro}{\HDorg@theCodelineNo}
%    \begin{macrocode}
\let\HDorg@theCodelineNo\theCodelineNo
%    \end{macrocode}
%    \end{macro}
%    \begin{macro}{\theCodelineNo}
%    \begin{macrocode}
\renewcommand*\theCodelineNo{%
  \settowidth{\HD@helplength}{\HDorg@theCodelineNo\ }%
  \ifdim\HD@helplength<\MacroIndent
    \addtolength{\HD@helplength}{-\MacroIndent}%
    \llap{%
      \begingroup\HD@target\endgroup
      \hspace*{-\HD@helplength}%
    }%
  \else
    \HD@target
  \fi
  \HDorg@theCodelineNo
}
%    \end{macrocode}
%    \end{macro}
%
%    \begin{macro}{\HDorg@index}
%    \begin{macrocode}
\AtBeginDocument{%
  \let\HDorg@index\index
}
%    \end{macrocode}
%    \end{macro}
%
% \subsection{\cs{SpecialUsageIndex}}
%
%    \begin{macro}{\HDorg@SpecialUsageIndex}
%    \begin{macrocode}
\let\HDorg@SpecialUsageIndex\SpecialUsageIndex
\renewcommand*\SpecialUsageIndex[1]{%
  \@bsphack
  \begingroup
    \HD@target
    \let\index\HDorg@index
    \let\HDorg@encapchar\encapchar
    \edef\encapchar usage{%
       \HDorg@encapchar hdclindex{\the\c@HD@hypercount}{usage}%
    }%
    \HDorg@SpecialUsageIndex{#1}%
  \endgroup
  \@esphack
}
%    \end{macrocode}
%    \end{macro}
%
% \subsection{\cs{autoindex}}
%
%    \begin{macrocode}
\begingroup\expandafter\expandafter\expandafter\endgroup
\expandafter\ifx\csname autoindex\endcsname\relax
\else
%    \end{macrocode}
%    \begin{macro}{\HDorg@autoindex}
%    \begin{macrocode}
  \let\HDorg@autoindex\autoindex
%    \end{macrocode}
%    \end{macro}
%    \begin{macro}{\autoindex}
%    \begin{macrocode}
  \renewcommand*\autoindex[1]{%
    \@bsphack
    \begingroup
      \HD@target
      \let\index\HDorg@index
      \let\HDorg@encapchar\encapchar
      \edef\encapchar usage{%
         \HDorg@encapchar hdclindex{\the\c@HD@hypercount}{usage}%
      }%
      \HDorg@autoindex{#1}%
    \endgroup
    \@esphack
  }%
\fi
%    \end{macrocode}
%    \end{macro}
%
% \subsection{\cs{SpecialEnvIndex}}
%
%    \begin{macro}{\HDorg@SpecialEnvIndex}
%    \begin{macrocode}
\let\HDorg@SpecialEnvIndex\SpecialEnvIndex
%    \end{macrocode}
%    \end{macro}
%    \begin{macro}{\SpecialEnvIndex}
%    \begin{macrocode}
\renewcommand*\SpecialEnvIndex[1]{%
  \@bsphack
  \begingroup
    \HD@target
    \let\HDorg@encapchar\encapchar
    \edef\encapchar usage{%
      \HDorg@encapchar hdclindex{\the\c@HD@hypercount}{usage}%
    }%
    \HDorg@SpecialEnvIndex{#1}%
  \endgroup
  \@esphack
}
%    \end{macrocode}
%    \end{macro}
%
% \subsection{\cs{SortIndex}}
%
%    \begin{macro}{\HDorg@SortIndex}
%    \begin{macrocode}
\let\HDorg@SortIndex\SortIndex
%    \end{macrocode}
%    \end{macro}
%    \begin{macro}{\SortIndex}
%    \begin{macrocode}
\renewcommand*{\SortIndex}[2]{%
  \@bsphack
  \begingroup
    \let\index\HD@page@wrindex
    \HDorg@SortIndex{#1}{#2}%
  \endgroup
  \@esphack
}
%    \end{macrocode}
%    \end{macro}
%
% \subsection{\cs{@wrglossary}}
%
%    \begin{macro}{\HDorg@wrglossary}
%    \begin{macrocode}
\let\HDorg@wrglossary\@wrglossary
%    \end{macrocode}
%    \end{macro}
%    \begin{macro}{\@wrglossary}
%    \begin{macrocode}
\def\@wrglossary#1{%
  \let\HDorg@encapchar\encapchar
  \def\encapchar##1\encapchar##2\@nil{%
    \HDorg@encapchar
    hdpindex{##1}%
  }%
  \HDorg@wrglossary{#1\encapchar\encapchar\@nil}%
}
%    \end{macrocode}
%    \end{macro}
%
% \subsection{\cs{MacroIndent}}
%
%    \cs{MacroIndent} automatically calculated.
%    \begin{macrocode}
\AfterLastShipout{%
  \if@filesw
    \immediate\write\@auxout{%
      \string\providecommand\string\HD@SetMacroIndent[1]{}%
    }%
    \immediate\write\@auxout{%
      \string\HD@SetMacroIndent{\number\value{CodelineNo}}%
    }%
  \fi
}
%    \end{macrocode}
%    \begin{macro}{\HD@SetMacroIndent}
%    \begin{macrocode}
\def\HD@SetMacroIndent#1{%
  \begingroup
    \value{CodelineNo}=#1\relax
    \settowidth{\MacroIndent}{\HDorg@theCodelineNo\ }%
    \global\MacroIndent\MacroIndent
  \endgroup
}
%    \end{macrocode}
%    \end{macro}
%
%    \begin{macrocode}
%</package>
%    \end{macrocode}
%
% \section{Installation}
%
% \subsection{Download}
%
% \paragraph{Package.} This package is available on
% CTAN\footnote{\url{http://ctan.org/pkg/hypdoc}}:
% \begin{description}
% \item[\CTAN{macros/latex/contrib/oberdiek/hypdoc.dtx}] The source file.
% \item[\CTAN{macros/latex/contrib/oberdiek/hypdoc.pdf}] Documentation.
% \end{description}
%
%
% \paragraph{Bundle.} All the packages of the bundle `oberdiek'
% are also available in a TDS compliant ZIP archive. There
% the packages are already unpacked and the documentation files
% are generated. The files and directories obey the TDS standard.
% \begin{description}
% \item[\CTAN{install/macros/latex/contrib/oberdiek.tds.zip}]
% \end{description}
% \emph{TDS} refers to the standard ``A Directory Structure
% for \TeX\ Files'' (\CTAN{tds/tds.pdf}). Directories
% with \xfile{texmf} in their name are usually organized this way.
%
% \subsection{Bundle installation}
%
% \paragraph{Unpacking.} Unpack the \xfile{oberdiek.tds.zip} in the
% TDS tree (also known as \xfile{texmf} tree) of your choice.
% Example (linux):
% \begin{quote}
%   |unzip oberdiek.tds.zip -d ~/texmf|
% \end{quote}
%
% \paragraph{Script installation.}
% Check the directory \xfile{TDS:scripts/oberdiek/} for
% scripts that need further installation steps.
% Package \xpackage{attachfile2} comes with the Perl script
% \xfile{pdfatfi.pl} that should be installed in such a way
% that it can be called as \texttt{pdfatfi}.
% Example (linux):
% \begin{quote}
%   |chmod +x scripts/oberdiek/pdfatfi.pl|\\
%   |cp scripts/oberdiek/pdfatfi.pl /usr/local/bin/|
% \end{quote}
%
% \subsection{Package installation}
%
% \paragraph{Unpacking.} The \xfile{.dtx} file is a self-extracting
% \docstrip\ archive. The files are extracted by running the
% \xfile{.dtx} through \plainTeX:
% \begin{quote}
%   \verb|tex hypdoc.dtx|
% \end{quote}
%
% \paragraph{TDS.} Now the different files must be moved into
% the different directories in your installation TDS tree
% (also known as \xfile{texmf} tree):
% \begin{quote}
% \def\t{^^A
% \begin{tabular}{@{}>{\ttfamily}l@{ $\rightarrow$ }>{\ttfamily}l@{}}
%   hypdoc.sty & tex/latex/oberdiek/hypdoc.sty\\
%   hypdoc.pdf & doc/latex/oberdiek/hypdoc.pdf\\
%   hypdoc.dtx & source/latex/oberdiek/hypdoc.dtx\\
% \end{tabular}^^A
% }^^A
% \sbox0{\t}^^A
% \ifdim\wd0>\linewidth
%   \begingroup
%     \advance\linewidth by\leftmargin
%     \advance\linewidth by\rightmargin
%   \edef\x{\endgroup
%     \def\noexpand\lw{\the\linewidth}^^A
%   }\x
%   \def\lwbox{^^A
%     \leavevmode
%     \hbox to \linewidth{^^A
%       \kern-\leftmargin\relax
%       \hss
%       \usebox0
%       \hss
%       \kern-\rightmargin\relax
%     }^^A
%   }^^A
%   \ifdim\wd0>\lw
%     \sbox0{\small\t}^^A
%     \ifdim\wd0>\linewidth
%       \ifdim\wd0>\lw
%         \sbox0{\footnotesize\t}^^A
%         \ifdim\wd0>\linewidth
%           \ifdim\wd0>\lw
%             \sbox0{\scriptsize\t}^^A
%             \ifdim\wd0>\linewidth
%               \ifdim\wd0>\lw
%                 \sbox0{\tiny\t}^^A
%                 \ifdim\wd0>\linewidth
%                   \lwbox
%                 \else
%                   \usebox0
%                 \fi
%               \else
%                 \lwbox
%               \fi
%             \else
%               \usebox0
%             \fi
%           \else
%             \lwbox
%           \fi
%         \else
%           \usebox0
%         \fi
%       \else
%         \lwbox
%       \fi
%     \else
%       \usebox0
%     \fi
%   \else
%     \lwbox
%   \fi
% \else
%   \usebox0
% \fi
% \end{quote}
% If you have a \xfile{docstrip.cfg} that configures and enables \docstrip's
% TDS installing feature, then some files can already be in the right
% place, see the documentation of \docstrip.
%
% \subsection{Refresh file name databases}
%
% If your \TeX~distribution
% (\teTeX, \mikTeX, \dots) relies on file name databases, you must refresh
% these. For example, \teTeX\ users run \verb|texhash| or
% \verb|mktexlsr|.
%
% \subsection{Some details for the interested}
%
% \paragraph{Attached source.}
%
% The PDF documentation on CTAN also includes the
% \xfile{.dtx} source file. It can be extracted by
% AcrobatReader 6 or higher. Another option is \textsf{pdftk},
% e.g. unpack the file into the current directory:
% \begin{quote}
%   \verb|pdftk hypdoc.pdf unpack_files output .|
% \end{quote}
%
% \paragraph{Unpacking with \LaTeX.}
% The \xfile{.dtx} chooses its action depending on the format:
% \begin{description}
% \item[\plainTeX:] Run \docstrip\ and extract the files.
% \item[\LaTeX:] Generate the documentation.
% \end{description}
% If you insist on using \LaTeX\ for \docstrip\ (really,
% \docstrip\ does not need \LaTeX), then inform the autodetect routine
% about your intention:
% \begin{quote}
%   \verb|latex \let\install=y\input{hypdoc.dtx}|
% \end{quote}
% Do not forget to quote the argument according to the demands
% of your shell.
%
% \paragraph{Generating the documentation.}
% You can use both the \xfile{.dtx} or the \xfile{.drv} to generate
% the documentation. The process can be configured by the
% configuration file \xfile{ltxdoc.cfg}. For instance, put this
% line into this file, if you want to have A4 as paper format:
% \begin{quote}
%   \verb|\PassOptionsToClass{a4paper}{article}|
% \end{quote}
% An example follows how to generate the
% documentation with pdf\LaTeX:
% \begin{quote}
%\begin{verbatim}
%pdflatex hypdoc.dtx
%makeindex -s gind.ist hypdoc.idx
%pdflatex hypdoc.dtx
%makeindex -s gind.ist hypdoc.idx
%pdflatex hypdoc.dtx
%\end{verbatim}
% \end{quote}
%
% \section{Catalogue}
%
% The following XML file can be used as source for the
% \href{http://mirror.ctan.org/help/Catalogue/catalogue.html}{\TeX\ Catalogue}.
% The elements \texttt{caption} and \texttt{description} are imported
% from the original XML file from the Catalogue.
% The name of the XML file in the Catalogue is \xfile{hypdoc.xml}.
%    \begin{macrocode}
%<*catalogue>
<?xml version='1.0' encoding='us-ascii'?>
<!DOCTYPE entry SYSTEM 'catalogue.dtd'>
<entry datestamp='$Date$' modifier='$Author$' id='hypdoc'>
  <name>hypdoc</name>
  <caption>Hyper extensions for doc.sty.</caption>
  <authorref id='auth:oberdiek'/>
  <copyright owner='Heiko Oberdiek' year='2002,2006,2007,2009-2011'/>
  <license type='lppl1.3'/>
  <version number='1.14'/>
  <description>
    This package adds hypertext features to the package
    <xref refid='doc'>doc</xref> that is used in the documentation
    system of <xref refid='latex'>LaTeX2e</xref>. Bookmarks are added
    and references are linked as far as possible.
    <p/>
    The package is part of the <xref refid='oberdiek'>oberdiek</xref>
    bundle.
  </description>
  <documentation details='Package documentation'
      href='ctan:/macros/latex/contrib/oberdiek/hypdoc.pdf'/>
  <ctan file='true' path='/macros/latex/contrib/oberdiek/hypdoc.dtx'/>
  <miktex location='oberdiek'/>
  <texlive location='oberdiek'/>
  <install path='/macros/latex/contrib/oberdiek/oberdiek.tds.zip'/>
</entry>
%</catalogue>
%    \end{macrocode}
%
% \begin{History}
%   \begin{Version}{2002/05/10 v1.0}
%   \item
%     First standalone version.
%   \end{Version}
%   \begin{Version}{2006/02/20 v1.1}
%   \item
%     Option \xoption{numbered} added.
%   \item
%     \cs{HD@savedest@init} fixed (empty \xext{hd} file triggered
%     first page with fl ligature).
%   \item
%     Bookmarks for \cs{listoffigures} and \cs{listoftables}.
%   \item
%     Documentation: features and usage added.
%   \item
%     LPPL 1.3.
%   \end{Version}
%   \begin{Version}{2006/04/27 v1.2}
%   \item
%     \cs{HD@target} fixed (\xfile{multicol.dtx}).
%   \end{Version}
%   \begin{Version}{2006/06/01 v1.3}
%   \item
%     Support for package \xpackage{amsdtx}'s \cs{autoindex} added.
%   \end{Version}
%   \begin{Version}{2006/06/02 v1.4}
%   \item
%     Bookmark fixed if there are several table of contents,
%     bibliographies, glossaries, \dots
%   \end{Version}
%   \begin{Version}{2007/04/11 v1.5}
%   \item
%     Line ends sanitized.
%   \end{Version}
%   \begin{Version}{2007/11/12 v1.6}
%   \item
%     Bug fix in \cs{listoffigures} (Axel Sommerfeldt).
%   \item
%     Markup added in implementation section.
%   \end{Version}
%   \begin{Version}{2009/12/10 v1.7}
%   \item
%     Use of package \xpackage{rerunfilecheck}.
%   \end{Version}
%   \begin{Version}{2010/02/03 v1.8}
%   \item
%     Fix for bug latex/4096 added.
%   \end{Version}
%   \begin{Version}{2010/03/26 v1.9}
%   \item
%     Definition of \cs{HD@SetMacroIndex} is provided in the \xext{aux} file.
%   \end{Version}
%   \begin{Version}{2011/05/05 v1.10}
%   \item
%     \cs{raisebox} in \cs{HD@target} with height and depth of \texttt{0pt}
%     (request of GL).
%   \end{Version}
%   \begin{Version}{2011/08/19 v1.11}
%   \item
%     Latest comma in option list for hyperref removed to avoid option clash
%     because of \hologo{LaTeX} 2011/06/27 with the new \cs{in@}.
%   \end{Version}
%   \begin{Version}{2016/05/16 v1.12}
%   \item
%     Documentation updates.
%   \end{Version}
%   \begin{Version}{2016/10/02 v1.13}
%   \item
%     Fixes to bookmarknames, also introduce \cs{changehistoryname}
%   \end{Version}
%   \begin{Version}{2018/09/07 v1.14}
%   \item
%     Do not load \xpackage{thumpdf}.
%   \end{Version}
% \end{History}
%
% \PrintIndex
%
% \Finale
\endinput
|
% \end{quote}
% Do not forget to quote the argument according to the demands
% of your shell.
%
% \paragraph{Generating the documentation.}
% You can use both the \xfile{.dtx} or the \xfile{.drv} to generate
% the documentation. The process can be configured by the
% configuration file \xfile{ltxdoc.cfg}. For instance, put this
% line into this file, if you want to have A4 as paper format:
% \begin{quote}
%   \verb|\PassOptionsToClass{a4paper}{article}|
% \end{quote}
% An example follows how to generate the
% documentation with pdf\LaTeX:
% \begin{quote}
%\begin{verbatim}
%pdflatex hypdoc.dtx
%makeindex -s gind.ist hypdoc.idx
%pdflatex hypdoc.dtx
%makeindex -s gind.ist hypdoc.idx
%pdflatex hypdoc.dtx
%\end{verbatim}
% \end{quote}
%
% \section{Catalogue}
%
% The following XML file can be used as source for the
% \href{http://mirror.ctan.org/help/Catalogue/catalogue.html}{\TeX\ Catalogue}.
% The elements \texttt{caption} and \texttt{description} are imported
% from the original XML file from the Catalogue.
% The name of the XML file in the Catalogue is \xfile{hypdoc.xml}.
%    \begin{macrocode}
%<*catalogue>
<?xml version='1.0' encoding='us-ascii'?>
<!DOCTYPE entry SYSTEM 'catalogue.dtd'>
<entry datestamp='$Date$' modifier='$Author$' id='hypdoc'>
  <name>hypdoc</name>
  <caption>Hyper extensions for doc.sty.</caption>
  <authorref id='auth:oberdiek'/>
  <copyright owner='Heiko Oberdiek' year='2002,2006,2007,2009-2011'/>
  <license type='lppl1.3'/>
  <version number='1.14'/>
  <description>
    This package adds hypertext features to the package
    <xref refid='doc'>doc</xref> that is used in the documentation
    system of <xref refid='latex'>LaTeX2e</xref>. Bookmarks are added
    and references are linked as far as possible.
    <p/>
    The package is part of the <xref refid='oberdiek'>oberdiek</xref>
    bundle.
  </description>
  <documentation details='Package documentation'
      href='ctan:/macros/latex/contrib/oberdiek/hypdoc.pdf'/>
  <ctan file='true' path='/macros/latex/contrib/oberdiek/hypdoc.dtx'/>
  <miktex location='oberdiek'/>
  <texlive location='oberdiek'/>
  <install path='/macros/latex/contrib/oberdiek/oberdiek.tds.zip'/>
</entry>
%</catalogue>
%    \end{macrocode}
%
% \begin{History}
%   \begin{Version}{2002/05/10 v1.0}
%   \item
%     First standalone version.
%   \end{Version}
%   \begin{Version}{2006/02/20 v1.1}
%   \item
%     Option \xoption{numbered} added.
%   \item
%     \cs{HD@savedest@init} fixed (empty \xext{hd} file triggered
%     first page with fl ligature).
%   \item
%     Bookmarks for \cs{listoffigures} and \cs{listoftables}.
%   \item
%     Documentation: features and usage added.
%   \item
%     LPPL 1.3.
%   \end{Version}
%   \begin{Version}{2006/04/27 v1.2}
%   \item
%     \cs{HD@target} fixed (\xfile{multicol.dtx}).
%   \end{Version}
%   \begin{Version}{2006/06/01 v1.3}
%   \item
%     Support for package \xpackage{amsdtx}'s \cs{autoindex} added.
%   \end{Version}
%   \begin{Version}{2006/06/02 v1.4}
%   \item
%     Bookmark fixed if there are several table of contents,
%     bibliographies, glossaries, \dots
%   \end{Version}
%   \begin{Version}{2007/04/11 v1.5}
%   \item
%     Line ends sanitized.
%   \end{Version}
%   \begin{Version}{2007/11/12 v1.6}
%   \item
%     Bug fix in \cs{listoffigures} (Axel Sommerfeldt).
%   \item
%     Markup added in implementation section.
%   \end{Version}
%   \begin{Version}{2009/12/10 v1.7}
%   \item
%     Use of package \xpackage{rerunfilecheck}.
%   \end{Version}
%   \begin{Version}{2010/02/03 v1.8}
%   \item
%     Fix for bug latex/4096 added.
%   \end{Version}
%   \begin{Version}{2010/03/26 v1.9}
%   \item
%     Definition of \cs{HD@SetMacroIndex} is provided in the \xext{aux} file.
%   \end{Version}
%   \begin{Version}{2011/05/05 v1.10}
%   \item
%     \cs{raisebox} in \cs{HD@target} with height and depth of \texttt{0pt}
%     (request of GL).
%   \end{Version}
%   \begin{Version}{2011/08/19 v1.11}
%   \item
%     Latest comma in option list for hyperref removed to avoid option clash
%     because of \hologo{LaTeX} 2011/06/27 with the new \cs{in@}.
%   \end{Version}
%   \begin{Version}{2016/05/16 v1.12}
%   \item
%     Documentation updates.
%   \end{Version}
%   \begin{Version}{2016/10/02 v1.13}
%   \item
%     Fixes to bookmarknames, also introduce \cs{changehistoryname}
%   \end{Version}
%   \begin{Version}{2018/09/07 v1.14}
%   \item
%     Do not load \xpackage{thumpdf}.
%   \end{Version}
% \end{History}
%
% \PrintIndex
%
% \Finale
\endinput

%        (quote the arguments according to the demands of your shell)
%
% Documentation:
%    (a) If hypdoc.drv is present:
%           latex hypdoc.drv
%    (b) Without hypdoc.drv:
%           latex hypdoc.dtx; ...
%    The class ltxdoc loads the configuration file ltxdoc.cfg
%    if available. Here you can specify further options, e.g.
%    use A4 as paper format:
%       \PassOptionsToClass{a4paper}{article}
%
%    Programm calls to get the documentation (example):
%       pdflatex hypdoc.dtx
%       makeindex -s gind.ist hypdoc.idx
%       pdflatex hypdoc.dtx
%       makeindex -s gind.ist hypdoc.idx
%       pdflatex hypdoc.dtx
%
% Installation:
%    TDS:tex/latex/oberdiek/hypdoc.sty
%    TDS:doc/latex/oberdiek/hypdoc.pdf
%    TDS:source/latex/oberdiek/hypdoc.dtx
%
%<*ignore>
\begingroup
  \catcode123=1 %
  \catcode125=2 %
  \def\x{LaTeX2e}%
\expandafter\endgroup
\ifcase 0\ifx\install y1\fi\expandafter
         \ifx\csname processbatchFile\endcsname\relax\else1\fi
         \ifx\fmtname\x\else 1\fi\relax
\else\csname fi\endcsname
%</ignore>
%<*install>
\input docstrip.tex
\Msg{************************************************************************}
\Msg{* Installation}
\Msg{* Package: hypdoc 2018/09/07 v1.14 Hyper extensions for doc.sty (HO)}
\Msg{************************************************************************}

\keepsilent
\askforoverwritefalse

\let\MetaPrefix\relax
\preamble

This is a generated file.

Project: hypdoc
Version: 2018/09/07 v1.14

Copyright (C) 2002, 2006, 2007, 2009-2011 by
   Heiko Oberdiek <heiko.oberdiek at googlemail.com>

This work may be distributed and/or modified under the
conditions of the LaTeX Project Public License, either
version 1.3c of this license or (at your option) any later
version. This version of this license is in
   http://www.latex-project.org/lppl/lppl-1-3c.txt
and the latest version of this license is in
   http://www.latex-project.org/lppl.txt
and version 1.3 or later is part of all distributions of
LaTeX version 2005/12/01 or later.

This work has the LPPL maintenance status "maintained".

This Current Maintainer of this work is Heiko Oberdiek.

This work consists of the main source file hypdoc.dtx
and the derived files
   hypdoc.sty, hypdoc.pdf, hypdoc.ins, hypdoc.drv.

\endpreamble
\let\MetaPrefix\DoubleperCent

\generate{%
  \file{hypdoc.ins}{\from{hypdoc.dtx}{install}}%
  \file{hypdoc.drv}{\from{hypdoc.dtx}{driver}}%
  \usedir{tex/latex/oberdiek}%
  \file{hypdoc.sty}{\from{hypdoc.dtx}{package}}%
  \nopreamble
  \nopostamble
%  \usedir{source/latex/oberdiek/catalogue}%
%  \file{hypdoc.xml}{\from{hypdoc.dtx}{catalogue}}%
}

\catcode32=13\relax% active space
\let =\space%
\Msg{************************************************************************}
\Msg{*}
\Msg{* To finish the installation you have to move the following}
\Msg{* file into a directory searched by TeX:}
\Msg{*}
\Msg{*     hypdoc.sty}
\Msg{*}
\Msg{* To produce the documentation run the file `hypdoc.drv'}
\Msg{* through LaTeX.}
\Msg{*}
\Msg{* Happy TeXing!}
\Msg{*}
\Msg{************************************************************************}

\endbatchfile
%</install>
%<*ignore>
\fi
%</ignore>
%<*driver>
\NeedsTeXFormat{LaTeX2e}
\ProvidesFile{hypdoc.drv}%
  [2018/09/07 v1.14 Hyper extensions for doc.sty (HO)]%
\documentclass{ltxdoc}
\usepackage{holtxdoc}[2011/11/22]
\begin{document}
  \DocInput{hypdoc.dtx}%
\end{document}
%</driver>
% \fi
%
%
% \CharacterTable
%  {Upper-case    \A\B\C\D\E\F\G\H\I\J\K\L\M\N\O\P\Q\R\S\T\U\V\W\X\Y\Z
%   Lower-case    \a\b\c\d\e\f\g\h\i\j\k\l\m\n\o\p\q\r\s\t\u\v\w\x\y\z
%   Digits        \0\1\2\3\4\5\6\7\8\9
%   Exclamation   \!     Double quote  \"     Hash (number) \#
%   Dollar        \$     Percent       \%     Ampersand     \&
%   Acute accent  \'     Left paren    \(     Right paren   \)
%   Asterisk      \*     Plus          \+     Comma         \,
%   Minus         \-     Point         \.     Solidus       \/
%   Colon         \:     Semicolon     \;     Less than     \<
%   Equals        \=     Greater than  \>     Question mark \?
%   Commercial at \@     Left bracket  \[     Backslash     \\
%   Right bracket \]     Circumflex    \^     Underscore    \_
%   Grave accent  \`     Left brace    \{     Vertical bar  \|
%   Right brace   \}     Tilde         \~}
%
% \GetFileInfo{hypdoc.drv}
%
% \title{The \xpackage{hypdoc} package}
% \date{2018/09/07 v1.14}
% \author{Heiko Oberdiek\thanks
% {Please report any issues at https://github.com/ho-tex/oberdiek/issues}\\
% \xemail{heiko.oberdiek at googlemail.com}}
%
% \maketitle
%
% \begin{abstract}
% This package adds hyper features to the package
% \xpackage{doc} that is used in the documentation
% system of \LaTeXe. Bookmarks are added and references
% are linked as far as possible.
% \end{abstract}
%
% \tableofcontents
%
% \section{Features}
%
% The package adds the following features:
% \begin{description}
% \item[Bookmarks:]
%   For supporting bookmarks package \verb|hyperref| is loaded.
%   Some \TeX\ code inside section titles can cause problems
%   during bookmark processing; these can be warnings, ugly
%   bookmark titles or even errors. With the help of
%   \verb|\pdfstringdefDisableCommands| or \verb|\texorpdfstring|
%   you can manage these things, see \verb|hyperref|'s package
%   documentation.
% \item[Additional bookmarks:]
%   Unnumbered bookmark entries are generated for:
%   \begin{itemize}
%   \item Table of contents
%   \item List of figures
%   \item List of tables
%   \item Index
%   \item Glossary
%   \end{itemize}
%   Option \texttt{numbered} controls, whether these section or
%   chapters should be numbered. The exception is the table
%   of contents if this it would be the first numbered
%   section. Then just a bookmark is added.
% \item[Metadata:]
%   The loaded package \verb|hyperref|, enforced by
%   option \verb|pdfusetitle|, tries to detect \verb|\title|
%   and \verb|\author| and sets the corresponding PDF
%   information fields. In case of problems see the previous
%   item about bookmarks.
%   These enries can be overwritten or new ones added by
%   \verb|\hypersetup| after the packages is loaded, see
%   documentation of package \verb|hyperref|.
% \item[References, links:] Because of package \verb|\hyperref|
%   |\ref| become links, also \verb|\url|, footnote referencing
%   are supported.
% \item[Index:]
%   The main part of this package deals with index links in order
%   to support the kind of index that package \verb|doc| provides.
% \item[Thumbnails:]
%   Package \verb|thumbpdf| may additionally be loaded for the case that the
%   PDF file also should contain thumbnails, see the documentation
%   of this package. But thumbnail embedding is not really
%   necessary, because nowadays AcrobatReader is able to
%   generate and view thumbnails on the fly.
% \end{description}
%
% \section{Usage}
%
% The purpose of this packages is to add PDF features to
% the documentation of LaTeX packages. Thus just load
% the package in the driver file that generates the
% documentation:
% \begin{quote}
%   \verb|\usepackage{hypdoc}|
% \end{quote}
% Package \verb|doc| is loaded by \verb|hypdoc|, thus you
% also can replace a \verb|\usepackage{doc}| with this line.
% At any case, the package must be loaded before macros of
% package \verb|doc| such as \verb|\CodelineIndex| or
% \verb|\PageIndex| are used.
%
% If the class \verb|ltxdoc| is used, you can also use
% the configuration file for this class to add
% package \verb|hypdoc|. Add the following line
% to the file \verb|ltxdoc.cfg| (or create a new one):
% \begin{quote}
%\begin{verbatim}
%\AtEndOfClass{\RequirePackage{hypdoc}}
%\end{verbatim}
% \end{quote}
%
% \subsection{Options}
%
% There is one option \texttt{numbered}. It controls,
% whether the starred versions of section or chapter
% inside \cs{tableofcontents}, \cs{listoffigures}, \dots
% should be replaced by the non-star versions.
%
% \subsection{Simple example}
% Example of a simple driver file \verb|foobar.drv|
% of a package \verb|foobar|. Often the driver file
% is packed in the file \verb|foobar.dtx| and can
% be extracted by docstrip or if an instruction file
% for docstrip is provided by:
% \begin{quote}
% \verb|tex foobar.ins|
% \end{quote}
% Note that although docstrip doesn't rely on \LaTeX,
% sometimes you need to run the \verb|.ins| file through
% \LaTeX. Oddly enough, often the only \LaTeX\ feature
% in use is \verb|\NeedsTeXFormat{LaTeX2e}| in the first line.
% \begin{quote}
%\begin{verbatim}
%\documentclass{ltxdoc}
%\usepackage{hypdoc}
%\RecordChanges
%\EnableCrossrefs
%\CodelineIndex
%\begin{document}
%\title{Title for package \texttt{foobar}}
%\author{Author of package \texttt{foobar}}
%\date{...}
%\maketitle
%\tableofcontents
%\DocInput{foobar.dtx}
%\PrintChanges
%\PrintIndex
%\end{document}
%\end{verbatim}
% \end{quote}
%
% Then compile it, for example with pdf\LaTeX.
% \begin{quote}
%\begin{verbatim}
%pdflatex foobar.drv
%makeindex -s gind.ist foobar.idx
%makeindex -s gglo.ist -o foobar.gls foobar.glo
%pdflatex foobar.drv
%\end{verbatim}
% \end{quote}
% You will need several cycles until the cross references
% and similar things are correct.
%
% \subsection{Example \texttt{source2e.tex}}
%
% The source code of the base of \LaTeX\ is available as
% a bunch of \verb|.dtx| files. \LaTeX\ provides the file
% \verb|source2e.tex| to merge them all together.
%
% Now you can either load package \verb|hypdoc| in a private copy
% of this file (please, respect the LPPL) or use the configuration
% file \verb|ltxdoc.cfg|. Example that also sets A4 paper size:
% \begin{quote}
%\begin{verbatim}
%\PassOptionsToClass{a4paper}{article}
%\AtEndOfClass{\RequirePackage{hypdoc}}
%\end{verbatim}
% \end{quote}
%
% Then \verb|source2e.tex| can be compiled by pdf\TeX, for example:
% \begin{quote}
%\begin{verbatim}
%pdflatex source2e
%makeindex -s gind.ist source2e.idx
%makeindex -s gglo.ist -o source2e.glo source2e.gls
%pdflatex source2e
%makeindex -s gind.ist source2e.idx
%makeindex -s gglo.ist -o source2e.glo source2e.gls
%pdflatex source2e
%\end{verbatim}
% \end{quote}
%
% \StopEventually{
% }
%
% \section{Implementation}
%
%    \begin{macrocode}
%<*package>
\NeedsTeXFormat{LaTeX2e}
\ProvidesPackage{hypdoc}%
  [2018/09/07 v1.14 Hyper extensions for doc.sty (HO)]%
%    \end{macrocode}
%    The package identification is done at the top of the |.dtx| file
%    in order to use only one identification string.
%
%    For unique command names this package uses |HD@| as prefix
%
% \subsection{Options}
%
%    \begin{macro}{\ifHD@numbered}
%    \begin{macrocode}
\newif\ifHD@numbered
%    \end{macrocode}
%    \end{macro}
%    \begin{macrocode}
\DeclareOption{numbered}{\HD@numberedtrue}
\ProcessOptions*\relax
%    \end{macrocode}
%
% \subsection{Package loading}
%
%    \begin{macrocode}
\RequirePackage{atveryend}[2010/03/24]
\RequirePackage{doc}
%    \end{macrocode}
%    \begin{macrocode}
\ifx\numexpr\@undefined
  \RequirePackage{calc}%
\fi
%    \end{macrocode}
%    \begin{macrocode}
\RequirePackage[%
  hyperindex=false,%
  pdfusetitle,%
  colorlinks,%
  pdfpagelabels%
]{hyperref}[2002/05/09]%
\hypersetup{%
  plainpages=false,%
  bookmarksopen,%
  bookmarksnumbered,%
  bookmarksopenlevel=0,%
  pdfstartview={FitBH \hypercalcbp{%
    \paperheight-\topmargin-1in-\headheight-\headsep
  }},%
}
%    \end{macrocode}
%
%    \begin{macrocode}
\newcounter{HD@unique}
%    \end{macrocode}
%    \begin{macro}{\theHD@unique}
%    \begin{macrocode}
\renewcommand{\theHD@unique}{%
  \number\value{HD@unique}%
}
%    \end{macrocode}
%    \end{macro}
%
%    \begin{macrocode}
\g@addto@macro\Hy@UseMaketitleInfos{%
  \@ifundefined{fileinfo}{}{%
    \ifx\@pdfsubject\@empty
      \hypersetup{pdfsubject={\fileinfo}}%
    \fi
  }%
}
%    \end{macrocode}
%
%    \begin{macrocode}
\pdfstringdefDisableCommands{%
  \let\thanks\@gobble
  \let\footnote\@gobble
  \def\cs#1{\textbackslash #1}%
  \let\normalfont\@empty
  \let\scshape\@empty
  \def\and{and }%
}
%    \end{macrocode}
%
% \subsubsection{Patch for bug latex/4096}
%
%    \begin{macrocode}
\begingroup\expandafter\expandafter\expandafter\endgroup
\expandafter\ifx\csname mod@math@codes\endcsname\relax
\else
  \def\HD@ModFix#1{%
    \mathcode`#1=`#1\relax
  }%
  \g@addto@macro\mod@math@codes{%
    \HD@ModFix\-%
    \HD@ModFix\+%
    \HD@ModFix\:%
    \HD@ModFix\=%
    \HD@ModFix\*%
  }%
\fi
%    \end{macrocode}
%
% \subsection{Bookmarks}
%
%    \begin{macro}{\HD@sectionpatch}
%    \begin{macrocode}
\def\HD@sectionpatch{%
  \ifHD@numbered
    \HD@@sectionpatch{section}%
    \HD@@sectionpatch{chapter}%
  \else
    \let\Hy@writebookmark\HD@disable@writebookmark
  \fi
}
%    \end{macrocode}
%    \end{macro}
%    \begin{macro}{\HD@@sectionpatch}
%    \begin{macrocode}
\def\HD@@sectionpatch#1{%
  \expandafter\let\csname HDorg@@#1\expandafter\endcsname
    \csname #1\endcsname
  \@namedef{#1}{%
    \@ifstar{\@nameuse{HDorg@@#1}}{\@nameuse{HDorg@@#1}}%
  }%
}
%    \end{macrocode}
%    \end{macro}
%    \begin{macro}{\HD@disable@writebookmark}
%    \begin{macrocode}
\def\HD@disable@writebookmark#1#2#3#4#5{}
%    \end{macrocode}
%    \end{macro}
%
%    \begin{macro}{\HD@guesstoclevel}
%    A wild guessing of the toclevel. There is the hope, that
%    the macro |#1| starts with a sectioning command.
%    \begin{macrocode}
\def\HD@guesstoclevel#1{%
  \expandafter\HD@@guesstoclevel#1\section\HD@end
}
%    \end{macrocode}
%    \end{macro}
%    \begin{macro}{\HD@@guesstoclevel}
%    \begin{macrocode}
\def\HD@@guesstoclevel#1#2\HD@end{%
  \@ifundefined{toclevel@\expandafter\@gobble\string#1\@empty}{%
    1%
  }{%
    \@nameuse{toclevel@\expandafter\@gobble\string#1\@empty}%
  }%
}
%    \end{macrocode}
%    \end{macro}
%
% \subsection{\cs{tableofcontents}}
%
%    \begin{macro}{\HDorg@tableofcontents}
%    \begin{macrocode}
\let\HDorg@tableofcontents\tableofcontents
%    \end{macrocode}
%    \end{macro}
%    \begin{macro}{\tableofcontents}
%    As first section the table of contents shouldn't
%    go into the toc itself. Also a number looks too funny then.
%    Just a bookmark is enough in this case.
%    \begin{macrocode}
\def\tableofcontents{%
  \begingroup
    \@ifundefined{c@section}{%
      \def\c@section{0}%
    }{}%
  \expandafter\endgroup
  \ifcase\c@section
    \stepcounter{HD@unique}%
    \pdfbookmark[\HD@guesstoclevel{\HDorg@tableofcontents}]%
        {\contentsname}{toc\theHD@unique}%
    \HDorg@tableofcontents
  \else
    \ifHD@numbered
    \else
      \stepcounter{HD@unique}%
      \pdfbookmark[\HD@guesstoclevel{\HDorg@tableofcontents}]%
          {\contentsname}{toc\theHD@unique}%
    \fi
    \begingroup
      \HD@sectionpatch
      \HDorg@tableofcontents
    \endgroup
  \fi
}
%    \end{macrocode}
%    \end{macro}
%
% \subsection{\cs{listoffigures}}
%
%    \begin{macro}{\HDorg@listoffigures}
%    \begin{macrocode}
\let\HDorg@listoffigures\listoffigures
%    \end{macrocode}
%    \end{macro}
%    \begin{macro}{\listoffigures}
%    \begin{macrocode}
\def\listoffigures{%
  \ifHD@numbered
  \else
    \stepcounter{HD@unique}%
    \pdfbookmark[\HD@guesstoclevel{\HDorg@listoffigures}]%
        {\listfigurename}{toc\theHD@unique}%
  \fi
  \begingroup
    \HD@sectionpatch
    \HDorg@listoffigures
  \endgroup
}
%    \end{macrocode}
%    \end{macro}
%
% \subsection{\cs{listoftables}}
%
%    \begin{macro}{\HDorg@listoftables}
%    \begin{macrocode}
\let\HDorg@listoftables\listoftables
%    \end{macrocode}
%    \end{macro}
%    \begin{macro}{\listoftables}
%    \begin{macrocode}
\def\listoftables{%
  \ifHD@numbered
  \else
    \stepcounter{HD@unique}%
    \pdfbookmark[\HD@guesstoclevel{\HDorg@listoftables}]%
        {\listtablename}{toc\theHD@unique}%
  \fi
  \begingroup
    \HD@sectionpatch
    \HDorg@listoftables
  \endgroup
}
%    \end{macrocode}
%    \end{macro}
%
% \subsection{\cs{thebibliography}}
%
%    \begin{macro}{\HDorg@thebibliography}
%    \begin{macrocode}
\let\HDorg@thebibliography\thebibliography
%    \end{macrocode}
%    \end{macro}
%    \begin{macro}{\thebibliography}
%    \begin{macrocode}
\def\thebibliography{%
  \ifHD@numbered
  \else
    \stepcounter{HD@unique}%
    \pdfbookmark[\HD@guesstoclevel{\HDorg@thebibliography}]%
      {\@ifundefined{chapter}{\refname}{\bibname}}{bib\theHD@unique}%
  \fi
  \HD@sectionpatch
  \HDorg@thebibliography
}
%    \end{macrocode}
%    \end{macro}
%
% \subsection{\cs{theindex}}
%
%    \begin{macro}{\HDorg@theindex}
%    \begin{macrocode}
\let\HDorg@theindex\theindex
%    \end{macrocode}
%    \end{macro}
%    \begin{macro}{\theindex}
%    \begin{macrocode}
\def\theindex{%
  \let\HDorg@index@prologue\index@prologue
  \let\HDorg@writebookmark\Hy@writebookmark
  \HD@sectionpatch
  \g@addto@macro\IndexParms{%
    \ifHD@numbered
    \else
      \let\Hy@writebookmark\HDorg@writebookmark
    \fi
    \let\bfseries\HD@bfseries
  }%
  \edef\HD@toclevel@index{\HD@guesstoclevel{\index@prologue}}%
  \begingroup
    \count@=\HD@toclevel@index\relax
    \advance\count@ by 1 %
    \xdef\HD@toclevel@subindex{\the\count@}%
  \endgroup
  \def\index@prologue{%
    \ifHD@numbered
    \else
      \begingroup
        \stepcounter{HD@unique}%
        \let\Hy@writebookmark\HDorg@writebookmark
        \pdfbookmark[\HD@toclevel@index]%
            {\indexname}{index\theHD@unique}%
      \endgroup
    \fi
    \HDorg@index@prologue
  }%
  \HDorg@theindex
}
%    \end{macrocode}
%    \end{macro}
%    \begin{macro}{\HDorg@bfseries}
%    \begin{macrocode}
\let\HDorg@bfseries\bfseries
%    \end{macrocode}
%    \end{macro}
%    \begin{macro}{\HD@bfseries}
%    \begin{macrocode}
\def\HD@bfseries{\futurelet\HD@hfil\HD@@bfseries}%
%    \end{macrocode}
%    \end{macro}
%    \begin{macro}{\HD@@bfseries}
%    \begin{macrocode}
\def\HD@@bfseries{%
  \HDorg@bfseries
  \ifx\HD@hfil\hfil
    \expandafter\HD@@@bfseries
  \fi
}
%    \end{macrocode}
%    \end{macro}
%    \begin{macro}{\HD@@@bfseries}
%    \begin{macrocode}
\def\HD@@@bfseries\hfil#1\hfil{%
  \ifx\\#1\\%
  \else
    \raisebox{\baselineskip}[0pt]{%
      \kern-\HD@margin\relax
      \pdfbookmark[\HD@toclevel@subindex]{#1}{HD.#1}%
      \kern\HD@margin\relax
    }%
  \fi
  \hfil#1\hfil
}%
%    \end{macrocode}
%    \end{macro}
%    \begin{macro}{\HD@margin}
%    \begin{macrocode}
\def\HD@margin{1mm}
%    \end{macrocode}
%    \end{macro}
%
% \subsection{\cs{theglossary}}
%
%    \begin{macro}{\HDorg@theglossary}
%    \begin{macrocode}
\let\HDorg@theglossary\theglossary
%    \end{macrocode}
%    \end{macro}
%    \begin{macro}{\theglossary}
%    \begin{macrocode}
\def\theglossary{%
  \let\HDorg@glossary@prologue\glossary@prologue
  \let\HDorg@writebookmark\Hy@writebookmark
  \HD@sectionpatch
  \def\glossary@prologue{%
    \ifHD@numbered
    \else
      \begingroup
        \let\Hy@writebookmark\HDorg@writebookmark
        \stepcounter{HD@unique}%
        \pdfbookmark[\HD@guesstoclevel{\HDorg@glossary@prologue}]%
            {\changehistoryname}{glossary\theHD@unique}%
      \endgroup
    \fi
    \HDorg@glossary@prologue
  }%
  \HDorg@theglossary
}
%    \end{macrocode}
%    \end{macro}
%    \begin{macro}{\glossaryname}
%    \begin{macrocode}
\providecommand*{\glossaryname}{Change History}
%    \end{macrocode}
%    \end{macro}
%
%    \begin{macro}{\glossaryname}
%    \begin{macrocode}
\providecommand*{\changehistoryname}{\glossaryname}
%    \end{macrocode}
%    \end{macro}
%
% \subsection{Index}
%
%    \begin{macrocode}
\newcounter{HD@hypercount}
%    \end{macrocode}
%    \begin{macro}{\HD@helplength}
%    \begin{macrocode}
\newlength{\HD@helplength}
%    \end{macrocode}
%    \end{macro}
%    \begin{macro}{\ifHD@savedest}
%    \begin{macrocode}
\newif\ifHD@savedest
\HD@savedesttrue
%    \end{macrocode}
%    \end{macro}
%    \begin{macrocode}
\AtBeginDocument{%
  \HD@savedest@init
}
%    \end{macrocode}
%    \begin{macrocode}
\RequirePackage{rerunfilecheck}[2009/12/10]
%    \end{macrocode}
%    \begin{macro}{\HD@savedest@init}
%    \begin{macrocode}
\begingroup
  \catcode`\^^M=\active
\@firstofone{\endgroup
  \def\HD@savedest@init{%
    \ifHD@savedest
      \let\HDorg@tilde~%
      \let~\HD@savedest@entry
      \edef\HDorg@catcodeM{\the\catcode`\^^M}%
      \catcode`\^^M=\active
      \def^^M{}%
      \InputIfFileExists{\jobname.hd}{}{}%
      \let~\HDorg@tilde
      \catcode`\^^M=\HDorg@catcodeM\relax
      \newwrite\HD@savedest@out
      \RerunFileCheck{\jobname.hd}{%
        \immediate\closeout\HD@savedest@out
      }{}%
      \immediate\openout\HD@savedest@out=\jobname.hd\relax
    \else
      \let\HD@savedest@add\@gobble
    \fi
  }%
%    \end{macrocode}
%    \end{macro}
%    \begin{macro}{\HD@savedest@entry}
%    \begin{macrocode}
  \def\HD@savedest@entry#1^^M{%
    \@namedef{HD.#1}{}%
  }%
}
%    \end{macrocode}
%    \end{macro}
%    \begin{macro}{\HD@savedest@add}
%    \begin{macrocode}
\def\HD@savedest@add#1{%
  \immediate\write\HD@savedest@out{\string~#1}%
}
%    \end{macrocode}
%    \end{macro}
%    \begin{macro}{\ifHD@targetraise}
%    \begin{macrocode}
\newif\ifHD@targetraise
\HD@targetraisetrue
%    \end{macrocode}
%    \end{macro}
%    \begin{macro}{\HD@target}
%    \begin{macrocode}
\def\HD@target{%
  \ifHD@targetraise
    \ifhmode
      \settoheight{\HD@helplength}{[H]}%
      \addtolength{\HD@helplength}{0.4ex}%
      \raisebox{\HD@helplength}[0pt][0pt]{%
        \HD@maketarget
      }%
    \else
      \HD@maketarget
    \fi
  \else
    \HD@maketarget
  \fi
}
%    \end{macrocode}
%    \end{macro}
%    \begin{macro}{\HD@maketarget}
%    \begin{macrocode}
\def\HD@maketarget{%
  \stepcounter{HD@hypercount}%
  \def\HD@next{%
    \llap{\hypertarget{HD.\the\c@HD@hypercount}{}\,}%
  }%
  \ifHD@savedest
    \expandafter\ifx\csname HD.\the\c@HD@hypercount\endcsname\@empty
    \else
      \let\HD@next\relax
    \fi
  \fi
  \HD@next
}
%    \end{macrocode}
%    \end{macro}
%
%    \begin{macro}{\CodelineIndex}
%    \begin{macrocode}
\g@addto@macro\CodelineIndex{%
  \let\special@index\HD@codeline@wrindex
}
%    \end{macrocode}
%    \end{macro}
%    \begin{macro}{\PageIndex}
%    \begin{macrocode}
\g@addto@macro\PageIndex{%
  \let\special@index\HD@page@wrindex
}
%    \end{macrocode}
%    \end{macro}
%
%    \begin{macro}{\HD@codeline@wrindex}
%    \begin{macrocode}
\def\HD@codeline@wrindex#1{%
  \begingroup
    \let\HDorg@encapchar\encapchar
    \def\encapchar##1\encapchar##2\@nil{%
      \HDorg@encapchar
      hdclindex{\the\c@HD@hypercount}{##1}%
    }%
    \codeline@wrindex{%
      #1\encapchar\encapchar\@nil
    }%
  \endgroup
}
%    \end{macrocode}
%    \end{macro}
%    \begin{macro}{\HD@page@wrindex}
%    \begin{macrocode}
\def\HD@page@wrindex#1{%
  \begingroup
    \let\HDorg@encapchar\encapchar
    \def\encapchar##1\encapchar##2\@nil{%
      \HDorg@encapchar
      hdpindex{##1}%
    }%
    \HDorg@index{#1\encapchar\encapchar\@nil}%
  \endgroup
}
%    \end{macrocode}
%    \end{macro}
%
%    \begin{macro}{\hdclindex}
%    \begin{macrocode}
\def\hdclindex#1#2#3{%
  \csname\ifx\\#2\\relax\else#2\fi\endcsname{%
    \hyperlink{HD.#1}{#3}%
    \HD@savedest@add{#1}%
  }%
}
%    \end{macrocode}
%    \end{macro}
%    \begin{macro}{\hdpindex}
%    \begin{macrocode}
\def\hdpindex#1#2{%
  \csname\ifx\\#1\\relax\else#1\fi\endcsname{%
    \hyperpage{#2}%
  }%
}
%    \end{macrocode}
%    \end{macro}
%
% \subsection{\cs{PrintDescribeMacro}}
%
%    \begin{macro}{\HDorg@PrintDescribeMacro}
%    \begin{macrocode}
\let\HDorg@PrintDescribeMacro\PrintDescribeMacro
%    \end{macrocode}
%    \end{macro}
%    \begin{macro}{\PrintDescribeMacro}
%    \begin{macrocode}
\renewcommand\PrintDescribeMacro[1]{%
  \begingroup
    \let\HDorg@MacroFont\MacroFont
    \def\MacroFont{%
      \HD@target
      \HDorg@MacroFont
    }%
    \HDorg@PrintDescribeMacro{#1}%
  \endgroup
}
%    \end{macrocode}
%    \end{macro}
%
% \subsection{\cs{PrintMacroName}}
%
%    \begin{macro}{\HDorg@PrintMacroName}
%    \begin{macrocode}
\let\HDorg@PrintMacroName\PrintMacroName
%    \end{macrocode}
%    \end{macro}
%    \begin{macro}{\PrintMacroName}
%    \begin{macrocode}
\renewcommand\PrintMacroName[1]{%
  \begingroup
    \let\HDorg@MacroFont\MacroFont
    \def\MacroFont{%
      \HD@target
      \HDorg@MacroFont
    }%
    \HDorg@PrintMacroName{#1}%
  \endgroup
}
%    \end{macrocode}
%    \end{macro}
%
% \subsection{\cs{theCodelineNo}}
%
%    \begin{macro}{\HDorg@theCodelineNo}
%    \begin{macrocode}
\let\HDorg@theCodelineNo\theCodelineNo
%    \end{macrocode}
%    \end{macro}
%    \begin{macro}{\theCodelineNo}
%    \begin{macrocode}
\renewcommand*\theCodelineNo{%
  \settowidth{\HD@helplength}{\HDorg@theCodelineNo\ }%
  \ifdim\HD@helplength<\MacroIndent
    \addtolength{\HD@helplength}{-\MacroIndent}%
    \llap{%
      \begingroup\HD@target\endgroup
      \hspace*{-\HD@helplength}%
    }%
  \else
    \HD@target
  \fi
  \HDorg@theCodelineNo
}
%    \end{macrocode}
%    \end{macro}
%
%    \begin{macro}{\HDorg@index}
%    \begin{macrocode}
\AtBeginDocument{%
  \let\HDorg@index\index
}
%    \end{macrocode}
%    \end{macro}
%
% \subsection{\cs{SpecialUsageIndex}}
%
%    \begin{macro}{\HDorg@SpecialUsageIndex}
%    \begin{macrocode}
\let\HDorg@SpecialUsageIndex\SpecialUsageIndex
\renewcommand*\SpecialUsageIndex[1]{%
  \@bsphack
  \begingroup
    \HD@target
    \let\index\HDorg@index
    \let\HDorg@encapchar\encapchar
    \edef\encapchar usage{%
       \HDorg@encapchar hdclindex{\the\c@HD@hypercount}{usage}%
    }%
    \HDorg@SpecialUsageIndex{#1}%
  \endgroup
  \@esphack
}
%    \end{macrocode}
%    \end{macro}
%
% \subsection{\cs{autoindex}}
%
%    \begin{macrocode}
\begingroup\expandafter\expandafter\expandafter\endgroup
\expandafter\ifx\csname autoindex\endcsname\relax
\else
%    \end{macrocode}
%    \begin{macro}{\HDorg@autoindex}
%    \begin{macrocode}
  \let\HDorg@autoindex\autoindex
%    \end{macrocode}
%    \end{macro}
%    \begin{macro}{\autoindex}
%    \begin{macrocode}
  \renewcommand*\autoindex[1]{%
    \@bsphack
    \begingroup
      \HD@target
      \let\index\HDorg@index
      \let\HDorg@encapchar\encapchar
      \edef\encapchar usage{%
         \HDorg@encapchar hdclindex{\the\c@HD@hypercount}{usage}%
      }%
      \HDorg@autoindex{#1}%
    \endgroup
    \@esphack
  }%
\fi
%    \end{macrocode}
%    \end{macro}
%
% \subsection{\cs{SpecialEnvIndex}}
%
%    \begin{macro}{\HDorg@SpecialEnvIndex}
%    \begin{macrocode}
\let\HDorg@SpecialEnvIndex\SpecialEnvIndex
%    \end{macrocode}
%    \end{macro}
%    \begin{macro}{\SpecialEnvIndex}
%    \begin{macrocode}
\renewcommand*\SpecialEnvIndex[1]{%
  \@bsphack
  \begingroup
    \HD@target
    \let\HDorg@encapchar\encapchar
    \edef\encapchar usage{%
      \HDorg@encapchar hdclindex{\the\c@HD@hypercount}{usage}%
    }%
    \HDorg@SpecialEnvIndex{#1}%
  \endgroup
  \@esphack
}
%    \end{macrocode}
%    \end{macro}
%
% \subsection{\cs{SortIndex}}
%
%    \begin{macro}{\HDorg@SortIndex}
%    \begin{macrocode}
\let\HDorg@SortIndex\SortIndex
%    \end{macrocode}
%    \end{macro}
%    \begin{macro}{\SortIndex}
%    \begin{macrocode}
\renewcommand*{\SortIndex}[2]{%
  \@bsphack
  \begingroup
    \let\index\HD@page@wrindex
    \HDorg@SortIndex{#1}{#2}%
  \endgroup
  \@esphack
}
%    \end{macrocode}
%    \end{macro}
%
% \subsection{\cs{@wrglossary}}
%
%    \begin{macro}{\HDorg@wrglossary}
%    \begin{macrocode}
\let\HDorg@wrglossary\@wrglossary
%    \end{macrocode}
%    \end{macro}
%    \begin{macro}{\@wrglossary}
%    \begin{macrocode}
\def\@wrglossary#1{%
  \let\HDorg@encapchar\encapchar
  \def\encapchar##1\encapchar##2\@nil{%
    \HDorg@encapchar
    hdpindex{##1}%
  }%
  \HDorg@wrglossary{#1\encapchar\encapchar\@nil}%
}
%    \end{macrocode}
%    \end{macro}
%
% \subsection{\cs{MacroIndent}}
%
%    \cs{MacroIndent} automatically calculated.
%    \begin{macrocode}
\AfterLastShipout{%
  \if@filesw
    \immediate\write\@auxout{%
      \string\providecommand\string\HD@SetMacroIndent[1]{}%
    }%
    \immediate\write\@auxout{%
      \string\HD@SetMacroIndent{\number\value{CodelineNo}}%
    }%
  \fi
}
%    \end{macrocode}
%    \begin{macro}{\HD@SetMacroIndent}
%    \begin{macrocode}
\def\HD@SetMacroIndent#1{%
  \begingroup
    \value{CodelineNo}=#1\relax
    \settowidth{\MacroIndent}{\HDorg@theCodelineNo\ }%
    \global\MacroIndent\MacroIndent
  \endgroup
}
%    \end{macrocode}
%    \end{macro}
%
%    \begin{macrocode}
%</package>
%    \end{macrocode}
%
% \section{Installation}
%
% \subsection{Download}
%
% \paragraph{Package.} This package is available on
% CTAN\footnote{\url{http://ctan.org/pkg/hypdoc}}:
% \begin{description}
% \item[\CTAN{macros/latex/contrib/oberdiek/hypdoc.dtx}] The source file.
% \item[\CTAN{macros/latex/contrib/oberdiek/hypdoc.pdf}] Documentation.
% \end{description}
%
%
% \paragraph{Bundle.} All the packages of the bundle `oberdiek'
% are also available in a TDS compliant ZIP archive. There
% the packages are already unpacked and the documentation files
% are generated. The files and directories obey the TDS standard.
% \begin{description}
% \item[\CTAN{install/macros/latex/contrib/oberdiek.tds.zip}]
% \end{description}
% \emph{TDS} refers to the standard ``A Directory Structure
% for \TeX\ Files'' (\CTAN{tds/tds.pdf}). Directories
% with \xfile{texmf} in their name are usually organized this way.
%
% \subsection{Bundle installation}
%
% \paragraph{Unpacking.} Unpack the \xfile{oberdiek.tds.zip} in the
% TDS tree (also known as \xfile{texmf} tree) of your choice.
% Example (linux):
% \begin{quote}
%   |unzip oberdiek.tds.zip -d ~/texmf|
% \end{quote}
%
% \paragraph{Script installation.}
% Check the directory \xfile{TDS:scripts/oberdiek/} for
% scripts that need further installation steps.
% Package \xpackage{attachfile2} comes with the Perl script
% \xfile{pdfatfi.pl} that should be installed in such a way
% that it can be called as \texttt{pdfatfi}.
% Example (linux):
% \begin{quote}
%   |chmod +x scripts/oberdiek/pdfatfi.pl|\\
%   |cp scripts/oberdiek/pdfatfi.pl /usr/local/bin/|
% \end{quote}
%
% \subsection{Package installation}
%
% \paragraph{Unpacking.} The \xfile{.dtx} file is a self-extracting
% \docstrip\ archive. The files are extracted by running the
% \xfile{.dtx} through \plainTeX:
% \begin{quote}
%   \verb|tex hypdoc.dtx|
% \end{quote}
%
% \paragraph{TDS.} Now the different files must be moved into
% the different directories in your installation TDS tree
% (also known as \xfile{texmf} tree):
% \begin{quote}
% \def\t{^^A
% \begin{tabular}{@{}>{\ttfamily}l@{ $\rightarrow$ }>{\ttfamily}l@{}}
%   hypdoc.sty & tex/latex/oberdiek/hypdoc.sty\\
%   hypdoc.pdf & doc/latex/oberdiek/hypdoc.pdf\\
%   hypdoc.dtx & source/latex/oberdiek/hypdoc.dtx\\
% \end{tabular}^^A
% }^^A
% \sbox0{\t}^^A
% \ifdim\wd0>\linewidth
%   \begingroup
%     \advance\linewidth by\leftmargin
%     \advance\linewidth by\rightmargin
%   \edef\x{\endgroup
%     \def\noexpand\lw{\the\linewidth}^^A
%   }\x
%   \def\lwbox{^^A
%     \leavevmode
%     \hbox to \linewidth{^^A
%       \kern-\leftmargin\relax
%       \hss
%       \usebox0
%       \hss
%       \kern-\rightmargin\relax
%     }^^A
%   }^^A
%   \ifdim\wd0>\lw
%     \sbox0{\small\t}^^A
%     \ifdim\wd0>\linewidth
%       \ifdim\wd0>\lw
%         \sbox0{\footnotesize\t}^^A
%         \ifdim\wd0>\linewidth
%           \ifdim\wd0>\lw
%             \sbox0{\scriptsize\t}^^A
%             \ifdim\wd0>\linewidth
%               \ifdim\wd0>\lw
%                 \sbox0{\tiny\t}^^A
%                 \ifdim\wd0>\linewidth
%                   \lwbox
%                 \else
%                   \usebox0
%                 \fi
%               \else
%                 \lwbox
%               \fi
%             \else
%               \usebox0
%             \fi
%           \else
%             \lwbox
%           \fi
%         \else
%           \usebox0
%         \fi
%       \else
%         \lwbox
%       \fi
%     \else
%       \usebox0
%     \fi
%   \else
%     \lwbox
%   \fi
% \else
%   \usebox0
% \fi
% \end{quote}
% If you have a \xfile{docstrip.cfg} that configures and enables \docstrip's
% TDS installing feature, then some files can already be in the right
% place, see the documentation of \docstrip.
%
% \subsection{Refresh file name databases}
%
% If your \TeX~distribution
% (\teTeX, \mikTeX, \dots) relies on file name databases, you must refresh
% these. For example, \teTeX\ users run \verb|texhash| or
% \verb|mktexlsr|.
%
% \subsection{Some details for the interested}
%
% \paragraph{Attached source.}
%
% The PDF documentation on CTAN also includes the
% \xfile{.dtx} source file. It can be extracted by
% AcrobatReader 6 or higher. Another option is \textsf{pdftk},
% e.g. unpack the file into the current directory:
% \begin{quote}
%   \verb|pdftk hypdoc.pdf unpack_files output .|
% \end{quote}
%
% \paragraph{Unpacking with \LaTeX.}
% The \xfile{.dtx} chooses its action depending on the format:
% \begin{description}
% \item[\plainTeX:] Run \docstrip\ and extract the files.
% \item[\LaTeX:] Generate the documentation.
% \end{description}
% If you insist on using \LaTeX\ for \docstrip\ (really,
% \docstrip\ does not need \LaTeX), then inform the autodetect routine
% about your intention:
% \begin{quote}
%   \verb|latex \let\install=y% \iffalse meta-comment
%
% File: hypdoc.dtx
% Version: 2018/09/07 v1.14
% Info: Hyper extensions for doc.sty
%
% Copyright (C) 2002, 2006, 2007, 2009-2011 by
%    Heiko Oberdiek <heiko.oberdiek at googlemail.com>
%    2016
%    https://github.com/ho-tex/oberdiek/issues
%
% This work may be distributed and/or modified under the
% conditions of the LaTeX Project Public License, either
% version 1.3c of this license or (at your option) any later
% version. This version of this license is in
%    http://www.latex-project.org/lppl/lppl-1-3c.txt
% and the latest version of this license is in
%    http://www.latex-project.org/lppl.txt
% and version 1.3 or later is part of all distributions of
% LaTeX version 2005/12/01 or later.
%
% This work has the LPPL maintenance status "maintained".
%
% This Current Maintainer of this work is Heiko Oberdiek.
%
% This work consists of the main source file hypdoc.dtx
% and the derived files
%    hypdoc.sty, hypdoc.pdf, hypdoc.ins, hypdoc.drv.
%
% Distribution:
%    CTAN:macros/latex/contrib/oberdiek/hypdoc.dtx
%    CTAN:macros/latex/contrib/oberdiek/hypdoc.pdf
%
% Unpacking:
%    (a) If hypdoc.ins is present:
%           tex hypdoc.ins
%    (b) Without hypdoc.ins:
%           tex hypdoc.dtx
%    (c) If you insist on using LaTeX
%           latex \let\install=y% \iffalse meta-comment
%
% File: hypdoc.dtx
% Version: 2018/09/07 v1.14
% Info: Hyper extensions for doc.sty
%
% Copyright (C) 2002, 2006, 2007, 2009-2011 by
%    Heiko Oberdiek <heiko.oberdiek at googlemail.com>
%    2016
%    https://github.com/ho-tex/oberdiek/issues
%
% This work may be distributed and/or modified under the
% conditions of the LaTeX Project Public License, either
% version 1.3c of this license or (at your option) any later
% version. This version of this license is in
%    http://www.latex-project.org/lppl/lppl-1-3c.txt
% and the latest version of this license is in
%    http://www.latex-project.org/lppl.txt
% and version 1.3 or later is part of all distributions of
% LaTeX version 2005/12/01 or later.
%
% This work has the LPPL maintenance status "maintained".
%
% This Current Maintainer of this work is Heiko Oberdiek.
%
% This work consists of the main source file hypdoc.dtx
% and the derived files
%    hypdoc.sty, hypdoc.pdf, hypdoc.ins, hypdoc.drv.
%
% Distribution:
%    CTAN:macros/latex/contrib/oberdiek/hypdoc.dtx
%    CTAN:macros/latex/contrib/oberdiek/hypdoc.pdf
%
% Unpacking:
%    (a) If hypdoc.ins is present:
%           tex hypdoc.ins
%    (b) Without hypdoc.ins:
%           tex hypdoc.dtx
%    (c) If you insist on using LaTeX
%           latex \let\install=y\input{hypdoc.dtx}
%        (quote the arguments according to the demands of your shell)
%
% Documentation:
%    (a) If hypdoc.drv is present:
%           latex hypdoc.drv
%    (b) Without hypdoc.drv:
%           latex hypdoc.dtx; ...
%    The class ltxdoc loads the configuration file ltxdoc.cfg
%    if available. Here you can specify further options, e.g.
%    use A4 as paper format:
%       \PassOptionsToClass{a4paper}{article}
%
%    Programm calls to get the documentation (example):
%       pdflatex hypdoc.dtx
%       makeindex -s gind.ist hypdoc.idx
%       pdflatex hypdoc.dtx
%       makeindex -s gind.ist hypdoc.idx
%       pdflatex hypdoc.dtx
%
% Installation:
%    TDS:tex/latex/oberdiek/hypdoc.sty
%    TDS:doc/latex/oberdiek/hypdoc.pdf
%    TDS:source/latex/oberdiek/hypdoc.dtx
%
%<*ignore>
\begingroup
  \catcode123=1 %
  \catcode125=2 %
  \def\x{LaTeX2e}%
\expandafter\endgroup
\ifcase 0\ifx\install y1\fi\expandafter
         \ifx\csname processbatchFile\endcsname\relax\else1\fi
         \ifx\fmtname\x\else 1\fi\relax
\else\csname fi\endcsname
%</ignore>
%<*install>
\input docstrip.tex
\Msg{************************************************************************}
\Msg{* Installation}
\Msg{* Package: hypdoc 2018/09/07 v1.14 Hyper extensions for doc.sty (HO)}
\Msg{************************************************************************}

\keepsilent
\askforoverwritefalse

\let\MetaPrefix\relax
\preamble

This is a generated file.

Project: hypdoc
Version: 2018/09/07 v1.14

Copyright (C) 2002, 2006, 2007, 2009-2011 by
   Heiko Oberdiek <heiko.oberdiek at googlemail.com>

This work may be distributed and/or modified under the
conditions of the LaTeX Project Public License, either
version 1.3c of this license or (at your option) any later
version. This version of this license is in
   http://www.latex-project.org/lppl/lppl-1-3c.txt
and the latest version of this license is in
   http://www.latex-project.org/lppl.txt
and version 1.3 or later is part of all distributions of
LaTeX version 2005/12/01 or later.

This work has the LPPL maintenance status "maintained".

This Current Maintainer of this work is Heiko Oberdiek.

This work consists of the main source file hypdoc.dtx
and the derived files
   hypdoc.sty, hypdoc.pdf, hypdoc.ins, hypdoc.drv.

\endpreamble
\let\MetaPrefix\DoubleperCent

\generate{%
  \file{hypdoc.ins}{\from{hypdoc.dtx}{install}}%
  \file{hypdoc.drv}{\from{hypdoc.dtx}{driver}}%
  \usedir{tex/latex/oberdiek}%
  \file{hypdoc.sty}{\from{hypdoc.dtx}{package}}%
  \nopreamble
  \nopostamble
%  \usedir{source/latex/oberdiek/catalogue}%
%  \file{hypdoc.xml}{\from{hypdoc.dtx}{catalogue}}%
}

\catcode32=13\relax% active space
\let =\space%
\Msg{************************************************************************}
\Msg{*}
\Msg{* To finish the installation you have to move the following}
\Msg{* file into a directory searched by TeX:}
\Msg{*}
\Msg{*     hypdoc.sty}
\Msg{*}
\Msg{* To produce the documentation run the file `hypdoc.drv'}
\Msg{* through LaTeX.}
\Msg{*}
\Msg{* Happy TeXing!}
\Msg{*}
\Msg{************************************************************************}

\endbatchfile
%</install>
%<*ignore>
\fi
%</ignore>
%<*driver>
\NeedsTeXFormat{LaTeX2e}
\ProvidesFile{hypdoc.drv}%
  [2018/09/07 v1.14 Hyper extensions for doc.sty (HO)]%
\documentclass{ltxdoc}
\usepackage{holtxdoc}[2011/11/22]
\begin{document}
  \DocInput{hypdoc.dtx}%
\end{document}
%</driver>
% \fi
%
%
% \CharacterTable
%  {Upper-case    \A\B\C\D\E\F\G\H\I\J\K\L\M\N\O\P\Q\R\S\T\U\V\W\X\Y\Z
%   Lower-case    \a\b\c\d\e\f\g\h\i\j\k\l\m\n\o\p\q\r\s\t\u\v\w\x\y\z
%   Digits        \0\1\2\3\4\5\6\7\8\9
%   Exclamation   \!     Double quote  \"     Hash (number) \#
%   Dollar        \$     Percent       \%     Ampersand     \&
%   Acute accent  \'     Left paren    \(     Right paren   \)
%   Asterisk      \*     Plus          \+     Comma         \,
%   Minus         \-     Point         \.     Solidus       \/
%   Colon         \:     Semicolon     \;     Less than     \<
%   Equals        \=     Greater than  \>     Question mark \?
%   Commercial at \@     Left bracket  \[     Backslash     \\
%   Right bracket \]     Circumflex    \^     Underscore    \_
%   Grave accent  \`     Left brace    \{     Vertical bar  \|
%   Right brace   \}     Tilde         \~}
%
% \GetFileInfo{hypdoc.drv}
%
% \title{The \xpackage{hypdoc} package}
% \date{2018/09/07 v1.14}
% \author{Heiko Oberdiek\thanks
% {Please report any issues at https://github.com/ho-tex/oberdiek/issues}\\
% \xemail{heiko.oberdiek at googlemail.com}}
%
% \maketitle
%
% \begin{abstract}
% This package adds hyper features to the package
% \xpackage{doc} that is used in the documentation
% system of \LaTeXe. Bookmarks are added and references
% are linked as far as possible.
% \end{abstract}
%
% \tableofcontents
%
% \section{Features}
%
% The package adds the following features:
% \begin{description}
% \item[Bookmarks:]
%   For supporting bookmarks package \verb|hyperref| is loaded.
%   Some \TeX\ code inside section titles can cause problems
%   during bookmark processing; these can be warnings, ugly
%   bookmark titles or even errors. With the help of
%   \verb|\pdfstringdefDisableCommands| or \verb|\texorpdfstring|
%   you can manage these things, see \verb|hyperref|'s package
%   documentation.
% \item[Additional bookmarks:]
%   Unnumbered bookmark entries are generated for:
%   \begin{itemize}
%   \item Table of contents
%   \item List of figures
%   \item List of tables
%   \item Index
%   \item Glossary
%   \end{itemize}
%   Option \texttt{numbered} controls, whether these section or
%   chapters should be numbered. The exception is the table
%   of contents if this it would be the first numbered
%   section. Then just a bookmark is added.
% \item[Metadata:]
%   The loaded package \verb|hyperref|, enforced by
%   option \verb|pdfusetitle|, tries to detect \verb|\title|
%   and \verb|\author| and sets the corresponding PDF
%   information fields. In case of problems see the previous
%   item about bookmarks.
%   These enries can be overwritten or new ones added by
%   \verb|\hypersetup| after the packages is loaded, see
%   documentation of package \verb|hyperref|.
% \item[References, links:] Because of package \verb|\hyperref|
%   |\ref| become links, also \verb|\url|, footnote referencing
%   are supported.
% \item[Index:]
%   The main part of this package deals with index links in order
%   to support the kind of index that package \verb|doc| provides.
% \item[Thumbnails:]
%   Package \verb|thumbpdf| may additionally be loaded for the case that the
%   PDF file also should contain thumbnails, see the documentation
%   of this package. But thumbnail embedding is not really
%   necessary, because nowadays AcrobatReader is able to
%   generate and view thumbnails on the fly.
% \end{description}
%
% \section{Usage}
%
% The purpose of this packages is to add PDF features to
% the documentation of LaTeX packages. Thus just load
% the package in the driver file that generates the
% documentation:
% \begin{quote}
%   \verb|\usepackage{hypdoc}|
% \end{quote}
% Package \verb|doc| is loaded by \verb|hypdoc|, thus you
% also can replace a \verb|\usepackage{doc}| with this line.
% At any case, the package must be loaded before macros of
% package \verb|doc| such as \verb|\CodelineIndex| or
% \verb|\PageIndex| are used.
%
% If the class \verb|ltxdoc| is used, you can also use
% the configuration file for this class to add
% package \verb|hypdoc|. Add the following line
% to the file \verb|ltxdoc.cfg| (or create a new one):
% \begin{quote}
%\begin{verbatim}
%\AtEndOfClass{\RequirePackage{hypdoc}}
%\end{verbatim}
% \end{quote}
%
% \subsection{Options}
%
% There is one option \texttt{numbered}. It controls,
% whether the starred versions of section or chapter
% inside \cs{tableofcontents}, \cs{listoffigures}, \dots
% should be replaced by the non-star versions.
%
% \subsection{Simple example}
% Example of a simple driver file \verb|foobar.drv|
% of a package \verb|foobar|. Often the driver file
% is packed in the file \verb|foobar.dtx| and can
% be extracted by docstrip or if an instruction file
% for docstrip is provided by:
% \begin{quote}
% \verb|tex foobar.ins|
% \end{quote}
% Note that although docstrip doesn't rely on \LaTeX,
% sometimes you need to run the \verb|.ins| file through
% \LaTeX. Oddly enough, often the only \LaTeX\ feature
% in use is \verb|\NeedsTeXFormat{LaTeX2e}| in the first line.
% \begin{quote}
%\begin{verbatim}
%\documentclass{ltxdoc}
%\usepackage{hypdoc}
%\RecordChanges
%\EnableCrossrefs
%\CodelineIndex
%\begin{document}
%\title{Title for package \texttt{foobar}}
%\author{Author of package \texttt{foobar}}
%\date{...}
%\maketitle
%\tableofcontents
%\DocInput{foobar.dtx}
%\PrintChanges
%\PrintIndex
%\end{document}
%\end{verbatim}
% \end{quote}
%
% Then compile it, for example with pdf\LaTeX.
% \begin{quote}
%\begin{verbatim}
%pdflatex foobar.drv
%makeindex -s gind.ist foobar.idx
%makeindex -s gglo.ist -o foobar.gls foobar.glo
%pdflatex foobar.drv
%\end{verbatim}
% \end{quote}
% You will need several cycles until the cross references
% and similar things are correct.
%
% \subsection{Example \texttt{source2e.tex}}
%
% The source code of the base of \LaTeX\ is available as
% a bunch of \verb|.dtx| files. \LaTeX\ provides the file
% \verb|source2e.tex| to merge them all together.
%
% Now you can either load package \verb|hypdoc| in a private copy
% of this file (please, respect the LPPL) or use the configuration
% file \verb|ltxdoc.cfg|. Example that also sets A4 paper size:
% \begin{quote}
%\begin{verbatim}
%\PassOptionsToClass{a4paper}{article}
%\AtEndOfClass{\RequirePackage{hypdoc}}
%\end{verbatim}
% \end{quote}
%
% Then \verb|source2e.tex| can be compiled by pdf\TeX, for example:
% \begin{quote}
%\begin{verbatim}
%pdflatex source2e
%makeindex -s gind.ist source2e.idx
%makeindex -s gglo.ist -o source2e.glo source2e.gls
%pdflatex source2e
%makeindex -s gind.ist source2e.idx
%makeindex -s gglo.ist -o source2e.glo source2e.gls
%pdflatex source2e
%\end{verbatim}
% \end{quote}
%
% \StopEventually{
% }
%
% \section{Implementation}
%
%    \begin{macrocode}
%<*package>
\NeedsTeXFormat{LaTeX2e}
\ProvidesPackage{hypdoc}%
  [2018/09/07 v1.14 Hyper extensions for doc.sty (HO)]%
%    \end{macrocode}
%    The package identification is done at the top of the |.dtx| file
%    in order to use only one identification string.
%
%    For unique command names this package uses |HD@| as prefix
%
% \subsection{Options}
%
%    \begin{macro}{\ifHD@numbered}
%    \begin{macrocode}
\newif\ifHD@numbered
%    \end{macrocode}
%    \end{macro}
%    \begin{macrocode}
\DeclareOption{numbered}{\HD@numberedtrue}
\ProcessOptions*\relax
%    \end{macrocode}
%
% \subsection{Package loading}
%
%    \begin{macrocode}
\RequirePackage{atveryend}[2010/03/24]
\RequirePackage{doc}
%    \end{macrocode}
%    \begin{macrocode}
\ifx\numexpr\@undefined
  \RequirePackage{calc}%
\fi
%    \end{macrocode}
%    \begin{macrocode}
\RequirePackage[%
  hyperindex=false,%
  pdfusetitle,%
  colorlinks,%
  pdfpagelabels%
]{hyperref}[2002/05/09]%
\hypersetup{%
  plainpages=false,%
  bookmarksopen,%
  bookmarksnumbered,%
  bookmarksopenlevel=0,%
  pdfstartview={FitBH \hypercalcbp{%
    \paperheight-\topmargin-1in-\headheight-\headsep
  }},%
}
%    \end{macrocode}
%
%    \begin{macrocode}
\newcounter{HD@unique}
%    \end{macrocode}
%    \begin{macro}{\theHD@unique}
%    \begin{macrocode}
\renewcommand{\theHD@unique}{%
  \number\value{HD@unique}%
}
%    \end{macrocode}
%    \end{macro}
%
%    \begin{macrocode}
\g@addto@macro\Hy@UseMaketitleInfos{%
  \@ifundefined{fileinfo}{}{%
    \ifx\@pdfsubject\@empty
      \hypersetup{pdfsubject={\fileinfo}}%
    \fi
  }%
}
%    \end{macrocode}
%
%    \begin{macrocode}
\pdfstringdefDisableCommands{%
  \let\thanks\@gobble
  \let\footnote\@gobble
  \def\cs#1{\textbackslash #1}%
  \let\normalfont\@empty
  \let\scshape\@empty
  \def\and{and }%
}
%    \end{macrocode}
%
% \subsubsection{Patch for bug latex/4096}
%
%    \begin{macrocode}
\begingroup\expandafter\expandafter\expandafter\endgroup
\expandafter\ifx\csname mod@math@codes\endcsname\relax
\else
  \def\HD@ModFix#1{%
    \mathcode`#1=`#1\relax
  }%
  \g@addto@macro\mod@math@codes{%
    \HD@ModFix\-%
    \HD@ModFix\+%
    \HD@ModFix\:%
    \HD@ModFix\=%
    \HD@ModFix\*%
  }%
\fi
%    \end{macrocode}
%
% \subsection{Bookmarks}
%
%    \begin{macro}{\HD@sectionpatch}
%    \begin{macrocode}
\def\HD@sectionpatch{%
  \ifHD@numbered
    \HD@@sectionpatch{section}%
    \HD@@sectionpatch{chapter}%
  \else
    \let\Hy@writebookmark\HD@disable@writebookmark
  \fi
}
%    \end{macrocode}
%    \end{macro}
%    \begin{macro}{\HD@@sectionpatch}
%    \begin{macrocode}
\def\HD@@sectionpatch#1{%
  \expandafter\let\csname HDorg@@#1\expandafter\endcsname
    \csname #1\endcsname
  \@namedef{#1}{%
    \@ifstar{\@nameuse{HDorg@@#1}}{\@nameuse{HDorg@@#1}}%
  }%
}
%    \end{macrocode}
%    \end{macro}
%    \begin{macro}{\HD@disable@writebookmark}
%    \begin{macrocode}
\def\HD@disable@writebookmark#1#2#3#4#5{}
%    \end{macrocode}
%    \end{macro}
%
%    \begin{macro}{\HD@guesstoclevel}
%    A wild guessing of the toclevel. There is the hope, that
%    the macro |#1| starts with a sectioning command.
%    \begin{macrocode}
\def\HD@guesstoclevel#1{%
  \expandafter\HD@@guesstoclevel#1\section\HD@end
}
%    \end{macrocode}
%    \end{macro}
%    \begin{macro}{\HD@@guesstoclevel}
%    \begin{macrocode}
\def\HD@@guesstoclevel#1#2\HD@end{%
  \@ifundefined{toclevel@\expandafter\@gobble\string#1\@empty}{%
    1%
  }{%
    \@nameuse{toclevel@\expandafter\@gobble\string#1\@empty}%
  }%
}
%    \end{macrocode}
%    \end{macro}
%
% \subsection{\cs{tableofcontents}}
%
%    \begin{macro}{\HDorg@tableofcontents}
%    \begin{macrocode}
\let\HDorg@tableofcontents\tableofcontents
%    \end{macrocode}
%    \end{macro}
%    \begin{macro}{\tableofcontents}
%    As first section the table of contents shouldn't
%    go into the toc itself. Also a number looks too funny then.
%    Just a bookmark is enough in this case.
%    \begin{macrocode}
\def\tableofcontents{%
  \begingroup
    \@ifundefined{c@section}{%
      \def\c@section{0}%
    }{}%
  \expandafter\endgroup
  \ifcase\c@section
    \stepcounter{HD@unique}%
    \pdfbookmark[\HD@guesstoclevel{\HDorg@tableofcontents}]%
        {\contentsname}{toc\theHD@unique}%
    \HDorg@tableofcontents
  \else
    \ifHD@numbered
    \else
      \stepcounter{HD@unique}%
      \pdfbookmark[\HD@guesstoclevel{\HDorg@tableofcontents}]%
          {\contentsname}{toc\theHD@unique}%
    \fi
    \begingroup
      \HD@sectionpatch
      \HDorg@tableofcontents
    \endgroup
  \fi
}
%    \end{macrocode}
%    \end{macro}
%
% \subsection{\cs{listoffigures}}
%
%    \begin{macro}{\HDorg@listoffigures}
%    \begin{macrocode}
\let\HDorg@listoffigures\listoffigures
%    \end{macrocode}
%    \end{macro}
%    \begin{macro}{\listoffigures}
%    \begin{macrocode}
\def\listoffigures{%
  \ifHD@numbered
  \else
    \stepcounter{HD@unique}%
    \pdfbookmark[\HD@guesstoclevel{\HDorg@listoffigures}]%
        {\listfigurename}{toc\theHD@unique}%
  \fi
  \begingroup
    \HD@sectionpatch
    \HDorg@listoffigures
  \endgroup
}
%    \end{macrocode}
%    \end{macro}
%
% \subsection{\cs{listoftables}}
%
%    \begin{macro}{\HDorg@listoftables}
%    \begin{macrocode}
\let\HDorg@listoftables\listoftables
%    \end{macrocode}
%    \end{macro}
%    \begin{macro}{\listoftables}
%    \begin{macrocode}
\def\listoftables{%
  \ifHD@numbered
  \else
    \stepcounter{HD@unique}%
    \pdfbookmark[\HD@guesstoclevel{\HDorg@listoftables}]%
        {\listtablename}{toc\theHD@unique}%
  \fi
  \begingroup
    \HD@sectionpatch
    \HDorg@listoftables
  \endgroup
}
%    \end{macrocode}
%    \end{macro}
%
% \subsection{\cs{thebibliography}}
%
%    \begin{macro}{\HDorg@thebibliography}
%    \begin{macrocode}
\let\HDorg@thebibliography\thebibliography
%    \end{macrocode}
%    \end{macro}
%    \begin{macro}{\thebibliography}
%    \begin{macrocode}
\def\thebibliography{%
  \ifHD@numbered
  \else
    \stepcounter{HD@unique}%
    \pdfbookmark[\HD@guesstoclevel{\HDorg@thebibliography}]%
      {\@ifundefined{chapter}{\refname}{\bibname}}{bib\theHD@unique}%
  \fi
  \HD@sectionpatch
  \HDorg@thebibliography
}
%    \end{macrocode}
%    \end{macro}
%
% \subsection{\cs{theindex}}
%
%    \begin{macro}{\HDorg@theindex}
%    \begin{macrocode}
\let\HDorg@theindex\theindex
%    \end{macrocode}
%    \end{macro}
%    \begin{macro}{\theindex}
%    \begin{macrocode}
\def\theindex{%
  \let\HDorg@index@prologue\index@prologue
  \let\HDorg@writebookmark\Hy@writebookmark
  \HD@sectionpatch
  \g@addto@macro\IndexParms{%
    \ifHD@numbered
    \else
      \let\Hy@writebookmark\HDorg@writebookmark
    \fi
    \let\bfseries\HD@bfseries
  }%
  \edef\HD@toclevel@index{\HD@guesstoclevel{\index@prologue}}%
  \begingroup
    \count@=\HD@toclevel@index\relax
    \advance\count@ by 1 %
    \xdef\HD@toclevel@subindex{\the\count@}%
  \endgroup
  \def\index@prologue{%
    \ifHD@numbered
    \else
      \begingroup
        \stepcounter{HD@unique}%
        \let\Hy@writebookmark\HDorg@writebookmark
        \pdfbookmark[\HD@toclevel@index]%
            {\indexname}{index\theHD@unique}%
      \endgroup
    \fi
    \HDorg@index@prologue
  }%
  \HDorg@theindex
}
%    \end{macrocode}
%    \end{macro}
%    \begin{macro}{\HDorg@bfseries}
%    \begin{macrocode}
\let\HDorg@bfseries\bfseries
%    \end{macrocode}
%    \end{macro}
%    \begin{macro}{\HD@bfseries}
%    \begin{macrocode}
\def\HD@bfseries{\futurelet\HD@hfil\HD@@bfseries}%
%    \end{macrocode}
%    \end{macro}
%    \begin{macro}{\HD@@bfseries}
%    \begin{macrocode}
\def\HD@@bfseries{%
  \HDorg@bfseries
  \ifx\HD@hfil\hfil
    \expandafter\HD@@@bfseries
  \fi
}
%    \end{macrocode}
%    \end{macro}
%    \begin{macro}{\HD@@@bfseries}
%    \begin{macrocode}
\def\HD@@@bfseries\hfil#1\hfil{%
  \ifx\\#1\\%
  \else
    \raisebox{\baselineskip}[0pt]{%
      \kern-\HD@margin\relax
      \pdfbookmark[\HD@toclevel@subindex]{#1}{HD.#1}%
      \kern\HD@margin\relax
    }%
  \fi
  \hfil#1\hfil
}%
%    \end{macrocode}
%    \end{macro}
%    \begin{macro}{\HD@margin}
%    \begin{macrocode}
\def\HD@margin{1mm}
%    \end{macrocode}
%    \end{macro}
%
% \subsection{\cs{theglossary}}
%
%    \begin{macro}{\HDorg@theglossary}
%    \begin{macrocode}
\let\HDorg@theglossary\theglossary
%    \end{macrocode}
%    \end{macro}
%    \begin{macro}{\theglossary}
%    \begin{macrocode}
\def\theglossary{%
  \let\HDorg@glossary@prologue\glossary@prologue
  \let\HDorg@writebookmark\Hy@writebookmark
  \HD@sectionpatch
  \def\glossary@prologue{%
    \ifHD@numbered
    \else
      \begingroup
        \let\Hy@writebookmark\HDorg@writebookmark
        \stepcounter{HD@unique}%
        \pdfbookmark[\HD@guesstoclevel{\HDorg@glossary@prologue}]%
            {\changehistoryname}{glossary\theHD@unique}%
      \endgroup
    \fi
    \HDorg@glossary@prologue
  }%
  \HDorg@theglossary
}
%    \end{macrocode}
%    \end{macro}
%    \begin{macro}{\glossaryname}
%    \begin{macrocode}
\providecommand*{\glossaryname}{Change History}
%    \end{macrocode}
%    \end{macro}
%
%    \begin{macro}{\glossaryname}
%    \begin{macrocode}
\providecommand*{\changehistoryname}{\glossaryname}
%    \end{macrocode}
%    \end{macro}
%
% \subsection{Index}
%
%    \begin{macrocode}
\newcounter{HD@hypercount}
%    \end{macrocode}
%    \begin{macro}{\HD@helplength}
%    \begin{macrocode}
\newlength{\HD@helplength}
%    \end{macrocode}
%    \end{macro}
%    \begin{macro}{\ifHD@savedest}
%    \begin{macrocode}
\newif\ifHD@savedest
\HD@savedesttrue
%    \end{macrocode}
%    \end{macro}
%    \begin{macrocode}
\AtBeginDocument{%
  \HD@savedest@init
}
%    \end{macrocode}
%    \begin{macrocode}
\RequirePackage{rerunfilecheck}[2009/12/10]
%    \end{macrocode}
%    \begin{macro}{\HD@savedest@init}
%    \begin{macrocode}
\begingroup
  \catcode`\^^M=\active
\@firstofone{\endgroup
  \def\HD@savedest@init{%
    \ifHD@savedest
      \let\HDorg@tilde~%
      \let~\HD@savedest@entry
      \edef\HDorg@catcodeM{\the\catcode`\^^M}%
      \catcode`\^^M=\active
      \def^^M{}%
      \InputIfFileExists{\jobname.hd}{}{}%
      \let~\HDorg@tilde
      \catcode`\^^M=\HDorg@catcodeM\relax
      \newwrite\HD@savedest@out
      \RerunFileCheck{\jobname.hd}{%
        \immediate\closeout\HD@savedest@out
      }{}%
      \immediate\openout\HD@savedest@out=\jobname.hd\relax
    \else
      \let\HD@savedest@add\@gobble
    \fi
  }%
%    \end{macrocode}
%    \end{macro}
%    \begin{macro}{\HD@savedest@entry}
%    \begin{macrocode}
  \def\HD@savedest@entry#1^^M{%
    \@namedef{HD.#1}{}%
  }%
}
%    \end{macrocode}
%    \end{macro}
%    \begin{macro}{\HD@savedest@add}
%    \begin{macrocode}
\def\HD@savedest@add#1{%
  \immediate\write\HD@savedest@out{\string~#1}%
}
%    \end{macrocode}
%    \end{macro}
%    \begin{macro}{\ifHD@targetraise}
%    \begin{macrocode}
\newif\ifHD@targetraise
\HD@targetraisetrue
%    \end{macrocode}
%    \end{macro}
%    \begin{macro}{\HD@target}
%    \begin{macrocode}
\def\HD@target{%
  \ifHD@targetraise
    \ifhmode
      \settoheight{\HD@helplength}{[H]}%
      \addtolength{\HD@helplength}{0.4ex}%
      \raisebox{\HD@helplength}[0pt][0pt]{%
        \HD@maketarget
      }%
    \else
      \HD@maketarget
    \fi
  \else
    \HD@maketarget
  \fi
}
%    \end{macrocode}
%    \end{macro}
%    \begin{macro}{\HD@maketarget}
%    \begin{macrocode}
\def\HD@maketarget{%
  \stepcounter{HD@hypercount}%
  \def\HD@next{%
    \llap{\hypertarget{HD.\the\c@HD@hypercount}{}\,}%
  }%
  \ifHD@savedest
    \expandafter\ifx\csname HD.\the\c@HD@hypercount\endcsname\@empty
    \else
      \let\HD@next\relax
    \fi
  \fi
  \HD@next
}
%    \end{macrocode}
%    \end{macro}
%
%    \begin{macro}{\CodelineIndex}
%    \begin{macrocode}
\g@addto@macro\CodelineIndex{%
  \let\special@index\HD@codeline@wrindex
}
%    \end{macrocode}
%    \end{macro}
%    \begin{macro}{\PageIndex}
%    \begin{macrocode}
\g@addto@macro\PageIndex{%
  \let\special@index\HD@page@wrindex
}
%    \end{macrocode}
%    \end{macro}
%
%    \begin{macro}{\HD@codeline@wrindex}
%    \begin{macrocode}
\def\HD@codeline@wrindex#1{%
  \begingroup
    \let\HDorg@encapchar\encapchar
    \def\encapchar##1\encapchar##2\@nil{%
      \HDorg@encapchar
      hdclindex{\the\c@HD@hypercount}{##1}%
    }%
    \codeline@wrindex{%
      #1\encapchar\encapchar\@nil
    }%
  \endgroup
}
%    \end{macrocode}
%    \end{macro}
%    \begin{macro}{\HD@page@wrindex}
%    \begin{macrocode}
\def\HD@page@wrindex#1{%
  \begingroup
    \let\HDorg@encapchar\encapchar
    \def\encapchar##1\encapchar##2\@nil{%
      \HDorg@encapchar
      hdpindex{##1}%
    }%
    \HDorg@index{#1\encapchar\encapchar\@nil}%
  \endgroup
}
%    \end{macrocode}
%    \end{macro}
%
%    \begin{macro}{\hdclindex}
%    \begin{macrocode}
\def\hdclindex#1#2#3{%
  \csname\ifx\\#2\\relax\else#2\fi\endcsname{%
    \hyperlink{HD.#1}{#3}%
    \HD@savedest@add{#1}%
  }%
}
%    \end{macrocode}
%    \end{macro}
%    \begin{macro}{\hdpindex}
%    \begin{macrocode}
\def\hdpindex#1#2{%
  \csname\ifx\\#1\\relax\else#1\fi\endcsname{%
    \hyperpage{#2}%
  }%
}
%    \end{macrocode}
%    \end{macro}
%
% \subsection{\cs{PrintDescribeMacro}}
%
%    \begin{macro}{\HDorg@PrintDescribeMacro}
%    \begin{macrocode}
\let\HDorg@PrintDescribeMacro\PrintDescribeMacro
%    \end{macrocode}
%    \end{macro}
%    \begin{macro}{\PrintDescribeMacro}
%    \begin{macrocode}
\renewcommand\PrintDescribeMacro[1]{%
  \begingroup
    \let\HDorg@MacroFont\MacroFont
    \def\MacroFont{%
      \HD@target
      \HDorg@MacroFont
    }%
    \HDorg@PrintDescribeMacro{#1}%
  \endgroup
}
%    \end{macrocode}
%    \end{macro}
%
% \subsection{\cs{PrintMacroName}}
%
%    \begin{macro}{\HDorg@PrintMacroName}
%    \begin{macrocode}
\let\HDorg@PrintMacroName\PrintMacroName
%    \end{macrocode}
%    \end{macro}
%    \begin{macro}{\PrintMacroName}
%    \begin{macrocode}
\renewcommand\PrintMacroName[1]{%
  \begingroup
    \let\HDorg@MacroFont\MacroFont
    \def\MacroFont{%
      \HD@target
      \HDorg@MacroFont
    }%
    \HDorg@PrintMacroName{#1}%
  \endgroup
}
%    \end{macrocode}
%    \end{macro}
%
% \subsection{\cs{theCodelineNo}}
%
%    \begin{macro}{\HDorg@theCodelineNo}
%    \begin{macrocode}
\let\HDorg@theCodelineNo\theCodelineNo
%    \end{macrocode}
%    \end{macro}
%    \begin{macro}{\theCodelineNo}
%    \begin{macrocode}
\renewcommand*\theCodelineNo{%
  \settowidth{\HD@helplength}{\HDorg@theCodelineNo\ }%
  \ifdim\HD@helplength<\MacroIndent
    \addtolength{\HD@helplength}{-\MacroIndent}%
    \llap{%
      \begingroup\HD@target\endgroup
      \hspace*{-\HD@helplength}%
    }%
  \else
    \HD@target
  \fi
  \HDorg@theCodelineNo
}
%    \end{macrocode}
%    \end{macro}
%
%    \begin{macro}{\HDorg@index}
%    \begin{macrocode}
\AtBeginDocument{%
  \let\HDorg@index\index
}
%    \end{macrocode}
%    \end{macro}
%
% \subsection{\cs{SpecialUsageIndex}}
%
%    \begin{macro}{\HDorg@SpecialUsageIndex}
%    \begin{macrocode}
\let\HDorg@SpecialUsageIndex\SpecialUsageIndex
\renewcommand*\SpecialUsageIndex[1]{%
  \@bsphack
  \begingroup
    \HD@target
    \let\index\HDorg@index
    \let\HDorg@encapchar\encapchar
    \edef\encapchar usage{%
       \HDorg@encapchar hdclindex{\the\c@HD@hypercount}{usage}%
    }%
    \HDorg@SpecialUsageIndex{#1}%
  \endgroup
  \@esphack
}
%    \end{macrocode}
%    \end{macro}
%
% \subsection{\cs{autoindex}}
%
%    \begin{macrocode}
\begingroup\expandafter\expandafter\expandafter\endgroup
\expandafter\ifx\csname autoindex\endcsname\relax
\else
%    \end{macrocode}
%    \begin{macro}{\HDorg@autoindex}
%    \begin{macrocode}
  \let\HDorg@autoindex\autoindex
%    \end{macrocode}
%    \end{macro}
%    \begin{macro}{\autoindex}
%    \begin{macrocode}
  \renewcommand*\autoindex[1]{%
    \@bsphack
    \begingroup
      \HD@target
      \let\index\HDorg@index
      \let\HDorg@encapchar\encapchar
      \edef\encapchar usage{%
         \HDorg@encapchar hdclindex{\the\c@HD@hypercount}{usage}%
      }%
      \HDorg@autoindex{#1}%
    \endgroup
    \@esphack
  }%
\fi
%    \end{macrocode}
%    \end{macro}
%
% \subsection{\cs{SpecialEnvIndex}}
%
%    \begin{macro}{\HDorg@SpecialEnvIndex}
%    \begin{macrocode}
\let\HDorg@SpecialEnvIndex\SpecialEnvIndex
%    \end{macrocode}
%    \end{macro}
%    \begin{macro}{\SpecialEnvIndex}
%    \begin{macrocode}
\renewcommand*\SpecialEnvIndex[1]{%
  \@bsphack
  \begingroup
    \HD@target
    \let\HDorg@encapchar\encapchar
    \edef\encapchar usage{%
      \HDorg@encapchar hdclindex{\the\c@HD@hypercount}{usage}%
    }%
    \HDorg@SpecialEnvIndex{#1}%
  \endgroup
  \@esphack
}
%    \end{macrocode}
%    \end{macro}
%
% \subsection{\cs{SortIndex}}
%
%    \begin{macro}{\HDorg@SortIndex}
%    \begin{macrocode}
\let\HDorg@SortIndex\SortIndex
%    \end{macrocode}
%    \end{macro}
%    \begin{macro}{\SortIndex}
%    \begin{macrocode}
\renewcommand*{\SortIndex}[2]{%
  \@bsphack
  \begingroup
    \let\index\HD@page@wrindex
    \HDorg@SortIndex{#1}{#2}%
  \endgroup
  \@esphack
}
%    \end{macrocode}
%    \end{macro}
%
% \subsection{\cs{@wrglossary}}
%
%    \begin{macro}{\HDorg@wrglossary}
%    \begin{macrocode}
\let\HDorg@wrglossary\@wrglossary
%    \end{macrocode}
%    \end{macro}
%    \begin{macro}{\@wrglossary}
%    \begin{macrocode}
\def\@wrglossary#1{%
  \let\HDorg@encapchar\encapchar
  \def\encapchar##1\encapchar##2\@nil{%
    \HDorg@encapchar
    hdpindex{##1}%
  }%
  \HDorg@wrglossary{#1\encapchar\encapchar\@nil}%
}
%    \end{macrocode}
%    \end{macro}
%
% \subsection{\cs{MacroIndent}}
%
%    \cs{MacroIndent} automatically calculated.
%    \begin{macrocode}
\AfterLastShipout{%
  \if@filesw
    \immediate\write\@auxout{%
      \string\providecommand\string\HD@SetMacroIndent[1]{}%
    }%
    \immediate\write\@auxout{%
      \string\HD@SetMacroIndent{\number\value{CodelineNo}}%
    }%
  \fi
}
%    \end{macrocode}
%    \begin{macro}{\HD@SetMacroIndent}
%    \begin{macrocode}
\def\HD@SetMacroIndent#1{%
  \begingroup
    \value{CodelineNo}=#1\relax
    \settowidth{\MacroIndent}{\HDorg@theCodelineNo\ }%
    \global\MacroIndent\MacroIndent
  \endgroup
}
%    \end{macrocode}
%    \end{macro}
%
%    \begin{macrocode}
%</package>
%    \end{macrocode}
%
% \section{Installation}
%
% \subsection{Download}
%
% \paragraph{Package.} This package is available on
% CTAN\footnote{\url{http://ctan.org/pkg/hypdoc}}:
% \begin{description}
% \item[\CTAN{macros/latex/contrib/oberdiek/hypdoc.dtx}] The source file.
% \item[\CTAN{macros/latex/contrib/oberdiek/hypdoc.pdf}] Documentation.
% \end{description}
%
%
% \paragraph{Bundle.} All the packages of the bundle `oberdiek'
% are also available in a TDS compliant ZIP archive. There
% the packages are already unpacked and the documentation files
% are generated. The files and directories obey the TDS standard.
% \begin{description}
% \item[\CTAN{install/macros/latex/contrib/oberdiek.tds.zip}]
% \end{description}
% \emph{TDS} refers to the standard ``A Directory Structure
% for \TeX\ Files'' (\CTAN{tds/tds.pdf}). Directories
% with \xfile{texmf} in their name are usually organized this way.
%
% \subsection{Bundle installation}
%
% \paragraph{Unpacking.} Unpack the \xfile{oberdiek.tds.zip} in the
% TDS tree (also known as \xfile{texmf} tree) of your choice.
% Example (linux):
% \begin{quote}
%   |unzip oberdiek.tds.zip -d ~/texmf|
% \end{quote}
%
% \paragraph{Script installation.}
% Check the directory \xfile{TDS:scripts/oberdiek/} for
% scripts that need further installation steps.
% Package \xpackage{attachfile2} comes with the Perl script
% \xfile{pdfatfi.pl} that should be installed in such a way
% that it can be called as \texttt{pdfatfi}.
% Example (linux):
% \begin{quote}
%   |chmod +x scripts/oberdiek/pdfatfi.pl|\\
%   |cp scripts/oberdiek/pdfatfi.pl /usr/local/bin/|
% \end{quote}
%
% \subsection{Package installation}
%
% \paragraph{Unpacking.} The \xfile{.dtx} file is a self-extracting
% \docstrip\ archive. The files are extracted by running the
% \xfile{.dtx} through \plainTeX:
% \begin{quote}
%   \verb|tex hypdoc.dtx|
% \end{quote}
%
% \paragraph{TDS.} Now the different files must be moved into
% the different directories in your installation TDS tree
% (also known as \xfile{texmf} tree):
% \begin{quote}
% \def\t{^^A
% \begin{tabular}{@{}>{\ttfamily}l@{ $\rightarrow$ }>{\ttfamily}l@{}}
%   hypdoc.sty & tex/latex/oberdiek/hypdoc.sty\\
%   hypdoc.pdf & doc/latex/oberdiek/hypdoc.pdf\\
%   hypdoc.dtx & source/latex/oberdiek/hypdoc.dtx\\
% \end{tabular}^^A
% }^^A
% \sbox0{\t}^^A
% \ifdim\wd0>\linewidth
%   \begingroup
%     \advance\linewidth by\leftmargin
%     \advance\linewidth by\rightmargin
%   \edef\x{\endgroup
%     \def\noexpand\lw{\the\linewidth}^^A
%   }\x
%   \def\lwbox{^^A
%     \leavevmode
%     \hbox to \linewidth{^^A
%       \kern-\leftmargin\relax
%       \hss
%       \usebox0
%       \hss
%       \kern-\rightmargin\relax
%     }^^A
%   }^^A
%   \ifdim\wd0>\lw
%     \sbox0{\small\t}^^A
%     \ifdim\wd0>\linewidth
%       \ifdim\wd0>\lw
%         \sbox0{\footnotesize\t}^^A
%         \ifdim\wd0>\linewidth
%           \ifdim\wd0>\lw
%             \sbox0{\scriptsize\t}^^A
%             \ifdim\wd0>\linewidth
%               \ifdim\wd0>\lw
%                 \sbox0{\tiny\t}^^A
%                 \ifdim\wd0>\linewidth
%                   \lwbox
%                 \else
%                   \usebox0
%                 \fi
%               \else
%                 \lwbox
%               \fi
%             \else
%               \usebox0
%             \fi
%           \else
%             \lwbox
%           \fi
%         \else
%           \usebox0
%         \fi
%       \else
%         \lwbox
%       \fi
%     \else
%       \usebox0
%     \fi
%   \else
%     \lwbox
%   \fi
% \else
%   \usebox0
% \fi
% \end{quote}
% If you have a \xfile{docstrip.cfg} that configures and enables \docstrip's
% TDS installing feature, then some files can already be in the right
% place, see the documentation of \docstrip.
%
% \subsection{Refresh file name databases}
%
% If your \TeX~distribution
% (\teTeX, \mikTeX, \dots) relies on file name databases, you must refresh
% these. For example, \teTeX\ users run \verb|texhash| or
% \verb|mktexlsr|.
%
% \subsection{Some details for the interested}
%
% \paragraph{Attached source.}
%
% The PDF documentation on CTAN also includes the
% \xfile{.dtx} source file. It can be extracted by
% AcrobatReader 6 or higher. Another option is \textsf{pdftk},
% e.g. unpack the file into the current directory:
% \begin{quote}
%   \verb|pdftk hypdoc.pdf unpack_files output .|
% \end{quote}
%
% \paragraph{Unpacking with \LaTeX.}
% The \xfile{.dtx} chooses its action depending on the format:
% \begin{description}
% \item[\plainTeX:] Run \docstrip\ and extract the files.
% \item[\LaTeX:] Generate the documentation.
% \end{description}
% If you insist on using \LaTeX\ for \docstrip\ (really,
% \docstrip\ does not need \LaTeX), then inform the autodetect routine
% about your intention:
% \begin{quote}
%   \verb|latex \let\install=y\input{hypdoc.dtx}|
% \end{quote}
% Do not forget to quote the argument according to the demands
% of your shell.
%
% \paragraph{Generating the documentation.}
% You can use both the \xfile{.dtx} or the \xfile{.drv} to generate
% the documentation. The process can be configured by the
% configuration file \xfile{ltxdoc.cfg}. For instance, put this
% line into this file, if you want to have A4 as paper format:
% \begin{quote}
%   \verb|\PassOptionsToClass{a4paper}{article}|
% \end{quote}
% An example follows how to generate the
% documentation with pdf\LaTeX:
% \begin{quote}
%\begin{verbatim}
%pdflatex hypdoc.dtx
%makeindex -s gind.ist hypdoc.idx
%pdflatex hypdoc.dtx
%makeindex -s gind.ist hypdoc.idx
%pdflatex hypdoc.dtx
%\end{verbatim}
% \end{quote}
%
% \section{Catalogue}
%
% The following XML file can be used as source for the
% \href{http://mirror.ctan.org/help/Catalogue/catalogue.html}{\TeX\ Catalogue}.
% The elements \texttt{caption} and \texttt{description} are imported
% from the original XML file from the Catalogue.
% The name of the XML file in the Catalogue is \xfile{hypdoc.xml}.
%    \begin{macrocode}
%<*catalogue>
<?xml version='1.0' encoding='us-ascii'?>
<!DOCTYPE entry SYSTEM 'catalogue.dtd'>
<entry datestamp='$Date$' modifier='$Author$' id='hypdoc'>
  <name>hypdoc</name>
  <caption>Hyper extensions for doc.sty.</caption>
  <authorref id='auth:oberdiek'/>
  <copyright owner='Heiko Oberdiek' year='2002,2006,2007,2009-2011'/>
  <license type='lppl1.3'/>
  <version number='1.14'/>
  <description>
    This package adds hypertext features to the package
    <xref refid='doc'>doc</xref> that is used in the documentation
    system of <xref refid='latex'>LaTeX2e</xref>. Bookmarks are added
    and references are linked as far as possible.
    <p/>
    The package is part of the <xref refid='oberdiek'>oberdiek</xref>
    bundle.
  </description>
  <documentation details='Package documentation'
      href='ctan:/macros/latex/contrib/oberdiek/hypdoc.pdf'/>
  <ctan file='true' path='/macros/latex/contrib/oberdiek/hypdoc.dtx'/>
  <miktex location='oberdiek'/>
  <texlive location='oberdiek'/>
  <install path='/macros/latex/contrib/oberdiek/oberdiek.tds.zip'/>
</entry>
%</catalogue>
%    \end{macrocode}
%
% \begin{History}
%   \begin{Version}{2002/05/10 v1.0}
%   \item
%     First standalone version.
%   \end{Version}
%   \begin{Version}{2006/02/20 v1.1}
%   \item
%     Option \xoption{numbered} added.
%   \item
%     \cs{HD@savedest@init} fixed (empty \xext{hd} file triggered
%     first page with fl ligature).
%   \item
%     Bookmarks for \cs{listoffigures} and \cs{listoftables}.
%   \item
%     Documentation: features and usage added.
%   \item
%     LPPL 1.3.
%   \end{Version}
%   \begin{Version}{2006/04/27 v1.2}
%   \item
%     \cs{HD@target} fixed (\xfile{multicol.dtx}).
%   \end{Version}
%   \begin{Version}{2006/06/01 v1.3}
%   \item
%     Support for package \xpackage{amsdtx}'s \cs{autoindex} added.
%   \end{Version}
%   \begin{Version}{2006/06/02 v1.4}
%   \item
%     Bookmark fixed if there are several table of contents,
%     bibliographies, glossaries, \dots
%   \end{Version}
%   \begin{Version}{2007/04/11 v1.5}
%   \item
%     Line ends sanitized.
%   \end{Version}
%   \begin{Version}{2007/11/12 v1.6}
%   \item
%     Bug fix in \cs{listoffigures} (Axel Sommerfeldt).
%   \item
%     Markup added in implementation section.
%   \end{Version}
%   \begin{Version}{2009/12/10 v1.7}
%   \item
%     Use of package \xpackage{rerunfilecheck}.
%   \end{Version}
%   \begin{Version}{2010/02/03 v1.8}
%   \item
%     Fix for bug latex/4096 added.
%   \end{Version}
%   \begin{Version}{2010/03/26 v1.9}
%   \item
%     Definition of \cs{HD@SetMacroIndex} is provided in the \xext{aux} file.
%   \end{Version}
%   \begin{Version}{2011/05/05 v1.10}
%   \item
%     \cs{raisebox} in \cs{HD@target} with height and depth of \texttt{0pt}
%     (request of GL).
%   \end{Version}
%   \begin{Version}{2011/08/19 v1.11}
%   \item
%     Latest comma in option list for hyperref removed to avoid option clash
%     because of \hologo{LaTeX} 2011/06/27 with the new \cs{in@}.
%   \end{Version}
%   \begin{Version}{2016/05/16 v1.12}
%   \item
%     Documentation updates.
%   \end{Version}
%   \begin{Version}{2016/10/02 v1.13}
%   \item
%     Fixes to bookmarknames, also introduce \cs{changehistoryname}
%   \end{Version}
%   \begin{Version}{2018/09/07 v1.14}
%   \item
%     Do not load \xpackage{thumpdf}.
%   \end{Version}
% \end{History}
%
% \PrintIndex
%
% \Finale
\endinput

%        (quote the arguments according to the demands of your shell)
%
% Documentation:
%    (a) If hypdoc.drv is present:
%           latex hypdoc.drv
%    (b) Without hypdoc.drv:
%           latex hypdoc.dtx; ...
%    The class ltxdoc loads the configuration file ltxdoc.cfg
%    if available. Here you can specify further options, e.g.
%    use A4 as paper format:
%       \PassOptionsToClass{a4paper}{article}
%
%    Programm calls to get the documentation (example):
%       pdflatex hypdoc.dtx
%       makeindex -s gind.ist hypdoc.idx
%       pdflatex hypdoc.dtx
%       makeindex -s gind.ist hypdoc.idx
%       pdflatex hypdoc.dtx
%
% Installation:
%    TDS:tex/latex/oberdiek/hypdoc.sty
%    TDS:doc/latex/oberdiek/hypdoc.pdf
%    TDS:source/latex/oberdiek/hypdoc.dtx
%
%<*ignore>
\begingroup
  \catcode123=1 %
  \catcode125=2 %
  \def\x{LaTeX2e}%
\expandafter\endgroup
\ifcase 0\ifx\install y1\fi\expandafter
         \ifx\csname processbatchFile\endcsname\relax\else1\fi
         \ifx\fmtname\x\else 1\fi\relax
\else\csname fi\endcsname
%</ignore>
%<*install>
\input docstrip.tex
\Msg{************************************************************************}
\Msg{* Installation}
\Msg{* Package: hypdoc 2018/09/07 v1.14 Hyper extensions for doc.sty (HO)}
\Msg{************************************************************************}

\keepsilent
\askforoverwritefalse

\let\MetaPrefix\relax
\preamble

This is a generated file.

Project: hypdoc
Version: 2018/09/07 v1.14

Copyright (C) 2002, 2006, 2007, 2009-2011 by
   Heiko Oberdiek <heiko.oberdiek at googlemail.com>

This work may be distributed and/or modified under the
conditions of the LaTeX Project Public License, either
version 1.3c of this license or (at your option) any later
version. This version of this license is in
   http://www.latex-project.org/lppl/lppl-1-3c.txt
and the latest version of this license is in
   http://www.latex-project.org/lppl.txt
and version 1.3 or later is part of all distributions of
LaTeX version 2005/12/01 or later.

This work has the LPPL maintenance status "maintained".

This Current Maintainer of this work is Heiko Oberdiek.

This work consists of the main source file hypdoc.dtx
and the derived files
   hypdoc.sty, hypdoc.pdf, hypdoc.ins, hypdoc.drv.

\endpreamble
\let\MetaPrefix\DoubleperCent

\generate{%
  \file{hypdoc.ins}{\from{hypdoc.dtx}{install}}%
  \file{hypdoc.drv}{\from{hypdoc.dtx}{driver}}%
  \usedir{tex/latex/oberdiek}%
  \file{hypdoc.sty}{\from{hypdoc.dtx}{package}}%
  \nopreamble
  \nopostamble
%  \usedir{source/latex/oberdiek/catalogue}%
%  \file{hypdoc.xml}{\from{hypdoc.dtx}{catalogue}}%
}

\catcode32=13\relax% active space
\let =\space%
\Msg{************************************************************************}
\Msg{*}
\Msg{* To finish the installation you have to move the following}
\Msg{* file into a directory searched by TeX:}
\Msg{*}
\Msg{*     hypdoc.sty}
\Msg{*}
\Msg{* To produce the documentation run the file `hypdoc.drv'}
\Msg{* through LaTeX.}
\Msg{*}
\Msg{* Happy TeXing!}
\Msg{*}
\Msg{************************************************************************}

\endbatchfile
%</install>
%<*ignore>
\fi
%</ignore>
%<*driver>
\NeedsTeXFormat{LaTeX2e}
\ProvidesFile{hypdoc.drv}%
  [2018/09/07 v1.14 Hyper extensions for doc.sty (HO)]%
\documentclass{ltxdoc}
\usepackage{holtxdoc}[2011/11/22]
\begin{document}
  \DocInput{hypdoc.dtx}%
\end{document}
%</driver>
% \fi
%
%
% \CharacterTable
%  {Upper-case    \A\B\C\D\E\F\G\H\I\J\K\L\M\N\O\P\Q\R\S\T\U\V\W\X\Y\Z
%   Lower-case    \a\b\c\d\e\f\g\h\i\j\k\l\m\n\o\p\q\r\s\t\u\v\w\x\y\z
%   Digits        \0\1\2\3\4\5\6\7\8\9
%   Exclamation   \!     Double quote  \"     Hash (number) \#
%   Dollar        \$     Percent       \%     Ampersand     \&
%   Acute accent  \'     Left paren    \(     Right paren   \)
%   Asterisk      \*     Plus          \+     Comma         \,
%   Minus         \-     Point         \.     Solidus       \/
%   Colon         \:     Semicolon     \;     Less than     \<
%   Equals        \=     Greater than  \>     Question mark \?
%   Commercial at \@     Left bracket  \[     Backslash     \\
%   Right bracket \]     Circumflex    \^     Underscore    \_
%   Grave accent  \`     Left brace    \{     Vertical bar  \|
%   Right brace   \}     Tilde         \~}
%
% \GetFileInfo{hypdoc.drv}
%
% \title{The \xpackage{hypdoc} package}
% \date{2018/09/07 v1.14}
% \author{Heiko Oberdiek\thanks
% {Please report any issues at https://github.com/ho-tex/oberdiek/issues}\\
% \xemail{heiko.oberdiek at googlemail.com}}
%
% \maketitle
%
% \begin{abstract}
% This package adds hyper features to the package
% \xpackage{doc} that is used in the documentation
% system of \LaTeXe. Bookmarks are added and references
% are linked as far as possible.
% \end{abstract}
%
% \tableofcontents
%
% \section{Features}
%
% The package adds the following features:
% \begin{description}
% \item[Bookmarks:]
%   For supporting bookmarks package \verb|hyperref| is loaded.
%   Some \TeX\ code inside section titles can cause problems
%   during bookmark processing; these can be warnings, ugly
%   bookmark titles or even errors. With the help of
%   \verb|\pdfstringdefDisableCommands| or \verb|\texorpdfstring|
%   you can manage these things, see \verb|hyperref|'s package
%   documentation.
% \item[Additional bookmarks:]
%   Unnumbered bookmark entries are generated for:
%   \begin{itemize}
%   \item Table of contents
%   \item List of figures
%   \item List of tables
%   \item Index
%   \item Glossary
%   \end{itemize}
%   Option \texttt{numbered} controls, whether these section or
%   chapters should be numbered. The exception is the table
%   of contents if this it would be the first numbered
%   section. Then just a bookmark is added.
% \item[Metadata:]
%   The loaded package \verb|hyperref|, enforced by
%   option \verb|pdfusetitle|, tries to detect \verb|\title|
%   and \verb|\author| and sets the corresponding PDF
%   information fields. In case of problems see the previous
%   item about bookmarks.
%   These enries can be overwritten or new ones added by
%   \verb|\hypersetup| after the packages is loaded, see
%   documentation of package \verb|hyperref|.
% \item[References, links:] Because of package \verb|\hyperref|
%   |\ref| become links, also \verb|\url|, footnote referencing
%   are supported.
% \item[Index:]
%   The main part of this package deals with index links in order
%   to support the kind of index that package \verb|doc| provides.
% \item[Thumbnails:]
%   Package \verb|thumbpdf| may additionally be loaded for the case that the
%   PDF file also should contain thumbnails, see the documentation
%   of this package. But thumbnail embedding is not really
%   necessary, because nowadays AcrobatReader is able to
%   generate and view thumbnails on the fly.
% \end{description}
%
% \section{Usage}
%
% The purpose of this packages is to add PDF features to
% the documentation of LaTeX packages. Thus just load
% the package in the driver file that generates the
% documentation:
% \begin{quote}
%   \verb|\usepackage{hypdoc}|
% \end{quote}
% Package \verb|doc| is loaded by \verb|hypdoc|, thus you
% also can replace a \verb|\usepackage{doc}| with this line.
% At any case, the package must be loaded before macros of
% package \verb|doc| such as \verb|\CodelineIndex| or
% \verb|\PageIndex| are used.
%
% If the class \verb|ltxdoc| is used, you can also use
% the configuration file for this class to add
% package \verb|hypdoc|. Add the following line
% to the file \verb|ltxdoc.cfg| (or create a new one):
% \begin{quote}
%\begin{verbatim}
%\AtEndOfClass{\RequirePackage{hypdoc}}
%\end{verbatim}
% \end{quote}
%
% \subsection{Options}
%
% There is one option \texttt{numbered}. It controls,
% whether the starred versions of section or chapter
% inside \cs{tableofcontents}, \cs{listoffigures}, \dots
% should be replaced by the non-star versions.
%
% \subsection{Simple example}
% Example of a simple driver file \verb|foobar.drv|
% of a package \verb|foobar|. Often the driver file
% is packed in the file \verb|foobar.dtx| and can
% be extracted by docstrip or if an instruction file
% for docstrip is provided by:
% \begin{quote}
% \verb|tex foobar.ins|
% \end{quote}
% Note that although docstrip doesn't rely on \LaTeX,
% sometimes you need to run the \verb|.ins| file through
% \LaTeX. Oddly enough, often the only \LaTeX\ feature
% in use is \verb|\NeedsTeXFormat{LaTeX2e}| in the first line.
% \begin{quote}
%\begin{verbatim}
%\documentclass{ltxdoc}
%\usepackage{hypdoc}
%\RecordChanges
%\EnableCrossrefs
%\CodelineIndex
%\begin{document}
%\title{Title for package \texttt{foobar}}
%\author{Author of package \texttt{foobar}}
%\date{...}
%\maketitle
%\tableofcontents
%\DocInput{foobar.dtx}
%\PrintChanges
%\PrintIndex
%\end{document}
%\end{verbatim}
% \end{quote}
%
% Then compile it, for example with pdf\LaTeX.
% \begin{quote}
%\begin{verbatim}
%pdflatex foobar.drv
%makeindex -s gind.ist foobar.idx
%makeindex -s gglo.ist -o foobar.gls foobar.glo
%pdflatex foobar.drv
%\end{verbatim}
% \end{quote}
% You will need several cycles until the cross references
% and similar things are correct.
%
% \subsection{Example \texttt{source2e.tex}}
%
% The source code of the base of \LaTeX\ is available as
% a bunch of \verb|.dtx| files. \LaTeX\ provides the file
% \verb|source2e.tex| to merge them all together.
%
% Now you can either load package \verb|hypdoc| in a private copy
% of this file (please, respect the LPPL) or use the configuration
% file \verb|ltxdoc.cfg|. Example that also sets A4 paper size:
% \begin{quote}
%\begin{verbatim}
%\PassOptionsToClass{a4paper}{article}
%\AtEndOfClass{\RequirePackage{hypdoc}}
%\end{verbatim}
% \end{quote}
%
% Then \verb|source2e.tex| can be compiled by pdf\TeX, for example:
% \begin{quote}
%\begin{verbatim}
%pdflatex source2e
%makeindex -s gind.ist source2e.idx
%makeindex -s gglo.ist -o source2e.glo source2e.gls
%pdflatex source2e
%makeindex -s gind.ist source2e.idx
%makeindex -s gglo.ist -o source2e.glo source2e.gls
%pdflatex source2e
%\end{verbatim}
% \end{quote}
%
% \StopEventually{
% }
%
% \section{Implementation}
%
%    \begin{macrocode}
%<*package>
\NeedsTeXFormat{LaTeX2e}
\ProvidesPackage{hypdoc}%
  [2018/09/07 v1.14 Hyper extensions for doc.sty (HO)]%
%    \end{macrocode}
%    The package identification is done at the top of the |.dtx| file
%    in order to use only one identification string.
%
%    For unique command names this package uses |HD@| as prefix
%
% \subsection{Options}
%
%    \begin{macro}{\ifHD@numbered}
%    \begin{macrocode}
\newif\ifHD@numbered
%    \end{macrocode}
%    \end{macro}
%    \begin{macrocode}
\DeclareOption{numbered}{\HD@numberedtrue}
\ProcessOptions*\relax
%    \end{macrocode}
%
% \subsection{Package loading}
%
%    \begin{macrocode}
\RequirePackage{atveryend}[2010/03/24]
\RequirePackage{doc}
%    \end{macrocode}
%    \begin{macrocode}
\ifx\numexpr\@undefined
  \RequirePackage{calc}%
\fi
%    \end{macrocode}
%    \begin{macrocode}
\RequirePackage[%
  hyperindex=false,%
  pdfusetitle,%
  colorlinks,%
  pdfpagelabels%
]{hyperref}[2002/05/09]%
\hypersetup{%
  plainpages=false,%
  bookmarksopen,%
  bookmarksnumbered,%
  bookmarksopenlevel=0,%
  pdfstartview={FitBH \hypercalcbp{%
    \paperheight-\topmargin-1in-\headheight-\headsep
  }},%
}
%    \end{macrocode}
%
%    \begin{macrocode}
\newcounter{HD@unique}
%    \end{macrocode}
%    \begin{macro}{\theHD@unique}
%    \begin{macrocode}
\renewcommand{\theHD@unique}{%
  \number\value{HD@unique}%
}
%    \end{macrocode}
%    \end{macro}
%
%    \begin{macrocode}
\g@addto@macro\Hy@UseMaketitleInfos{%
  \@ifundefined{fileinfo}{}{%
    \ifx\@pdfsubject\@empty
      \hypersetup{pdfsubject={\fileinfo}}%
    \fi
  }%
}
%    \end{macrocode}
%
%    \begin{macrocode}
\pdfstringdefDisableCommands{%
  \let\thanks\@gobble
  \let\footnote\@gobble
  \def\cs#1{\textbackslash #1}%
  \let\normalfont\@empty
  \let\scshape\@empty
  \def\and{and }%
}
%    \end{macrocode}
%
% \subsubsection{Patch for bug latex/4096}
%
%    \begin{macrocode}
\begingroup\expandafter\expandafter\expandafter\endgroup
\expandafter\ifx\csname mod@math@codes\endcsname\relax
\else
  \def\HD@ModFix#1{%
    \mathcode`#1=`#1\relax
  }%
  \g@addto@macro\mod@math@codes{%
    \HD@ModFix\-%
    \HD@ModFix\+%
    \HD@ModFix\:%
    \HD@ModFix\=%
    \HD@ModFix\*%
  }%
\fi
%    \end{macrocode}
%
% \subsection{Bookmarks}
%
%    \begin{macro}{\HD@sectionpatch}
%    \begin{macrocode}
\def\HD@sectionpatch{%
  \ifHD@numbered
    \HD@@sectionpatch{section}%
    \HD@@sectionpatch{chapter}%
  \else
    \let\Hy@writebookmark\HD@disable@writebookmark
  \fi
}
%    \end{macrocode}
%    \end{macro}
%    \begin{macro}{\HD@@sectionpatch}
%    \begin{macrocode}
\def\HD@@sectionpatch#1{%
  \expandafter\let\csname HDorg@@#1\expandafter\endcsname
    \csname #1\endcsname
  \@namedef{#1}{%
    \@ifstar{\@nameuse{HDorg@@#1}}{\@nameuse{HDorg@@#1}}%
  }%
}
%    \end{macrocode}
%    \end{macro}
%    \begin{macro}{\HD@disable@writebookmark}
%    \begin{macrocode}
\def\HD@disable@writebookmark#1#2#3#4#5{}
%    \end{macrocode}
%    \end{macro}
%
%    \begin{macro}{\HD@guesstoclevel}
%    A wild guessing of the toclevel. There is the hope, that
%    the macro |#1| starts with a sectioning command.
%    \begin{macrocode}
\def\HD@guesstoclevel#1{%
  \expandafter\HD@@guesstoclevel#1\section\HD@end
}
%    \end{macrocode}
%    \end{macro}
%    \begin{macro}{\HD@@guesstoclevel}
%    \begin{macrocode}
\def\HD@@guesstoclevel#1#2\HD@end{%
  \@ifundefined{toclevel@\expandafter\@gobble\string#1\@empty}{%
    1%
  }{%
    \@nameuse{toclevel@\expandafter\@gobble\string#1\@empty}%
  }%
}
%    \end{macrocode}
%    \end{macro}
%
% \subsection{\cs{tableofcontents}}
%
%    \begin{macro}{\HDorg@tableofcontents}
%    \begin{macrocode}
\let\HDorg@tableofcontents\tableofcontents
%    \end{macrocode}
%    \end{macro}
%    \begin{macro}{\tableofcontents}
%    As first section the table of contents shouldn't
%    go into the toc itself. Also a number looks too funny then.
%    Just a bookmark is enough in this case.
%    \begin{macrocode}
\def\tableofcontents{%
  \begingroup
    \@ifundefined{c@section}{%
      \def\c@section{0}%
    }{}%
  \expandafter\endgroup
  \ifcase\c@section
    \stepcounter{HD@unique}%
    \pdfbookmark[\HD@guesstoclevel{\HDorg@tableofcontents}]%
        {\contentsname}{toc\theHD@unique}%
    \HDorg@tableofcontents
  \else
    \ifHD@numbered
    \else
      \stepcounter{HD@unique}%
      \pdfbookmark[\HD@guesstoclevel{\HDorg@tableofcontents}]%
          {\contentsname}{toc\theHD@unique}%
    \fi
    \begingroup
      \HD@sectionpatch
      \HDorg@tableofcontents
    \endgroup
  \fi
}
%    \end{macrocode}
%    \end{macro}
%
% \subsection{\cs{listoffigures}}
%
%    \begin{macro}{\HDorg@listoffigures}
%    \begin{macrocode}
\let\HDorg@listoffigures\listoffigures
%    \end{macrocode}
%    \end{macro}
%    \begin{macro}{\listoffigures}
%    \begin{macrocode}
\def\listoffigures{%
  \ifHD@numbered
  \else
    \stepcounter{HD@unique}%
    \pdfbookmark[\HD@guesstoclevel{\HDorg@listoffigures}]%
        {\listfigurename}{toc\theHD@unique}%
  \fi
  \begingroup
    \HD@sectionpatch
    \HDorg@listoffigures
  \endgroup
}
%    \end{macrocode}
%    \end{macro}
%
% \subsection{\cs{listoftables}}
%
%    \begin{macro}{\HDorg@listoftables}
%    \begin{macrocode}
\let\HDorg@listoftables\listoftables
%    \end{macrocode}
%    \end{macro}
%    \begin{macro}{\listoftables}
%    \begin{macrocode}
\def\listoftables{%
  \ifHD@numbered
  \else
    \stepcounter{HD@unique}%
    \pdfbookmark[\HD@guesstoclevel{\HDorg@listoftables}]%
        {\listtablename}{toc\theHD@unique}%
  \fi
  \begingroup
    \HD@sectionpatch
    \HDorg@listoftables
  \endgroup
}
%    \end{macrocode}
%    \end{macro}
%
% \subsection{\cs{thebibliography}}
%
%    \begin{macro}{\HDorg@thebibliography}
%    \begin{macrocode}
\let\HDorg@thebibliography\thebibliography
%    \end{macrocode}
%    \end{macro}
%    \begin{macro}{\thebibliography}
%    \begin{macrocode}
\def\thebibliography{%
  \ifHD@numbered
  \else
    \stepcounter{HD@unique}%
    \pdfbookmark[\HD@guesstoclevel{\HDorg@thebibliography}]%
      {\@ifundefined{chapter}{\refname}{\bibname}}{bib\theHD@unique}%
  \fi
  \HD@sectionpatch
  \HDorg@thebibliography
}
%    \end{macrocode}
%    \end{macro}
%
% \subsection{\cs{theindex}}
%
%    \begin{macro}{\HDorg@theindex}
%    \begin{macrocode}
\let\HDorg@theindex\theindex
%    \end{macrocode}
%    \end{macro}
%    \begin{macro}{\theindex}
%    \begin{macrocode}
\def\theindex{%
  \let\HDorg@index@prologue\index@prologue
  \let\HDorg@writebookmark\Hy@writebookmark
  \HD@sectionpatch
  \g@addto@macro\IndexParms{%
    \ifHD@numbered
    \else
      \let\Hy@writebookmark\HDorg@writebookmark
    \fi
    \let\bfseries\HD@bfseries
  }%
  \edef\HD@toclevel@index{\HD@guesstoclevel{\index@prologue}}%
  \begingroup
    \count@=\HD@toclevel@index\relax
    \advance\count@ by 1 %
    \xdef\HD@toclevel@subindex{\the\count@}%
  \endgroup
  \def\index@prologue{%
    \ifHD@numbered
    \else
      \begingroup
        \stepcounter{HD@unique}%
        \let\Hy@writebookmark\HDorg@writebookmark
        \pdfbookmark[\HD@toclevel@index]%
            {\indexname}{index\theHD@unique}%
      \endgroup
    \fi
    \HDorg@index@prologue
  }%
  \HDorg@theindex
}
%    \end{macrocode}
%    \end{macro}
%    \begin{macro}{\HDorg@bfseries}
%    \begin{macrocode}
\let\HDorg@bfseries\bfseries
%    \end{macrocode}
%    \end{macro}
%    \begin{macro}{\HD@bfseries}
%    \begin{macrocode}
\def\HD@bfseries{\futurelet\HD@hfil\HD@@bfseries}%
%    \end{macrocode}
%    \end{macro}
%    \begin{macro}{\HD@@bfseries}
%    \begin{macrocode}
\def\HD@@bfseries{%
  \HDorg@bfseries
  \ifx\HD@hfil\hfil
    \expandafter\HD@@@bfseries
  \fi
}
%    \end{macrocode}
%    \end{macro}
%    \begin{macro}{\HD@@@bfseries}
%    \begin{macrocode}
\def\HD@@@bfseries\hfil#1\hfil{%
  \ifx\\#1\\%
  \else
    \raisebox{\baselineskip}[0pt]{%
      \kern-\HD@margin\relax
      \pdfbookmark[\HD@toclevel@subindex]{#1}{HD.#1}%
      \kern\HD@margin\relax
    }%
  \fi
  \hfil#1\hfil
}%
%    \end{macrocode}
%    \end{macro}
%    \begin{macro}{\HD@margin}
%    \begin{macrocode}
\def\HD@margin{1mm}
%    \end{macrocode}
%    \end{macro}
%
% \subsection{\cs{theglossary}}
%
%    \begin{macro}{\HDorg@theglossary}
%    \begin{macrocode}
\let\HDorg@theglossary\theglossary
%    \end{macrocode}
%    \end{macro}
%    \begin{macro}{\theglossary}
%    \begin{macrocode}
\def\theglossary{%
  \let\HDorg@glossary@prologue\glossary@prologue
  \let\HDorg@writebookmark\Hy@writebookmark
  \HD@sectionpatch
  \def\glossary@prologue{%
    \ifHD@numbered
    \else
      \begingroup
        \let\Hy@writebookmark\HDorg@writebookmark
        \stepcounter{HD@unique}%
        \pdfbookmark[\HD@guesstoclevel{\HDorg@glossary@prologue}]%
            {\changehistoryname}{glossary\theHD@unique}%
      \endgroup
    \fi
    \HDorg@glossary@prologue
  }%
  \HDorg@theglossary
}
%    \end{macrocode}
%    \end{macro}
%    \begin{macro}{\glossaryname}
%    \begin{macrocode}
\providecommand*{\glossaryname}{Change History}
%    \end{macrocode}
%    \end{macro}
%
%    \begin{macro}{\glossaryname}
%    \begin{macrocode}
\providecommand*{\changehistoryname}{\glossaryname}
%    \end{macrocode}
%    \end{macro}
%
% \subsection{Index}
%
%    \begin{macrocode}
\newcounter{HD@hypercount}
%    \end{macrocode}
%    \begin{macro}{\HD@helplength}
%    \begin{macrocode}
\newlength{\HD@helplength}
%    \end{macrocode}
%    \end{macro}
%    \begin{macro}{\ifHD@savedest}
%    \begin{macrocode}
\newif\ifHD@savedest
\HD@savedesttrue
%    \end{macrocode}
%    \end{macro}
%    \begin{macrocode}
\AtBeginDocument{%
  \HD@savedest@init
}
%    \end{macrocode}
%    \begin{macrocode}
\RequirePackage{rerunfilecheck}[2009/12/10]
%    \end{macrocode}
%    \begin{macro}{\HD@savedest@init}
%    \begin{macrocode}
\begingroup
  \catcode`\^^M=\active
\@firstofone{\endgroup
  \def\HD@savedest@init{%
    \ifHD@savedest
      \let\HDorg@tilde~%
      \let~\HD@savedest@entry
      \edef\HDorg@catcodeM{\the\catcode`\^^M}%
      \catcode`\^^M=\active
      \def^^M{}%
      \InputIfFileExists{\jobname.hd}{}{}%
      \let~\HDorg@tilde
      \catcode`\^^M=\HDorg@catcodeM\relax
      \newwrite\HD@savedest@out
      \RerunFileCheck{\jobname.hd}{%
        \immediate\closeout\HD@savedest@out
      }{}%
      \immediate\openout\HD@savedest@out=\jobname.hd\relax
    \else
      \let\HD@savedest@add\@gobble
    \fi
  }%
%    \end{macrocode}
%    \end{macro}
%    \begin{macro}{\HD@savedest@entry}
%    \begin{macrocode}
  \def\HD@savedest@entry#1^^M{%
    \@namedef{HD.#1}{}%
  }%
}
%    \end{macrocode}
%    \end{macro}
%    \begin{macro}{\HD@savedest@add}
%    \begin{macrocode}
\def\HD@savedest@add#1{%
  \immediate\write\HD@savedest@out{\string~#1}%
}
%    \end{macrocode}
%    \end{macro}
%    \begin{macro}{\ifHD@targetraise}
%    \begin{macrocode}
\newif\ifHD@targetraise
\HD@targetraisetrue
%    \end{macrocode}
%    \end{macro}
%    \begin{macro}{\HD@target}
%    \begin{macrocode}
\def\HD@target{%
  \ifHD@targetraise
    \ifhmode
      \settoheight{\HD@helplength}{[H]}%
      \addtolength{\HD@helplength}{0.4ex}%
      \raisebox{\HD@helplength}[0pt][0pt]{%
        \HD@maketarget
      }%
    \else
      \HD@maketarget
    \fi
  \else
    \HD@maketarget
  \fi
}
%    \end{macrocode}
%    \end{macro}
%    \begin{macro}{\HD@maketarget}
%    \begin{macrocode}
\def\HD@maketarget{%
  \stepcounter{HD@hypercount}%
  \def\HD@next{%
    \llap{\hypertarget{HD.\the\c@HD@hypercount}{}\,}%
  }%
  \ifHD@savedest
    \expandafter\ifx\csname HD.\the\c@HD@hypercount\endcsname\@empty
    \else
      \let\HD@next\relax
    \fi
  \fi
  \HD@next
}
%    \end{macrocode}
%    \end{macro}
%
%    \begin{macro}{\CodelineIndex}
%    \begin{macrocode}
\g@addto@macro\CodelineIndex{%
  \let\special@index\HD@codeline@wrindex
}
%    \end{macrocode}
%    \end{macro}
%    \begin{macro}{\PageIndex}
%    \begin{macrocode}
\g@addto@macro\PageIndex{%
  \let\special@index\HD@page@wrindex
}
%    \end{macrocode}
%    \end{macro}
%
%    \begin{macro}{\HD@codeline@wrindex}
%    \begin{macrocode}
\def\HD@codeline@wrindex#1{%
  \begingroup
    \let\HDorg@encapchar\encapchar
    \def\encapchar##1\encapchar##2\@nil{%
      \HDorg@encapchar
      hdclindex{\the\c@HD@hypercount}{##1}%
    }%
    \codeline@wrindex{%
      #1\encapchar\encapchar\@nil
    }%
  \endgroup
}
%    \end{macrocode}
%    \end{macro}
%    \begin{macro}{\HD@page@wrindex}
%    \begin{macrocode}
\def\HD@page@wrindex#1{%
  \begingroup
    \let\HDorg@encapchar\encapchar
    \def\encapchar##1\encapchar##2\@nil{%
      \HDorg@encapchar
      hdpindex{##1}%
    }%
    \HDorg@index{#1\encapchar\encapchar\@nil}%
  \endgroup
}
%    \end{macrocode}
%    \end{macro}
%
%    \begin{macro}{\hdclindex}
%    \begin{macrocode}
\def\hdclindex#1#2#3{%
  \csname\ifx\\#2\\relax\else#2\fi\endcsname{%
    \hyperlink{HD.#1}{#3}%
    \HD@savedest@add{#1}%
  }%
}
%    \end{macrocode}
%    \end{macro}
%    \begin{macro}{\hdpindex}
%    \begin{macrocode}
\def\hdpindex#1#2{%
  \csname\ifx\\#1\\relax\else#1\fi\endcsname{%
    \hyperpage{#2}%
  }%
}
%    \end{macrocode}
%    \end{macro}
%
% \subsection{\cs{PrintDescribeMacro}}
%
%    \begin{macro}{\HDorg@PrintDescribeMacro}
%    \begin{macrocode}
\let\HDorg@PrintDescribeMacro\PrintDescribeMacro
%    \end{macrocode}
%    \end{macro}
%    \begin{macro}{\PrintDescribeMacro}
%    \begin{macrocode}
\renewcommand\PrintDescribeMacro[1]{%
  \begingroup
    \let\HDorg@MacroFont\MacroFont
    \def\MacroFont{%
      \HD@target
      \HDorg@MacroFont
    }%
    \HDorg@PrintDescribeMacro{#1}%
  \endgroup
}
%    \end{macrocode}
%    \end{macro}
%
% \subsection{\cs{PrintMacroName}}
%
%    \begin{macro}{\HDorg@PrintMacroName}
%    \begin{macrocode}
\let\HDorg@PrintMacroName\PrintMacroName
%    \end{macrocode}
%    \end{macro}
%    \begin{macro}{\PrintMacroName}
%    \begin{macrocode}
\renewcommand\PrintMacroName[1]{%
  \begingroup
    \let\HDorg@MacroFont\MacroFont
    \def\MacroFont{%
      \HD@target
      \HDorg@MacroFont
    }%
    \HDorg@PrintMacroName{#1}%
  \endgroup
}
%    \end{macrocode}
%    \end{macro}
%
% \subsection{\cs{theCodelineNo}}
%
%    \begin{macro}{\HDorg@theCodelineNo}
%    \begin{macrocode}
\let\HDorg@theCodelineNo\theCodelineNo
%    \end{macrocode}
%    \end{macro}
%    \begin{macro}{\theCodelineNo}
%    \begin{macrocode}
\renewcommand*\theCodelineNo{%
  \settowidth{\HD@helplength}{\HDorg@theCodelineNo\ }%
  \ifdim\HD@helplength<\MacroIndent
    \addtolength{\HD@helplength}{-\MacroIndent}%
    \llap{%
      \begingroup\HD@target\endgroup
      \hspace*{-\HD@helplength}%
    }%
  \else
    \HD@target
  \fi
  \HDorg@theCodelineNo
}
%    \end{macrocode}
%    \end{macro}
%
%    \begin{macro}{\HDorg@index}
%    \begin{macrocode}
\AtBeginDocument{%
  \let\HDorg@index\index
}
%    \end{macrocode}
%    \end{macro}
%
% \subsection{\cs{SpecialUsageIndex}}
%
%    \begin{macro}{\HDorg@SpecialUsageIndex}
%    \begin{macrocode}
\let\HDorg@SpecialUsageIndex\SpecialUsageIndex
\renewcommand*\SpecialUsageIndex[1]{%
  \@bsphack
  \begingroup
    \HD@target
    \let\index\HDorg@index
    \let\HDorg@encapchar\encapchar
    \edef\encapchar usage{%
       \HDorg@encapchar hdclindex{\the\c@HD@hypercount}{usage}%
    }%
    \HDorg@SpecialUsageIndex{#1}%
  \endgroup
  \@esphack
}
%    \end{macrocode}
%    \end{macro}
%
% \subsection{\cs{autoindex}}
%
%    \begin{macrocode}
\begingroup\expandafter\expandafter\expandafter\endgroup
\expandafter\ifx\csname autoindex\endcsname\relax
\else
%    \end{macrocode}
%    \begin{macro}{\HDorg@autoindex}
%    \begin{macrocode}
  \let\HDorg@autoindex\autoindex
%    \end{macrocode}
%    \end{macro}
%    \begin{macro}{\autoindex}
%    \begin{macrocode}
  \renewcommand*\autoindex[1]{%
    \@bsphack
    \begingroup
      \HD@target
      \let\index\HDorg@index
      \let\HDorg@encapchar\encapchar
      \edef\encapchar usage{%
         \HDorg@encapchar hdclindex{\the\c@HD@hypercount}{usage}%
      }%
      \HDorg@autoindex{#1}%
    \endgroup
    \@esphack
  }%
\fi
%    \end{macrocode}
%    \end{macro}
%
% \subsection{\cs{SpecialEnvIndex}}
%
%    \begin{macro}{\HDorg@SpecialEnvIndex}
%    \begin{macrocode}
\let\HDorg@SpecialEnvIndex\SpecialEnvIndex
%    \end{macrocode}
%    \end{macro}
%    \begin{macro}{\SpecialEnvIndex}
%    \begin{macrocode}
\renewcommand*\SpecialEnvIndex[1]{%
  \@bsphack
  \begingroup
    \HD@target
    \let\HDorg@encapchar\encapchar
    \edef\encapchar usage{%
      \HDorg@encapchar hdclindex{\the\c@HD@hypercount}{usage}%
    }%
    \HDorg@SpecialEnvIndex{#1}%
  \endgroup
  \@esphack
}
%    \end{macrocode}
%    \end{macro}
%
% \subsection{\cs{SortIndex}}
%
%    \begin{macro}{\HDorg@SortIndex}
%    \begin{macrocode}
\let\HDorg@SortIndex\SortIndex
%    \end{macrocode}
%    \end{macro}
%    \begin{macro}{\SortIndex}
%    \begin{macrocode}
\renewcommand*{\SortIndex}[2]{%
  \@bsphack
  \begingroup
    \let\index\HD@page@wrindex
    \HDorg@SortIndex{#1}{#2}%
  \endgroup
  \@esphack
}
%    \end{macrocode}
%    \end{macro}
%
% \subsection{\cs{@wrglossary}}
%
%    \begin{macro}{\HDorg@wrglossary}
%    \begin{macrocode}
\let\HDorg@wrglossary\@wrglossary
%    \end{macrocode}
%    \end{macro}
%    \begin{macro}{\@wrglossary}
%    \begin{macrocode}
\def\@wrglossary#1{%
  \let\HDorg@encapchar\encapchar
  \def\encapchar##1\encapchar##2\@nil{%
    \HDorg@encapchar
    hdpindex{##1}%
  }%
  \HDorg@wrglossary{#1\encapchar\encapchar\@nil}%
}
%    \end{macrocode}
%    \end{macro}
%
% \subsection{\cs{MacroIndent}}
%
%    \cs{MacroIndent} automatically calculated.
%    \begin{macrocode}
\AfterLastShipout{%
  \if@filesw
    \immediate\write\@auxout{%
      \string\providecommand\string\HD@SetMacroIndent[1]{}%
    }%
    \immediate\write\@auxout{%
      \string\HD@SetMacroIndent{\number\value{CodelineNo}}%
    }%
  \fi
}
%    \end{macrocode}
%    \begin{macro}{\HD@SetMacroIndent}
%    \begin{macrocode}
\def\HD@SetMacroIndent#1{%
  \begingroup
    \value{CodelineNo}=#1\relax
    \settowidth{\MacroIndent}{\HDorg@theCodelineNo\ }%
    \global\MacroIndent\MacroIndent
  \endgroup
}
%    \end{macrocode}
%    \end{macro}
%
%    \begin{macrocode}
%</package>
%    \end{macrocode}
%
% \section{Installation}
%
% \subsection{Download}
%
% \paragraph{Package.} This package is available on
% CTAN\footnote{\url{http://ctan.org/pkg/hypdoc}}:
% \begin{description}
% \item[\CTAN{macros/latex/contrib/oberdiek/hypdoc.dtx}] The source file.
% \item[\CTAN{macros/latex/contrib/oberdiek/hypdoc.pdf}] Documentation.
% \end{description}
%
%
% \paragraph{Bundle.} All the packages of the bundle `oberdiek'
% are also available in a TDS compliant ZIP archive. There
% the packages are already unpacked and the documentation files
% are generated. The files and directories obey the TDS standard.
% \begin{description}
% \item[\CTAN{install/macros/latex/contrib/oberdiek.tds.zip}]
% \end{description}
% \emph{TDS} refers to the standard ``A Directory Structure
% for \TeX\ Files'' (\CTAN{tds/tds.pdf}). Directories
% with \xfile{texmf} in their name are usually organized this way.
%
% \subsection{Bundle installation}
%
% \paragraph{Unpacking.} Unpack the \xfile{oberdiek.tds.zip} in the
% TDS tree (also known as \xfile{texmf} tree) of your choice.
% Example (linux):
% \begin{quote}
%   |unzip oberdiek.tds.zip -d ~/texmf|
% \end{quote}
%
% \paragraph{Script installation.}
% Check the directory \xfile{TDS:scripts/oberdiek/} for
% scripts that need further installation steps.
% Package \xpackage{attachfile2} comes with the Perl script
% \xfile{pdfatfi.pl} that should be installed in such a way
% that it can be called as \texttt{pdfatfi}.
% Example (linux):
% \begin{quote}
%   |chmod +x scripts/oberdiek/pdfatfi.pl|\\
%   |cp scripts/oberdiek/pdfatfi.pl /usr/local/bin/|
% \end{quote}
%
% \subsection{Package installation}
%
% \paragraph{Unpacking.} The \xfile{.dtx} file is a self-extracting
% \docstrip\ archive. The files are extracted by running the
% \xfile{.dtx} through \plainTeX:
% \begin{quote}
%   \verb|tex hypdoc.dtx|
% \end{quote}
%
% \paragraph{TDS.} Now the different files must be moved into
% the different directories in your installation TDS tree
% (also known as \xfile{texmf} tree):
% \begin{quote}
% \def\t{^^A
% \begin{tabular}{@{}>{\ttfamily}l@{ $\rightarrow$ }>{\ttfamily}l@{}}
%   hypdoc.sty & tex/latex/oberdiek/hypdoc.sty\\
%   hypdoc.pdf & doc/latex/oberdiek/hypdoc.pdf\\
%   hypdoc.dtx & source/latex/oberdiek/hypdoc.dtx\\
% \end{tabular}^^A
% }^^A
% \sbox0{\t}^^A
% \ifdim\wd0>\linewidth
%   \begingroup
%     \advance\linewidth by\leftmargin
%     \advance\linewidth by\rightmargin
%   \edef\x{\endgroup
%     \def\noexpand\lw{\the\linewidth}^^A
%   }\x
%   \def\lwbox{^^A
%     \leavevmode
%     \hbox to \linewidth{^^A
%       \kern-\leftmargin\relax
%       \hss
%       \usebox0
%       \hss
%       \kern-\rightmargin\relax
%     }^^A
%   }^^A
%   \ifdim\wd0>\lw
%     \sbox0{\small\t}^^A
%     \ifdim\wd0>\linewidth
%       \ifdim\wd0>\lw
%         \sbox0{\footnotesize\t}^^A
%         \ifdim\wd0>\linewidth
%           \ifdim\wd0>\lw
%             \sbox0{\scriptsize\t}^^A
%             \ifdim\wd0>\linewidth
%               \ifdim\wd0>\lw
%                 \sbox0{\tiny\t}^^A
%                 \ifdim\wd0>\linewidth
%                   \lwbox
%                 \else
%                   \usebox0
%                 \fi
%               \else
%                 \lwbox
%               \fi
%             \else
%               \usebox0
%             \fi
%           \else
%             \lwbox
%           \fi
%         \else
%           \usebox0
%         \fi
%       \else
%         \lwbox
%       \fi
%     \else
%       \usebox0
%     \fi
%   \else
%     \lwbox
%   \fi
% \else
%   \usebox0
% \fi
% \end{quote}
% If you have a \xfile{docstrip.cfg} that configures and enables \docstrip's
% TDS installing feature, then some files can already be in the right
% place, see the documentation of \docstrip.
%
% \subsection{Refresh file name databases}
%
% If your \TeX~distribution
% (\teTeX, \mikTeX, \dots) relies on file name databases, you must refresh
% these. For example, \teTeX\ users run \verb|texhash| or
% \verb|mktexlsr|.
%
% \subsection{Some details for the interested}
%
% \paragraph{Attached source.}
%
% The PDF documentation on CTAN also includes the
% \xfile{.dtx} source file. It can be extracted by
% AcrobatReader 6 or higher. Another option is \textsf{pdftk},
% e.g. unpack the file into the current directory:
% \begin{quote}
%   \verb|pdftk hypdoc.pdf unpack_files output .|
% \end{quote}
%
% \paragraph{Unpacking with \LaTeX.}
% The \xfile{.dtx} chooses its action depending on the format:
% \begin{description}
% \item[\plainTeX:] Run \docstrip\ and extract the files.
% \item[\LaTeX:] Generate the documentation.
% \end{description}
% If you insist on using \LaTeX\ for \docstrip\ (really,
% \docstrip\ does not need \LaTeX), then inform the autodetect routine
% about your intention:
% \begin{quote}
%   \verb|latex \let\install=y% \iffalse meta-comment
%
% File: hypdoc.dtx
% Version: 2018/09/07 v1.14
% Info: Hyper extensions for doc.sty
%
% Copyright (C) 2002, 2006, 2007, 2009-2011 by
%    Heiko Oberdiek <heiko.oberdiek at googlemail.com>
%    2016
%    https://github.com/ho-tex/oberdiek/issues
%
% This work may be distributed and/or modified under the
% conditions of the LaTeX Project Public License, either
% version 1.3c of this license or (at your option) any later
% version. This version of this license is in
%    http://www.latex-project.org/lppl/lppl-1-3c.txt
% and the latest version of this license is in
%    http://www.latex-project.org/lppl.txt
% and version 1.3 or later is part of all distributions of
% LaTeX version 2005/12/01 or later.
%
% This work has the LPPL maintenance status "maintained".
%
% This Current Maintainer of this work is Heiko Oberdiek.
%
% This work consists of the main source file hypdoc.dtx
% and the derived files
%    hypdoc.sty, hypdoc.pdf, hypdoc.ins, hypdoc.drv.
%
% Distribution:
%    CTAN:macros/latex/contrib/oberdiek/hypdoc.dtx
%    CTAN:macros/latex/contrib/oberdiek/hypdoc.pdf
%
% Unpacking:
%    (a) If hypdoc.ins is present:
%           tex hypdoc.ins
%    (b) Without hypdoc.ins:
%           tex hypdoc.dtx
%    (c) If you insist on using LaTeX
%           latex \let\install=y\input{hypdoc.dtx}
%        (quote the arguments according to the demands of your shell)
%
% Documentation:
%    (a) If hypdoc.drv is present:
%           latex hypdoc.drv
%    (b) Without hypdoc.drv:
%           latex hypdoc.dtx; ...
%    The class ltxdoc loads the configuration file ltxdoc.cfg
%    if available. Here you can specify further options, e.g.
%    use A4 as paper format:
%       \PassOptionsToClass{a4paper}{article}
%
%    Programm calls to get the documentation (example):
%       pdflatex hypdoc.dtx
%       makeindex -s gind.ist hypdoc.idx
%       pdflatex hypdoc.dtx
%       makeindex -s gind.ist hypdoc.idx
%       pdflatex hypdoc.dtx
%
% Installation:
%    TDS:tex/latex/oberdiek/hypdoc.sty
%    TDS:doc/latex/oberdiek/hypdoc.pdf
%    TDS:source/latex/oberdiek/hypdoc.dtx
%
%<*ignore>
\begingroup
  \catcode123=1 %
  \catcode125=2 %
  \def\x{LaTeX2e}%
\expandafter\endgroup
\ifcase 0\ifx\install y1\fi\expandafter
         \ifx\csname processbatchFile\endcsname\relax\else1\fi
         \ifx\fmtname\x\else 1\fi\relax
\else\csname fi\endcsname
%</ignore>
%<*install>
\input docstrip.tex
\Msg{************************************************************************}
\Msg{* Installation}
\Msg{* Package: hypdoc 2018/09/07 v1.14 Hyper extensions for doc.sty (HO)}
\Msg{************************************************************************}

\keepsilent
\askforoverwritefalse

\let\MetaPrefix\relax
\preamble

This is a generated file.

Project: hypdoc
Version: 2018/09/07 v1.14

Copyright (C) 2002, 2006, 2007, 2009-2011 by
   Heiko Oberdiek <heiko.oberdiek at googlemail.com>

This work may be distributed and/or modified under the
conditions of the LaTeX Project Public License, either
version 1.3c of this license or (at your option) any later
version. This version of this license is in
   http://www.latex-project.org/lppl/lppl-1-3c.txt
and the latest version of this license is in
   http://www.latex-project.org/lppl.txt
and version 1.3 or later is part of all distributions of
LaTeX version 2005/12/01 or later.

This work has the LPPL maintenance status "maintained".

This Current Maintainer of this work is Heiko Oberdiek.

This work consists of the main source file hypdoc.dtx
and the derived files
   hypdoc.sty, hypdoc.pdf, hypdoc.ins, hypdoc.drv.

\endpreamble
\let\MetaPrefix\DoubleperCent

\generate{%
  \file{hypdoc.ins}{\from{hypdoc.dtx}{install}}%
  \file{hypdoc.drv}{\from{hypdoc.dtx}{driver}}%
  \usedir{tex/latex/oberdiek}%
  \file{hypdoc.sty}{\from{hypdoc.dtx}{package}}%
  \nopreamble
  \nopostamble
%  \usedir{source/latex/oberdiek/catalogue}%
%  \file{hypdoc.xml}{\from{hypdoc.dtx}{catalogue}}%
}

\catcode32=13\relax% active space
\let =\space%
\Msg{************************************************************************}
\Msg{*}
\Msg{* To finish the installation you have to move the following}
\Msg{* file into a directory searched by TeX:}
\Msg{*}
\Msg{*     hypdoc.sty}
\Msg{*}
\Msg{* To produce the documentation run the file `hypdoc.drv'}
\Msg{* through LaTeX.}
\Msg{*}
\Msg{* Happy TeXing!}
\Msg{*}
\Msg{************************************************************************}

\endbatchfile
%</install>
%<*ignore>
\fi
%</ignore>
%<*driver>
\NeedsTeXFormat{LaTeX2e}
\ProvidesFile{hypdoc.drv}%
  [2018/09/07 v1.14 Hyper extensions for doc.sty (HO)]%
\documentclass{ltxdoc}
\usepackage{holtxdoc}[2011/11/22]
\begin{document}
  \DocInput{hypdoc.dtx}%
\end{document}
%</driver>
% \fi
%
%
% \CharacterTable
%  {Upper-case    \A\B\C\D\E\F\G\H\I\J\K\L\M\N\O\P\Q\R\S\T\U\V\W\X\Y\Z
%   Lower-case    \a\b\c\d\e\f\g\h\i\j\k\l\m\n\o\p\q\r\s\t\u\v\w\x\y\z
%   Digits        \0\1\2\3\4\5\6\7\8\9
%   Exclamation   \!     Double quote  \"     Hash (number) \#
%   Dollar        \$     Percent       \%     Ampersand     \&
%   Acute accent  \'     Left paren    \(     Right paren   \)
%   Asterisk      \*     Plus          \+     Comma         \,
%   Minus         \-     Point         \.     Solidus       \/
%   Colon         \:     Semicolon     \;     Less than     \<
%   Equals        \=     Greater than  \>     Question mark \?
%   Commercial at \@     Left bracket  \[     Backslash     \\
%   Right bracket \]     Circumflex    \^     Underscore    \_
%   Grave accent  \`     Left brace    \{     Vertical bar  \|
%   Right brace   \}     Tilde         \~}
%
% \GetFileInfo{hypdoc.drv}
%
% \title{The \xpackage{hypdoc} package}
% \date{2018/09/07 v1.14}
% \author{Heiko Oberdiek\thanks
% {Please report any issues at https://github.com/ho-tex/oberdiek/issues}\\
% \xemail{heiko.oberdiek at googlemail.com}}
%
% \maketitle
%
% \begin{abstract}
% This package adds hyper features to the package
% \xpackage{doc} that is used in the documentation
% system of \LaTeXe. Bookmarks are added and references
% are linked as far as possible.
% \end{abstract}
%
% \tableofcontents
%
% \section{Features}
%
% The package adds the following features:
% \begin{description}
% \item[Bookmarks:]
%   For supporting bookmarks package \verb|hyperref| is loaded.
%   Some \TeX\ code inside section titles can cause problems
%   during bookmark processing; these can be warnings, ugly
%   bookmark titles or even errors. With the help of
%   \verb|\pdfstringdefDisableCommands| or \verb|\texorpdfstring|
%   you can manage these things, see \verb|hyperref|'s package
%   documentation.
% \item[Additional bookmarks:]
%   Unnumbered bookmark entries are generated for:
%   \begin{itemize}
%   \item Table of contents
%   \item List of figures
%   \item List of tables
%   \item Index
%   \item Glossary
%   \end{itemize}
%   Option \texttt{numbered} controls, whether these section or
%   chapters should be numbered. The exception is the table
%   of contents if this it would be the first numbered
%   section. Then just a bookmark is added.
% \item[Metadata:]
%   The loaded package \verb|hyperref|, enforced by
%   option \verb|pdfusetitle|, tries to detect \verb|\title|
%   and \verb|\author| and sets the corresponding PDF
%   information fields. In case of problems see the previous
%   item about bookmarks.
%   These enries can be overwritten or new ones added by
%   \verb|\hypersetup| after the packages is loaded, see
%   documentation of package \verb|hyperref|.
% \item[References, links:] Because of package \verb|\hyperref|
%   |\ref| become links, also \verb|\url|, footnote referencing
%   are supported.
% \item[Index:]
%   The main part of this package deals with index links in order
%   to support the kind of index that package \verb|doc| provides.
% \item[Thumbnails:]
%   Package \verb|thumbpdf| may additionally be loaded for the case that the
%   PDF file also should contain thumbnails, see the documentation
%   of this package. But thumbnail embedding is not really
%   necessary, because nowadays AcrobatReader is able to
%   generate and view thumbnails on the fly.
% \end{description}
%
% \section{Usage}
%
% The purpose of this packages is to add PDF features to
% the documentation of LaTeX packages. Thus just load
% the package in the driver file that generates the
% documentation:
% \begin{quote}
%   \verb|\usepackage{hypdoc}|
% \end{quote}
% Package \verb|doc| is loaded by \verb|hypdoc|, thus you
% also can replace a \verb|\usepackage{doc}| with this line.
% At any case, the package must be loaded before macros of
% package \verb|doc| such as \verb|\CodelineIndex| or
% \verb|\PageIndex| are used.
%
% If the class \verb|ltxdoc| is used, you can also use
% the configuration file for this class to add
% package \verb|hypdoc|. Add the following line
% to the file \verb|ltxdoc.cfg| (or create a new one):
% \begin{quote}
%\begin{verbatim}
%\AtEndOfClass{\RequirePackage{hypdoc}}
%\end{verbatim}
% \end{quote}
%
% \subsection{Options}
%
% There is one option \texttt{numbered}. It controls,
% whether the starred versions of section or chapter
% inside \cs{tableofcontents}, \cs{listoffigures}, \dots
% should be replaced by the non-star versions.
%
% \subsection{Simple example}
% Example of a simple driver file \verb|foobar.drv|
% of a package \verb|foobar|. Often the driver file
% is packed in the file \verb|foobar.dtx| and can
% be extracted by docstrip or if an instruction file
% for docstrip is provided by:
% \begin{quote}
% \verb|tex foobar.ins|
% \end{quote}
% Note that although docstrip doesn't rely on \LaTeX,
% sometimes you need to run the \verb|.ins| file through
% \LaTeX. Oddly enough, often the only \LaTeX\ feature
% in use is \verb|\NeedsTeXFormat{LaTeX2e}| in the first line.
% \begin{quote}
%\begin{verbatim}
%\documentclass{ltxdoc}
%\usepackage{hypdoc}
%\RecordChanges
%\EnableCrossrefs
%\CodelineIndex
%\begin{document}
%\title{Title for package \texttt{foobar}}
%\author{Author of package \texttt{foobar}}
%\date{...}
%\maketitle
%\tableofcontents
%\DocInput{foobar.dtx}
%\PrintChanges
%\PrintIndex
%\end{document}
%\end{verbatim}
% \end{quote}
%
% Then compile it, for example with pdf\LaTeX.
% \begin{quote}
%\begin{verbatim}
%pdflatex foobar.drv
%makeindex -s gind.ist foobar.idx
%makeindex -s gglo.ist -o foobar.gls foobar.glo
%pdflatex foobar.drv
%\end{verbatim}
% \end{quote}
% You will need several cycles until the cross references
% and similar things are correct.
%
% \subsection{Example \texttt{source2e.tex}}
%
% The source code of the base of \LaTeX\ is available as
% a bunch of \verb|.dtx| files. \LaTeX\ provides the file
% \verb|source2e.tex| to merge them all together.
%
% Now you can either load package \verb|hypdoc| in a private copy
% of this file (please, respect the LPPL) or use the configuration
% file \verb|ltxdoc.cfg|. Example that also sets A4 paper size:
% \begin{quote}
%\begin{verbatim}
%\PassOptionsToClass{a4paper}{article}
%\AtEndOfClass{\RequirePackage{hypdoc}}
%\end{verbatim}
% \end{quote}
%
% Then \verb|source2e.tex| can be compiled by pdf\TeX, for example:
% \begin{quote}
%\begin{verbatim}
%pdflatex source2e
%makeindex -s gind.ist source2e.idx
%makeindex -s gglo.ist -o source2e.glo source2e.gls
%pdflatex source2e
%makeindex -s gind.ist source2e.idx
%makeindex -s gglo.ist -o source2e.glo source2e.gls
%pdflatex source2e
%\end{verbatim}
% \end{quote}
%
% \StopEventually{
% }
%
% \section{Implementation}
%
%    \begin{macrocode}
%<*package>
\NeedsTeXFormat{LaTeX2e}
\ProvidesPackage{hypdoc}%
  [2018/09/07 v1.14 Hyper extensions for doc.sty (HO)]%
%    \end{macrocode}
%    The package identification is done at the top of the |.dtx| file
%    in order to use only one identification string.
%
%    For unique command names this package uses |HD@| as prefix
%
% \subsection{Options}
%
%    \begin{macro}{\ifHD@numbered}
%    \begin{macrocode}
\newif\ifHD@numbered
%    \end{macrocode}
%    \end{macro}
%    \begin{macrocode}
\DeclareOption{numbered}{\HD@numberedtrue}
\ProcessOptions*\relax
%    \end{macrocode}
%
% \subsection{Package loading}
%
%    \begin{macrocode}
\RequirePackage{atveryend}[2010/03/24]
\RequirePackage{doc}
%    \end{macrocode}
%    \begin{macrocode}
\ifx\numexpr\@undefined
  \RequirePackage{calc}%
\fi
%    \end{macrocode}
%    \begin{macrocode}
\RequirePackage[%
  hyperindex=false,%
  pdfusetitle,%
  colorlinks,%
  pdfpagelabels%
]{hyperref}[2002/05/09]%
\hypersetup{%
  plainpages=false,%
  bookmarksopen,%
  bookmarksnumbered,%
  bookmarksopenlevel=0,%
  pdfstartview={FitBH \hypercalcbp{%
    \paperheight-\topmargin-1in-\headheight-\headsep
  }},%
}
%    \end{macrocode}
%
%    \begin{macrocode}
\newcounter{HD@unique}
%    \end{macrocode}
%    \begin{macro}{\theHD@unique}
%    \begin{macrocode}
\renewcommand{\theHD@unique}{%
  \number\value{HD@unique}%
}
%    \end{macrocode}
%    \end{macro}
%
%    \begin{macrocode}
\g@addto@macro\Hy@UseMaketitleInfos{%
  \@ifundefined{fileinfo}{}{%
    \ifx\@pdfsubject\@empty
      \hypersetup{pdfsubject={\fileinfo}}%
    \fi
  }%
}
%    \end{macrocode}
%
%    \begin{macrocode}
\pdfstringdefDisableCommands{%
  \let\thanks\@gobble
  \let\footnote\@gobble
  \def\cs#1{\textbackslash #1}%
  \let\normalfont\@empty
  \let\scshape\@empty
  \def\and{and }%
}
%    \end{macrocode}
%
% \subsubsection{Patch for bug latex/4096}
%
%    \begin{macrocode}
\begingroup\expandafter\expandafter\expandafter\endgroup
\expandafter\ifx\csname mod@math@codes\endcsname\relax
\else
  \def\HD@ModFix#1{%
    \mathcode`#1=`#1\relax
  }%
  \g@addto@macro\mod@math@codes{%
    \HD@ModFix\-%
    \HD@ModFix\+%
    \HD@ModFix\:%
    \HD@ModFix\=%
    \HD@ModFix\*%
  }%
\fi
%    \end{macrocode}
%
% \subsection{Bookmarks}
%
%    \begin{macro}{\HD@sectionpatch}
%    \begin{macrocode}
\def\HD@sectionpatch{%
  \ifHD@numbered
    \HD@@sectionpatch{section}%
    \HD@@sectionpatch{chapter}%
  \else
    \let\Hy@writebookmark\HD@disable@writebookmark
  \fi
}
%    \end{macrocode}
%    \end{macro}
%    \begin{macro}{\HD@@sectionpatch}
%    \begin{macrocode}
\def\HD@@sectionpatch#1{%
  \expandafter\let\csname HDorg@@#1\expandafter\endcsname
    \csname #1\endcsname
  \@namedef{#1}{%
    \@ifstar{\@nameuse{HDorg@@#1}}{\@nameuse{HDorg@@#1}}%
  }%
}
%    \end{macrocode}
%    \end{macro}
%    \begin{macro}{\HD@disable@writebookmark}
%    \begin{macrocode}
\def\HD@disable@writebookmark#1#2#3#4#5{}
%    \end{macrocode}
%    \end{macro}
%
%    \begin{macro}{\HD@guesstoclevel}
%    A wild guessing of the toclevel. There is the hope, that
%    the macro |#1| starts with a sectioning command.
%    \begin{macrocode}
\def\HD@guesstoclevel#1{%
  \expandafter\HD@@guesstoclevel#1\section\HD@end
}
%    \end{macrocode}
%    \end{macro}
%    \begin{macro}{\HD@@guesstoclevel}
%    \begin{macrocode}
\def\HD@@guesstoclevel#1#2\HD@end{%
  \@ifundefined{toclevel@\expandafter\@gobble\string#1\@empty}{%
    1%
  }{%
    \@nameuse{toclevel@\expandafter\@gobble\string#1\@empty}%
  }%
}
%    \end{macrocode}
%    \end{macro}
%
% \subsection{\cs{tableofcontents}}
%
%    \begin{macro}{\HDorg@tableofcontents}
%    \begin{macrocode}
\let\HDorg@tableofcontents\tableofcontents
%    \end{macrocode}
%    \end{macro}
%    \begin{macro}{\tableofcontents}
%    As first section the table of contents shouldn't
%    go into the toc itself. Also a number looks too funny then.
%    Just a bookmark is enough in this case.
%    \begin{macrocode}
\def\tableofcontents{%
  \begingroup
    \@ifundefined{c@section}{%
      \def\c@section{0}%
    }{}%
  \expandafter\endgroup
  \ifcase\c@section
    \stepcounter{HD@unique}%
    \pdfbookmark[\HD@guesstoclevel{\HDorg@tableofcontents}]%
        {\contentsname}{toc\theHD@unique}%
    \HDorg@tableofcontents
  \else
    \ifHD@numbered
    \else
      \stepcounter{HD@unique}%
      \pdfbookmark[\HD@guesstoclevel{\HDorg@tableofcontents}]%
          {\contentsname}{toc\theHD@unique}%
    \fi
    \begingroup
      \HD@sectionpatch
      \HDorg@tableofcontents
    \endgroup
  \fi
}
%    \end{macrocode}
%    \end{macro}
%
% \subsection{\cs{listoffigures}}
%
%    \begin{macro}{\HDorg@listoffigures}
%    \begin{macrocode}
\let\HDorg@listoffigures\listoffigures
%    \end{macrocode}
%    \end{macro}
%    \begin{macro}{\listoffigures}
%    \begin{macrocode}
\def\listoffigures{%
  \ifHD@numbered
  \else
    \stepcounter{HD@unique}%
    \pdfbookmark[\HD@guesstoclevel{\HDorg@listoffigures}]%
        {\listfigurename}{toc\theHD@unique}%
  \fi
  \begingroup
    \HD@sectionpatch
    \HDorg@listoffigures
  \endgroup
}
%    \end{macrocode}
%    \end{macro}
%
% \subsection{\cs{listoftables}}
%
%    \begin{macro}{\HDorg@listoftables}
%    \begin{macrocode}
\let\HDorg@listoftables\listoftables
%    \end{macrocode}
%    \end{macro}
%    \begin{macro}{\listoftables}
%    \begin{macrocode}
\def\listoftables{%
  \ifHD@numbered
  \else
    \stepcounter{HD@unique}%
    \pdfbookmark[\HD@guesstoclevel{\HDorg@listoftables}]%
        {\listtablename}{toc\theHD@unique}%
  \fi
  \begingroup
    \HD@sectionpatch
    \HDorg@listoftables
  \endgroup
}
%    \end{macrocode}
%    \end{macro}
%
% \subsection{\cs{thebibliography}}
%
%    \begin{macro}{\HDorg@thebibliography}
%    \begin{macrocode}
\let\HDorg@thebibliography\thebibliography
%    \end{macrocode}
%    \end{macro}
%    \begin{macro}{\thebibliography}
%    \begin{macrocode}
\def\thebibliography{%
  \ifHD@numbered
  \else
    \stepcounter{HD@unique}%
    \pdfbookmark[\HD@guesstoclevel{\HDorg@thebibliography}]%
      {\@ifundefined{chapter}{\refname}{\bibname}}{bib\theHD@unique}%
  \fi
  \HD@sectionpatch
  \HDorg@thebibliography
}
%    \end{macrocode}
%    \end{macro}
%
% \subsection{\cs{theindex}}
%
%    \begin{macro}{\HDorg@theindex}
%    \begin{macrocode}
\let\HDorg@theindex\theindex
%    \end{macrocode}
%    \end{macro}
%    \begin{macro}{\theindex}
%    \begin{macrocode}
\def\theindex{%
  \let\HDorg@index@prologue\index@prologue
  \let\HDorg@writebookmark\Hy@writebookmark
  \HD@sectionpatch
  \g@addto@macro\IndexParms{%
    \ifHD@numbered
    \else
      \let\Hy@writebookmark\HDorg@writebookmark
    \fi
    \let\bfseries\HD@bfseries
  }%
  \edef\HD@toclevel@index{\HD@guesstoclevel{\index@prologue}}%
  \begingroup
    \count@=\HD@toclevel@index\relax
    \advance\count@ by 1 %
    \xdef\HD@toclevel@subindex{\the\count@}%
  \endgroup
  \def\index@prologue{%
    \ifHD@numbered
    \else
      \begingroup
        \stepcounter{HD@unique}%
        \let\Hy@writebookmark\HDorg@writebookmark
        \pdfbookmark[\HD@toclevel@index]%
            {\indexname}{index\theHD@unique}%
      \endgroup
    \fi
    \HDorg@index@prologue
  }%
  \HDorg@theindex
}
%    \end{macrocode}
%    \end{macro}
%    \begin{macro}{\HDorg@bfseries}
%    \begin{macrocode}
\let\HDorg@bfseries\bfseries
%    \end{macrocode}
%    \end{macro}
%    \begin{macro}{\HD@bfseries}
%    \begin{macrocode}
\def\HD@bfseries{\futurelet\HD@hfil\HD@@bfseries}%
%    \end{macrocode}
%    \end{macro}
%    \begin{macro}{\HD@@bfseries}
%    \begin{macrocode}
\def\HD@@bfseries{%
  \HDorg@bfseries
  \ifx\HD@hfil\hfil
    \expandafter\HD@@@bfseries
  \fi
}
%    \end{macrocode}
%    \end{macro}
%    \begin{macro}{\HD@@@bfseries}
%    \begin{macrocode}
\def\HD@@@bfseries\hfil#1\hfil{%
  \ifx\\#1\\%
  \else
    \raisebox{\baselineskip}[0pt]{%
      \kern-\HD@margin\relax
      \pdfbookmark[\HD@toclevel@subindex]{#1}{HD.#1}%
      \kern\HD@margin\relax
    }%
  \fi
  \hfil#1\hfil
}%
%    \end{macrocode}
%    \end{macro}
%    \begin{macro}{\HD@margin}
%    \begin{macrocode}
\def\HD@margin{1mm}
%    \end{macrocode}
%    \end{macro}
%
% \subsection{\cs{theglossary}}
%
%    \begin{macro}{\HDorg@theglossary}
%    \begin{macrocode}
\let\HDorg@theglossary\theglossary
%    \end{macrocode}
%    \end{macro}
%    \begin{macro}{\theglossary}
%    \begin{macrocode}
\def\theglossary{%
  \let\HDorg@glossary@prologue\glossary@prologue
  \let\HDorg@writebookmark\Hy@writebookmark
  \HD@sectionpatch
  \def\glossary@prologue{%
    \ifHD@numbered
    \else
      \begingroup
        \let\Hy@writebookmark\HDorg@writebookmark
        \stepcounter{HD@unique}%
        \pdfbookmark[\HD@guesstoclevel{\HDorg@glossary@prologue}]%
            {\changehistoryname}{glossary\theHD@unique}%
      \endgroup
    \fi
    \HDorg@glossary@prologue
  }%
  \HDorg@theglossary
}
%    \end{macrocode}
%    \end{macro}
%    \begin{macro}{\glossaryname}
%    \begin{macrocode}
\providecommand*{\glossaryname}{Change History}
%    \end{macrocode}
%    \end{macro}
%
%    \begin{macro}{\glossaryname}
%    \begin{macrocode}
\providecommand*{\changehistoryname}{\glossaryname}
%    \end{macrocode}
%    \end{macro}
%
% \subsection{Index}
%
%    \begin{macrocode}
\newcounter{HD@hypercount}
%    \end{macrocode}
%    \begin{macro}{\HD@helplength}
%    \begin{macrocode}
\newlength{\HD@helplength}
%    \end{macrocode}
%    \end{macro}
%    \begin{macro}{\ifHD@savedest}
%    \begin{macrocode}
\newif\ifHD@savedest
\HD@savedesttrue
%    \end{macrocode}
%    \end{macro}
%    \begin{macrocode}
\AtBeginDocument{%
  \HD@savedest@init
}
%    \end{macrocode}
%    \begin{macrocode}
\RequirePackage{rerunfilecheck}[2009/12/10]
%    \end{macrocode}
%    \begin{macro}{\HD@savedest@init}
%    \begin{macrocode}
\begingroup
  \catcode`\^^M=\active
\@firstofone{\endgroup
  \def\HD@savedest@init{%
    \ifHD@savedest
      \let\HDorg@tilde~%
      \let~\HD@savedest@entry
      \edef\HDorg@catcodeM{\the\catcode`\^^M}%
      \catcode`\^^M=\active
      \def^^M{}%
      \InputIfFileExists{\jobname.hd}{}{}%
      \let~\HDorg@tilde
      \catcode`\^^M=\HDorg@catcodeM\relax
      \newwrite\HD@savedest@out
      \RerunFileCheck{\jobname.hd}{%
        \immediate\closeout\HD@savedest@out
      }{}%
      \immediate\openout\HD@savedest@out=\jobname.hd\relax
    \else
      \let\HD@savedest@add\@gobble
    \fi
  }%
%    \end{macrocode}
%    \end{macro}
%    \begin{macro}{\HD@savedest@entry}
%    \begin{macrocode}
  \def\HD@savedest@entry#1^^M{%
    \@namedef{HD.#1}{}%
  }%
}
%    \end{macrocode}
%    \end{macro}
%    \begin{macro}{\HD@savedest@add}
%    \begin{macrocode}
\def\HD@savedest@add#1{%
  \immediate\write\HD@savedest@out{\string~#1}%
}
%    \end{macrocode}
%    \end{macro}
%    \begin{macro}{\ifHD@targetraise}
%    \begin{macrocode}
\newif\ifHD@targetraise
\HD@targetraisetrue
%    \end{macrocode}
%    \end{macro}
%    \begin{macro}{\HD@target}
%    \begin{macrocode}
\def\HD@target{%
  \ifHD@targetraise
    \ifhmode
      \settoheight{\HD@helplength}{[H]}%
      \addtolength{\HD@helplength}{0.4ex}%
      \raisebox{\HD@helplength}[0pt][0pt]{%
        \HD@maketarget
      }%
    \else
      \HD@maketarget
    \fi
  \else
    \HD@maketarget
  \fi
}
%    \end{macrocode}
%    \end{macro}
%    \begin{macro}{\HD@maketarget}
%    \begin{macrocode}
\def\HD@maketarget{%
  \stepcounter{HD@hypercount}%
  \def\HD@next{%
    \llap{\hypertarget{HD.\the\c@HD@hypercount}{}\,}%
  }%
  \ifHD@savedest
    \expandafter\ifx\csname HD.\the\c@HD@hypercount\endcsname\@empty
    \else
      \let\HD@next\relax
    \fi
  \fi
  \HD@next
}
%    \end{macrocode}
%    \end{macro}
%
%    \begin{macro}{\CodelineIndex}
%    \begin{macrocode}
\g@addto@macro\CodelineIndex{%
  \let\special@index\HD@codeline@wrindex
}
%    \end{macrocode}
%    \end{macro}
%    \begin{macro}{\PageIndex}
%    \begin{macrocode}
\g@addto@macro\PageIndex{%
  \let\special@index\HD@page@wrindex
}
%    \end{macrocode}
%    \end{macro}
%
%    \begin{macro}{\HD@codeline@wrindex}
%    \begin{macrocode}
\def\HD@codeline@wrindex#1{%
  \begingroup
    \let\HDorg@encapchar\encapchar
    \def\encapchar##1\encapchar##2\@nil{%
      \HDorg@encapchar
      hdclindex{\the\c@HD@hypercount}{##1}%
    }%
    \codeline@wrindex{%
      #1\encapchar\encapchar\@nil
    }%
  \endgroup
}
%    \end{macrocode}
%    \end{macro}
%    \begin{macro}{\HD@page@wrindex}
%    \begin{macrocode}
\def\HD@page@wrindex#1{%
  \begingroup
    \let\HDorg@encapchar\encapchar
    \def\encapchar##1\encapchar##2\@nil{%
      \HDorg@encapchar
      hdpindex{##1}%
    }%
    \HDorg@index{#1\encapchar\encapchar\@nil}%
  \endgroup
}
%    \end{macrocode}
%    \end{macro}
%
%    \begin{macro}{\hdclindex}
%    \begin{macrocode}
\def\hdclindex#1#2#3{%
  \csname\ifx\\#2\\relax\else#2\fi\endcsname{%
    \hyperlink{HD.#1}{#3}%
    \HD@savedest@add{#1}%
  }%
}
%    \end{macrocode}
%    \end{macro}
%    \begin{macro}{\hdpindex}
%    \begin{macrocode}
\def\hdpindex#1#2{%
  \csname\ifx\\#1\\relax\else#1\fi\endcsname{%
    \hyperpage{#2}%
  }%
}
%    \end{macrocode}
%    \end{macro}
%
% \subsection{\cs{PrintDescribeMacro}}
%
%    \begin{macro}{\HDorg@PrintDescribeMacro}
%    \begin{macrocode}
\let\HDorg@PrintDescribeMacro\PrintDescribeMacro
%    \end{macrocode}
%    \end{macro}
%    \begin{macro}{\PrintDescribeMacro}
%    \begin{macrocode}
\renewcommand\PrintDescribeMacro[1]{%
  \begingroup
    \let\HDorg@MacroFont\MacroFont
    \def\MacroFont{%
      \HD@target
      \HDorg@MacroFont
    }%
    \HDorg@PrintDescribeMacro{#1}%
  \endgroup
}
%    \end{macrocode}
%    \end{macro}
%
% \subsection{\cs{PrintMacroName}}
%
%    \begin{macro}{\HDorg@PrintMacroName}
%    \begin{macrocode}
\let\HDorg@PrintMacroName\PrintMacroName
%    \end{macrocode}
%    \end{macro}
%    \begin{macro}{\PrintMacroName}
%    \begin{macrocode}
\renewcommand\PrintMacroName[1]{%
  \begingroup
    \let\HDorg@MacroFont\MacroFont
    \def\MacroFont{%
      \HD@target
      \HDorg@MacroFont
    }%
    \HDorg@PrintMacroName{#1}%
  \endgroup
}
%    \end{macrocode}
%    \end{macro}
%
% \subsection{\cs{theCodelineNo}}
%
%    \begin{macro}{\HDorg@theCodelineNo}
%    \begin{macrocode}
\let\HDorg@theCodelineNo\theCodelineNo
%    \end{macrocode}
%    \end{macro}
%    \begin{macro}{\theCodelineNo}
%    \begin{macrocode}
\renewcommand*\theCodelineNo{%
  \settowidth{\HD@helplength}{\HDorg@theCodelineNo\ }%
  \ifdim\HD@helplength<\MacroIndent
    \addtolength{\HD@helplength}{-\MacroIndent}%
    \llap{%
      \begingroup\HD@target\endgroup
      \hspace*{-\HD@helplength}%
    }%
  \else
    \HD@target
  \fi
  \HDorg@theCodelineNo
}
%    \end{macrocode}
%    \end{macro}
%
%    \begin{macro}{\HDorg@index}
%    \begin{macrocode}
\AtBeginDocument{%
  \let\HDorg@index\index
}
%    \end{macrocode}
%    \end{macro}
%
% \subsection{\cs{SpecialUsageIndex}}
%
%    \begin{macro}{\HDorg@SpecialUsageIndex}
%    \begin{macrocode}
\let\HDorg@SpecialUsageIndex\SpecialUsageIndex
\renewcommand*\SpecialUsageIndex[1]{%
  \@bsphack
  \begingroup
    \HD@target
    \let\index\HDorg@index
    \let\HDorg@encapchar\encapchar
    \edef\encapchar usage{%
       \HDorg@encapchar hdclindex{\the\c@HD@hypercount}{usage}%
    }%
    \HDorg@SpecialUsageIndex{#1}%
  \endgroup
  \@esphack
}
%    \end{macrocode}
%    \end{macro}
%
% \subsection{\cs{autoindex}}
%
%    \begin{macrocode}
\begingroup\expandafter\expandafter\expandafter\endgroup
\expandafter\ifx\csname autoindex\endcsname\relax
\else
%    \end{macrocode}
%    \begin{macro}{\HDorg@autoindex}
%    \begin{macrocode}
  \let\HDorg@autoindex\autoindex
%    \end{macrocode}
%    \end{macro}
%    \begin{macro}{\autoindex}
%    \begin{macrocode}
  \renewcommand*\autoindex[1]{%
    \@bsphack
    \begingroup
      \HD@target
      \let\index\HDorg@index
      \let\HDorg@encapchar\encapchar
      \edef\encapchar usage{%
         \HDorg@encapchar hdclindex{\the\c@HD@hypercount}{usage}%
      }%
      \HDorg@autoindex{#1}%
    \endgroup
    \@esphack
  }%
\fi
%    \end{macrocode}
%    \end{macro}
%
% \subsection{\cs{SpecialEnvIndex}}
%
%    \begin{macro}{\HDorg@SpecialEnvIndex}
%    \begin{macrocode}
\let\HDorg@SpecialEnvIndex\SpecialEnvIndex
%    \end{macrocode}
%    \end{macro}
%    \begin{macro}{\SpecialEnvIndex}
%    \begin{macrocode}
\renewcommand*\SpecialEnvIndex[1]{%
  \@bsphack
  \begingroup
    \HD@target
    \let\HDorg@encapchar\encapchar
    \edef\encapchar usage{%
      \HDorg@encapchar hdclindex{\the\c@HD@hypercount}{usage}%
    }%
    \HDorg@SpecialEnvIndex{#1}%
  \endgroup
  \@esphack
}
%    \end{macrocode}
%    \end{macro}
%
% \subsection{\cs{SortIndex}}
%
%    \begin{macro}{\HDorg@SortIndex}
%    \begin{macrocode}
\let\HDorg@SortIndex\SortIndex
%    \end{macrocode}
%    \end{macro}
%    \begin{macro}{\SortIndex}
%    \begin{macrocode}
\renewcommand*{\SortIndex}[2]{%
  \@bsphack
  \begingroup
    \let\index\HD@page@wrindex
    \HDorg@SortIndex{#1}{#2}%
  \endgroup
  \@esphack
}
%    \end{macrocode}
%    \end{macro}
%
% \subsection{\cs{@wrglossary}}
%
%    \begin{macro}{\HDorg@wrglossary}
%    \begin{macrocode}
\let\HDorg@wrglossary\@wrglossary
%    \end{macrocode}
%    \end{macro}
%    \begin{macro}{\@wrglossary}
%    \begin{macrocode}
\def\@wrglossary#1{%
  \let\HDorg@encapchar\encapchar
  \def\encapchar##1\encapchar##2\@nil{%
    \HDorg@encapchar
    hdpindex{##1}%
  }%
  \HDorg@wrglossary{#1\encapchar\encapchar\@nil}%
}
%    \end{macrocode}
%    \end{macro}
%
% \subsection{\cs{MacroIndent}}
%
%    \cs{MacroIndent} automatically calculated.
%    \begin{macrocode}
\AfterLastShipout{%
  \if@filesw
    \immediate\write\@auxout{%
      \string\providecommand\string\HD@SetMacroIndent[1]{}%
    }%
    \immediate\write\@auxout{%
      \string\HD@SetMacroIndent{\number\value{CodelineNo}}%
    }%
  \fi
}
%    \end{macrocode}
%    \begin{macro}{\HD@SetMacroIndent}
%    \begin{macrocode}
\def\HD@SetMacroIndent#1{%
  \begingroup
    \value{CodelineNo}=#1\relax
    \settowidth{\MacroIndent}{\HDorg@theCodelineNo\ }%
    \global\MacroIndent\MacroIndent
  \endgroup
}
%    \end{macrocode}
%    \end{macro}
%
%    \begin{macrocode}
%</package>
%    \end{macrocode}
%
% \section{Installation}
%
% \subsection{Download}
%
% \paragraph{Package.} This package is available on
% CTAN\footnote{\url{http://ctan.org/pkg/hypdoc}}:
% \begin{description}
% \item[\CTAN{macros/latex/contrib/oberdiek/hypdoc.dtx}] The source file.
% \item[\CTAN{macros/latex/contrib/oberdiek/hypdoc.pdf}] Documentation.
% \end{description}
%
%
% \paragraph{Bundle.} All the packages of the bundle `oberdiek'
% are also available in a TDS compliant ZIP archive. There
% the packages are already unpacked and the documentation files
% are generated. The files and directories obey the TDS standard.
% \begin{description}
% \item[\CTAN{install/macros/latex/contrib/oberdiek.tds.zip}]
% \end{description}
% \emph{TDS} refers to the standard ``A Directory Structure
% for \TeX\ Files'' (\CTAN{tds/tds.pdf}). Directories
% with \xfile{texmf} in their name are usually organized this way.
%
% \subsection{Bundle installation}
%
% \paragraph{Unpacking.} Unpack the \xfile{oberdiek.tds.zip} in the
% TDS tree (also known as \xfile{texmf} tree) of your choice.
% Example (linux):
% \begin{quote}
%   |unzip oberdiek.tds.zip -d ~/texmf|
% \end{quote}
%
% \paragraph{Script installation.}
% Check the directory \xfile{TDS:scripts/oberdiek/} for
% scripts that need further installation steps.
% Package \xpackage{attachfile2} comes with the Perl script
% \xfile{pdfatfi.pl} that should be installed in such a way
% that it can be called as \texttt{pdfatfi}.
% Example (linux):
% \begin{quote}
%   |chmod +x scripts/oberdiek/pdfatfi.pl|\\
%   |cp scripts/oberdiek/pdfatfi.pl /usr/local/bin/|
% \end{quote}
%
% \subsection{Package installation}
%
% \paragraph{Unpacking.} The \xfile{.dtx} file is a self-extracting
% \docstrip\ archive. The files are extracted by running the
% \xfile{.dtx} through \plainTeX:
% \begin{quote}
%   \verb|tex hypdoc.dtx|
% \end{quote}
%
% \paragraph{TDS.} Now the different files must be moved into
% the different directories in your installation TDS tree
% (also known as \xfile{texmf} tree):
% \begin{quote}
% \def\t{^^A
% \begin{tabular}{@{}>{\ttfamily}l@{ $\rightarrow$ }>{\ttfamily}l@{}}
%   hypdoc.sty & tex/latex/oberdiek/hypdoc.sty\\
%   hypdoc.pdf & doc/latex/oberdiek/hypdoc.pdf\\
%   hypdoc.dtx & source/latex/oberdiek/hypdoc.dtx\\
% \end{tabular}^^A
% }^^A
% \sbox0{\t}^^A
% \ifdim\wd0>\linewidth
%   \begingroup
%     \advance\linewidth by\leftmargin
%     \advance\linewidth by\rightmargin
%   \edef\x{\endgroup
%     \def\noexpand\lw{\the\linewidth}^^A
%   }\x
%   \def\lwbox{^^A
%     \leavevmode
%     \hbox to \linewidth{^^A
%       \kern-\leftmargin\relax
%       \hss
%       \usebox0
%       \hss
%       \kern-\rightmargin\relax
%     }^^A
%   }^^A
%   \ifdim\wd0>\lw
%     \sbox0{\small\t}^^A
%     \ifdim\wd0>\linewidth
%       \ifdim\wd0>\lw
%         \sbox0{\footnotesize\t}^^A
%         \ifdim\wd0>\linewidth
%           \ifdim\wd0>\lw
%             \sbox0{\scriptsize\t}^^A
%             \ifdim\wd0>\linewidth
%               \ifdim\wd0>\lw
%                 \sbox0{\tiny\t}^^A
%                 \ifdim\wd0>\linewidth
%                   \lwbox
%                 \else
%                   \usebox0
%                 \fi
%               \else
%                 \lwbox
%               \fi
%             \else
%               \usebox0
%             \fi
%           \else
%             \lwbox
%           \fi
%         \else
%           \usebox0
%         \fi
%       \else
%         \lwbox
%       \fi
%     \else
%       \usebox0
%     \fi
%   \else
%     \lwbox
%   \fi
% \else
%   \usebox0
% \fi
% \end{quote}
% If you have a \xfile{docstrip.cfg} that configures and enables \docstrip's
% TDS installing feature, then some files can already be in the right
% place, see the documentation of \docstrip.
%
% \subsection{Refresh file name databases}
%
% If your \TeX~distribution
% (\teTeX, \mikTeX, \dots) relies on file name databases, you must refresh
% these. For example, \teTeX\ users run \verb|texhash| or
% \verb|mktexlsr|.
%
% \subsection{Some details for the interested}
%
% \paragraph{Attached source.}
%
% The PDF documentation on CTAN also includes the
% \xfile{.dtx} source file. It can be extracted by
% AcrobatReader 6 or higher. Another option is \textsf{pdftk},
% e.g. unpack the file into the current directory:
% \begin{quote}
%   \verb|pdftk hypdoc.pdf unpack_files output .|
% \end{quote}
%
% \paragraph{Unpacking with \LaTeX.}
% The \xfile{.dtx} chooses its action depending on the format:
% \begin{description}
% \item[\plainTeX:] Run \docstrip\ and extract the files.
% \item[\LaTeX:] Generate the documentation.
% \end{description}
% If you insist on using \LaTeX\ for \docstrip\ (really,
% \docstrip\ does not need \LaTeX), then inform the autodetect routine
% about your intention:
% \begin{quote}
%   \verb|latex \let\install=y\input{hypdoc.dtx}|
% \end{quote}
% Do not forget to quote the argument according to the demands
% of your shell.
%
% \paragraph{Generating the documentation.}
% You can use both the \xfile{.dtx} or the \xfile{.drv} to generate
% the documentation. The process can be configured by the
% configuration file \xfile{ltxdoc.cfg}. For instance, put this
% line into this file, if you want to have A4 as paper format:
% \begin{quote}
%   \verb|\PassOptionsToClass{a4paper}{article}|
% \end{quote}
% An example follows how to generate the
% documentation with pdf\LaTeX:
% \begin{quote}
%\begin{verbatim}
%pdflatex hypdoc.dtx
%makeindex -s gind.ist hypdoc.idx
%pdflatex hypdoc.dtx
%makeindex -s gind.ist hypdoc.idx
%pdflatex hypdoc.dtx
%\end{verbatim}
% \end{quote}
%
% \section{Catalogue}
%
% The following XML file can be used as source for the
% \href{http://mirror.ctan.org/help/Catalogue/catalogue.html}{\TeX\ Catalogue}.
% The elements \texttt{caption} and \texttt{description} are imported
% from the original XML file from the Catalogue.
% The name of the XML file in the Catalogue is \xfile{hypdoc.xml}.
%    \begin{macrocode}
%<*catalogue>
<?xml version='1.0' encoding='us-ascii'?>
<!DOCTYPE entry SYSTEM 'catalogue.dtd'>
<entry datestamp='$Date$' modifier='$Author$' id='hypdoc'>
  <name>hypdoc</name>
  <caption>Hyper extensions for doc.sty.</caption>
  <authorref id='auth:oberdiek'/>
  <copyright owner='Heiko Oberdiek' year='2002,2006,2007,2009-2011'/>
  <license type='lppl1.3'/>
  <version number='1.14'/>
  <description>
    This package adds hypertext features to the package
    <xref refid='doc'>doc</xref> that is used in the documentation
    system of <xref refid='latex'>LaTeX2e</xref>. Bookmarks are added
    and references are linked as far as possible.
    <p/>
    The package is part of the <xref refid='oberdiek'>oberdiek</xref>
    bundle.
  </description>
  <documentation details='Package documentation'
      href='ctan:/macros/latex/contrib/oberdiek/hypdoc.pdf'/>
  <ctan file='true' path='/macros/latex/contrib/oberdiek/hypdoc.dtx'/>
  <miktex location='oberdiek'/>
  <texlive location='oberdiek'/>
  <install path='/macros/latex/contrib/oberdiek/oberdiek.tds.zip'/>
</entry>
%</catalogue>
%    \end{macrocode}
%
% \begin{History}
%   \begin{Version}{2002/05/10 v1.0}
%   \item
%     First standalone version.
%   \end{Version}
%   \begin{Version}{2006/02/20 v1.1}
%   \item
%     Option \xoption{numbered} added.
%   \item
%     \cs{HD@savedest@init} fixed (empty \xext{hd} file triggered
%     first page with fl ligature).
%   \item
%     Bookmarks for \cs{listoffigures} and \cs{listoftables}.
%   \item
%     Documentation: features and usage added.
%   \item
%     LPPL 1.3.
%   \end{Version}
%   \begin{Version}{2006/04/27 v1.2}
%   \item
%     \cs{HD@target} fixed (\xfile{multicol.dtx}).
%   \end{Version}
%   \begin{Version}{2006/06/01 v1.3}
%   \item
%     Support for package \xpackage{amsdtx}'s \cs{autoindex} added.
%   \end{Version}
%   \begin{Version}{2006/06/02 v1.4}
%   \item
%     Bookmark fixed if there are several table of contents,
%     bibliographies, glossaries, \dots
%   \end{Version}
%   \begin{Version}{2007/04/11 v1.5}
%   \item
%     Line ends sanitized.
%   \end{Version}
%   \begin{Version}{2007/11/12 v1.6}
%   \item
%     Bug fix in \cs{listoffigures} (Axel Sommerfeldt).
%   \item
%     Markup added in implementation section.
%   \end{Version}
%   \begin{Version}{2009/12/10 v1.7}
%   \item
%     Use of package \xpackage{rerunfilecheck}.
%   \end{Version}
%   \begin{Version}{2010/02/03 v1.8}
%   \item
%     Fix for bug latex/4096 added.
%   \end{Version}
%   \begin{Version}{2010/03/26 v1.9}
%   \item
%     Definition of \cs{HD@SetMacroIndex} is provided in the \xext{aux} file.
%   \end{Version}
%   \begin{Version}{2011/05/05 v1.10}
%   \item
%     \cs{raisebox} in \cs{HD@target} with height and depth of \texttt{0pt}
%     (request of GL).
%   \end{Version}
%   \begin{Version}{2011/08/19 v1.11}
%   \item
%     Latest comma in option list for hyperref removed to avoid option clash
%     because of \hologo{LaTeX} 2011/06/27 with the new \cs{in@}.
%   \end{Version}
%   \begin{Version}{2016/05/16 v1.12}
%   \item
%     Documentation updates.
%   \end{Version}
%   \begin{Version}{2016/10/02 v1.13}
%   \item
%     Fixes to bookmarknames, also introduce \cs{changehistoryname}
%   \end{Version}
%   \begin{Version}{2018/09/07 v1.14}
%   \item
%     Do not load \xpackage{thumpdf}.
%   \end{Version}
% \end{History}
%
% \PrintIndex
%
% \Finale
\endinput
|
% \end{quote}
% Do not forget to quote the argument according to the demands
% of your shell.
%
% \paragraph{Generating the documentation.}
% You can use both the \xfile{.dtx} or the \xfile{.drv} to generate
% the documentation. The process can be configured by the
% configuration file \xfile{ltxdoc.cfg}. For instance, put this
% line into this file, if you want to have A4 as paper format:
% \begin{quote}
%   \verb|\PassOptionsToClass{a4paper}{article}|
% \end{quote}
% An example follows how to generate the
% documentation with pdf\LaTeX:
% \begin{quote}
%\begin{verbatim}
%pdflatex hypdoc.dtx
%makeindex -s gind.ist hypdoc.idx
%pdflatex hypdoc.dtx
%makeindex -s gind.ist hypdoc.idx
%pdflatex hypdoc.dtx
%\end{verbatim}
% \end{quote}
%
% \section{Catalogue}
%
% The following XML file can be used as source for the
% \href{http://mirror.ctan.org/help/Catalogue/catalogue.html}{\TeX\ Catalogue}.
% The elements \texttt{caption} and \texttt{description} are imported
% from the original XML file from the Catalogue.
% The name of the XML file in the Catalogue is \xfile{hypdoc.xml}.
%    \begin{macrocode}
%<*catalogue>
<?xml version='1.0' encoding='us-ascii'?>
<!DOCTYPE entry SYSTEM 'catalogue.dtd'>
<entry datestamp='$Date$' modifier='$Author$' id='hypdoc'>
  <name>hypdoc</name>
  <caption>Hyper extensions for doc.sty.</caption>
  <authorref id='auth:oberdiek'/>
  <copyright owner='Heiko Oberdiek' year='2002,2006,2007,2009-2011'/>
  <license type='lppl1.3'/>
  <version number='1.14'/>
  <description>
    This package adds hypertext features to the package
    <xref refid='doc'>doc</xref> that is used in the documentation
    system of <xref refid='latex'>LaTeX2e</xref>. Bookmarks are added
    and references are linked as far as possible.
    <p/>
    The package is part of the <xref refid='oberdiek'>oberdiek</xref>
    bundle.
  </description>
  <documentation details='Package documentation'
      href='ctan:/macros/latex/contrib/oberdiek/hypdoc.pdf'/>
  <ctan file='true' path='/macros/latex/contrib/oberdiek/hypdoc.dtx'/>
  <miktex location='oberdiek'/>
  <texlive location='oberdiek'/>
  <install path='/macros/latex/contrib/oberdiek/oberdiek.tds.zip'/>
</entry>
%</catalogue>
%    \end{macrocode}
%
% \begin{History}
%   \begin{Version}{2002/05/10 v1.0}
%   \item
%     First standalone version.
%   \end{Version}
%   \begin{Version}{2006/02/20 v1.1}
%   \item
%     Option \xoption{numbered} added.
%   \item
%     \cs{HD@savedest@init} fixed (empty \xext{hd} file triggered
%     first page with fl ligature).
%   \item
%     Bookmarks for \cs{listoffigures} and \cs{listoftables}.
%   \item
%     Documentation: features and usage added.
%   \item
%     LPPL 1.3.
%   \end{Version}
%   \begin{Version}{2006/04/27 v1.2}
%   \item
%     \cs{HD@target} fixed (\xfile{multicol.dtx}).
%   \end{Version}
%   \begin{Version}{2006/06/01 v1.3}
%   \item
%     Support for package \xpackage{amsdtx}'s \cs{autoindex} added.
%   \end{Version}
%   \begin{Version}{2006/06/02 v1.4}
%   \item
%     Bookmark fixed if there are several table of contents,
%     bibliographies, glossaries, \dots
%   \end{Version}
%   \begin{Version}{2007/04/11 v1.5}
%   \item
%     Line ends sanitized.
%   \end{Version}
%   \begin{Version}{2007/11/12 v1.6}
%   \item
%     Bug fix in \cs{listoffigures} (Axel Sommerfeldt).
%   \item
%     Markup added in implementation section.
%   \end{Version}
%   \begin{Version}{2009/12/10 v1.7}
%   \item
%     Use of package \xpackage{rerunfilecheck}.
%   \end{Version}
%   \begin{Version}{2010/02/03 v1.8}
%   \item
%     Fix for bug latex/4096 added.
%   \end{Version}
%   \begin{Version}{2010/03/26 v1.9}
%   \item
%     Definition of \cs{HD@SetMacroIndex} is provided in the \xext{aux} file.
%   \end{Version}
%   \begin{Version}{2011/05/05 v1.10}
%   \item
%     \cs{raisebox} in \cs{HD@target} with height and depth of \texttt{0pt}
%     (request of GL).
%   \end{Version}
%   \begin{Version}{2011/08/19 v1.11}
%   \item
%     Latest comma in option list for hyperref removed to avoid option clash
%     because of \hologo{LaTeX} 2011/06/27 with the new \cs{in@}.
%   \end{Version}
%   \begin{Version}{2016/05/16 v1.12}
%   \item
%     Documentation updates.
%   \end{Version}
%   \begin{Version}{2016/10/02 v1.13}
%   \item
%     Fixes to bookmarknames, also introduce \cs{changehistoryname}
%   \end{Version}
%   \begin{Version}{2018/09/07 v1.14}
%   \item
%     Do not load \xpackage{thumpdf}.
%   \end{Version}
% \end{History}
%
% \PrintIndex
%
% \Finale
\endinput
|
% \end{quote}
% Do not forget to quote the argument according to the demands
% of your shell.
%
% \paragraph{Generating the documentation.}
% You can use both the \xfile{.dtx} or the \xfile{.drv} to generate
% the documentation. The process can be configured by the
% configuration file \xfile{ltxdoc.cfg}. For instance, put this
% line into this file, if you want to have A4 as paper format:
% \begin{quote}
%   \verb|\PassOptionsToClass{a4paper}{article}|
% \end{quote}
% An example follows how to generate the
% documentation with pdf\LaTeX:
% \begin{quote}
%\begin{verbatim}
%pdflatex hypdoc.dtx
%makeindex -s gind.ist hypdoc.idx
%pdflatex hypdoc.dtx
%makeindex -s gind.ist hypdoc.idx
%pdflatex hypdoc.dtx
%\end{verbatim}
% \end{quote}
%
% \section{Catalogue}
%
% The following XML file can be used as source for the
% \href{http://mirror.ctan.org/help/Catalogue/catalogue.html}{\TeX\ Catalogue}.
% The elements \texttt{caption} and \texttt{description} are imported
% from the original XML file from the Catalogue.
% The name of the XML file in the Catalogue is \xfile{hypdoc.xml}.
%    \begin{macrocode}
%<*catalogue>
<?xml version='1.0' encoding='us-ascii'?>
<!DOCTYPE entry SYSTEM 'catalogue.dtd'>
<entry datestamp='$Date$' modifier='$Author$' id='hypdoc'>
  <name>hypdoc</name>
  <caption>Hyper extensions for doc.sty.</caption>
  <authorref id='auth:oberdiek'/>
  <copyright owner='Heiko Oberdiek' year='2002,2006,2007,2009-2011'/>
  <license type='lppl1.3'/>
  <version number='1.14'/>
  <description>
    This package adds hypertext features to the package
    <xref refid='doc'>doc</xref> that is used in the documentation
    system of <xref refid='latex'>LaTeX2e</xref>. Bookmarks are added
    and references are linked as far as possible.
    <p/>
    The package is part of the <xref refid='oberdiek'>oberdiek</xref>
    bundle.
  </description>
  <documentation details='Package documentation'
      href='ctan:/macros/latex/contrib/oberdiek/hypdoc.pdf'/>
  <ctan file='true' path='/macros/latex/contrib/oberdiek/hypdoc.dtx'/>
  <miktex location='oberdiek'/>
  <texlive location='oberdiek'/>
  <install path='/macros/latex/contrib/oberdiek/oberdiek.tds.zip'/>
</entry>
%</catalogue>
%    \end{macrocode}
%
% \begin{History}
%   \begin{Version}{2002/05/10 v1.0}
%   \item
%     First standalone version.
%   \end{Version}
%   \begin{Version}{2006/02/20 v1.1}
%   \item
%     Option \xoption{numbered} added.
%   \item
%     \cs{HD@savedest@init} fixed (empty \xext{hd} file triggered
%     first page with fl ligature).
%   \item
%     Bookmarks for \cs{listoffigures} and \cs{listoftables}.
%   \item
%     Documentation: features and usage added.
%   \item
%     LPPL 1.3.
%   \end{Version}
%   \begin{Version}{2006/04/27 v1.2}
%   \item
%     \cs{HD@target} fixed (\xfile{multicol.dtx}).
%   \end{Version}
%   \begin{Version}{2006/06/01 v1.3}
%   \item
%     Support for package \xpackage{amsdtx}'s \cs{autoindex} added.
%   \end{Version}
%   \begin{Version}{2006/06/02 v1.4}
%   \item
%     Bookmark fixed if there are several table of contents,
%     bibliographies, glossaries, \dots
%   \end{Version}
%   \begin{Version}{2007/04/11 v1.5}
%   \item
%     Line ends sanitized.
%   \end{Version}
%   \begin{Version}{2007/11/12 v1.6}
%   \item
%     Bug fix in \cs{listoffigures} (Axel Sommerfeldt).
%   \item
%     Markup added in implementation section.
%   \end{Version}
%   \begin{Version}{2009/12/10 v1.7}
%   \item
%     Use of package \xpackage{rerunfilecheck}.
%   \end{Version}
%   \begin{Version}{2010/02/03 v1.8}
%   \item
%     Fix for bug latex/4096 added.
%   \end{Version}
%   \begin{Version}{2010/03/26 v1.9}
%   \item
%     Definition of \cs{HD@SetMacroIndex} is provided in the \xext{aux} file.
%   \end{Version}
%   \begin{Version}{2011/05/05 v1.10}
%   \item
%     \cs{raisebox} in \cs{HD@target} with height and depth of \texttt{0pt}
%     (request of GL).
%   \end{Version}
%   \begin{Version}{2011/08/19 v1.11}
%   \item
%     Latest comma in option list for hyperref removed to avoid option clash
%     because of \hologo{LaTeX} 2011/06/27 with the new \cs{in@}.
%   \end{Version}
%   \begin{Version}{2016/05/16 v1.12}
%   \item
%     Documentation updates.
%   \end{Version}
%   \begin{Version}{2016/10/02 v1.13}
%   \item
%     Fixes to bookmarknames, also introduce \cs{changehistoryname}
%   \end{Version}
%   \begin{Version}{2018/09/07 v1.14}
%   \item
%     Do not load \xpackage{thumpdf}.
%   \end{Version}
% \end{History}
%
% \PrintIndex
%
% \Finale
\endinput
|
% \end{quote}
% Do not forget to quote the argument according to the demands
% of your shell.
%
% \paragraph{Generating the documentation.}
% You can use both the \xfile{.dtx} or the \xfile{.drv} to generate
% the documentation. The process can be configured by the
% configuration file \xfile{ltxdoc.cfg}. For instance, put this
% line into this file, if you want to have A4 as paper format:
% \begin{quote}
%   \verb|\PassOptionsToClass{a4paper}{article}|
% \end{quote}
% An example follows how to generate the
% documentation with pdf\LaTeX:
% \begin{quote}
%\begin{verbatim}
%pdflatex hypdoc.dtx
%makeindex -s gind.ist hypdoc.idx
%pdflatex hypdoc.dtx
%makeindex -s gind.ist hypdoc.idx
%pdflatex hypdoc.dtx
%\end{verbatim}
% \end{quote}
%
% \section{Catalogue}
%
% The following XML file can be used as source for the
% \href{http://mirror.ctan.org/help/Catalogue/catalogue.html}{\TeX\ Catalogue}.
% The elements \texttt{caption} and \texttt{description} are imported
% from the original XML file from the Catalogue.
% The name of the XML file in the Catalogue is \xfile{hypdoc.xml}.
%    \begin{macrocode}
%<*catalogue>
<?xml version='1.0' encoding='us-ascii'?>
<!DOCTYPE entry SYSTEM 'catalogue.dtd'>
<entry datestamp='$Date$' modifier='$Author$' id='hypdoc'>
  <name>hypdoc</name>
  <caption>Hyper extensions for doc.sty.</caption>
  <authorref id='auth:oberdiek'/>
  <copyright owner='Heiko Oberdiek' year='2002,2006,2007,2009-2011'/>
  <license type='lppl1.3'/>
  <version number='1.14'/>
  <description>
    This package adds hypertext features to the package
    <xref refid='doc'>doc</xref> that is used in the documentation
    system of <xref refid='latex'>LaTeX2e</xref>. Bookmarks are added
    and references are linked as far as possible.
    <p/>
    The package is part of the <xref refid='oberdiek'>oberdiek</xref>
    bundle.
  </description>
  <documentation details='Package documentation'
      href='ctan:/macros/latex/contrib/oberdiek/hypdoc.pdf'/>
  <ctan file='true' path='/macros/latex/contrib/oberdiek/hypdoc.dtx'/>
  <miktex location='oberdiek'/>
  <texlive location='oberdiek'/>
  <install path='/macros/latex/contrib/oberdiek/oberdiek.tds.zip'/>
</entry>
%</catalogue>
%    \end{macrocode}
%
% \begin{History}
%   \begin{Version}{2002/05/10 v1.0}
%   \item
%     First standalone version.
%   \end{Version}
%   \begin{Version}{2006/02/20 v1.1}
%   \item
%     Option \xoption{numbered} added.
%   \item
%     \cs{HD@savedest@init} fixed (empty \xext{hd} file triggered
%     first page with fl ligature).
%   \item
%     Bookmarks for \cs{listoffigures} and \cs{listoftables}.
%   \item
%     Documentation: features and usage added.
%   \item
%     LPPL 1.3.
%   \end{Version}
%   \begin{Version}{2006/04/27 v1.2}
%   \item
%     \cs{HD@target} fixed (\xfile{multicol.dtx}).
%   \end{Version}
%   \begin{Version}{2006/06/01 v1.3}
%   \item
%     Support for package \xpackage{amsdtx}'s \cs{autoindex} added.
%   \end{Version}
%   \begin{Version}{2006/06/02 v1.4}
%   \item
%     Bookmark fixed if there are several table of contents,
%     bibliographies, glossaries, \dots
%   \end{Version}
%   \begin{Version}{2007/04/11 v1.5}
%   \item
%     Line ends sanitized.
%   \end{Version}
%   \begin{Version}{2007/11/12 v1.6}
%   \item
%     Bug fix in \cs{listoffigures} (Axel Sommerfeldt).
%   \item
%     Markup added in implementation section.
%   \end{Version}
%   \begin{Version}{2009/12/10 v1.7}
%   \item
%     Use of package \xpackage{rerunfilecheck}.
%   \end{Version}
%   \begin{Version}{2010/02/03 v1.8}
%   \item
%     Fix for bug latex/4096 added.
%   \end{Version}
%   \begin{Version}{2010/03/26 v1.9}
%   \item
%     Definition of \cs{HD@SetMacroIndex} is provided in the \xext{aux} file.
%   \end{Version}
%   \begin{Version}{2011/05/05 v1.10}
%   \item
%     \cs{raisebox} in \cs{HD@target} with height and depth of \texttt{0pt}
%     (request of GL).
%   \end{Version}
%   \begin{Version}{2011/08/19 v1.11}
%   \item
%     Latest comma in option list for hyperref removed to avoid option clash
%     because of \hologo{LaTeX} 2011/06/27 with the new \cs{in@}.
%   \end{Version}
%   \begin{Version}{2016/05/16 v1.12}
%   \item
%     Documentation updates.
%   \end{Version}
%   \begin{Version}{2016/10/02 v1.13}
%   \item
%     Fixes to bookmarknames, also introduce \cs{changehistoryname}
%   \end{Version}
%   \begin{Version}{2018/09/07 v1.14}
%   \item
%     Do not load \xpackage{thumpdf}.
%   \end{Version}
% \end{History}
%
% \PrintIndex
%
% \Finale
\endinput
